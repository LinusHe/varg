%%%%%%%%%%%%  Generated using docx2latex.com  %%%%%%%%%%%%%%

%%%%%%%%%%%%  v2.0.0-beta  %%%%%%%%%%%%%%

\documentclass[12pt]{article}
\usepackage{amsmath}
\usepackage{latexsym}
\usepackage{amsfonts}
\usepackage[normalem]{ulem}
\usepackage{soul}
\usepackage{array}
\usepackage{amssymb}
\usepackage{extarrows}
\usepackage{graphicx}
\usepackage[backend=biber,
style=numeric,
sorting=none,
isbn=false,
doi=false,
url=false,
]{biblatex}\addbibresource{bibliography.bib}

\usepackage{subfig}
\usepackage{wrapfig}
\usepackage{wasysym}
\usepackage{enumitem}
\usepackage{adjustbox}
\usepackage{ragged2e}
\usepackage[svgnames,table]{xcolor}
\usepackage{tikz}
\usepackage{longtable}
\usepackage{changepage}
\usepackage{setspace}
\usepackage{hhline}
\usepackage{multicol}
\usepackage{tabto}
\usepackage{float}
\usepackage{multirow}
\usepackage{makecell}
\usepackage{fancyhdr}
\usepackage[toc,page]{appendix}
\usepackage[hidelinks]{hyperref}
\usetikzlibrary{shapes.symbols,shapes.geometric,shadows,arrows.meta}
\tikzset{>={Latex[width=1.5mm,length=2mm]}}
\usepackage{flowchart}\usepackage[paperheight=11.69in,paperwidth=8.27in,left=0.98in,right=0.98in,top=0.98in,bottom=0.79in,headheight=1in]{geometry}
\usepackage[utf8]{inputenc}
\usepackage[T1]{fontenc}
\usepackage[german]{babel}
\TabPositions{0.49in,0.98in,1.47in,1.96in,2.45in,2.94in,3.43in,3.92in,4.41in,4.9in,5.39in,5.88in,}

\urlstyle{same}

\renewcommand{\_}{\kern-1.5pt\textunderscore\kern-1.5pt}

 %%%%%%%%%%%%  Set Depths for Sections  %%%%%%%%%%%%%%

% 1) Section
% 1.1) SubSection
% 1.1.1) SubSubSection
% 1.1.1.1) Paragraph
% 1.1.1.1.1) Subparagraph


\setcounter{tocdepth}{5}
\setcounter{secnumdepth}{5}


 %%%%%%%%%%%%  Set Depths for Nested Lists created by \begin{enumerate}  %%%%%%%%%%%%%%


\setlistdepth{9}
\renewlist{enumerate}{enumerate}{9}
		\setlist[enumerate,1]{label=\arabic*)}
		\setlist[enumerate,2]{label=\alph*)}
		\setlist[enumerate,3]{label=(\roman*)}
		\setlist[enumerate,4]{label=(\arabic*)}
		\setlist[enumerate,5]{label=(\Alph*)}
		\setlist[enumerate,6]{label=(\Roman*)}
		\setlist[enumerate,7]{label=\arabic*}
		\setlist[enumerate,8]{label=\alph*}
		\setlist[enumerate,9]{label=\roman*}

\renewlist{itemize}{itemize}{9}
		\setlist[itemize]{label=$\cdot$}
		\setlist[itemize,1]{label=\textbullet}
		\setlist[itemize,2]{label=$\circ$}
		\setlist[itemize,3]{label=$\ast$}
		\setlist[itemize,4]{label=$\dagger$}
		\setlist[itemize,5]{label=$\triangleright$}
		\setlist[itemize,6]{label=$\bigstar$}
		\setlist[itemize,7]{label=$\blacklozenge$}
		\setlist[itemize,8]{label=$\prime$}

\setlength{\topsep}{0pt}\setlength{\parskip}{8.04pt}
\setlength{\parindent}{0pt}

 %%%%%%%%%%%%  This sets linespacing (verticle gap between Lines) Default=1 %%%%%%%%%%%%%%


\renewcommand{\arraystretch}{1.3}


%%%%%%%%%%%%%%%%%%%% Document code starts here %%%%%%%%%%%%%%%%%%%%



\begin{document}
Coding Style\par


\vspace{\baselineskip}
Beim Schreiben unseres Programmcodes haben wir uns an einigen Coding Conventions gehalten.\par

\begin{itemize}
	\item Zeilenlänge: maximal 80 Zeichen\par

	\item Kommentare und Dokumentation\par

\begin{itemize}
	\item Kommentare auf Englisch\par

	\item Klassen und Methoden in kurzen, prägnanten Sätzen beschreiben\par

	\item Unnötige Kommentare vermeiden\par

	\item Kommentare aktuell halten\par


\end{itemize}
	\item Einrückung und Zeilenumbrüche\par

\begin{itemize}
	\item 2 Leerzeichen statt Tabulator\par

	\item ‚$ \{ $ ‘ hinter Methodendeklaration\par

	\item ‚$ \} $ ‘ in neuer Zeile auf gleiche Einrückungsebene\par

	\item Optionale Zeilenumbrüche für Übersichtlichkeit\par

	\item Nur ein Import pro Zeile\par


\end{itemize}
	\item Leerzeichen\par

\begin{itemize}
	\item Vor und nach binären Operationen\par

\begin{itemize}
	\item Ausnahme nur im Fall von Verdeutlichung unterschiedlicher Prioritäten\par


\end{itemize}
	\item Keine Leerzeichen vor und nach Klammern\par

	\item Keine Leerzeichen vor Kommata und Semikolon\par

	\item Leerzeichen nach Kommata\par

	\item Keine Leerzeichen am Zeilenende\par


\end{itemize}
	\item Konsistentes Benennungsschema\par

\begin{itemize}
	\item Deskriptive Namen verwenden\par

	\item mixedCase für Variablen\par

	\item GROßSCHREIBUNG für Konstanten\par

	\item Keine Umlaute\par

	\item Reservierte Schlüsselwörter beachten\par

	\item Immer auf Englisch\par

	\item Bezeichner von Booleanwerte sollen Zustand beschreiben, der wahr oder falsch sein kann\par

	\item Hilfsvariablen möglichst gleich benennen\par

	\item Übergabe von Attributen an Konstruktoren\par

\begin{itemize}
	\item ‚length‘ als Attribut, ‚\_length‘ als Argument\par


\end{itemize}
\end{itemize}
	\item Textcodierung UTF-8
\end{itemize}\par


\printbibliography
\end{document}