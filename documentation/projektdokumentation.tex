\documentclass[twoside]{report}

% ------
% Umlaute
\usepackage{ifluatex,ifxetex}
\ifluatex
  \usepackage{fontspec}
\else
  \ifxetex
    \usepackage{fontspec}
  \else
    \usepackage{selinput}
    \SelectInputMappings{
      adieresis={ä},
      germandbls={ß},
    }
    \usepackage[T1]{fontenc}
    %\usepackage{textcomp}% optional
    %\usepackage{lmodern}
  \fi
\fi

% ------
% Paper auf Deutsch
\usepackage[ngerman]{babel}



% ------
% Page layout
\usepackage[hmarginratio=1:1,top=32mm,columnsep=20pt]{geometry}
\usepackage[font=it]{caption}
\usepackage{paralist}
%\usepackage{multicol}


% ------
% Abstract
\usepackage{abstract}
	\renewcommand{\abstractnamefont}{\normalfont\bfseries}
	\renewcommand{\abstracttextfont}{\normalfont\small\itshape}


% ------
% Titling (section/subsection)
\usepackage{titlesec}
\renewcommand\thesection{\Roman{section}}
\titleformat{\section}[block]{\Large\scshape\bfseries}{\thesection.}{1em}{}
\setcounter{secnumdepth}{3}

% ------
% Tabellen über Seitenumbrüche hinweg
\usepackage{longtable}

% ------
% Header/footer
\usepackage{fancyhdr}
	\pagestyle{fancy}
	\fancyhead{}
	\fancyfoot{}
	\fancyhead[C]{Projektdokumentation $\bullet$ PROJEKTNAME $\bullet$ SS17$+$WS17/18}
	\fancyfoot[RO,LE]{}


% ------
% Clickable URLs (optional)
% \usepackage{hyperref}

% ------
% Literaturverweise mit Bibtex einbinden
\usepackage[authoryear,sectionbib,round]{natbib}

% ------
% Bilder laden
\usepackage[pdftex]{graphicx}

% ------
% Maketitle metadata
\title{\vspace{-5mm}%
	\fontsize{24pt}{10pt}\selectfont
	\textbf{Projektdokumentation}
	}	
\author{%
        % alle Autoren hier listen
        % 
	\large
	\textsc{Autor I -- E-Mail} \\[2mm]
	\textsc{Autor II -- E-Mail} \\[2mm]
	\normalsize	HTWK Leipzig 
	}
\date{}



%%%%%%%%%%%%%%%%%%%%%%%%
\begin{document}


% -------
% Titel und Abstract über beide Spalten
%\twocolumn[
%\begin{@twocolumnfalse}

\maketitle
\thispagestyle{fancy}

\tableofcontents

%%%%
%%%% Die Struktur des Dokuments bitte nicht aendern!!!
%%%%

\section{Anforderungsspezifikation}

\subsection{Initiale Kundenvorgaben}
{\small Autor: xxx}

Maecenas sed ultricies felis. Sed imperdiet dictum arcu a egestas.
In sapien ante, ultricies quis pellentesque ut, fringilla id sem. Proin justo libero, dapibus consequat auctor at, euismod et erat. Sed ut ipsum erat, iaculis vehicula lorem. Cras non dolor id libero blandit ornare. Pellentesque luctus fermentum eros ut posuere. Suspendisse rutrum suscipit massa sit amet molestie. Donec suscipit lacinia diam, eu posuere libero rutrum sed. Nam blandit lorem sit amet dolor vestibulum in lacinia purus varius. Ut tortor massa, rhoncus ut auctor eget, vestibulum ut justo.


\subsection{Produktvision}
{\small Autor: Alex Hofmann}
\\

\noindent Product Vision Board: \\
\begin{tabular}{|p{50mm}|p{50mm}|p{50mm}|}
  \hline
  \textbf{Target Group}                                                  & \textbf{Needs}                                                                                                                        & \textbf{Product}                                                                                                                                                                                                 \\
  \hline
  -Maschinenbau-Studenten \newline Maschinenbau-Profs \newline -Lehrende & Vgl. zu händisch: \newline einheitlicher, schneller \newline -plattformunabhängig \newline -Open Source \newline -Einfach zu bedienen & -Webanwendung \newline -Als Graph \newline $\rightarrow$ quasi als Baukasten \newline $\rightarrow$ Kantengewichtung, Bausteine wählbar \newline -Import/Export von Modellen \newline Normalisierung des Graphen \\
  \hline
\end{tabular}
\\

\noindent Die Webanwendung VarG ist entwickelt für Lehrende und Lernende aus dem Maschinenbau Bachelorstudiengang.
Diese erleichtert die einheitliche Erstellung, Bearbeitung, Optimierung sowie Im- bzw. Exportierung von sogenannten Variantenfolgegraphen. Darunter ist eine graphische Übersicht zu verstehen, die die möglichen Varianten eines Produktionsprozesses für ein Werkstück darstellt.



% Das hier ist ein Absatz, der die Grafik in Abbildung~\ref{fig:bild1} detailliert erläutert, erklärt und interpretiert.

% \begin{figure}[b]
%   \centering
%   \includegraphics[width=4.5cm]{bspbild1.png}
%   \caption{Beispiel für ein einspaltiges Bild}
%   \label{fig:bild1}
% \end{figure}


\subsection{Liste der funktionalen Anforderungen}

XXX

%
% soll der Inhalt dieser Subsection in einer separaten Datei
% (z.B. listefunktional.tex) liegen, dann kann dies mit dem
% folgenden Kommando geschehen.
%
% \input{listefunktional}

\subsection{Liste der nicht-funktionalen Anforderungen}
{\small Autor: xxx}

XXX

\subsection{Weitere Zuarbeiten zum Produktvisions-Workshop}

XXX

\subsubsection{Zuarbeit von Autor X}
XXX
\subsubsection{Zuarbeit von Autor Y}
XXX

\subsection{Liste der Kundengespräche mit Ergebnissen}
{\small Autor: xxx}

XXX



\section{Architektur und Entwurf}

\subsection{Zuarbeiten der Teammitglieder}

XXX

\subsection{Entscheidungen des Technologieworkshops}
{\small Autor: xxx}

XXX

\subsection{Überblick über Architektur}
{\small Autor: xxx}

XXX

\subsection{Definierte Schnittstellen}
{\small Autor: xxx}

XXX

\subsection{Liste der Architekturentscheidungen}
{\small Autor: xxx}

XXX (bewusste und unbewusste Entscheidungen mit zeitlicher Einordnung)



\section{Prozess- und Implementationsvorgaben}

\subsection{Definition of Done}
{\small Autor: xxx}

XXX

\subsection{Coding Style}
{\small Autor: xxx}

XXX

\subsection{Zu nutzende Werkzeuge}
{\small Autor: xxx}

XXX

\newpage

%%%%%%%%%%%%
%% Abschnitt mit den Sprints beginnt hier
%%%%%%%%%%%%

\section{Sprint 1}


\subsection{Ziel des Sprints}
{\small Autor: Erik Heldt}

Der erste Sprint des VarG-Projekts lief vom 05.12.2019 bis zum 16.12.2019. Ziel war es, eine fundamentale Struktur und grundlegende Funktionalitäten für die Anwendung zu entwickeln, auf denen man später weiter aufbauen kann. Währenddessen konnte man allgemeine Erfahrungen mit dem Ablauf eines Sprints machen.

\subsection{User-Stories des Sprint-Backlogs}
{\small Autor: Erik Heldt}

\textbf{Grundstruktur}
Die Anwendung sollte zu Beginn ein grundlegendes Fundament aufweisen, damit sich alle Teammitglieder vorstellen können, wie am Ende das Programm aussehen soll. Dazu gehörte zu Beginn das Design der Startseite mit dem VarGraph im Zentrum und der Einbindung von Cytoscape in die Programmstruktur.

\textbf{Datenstruktur für Knoten}
Es sollte mit Hilfe von Cytoscape herausgefunden werden, wie man Knoten im Programmcode hinzufügen und speichern kann. Dafür sollte dann eine Datei im Programm angelegt werden.

\textbf{Knoten zu bestehender Datenstruktur hinzufügen}
Die Anwendung sollte eine einfache Funktionalität zum Erstellen neuer Knoten aka Produktionsschritte erhalten, um sich mit den Cytoscape-Funktionen näher vertraut zu machen. Hier war erstmal noch keine graphische Darstellung in der GUI notwendig, es reichte per Console logs zu testen.

\textbf{Darstellung eines Graphen in Weboberfläche}
In der Anwendung sollte zunächst ein statischer Graph mit Hilfe einer Cytoscape-Datenstruktur sichtbar dargestellt werden, damit man sehen konnte, wie so ein „CytoGraph“ überhaupt aussieht. User-Interaktion war hier noch nicht notwendig.

\textbf{Kanten anlegen}
Zusätzlich zu Knoten sollten auch Kanten zwischen bestehenden Knoten hinzugefügt werden können. Diese Kanten sollten mit verschiedenen Attributen in der Cytoscape-Datenstruktur gespeichert werden.

\textbf{Berechnung verschiedener Eigenschaften}
Anhand der mit den Kanten gespeicherten Attribute sollte eine Funktionalität entwickelt werden, welche die Gesamtkosten (Auswahl von Geld oder Zeit) aller unterschiedlichen Pfade berechnen und anzeigen sollte. Dies war der erste Schritt in Richtung Optimierung, d.h. später sollte diese Funktionalität automatisch den günstigsten Pfad herausfinden und anzeigen.

\subsection{Liste der durchgeführten Meetings}
{\small Autor: Erik Heldt}

\begin{itemize}
	\item Planning - 05.12.2019
	\item Weekly Scrum 1 - 09.12.2019
	\item Weekly Scrum 2 - 12.12.2019
	\item Review - 16.12.2019
	\item Retrospektive - 19.12.2019
\end{itemize}

\subsection{Ergebnisse des Planning-Meetings}
{\small Autor: Erik Heldt}

Im Planning-Meeting erklärten die Projektmanager zu Beginn noch einmal kurz, wie ein Sprint im Allgemeinen abläuft und haben auf die Bedeutsamkeit der Coding Guidelines hingewiesen. Anschließend wurden die ersten User-Stories vom Project Owner vorgestellt und von den Bachelorstudenten per Finger-System in ihrer Komplexität eingeschätzt. Weiterhin wurde festgelegt, dass die Bachelorstudenten während des Sprints die User-Stories selbst in Tasks aufteilen und diese dann bearbeiten sollen.

\subsection{Aufgewendete Arbeitszeit pro Person$+$Arbeitspaket}
{\small Autor: xxx}

\begin{longtable}{|p{4cm}|l|l|l|l|l|}
        \hline
	Arbeitspaket & Person & Start & Ende & h & Artefakt\\
        \hline
	Vue.js "Getting Started" Tutorial durcharbeiten (für alle) & Buchmann, Lennart & 07.12.19 & 07.12.19 & 3 & Tutorial abgeschlossen\\ \hline
	Beispielgraph erstellen & Buxel, Nils & 09.12.19 & 09.12.19 & 1 & index.js\\ \hline
	Kürzesten Weg mit A*-Algorithm berechnen u anzeigen lassen & Buxel, Nils &16.12.19 & 16.12.19 & 1 & index.js\\ \hline
	Funktionen zu Buttons hinzufügen & Gwozdz, Jonas & 14.12.19 & 16.12.19 & 4 & MenuControls.vue\\ \hline
	Task: Einbindung in Vue-Dateistruktur & Heldt, Erik & 15.12.19 & 15.12.19 & 3 & MenuControls.vue, BasicData.js\\ \hline
	Graphenanordnung & Heldt, Erik & 05.12.19 & 05.12.19 & 3 & Graphenanordnung.pdf\\ \hline
	Vue.js "Getting Started" Tutorial durcharbeiten (für alle) & Heldt, Erik & 11.12.19 & 11.12.19 & 2 & Tutorial abgeschlossen\\ \hline
	Funktionen zu Buttons hinzufügen & Henning, Tim & 10.12.19 & 10.12.19 & 2 & MenuControls.vue\\ \hline
	Vue.js "Getting Started" Tutorial durcharbeiten (für alle) & Henning, Tim & 06.12.19 & 06.12.19 & 3 & Tutorial abgeschlossen\\ \hline
	Einbindung von Cytoscape in Vue & Herterich, Linus & 10.12.19 & 10.12.19 & 4 & index.js\\ \hline
	Buttons für Knoten und Kantenerstellung & Herterich, Linus & 13.12.19 & 13.12.19 & 3 & CreateControls.vue\\ \hline
	Knoten zu Graph hinzufügen & Herterich, Linus & 16.12.19 & 16.12.19 & 2,5 & index.js, CreateControls.vue\\ \hline
	Grundstruktur aufbauen & Herterich, Linus & 05.12.19 & 07.12.19 & 9,5 & Vue-Dateistruktur, sämtliche Startkomponenten\\ \hline
	Task: Basic Datenstruktur & Hohlfeld, Julius & 15.12.19 & 15.12.19 & 8 & BasicData.js, MenuControls.vue\\ \hline
      \end{longtable}

\subsection{Konkrete Code-Qualität im Sprint}
{\small Autor: Erik Heldt}

Zu Beginn wurde viel experimentiert und hauptsächlich sollte der Code erstmal ein funktionierendes Programm erzeugen, weswegen weniger auf die Qualität geachtet wurde. Trotzdem wurde sich größtenteils an die Coding Conventions gehalten und bereits einige Kommentare verfasst.

\subsection{Konkrete Test-Überdeckung im Sprint}
{\small Autor: Erik Heldt}

Da der erste Sprint größtenteils nur zur Erstellung einer grundlegenden Datenstruktur und zur Einarbeitung in JavaScript und den genutzten Frameworks bzw. Bibliotheken gedient hat, gab es noch keine Tests.

\subsection{Ergebnisse des Reviews}
{\small Autor: Erik Heldt}

Im ersten Review-Meeting stellten die Bachelorstudenten ihre Ergebnisse aus dem Sprint vor und die Manager gaben ihr Feedback dazu. Da sich die meisten Teammitglieder noch nicht richtig in Vue.js und Cytoscape einarbeiten konnten und teilweise große Schwierigkeiten mit den Frameworks hatten, gab es noch viele offene Aufgaben und nicht jeder hatte etwas vorzuzeigen.
Als erstes stellten Julius H. und Erik die Datenstruktur für die Knoten vor. Weiterhin zeigte Julius, wie ein Knoten in der Anwendung dargestellt wird und dass dieser durch ungeschickte Verschiebung und Skalierung aus der GUI verschwinden kann. Deshalb kamen Vorschläge, zukünftig den Zoom zu limitieren und das grundsätzliche Graph-Layout nochmal zu überarbeiten.
Um allen den Einstieg in die neuen Programmiersprachen und Bibliotheken etwas zu vereinfachen, stellte daraufhin Linus die Grundstruktur vor und erklärte noch einmal genau die einzelnen Elemente in der Dateistruktur. Weiterhin zeigte er, wie man ESLint-Fehler bei der Konsolenausgabe verhindern kann.
Danach wurde zwischen den Managern und den Bachelorstudenten noch die zukünftige Berechnung der kürzesten Wege und die unbearbeiteten User-Stories besprochen und dass diese in den nächsten Sprint mit einfließen werden.
Zum Schluss wurden noch ein paar allgemeine Fragen zum Testen und zu Git geklärt.

\subsection{Ergebnisse der Retrospektive}
{\small Autor: Erik Heldt}

In der Retrospektive konnte jedes Teammitglied vor an die Tafel gehen und verschiedene Aspekte des Sprints mit einem Strich in einer Tabelle bewerten.
Die Bewertung ging ausgeglichen aus. Die Gruppenleistung und das Gesamtergebnis waren gut, aber die Einzelleistungen der meisten Teammitglieder nicht. Viele Aufgaben blieben offen und wurden nicht erledigt, wozu in der Diskussion verschiedene Gründe angeführt wurden. Einerseits war es für die meisten schwer, sich selbst in die neue Programmierumgebung samt den Frameworks und Bibliotheken einzuarbeiten. Andererseits wussten viele nicht, was und wie viel sie machen sollten, was auf die nicht festgelegte Aufgabenzuteilung im Planning und die schlechte Kommunikation im Team während des Sprints zurückgeführt wurde. Letzteres Problem plante man damit zu lösen, in zukünftigen Plannings immer direkt Verantwortliche für bestimmte User-Stories festzulegen und entsprechende Tickets sofort im Anschluss zu erstellen und zuzuweisen.
Beim Thema der Daily Meetings ist man zu dem Schluss gekommen, dass diese wenn möglich immer persönlich bleiben sollten und nur in Ausnahmefällen online z.B. über Discord stattfinden sollten. Weiterhin wurde diskutiert, ob die Zeitspanne zwischen Donnerstag und Montag evtl. zu kurz ist, um schon weitreichende Ergebnisse zu erzielen, da am Wochenende einige Teammitglieder nicht programmieren können. Deshalb sollten die ersten Meetings beim nächsten Sprint stattdessen Montag und Donnerstag stattfinden.
Ein weiterer Themenpunkt war die Organisation im Git. Es wurde festgelegt, dass der Master-Branch während des Sprints unberührt bleiben sollte, da dieser immer lauffähig sein muss. Stattdessen sollte sich jeder seinen eigenen Branch erstellen und diesen nach Abschluss der eigenen Aufgaben auf den neuen Developer-Branch namens "targetbranch" mergen. Am Ende jedes Sprints würde dann der Developer-Branch mit dem Master-Branch gemerged werden.

\subsection{Abschließende Einschätzung des Product-Owners}
{\small Autor: xxx}

XXX

\subsection{Abschließende Einschätzung des Software-Architekten}
{\small Autor: xxx}

XXX

\subsection{Abschließende Einschätzung des Team-Managers}
{\small Autor: xxx}

XXX


\newpage

\section{Sprint 2}


\subsection{Ziel des Sprints}
{\small Autor: Linus Herterich}

Nachdem im ersten Sprint hauptsächlich die Grundstruktur sowie erste Datenstrukturen entworfen wurden,
war es nun wichtig, dass sich das gesamte Team im Sprint 2 mit der Projektstruktur (besonders mit dem Framework Vue)
auseinandersetzt und erste UserStories direkt am Code umsetzt. Zudem blieben einige Tickets noch vom letzten Sprint offen,
welche nun auch bearbeitet werden sollten.

\subsection{User-Stories des Sprint-Backlogs}
{\small Autor: Linus Herterich}

\begin{itemize}
  \item \textbf{Designumsetzung nach Adobe Preview}
        \\\textit{
          Als Benutzer der WebApplikation möchte ich ein ansehnliche und intuitive
          Oberflächengesstaltung haben, damit ich die Applikation gerne verwende.}
  \item \textbf{Authentifizierung eines Nutzers}
        \\\textit{
          Als Nutzer möchte ich mich in die Web Applikation einloggen können,
          damit nicht jeder meine erzeugten Graphen einsehen kann.}
  \item \textbf{Logische verknüpfung zwischen Knoten erstellen}
        \\ (wurde in Sprint 1 nicht abgeschlossen)
        \\\textit{
          Ein Nutzer muss eine Abfolge der Knoten definieren können,
          damit ersichtlich wird welcher Produktionsschritt auf den nächsten folgt}
  \item \textbf{Berechnung der Eingenschaften des Gesamtgraphs}
        \\ (wurde in Sprint 1 nicht abgeschlossen)
        \\\textit{
          Ein Nutzer der Webanwendung VarG muss die berechneten gesamt Eigenschaften
          jedes Zusammenhängendes Pfades ausgeben lassen können um eine Auswahl
          eines Pfades zu treffen.}
  \item \textbf{Datenstruktur Ausarbeiten \& Knoten zu einer vorhandenen Datenstruktur hinzufügen}
        \\ (wurde in Sprint 1 nicht abgeschlossen)
        \\\textit{
          Als Nutzer möchte ich Knoten zu der Datenstruktur hinzufügen können
          um die möglichen Produktionsschritte des Werkstücks überblicken zu können}

\end{itemize}

\subsection{Liste der durchgeführten Meetings}
{\small Autor: Linus Herterich}

\begin{itemize}
  \item 19.12.2019: Planning Meeting
  \item 23.12.2019: Daily Meeting (in Discord)
  \item 28.12.2019: Daily Meeting (in Discord)
  \item 05.01.2020: Review Meeting
  \item 06.01.2020: Retrospektive
\end{itemize}

\subsection{Ergebnisse des Planning-Meetings}
{\small Autor: Linus Herterich}

Neben der Aufgabenverteilung wurde im Planning darüber gesprochen, dass die Arbeitsaufteilung im letzten
Sprint nicht gut geklappt hat. Es wurde anschließend beschlossen im nächsten Sprint die User-Stories direkt
an Studenten zuzuweisen, damit jeder einen Teilbereich hat, den er bearbeiten muss.
\\ Desweiteren wurde eine Änderung im Git angekündigt. In Zukunft müsse der "Master"\--Branch während eines Sprints
immer gleich bleiben und Funktionalitäten werden auf einen "Developer"\--Branch gemerged. Am Ende des Sprints
wird dann der "Developer"\--Branch auf den "Master"\--Branch gemerged. wichtig ist, dass der "Master"\--Branch zu jedem
Zeitpunkt lauffähig ist.
\\ Für den folgenden Sprint wurde beschlossen, die Daily Meetings online (auf einem Discord Server) abzuhalten,
da viele Studenten über die Weihnachtsferien in der Heimat sind und somit ein persönliches wöchentliches treffen
nicht möglich wäre.

\subsection{Aufgewendete Arbeitszeit pro Person$+$Arbeitspaket}
{\small Autor: Linus Herterich}

\begin{longtable}{|p{4cm}|p{2cm}|p{1.2cm}|p{1.2cm}|p{0.7cm}|p{3.8cm}|}
  \hline
  Arbeitspaket                                                          & Person                & Start    & Ende     & h     & Artefakt                                                    \\
  \hline
  UI: Login                                                             & Berger, Matthias      & 22.12.19 & 22.12.19 & 3,5   & Login Funktionalität \& Design                              \\ \hline
  UI: Login                                                             & Buchmann, Lennart     & 22.12.19 & 22.12.19 & 6     & Login Funktionalität \& Design                              \\ \hline
  UI: Grapheneditor                                                     & Gwozdz, Jonas         & 23.12.19 & 04.01.20 & 9     & GraphHeader.vue, Toolbar.vue                                \\ \hline
  Task: Einbindung in Vue-Dateistruktur                                 & Heldt, Erik           & 19.12.19 & 19.12.19 & 0,25  & BasicData.js                                                \\ \hline
  Abrufbaren Knoten in Graph einfügen                                   & Heldt, Erik           & 23.12.19 & 26.12.19 & 3,5   & BasicData.js, TestDatabase.js                               \\ \hline
  Testdatenbank mit Speichern und Laden                                 & Heldt, Erik           & 27.12.19 & 27.12.19 & 3,5   & TestDatabase.js                                             \\ \hline
  Highlighting eines kürzesten Pfades nach Anwendung des A* Algorithmus & Henning, Tim          & 24.12.19 & 03.01.20 & 9     & OptimizeControls.vue, index.js -> Graph Highlighting        \\ \hline
  Protokoll: Meeting 19.12.19                                           & Herterich, Linus      & 19.12.19 & 19.12.19 & 1     & meeting\_19\_12\_19.pdf                                     \\ \hline
  UI: Login                                                             & Herterich, Linus      & 20.12.19 & 20.12.19 & 5     & LoginForm.vue, Login.vue                                    \\ \hline
  UI: Home                                                              & Herterich, Linus      & 23.12.19 & 23.12.19 & 7     & HomeMenu.vue (component), Home.vue (view), Menu.vue (view)  \\ \hline
  UI: Neuer Graph                                                       & Herterich, Linus      & 28.12.19 & 28.12.19 & 1,5   & NewGraph.vue (view), NewGraph.vue (component)               \\ \hline
  UI: Grapheneditor                                                     & Herterich, Linus      & 02.01.20 & 04.01.20 & 11,75 & Graph.vue (view), zahlreiche components                     \\ \hline
  Graph zu Datenstruktur hinzufügen                                     & Hohlfeld, Julius      & 21.12.19 & 23.12.19 & 4     & BasicData.js, TestDatabase.js                               \\ \hline
  Testdatenbank mit Speichern und Laden                                 & Hohlfeld, Julius      & 27.12.19 & 03.01.20 & 8     & BasicData.js, TestDatabase.js, index.js, JSonPersistence.js \\ \hline
  Mergen und Anpassen                                                   & Hohlfeld, Julius      & 04.01.20 & 04.01.20 & 2     & Bugs entfernt \& Mergekonflikte behoben                     \\ \hline
  UI: Datenbank-Import Fenster                                          & Karkoutli, Alaa Aldin & 31.01.20 & 04.01.20 & 12,5  & Database.vue (view), DatabaseForm.vue (component)           \\ \hline
  Kanten zu Graph hinzufügen                                            & Koch, David           & 23.12.20 & 04.01.20 & 10    & Änderungen an index.js, CreateControls.vue (component)      \\ \hline
\end{longtable}

\subsection{Konkrete Code-Qualität im Sprint}
{\small Autor: Linus Herterich}

Es wurde sich größtenteils an die Coding-Guidelines gehalten. An wichtigen Stellen sowie vor jeder Funktion wurden Kommentare
geschrieben. Die Trennung zwischen Views und Components sowie die Auslagerung der Style-Dateien wurde ebenfalls eingehalten.

\subsection{Konkrete Test-Überdeckung im Sprint}
{\small Autor: Linus Herterich}

Ein Student wurde beauftragt bis zum Ende des Sprints ein geeignetes Test-Framework zu finden.
Somit wurden während des Sprints noch keine Tests geschrieben.

\subsection{Ergebnisse des Reviews}
{\small Autor: Linus Herterich}

Es wurden fast alle UserStories umgesetzt. Somit war der zweite Sprint erfolgreich.
Alle Studenten konnten sich in das Projekt einarbeiten und haben die Strukturierung
größtenteils verstanden und eingehalten.
\\ Das User-Interface wurde nach der Designvorlage umgesetzt und die ersten Graphen-Funktionen
(Hinzufügen von Knoten und Kanten \& Optimieren) funktionieren bereits.
\\ Da noch nicht feststeht, wo die Software gehostet werden soll und wie die Datenbank-Funktionalität
umgesetzt werden soll, wurde zunächst eine lokale Speicherlösung als "Datenbank" verwendet. Somit konnten
die Speichern- und Laden-Funktionen erfolgreich implementiert werden.
\\ Die Login-Funktionalität ist derzeit nur sporadisch eingerichtet und wird finalisiert,
sobald feststeht, wie die Authentifizierung der Nutzer erfolgen soll (Anbindung an HTWK Login?).
\\ Leider ist immernoch kein geeignetes Testframework gefunden worden, mit dem sich sowohl Vue.js
als auch cytoscape (Graphen-Funktionalitäten) testen lassen.

\subsection{Ergebnisse der Retrospektive}
{\small Autor: Linus Herterich}

Das Happiness-Barometer für diesen Sprint ist sehr gut ausgefallen. Das liegt hauptsächlich an der guten Aufgabenverteilung
sowie an den großen Erfolgen, die diesen Sprint erzielt wurden.
\\ Kritisiert wurde die die Kommunikation gegen Ende des Sprints. Das finale Mergen aller Branches war zu hektisch und unsicher.
\\ Es wurde sich darauf geeinigt in Zukunft zwei Dailies pro Woche abzuhalten und das letzte Meeting eines Sprints zum gemeinsamen Mergen zu verwenden.

\subsection{Abschließende Einschätzung des Product-Owners}
{\small Autor: Manuel Eckert}

Aus den bei dem Planning-Meeting vorgestellten User-Stories ergaben sich drei Subteams. Diese teilten sich in die Bereiche Login, UI-Design und Graph-Funktionalitäten auf. Damit wurde das konkretere Aufteilen der User-Stories auf Subteams umgesetzt. \\
Dies hatte einen positiven Einfluss auf die Anzahl der erfolgreich abgeschlossen Aufgaben. \\
Während des Reviews wurden fehlende Code Kommentare und eine zu niedrige Testabdeckung benängelt.


\subsection{Abschließende Einschätzung des Software-Architekten}
{\small Autor: Julius Jolig}

In diesem Sprint wurden bereits mehr Kommentare im Code verfasst, aber hier ist noch Luft nach oben. Die Bachelorstudenten haben sich gut mit Vue.js und cytoscape vertraut gemacht und gute Ergebnisse erzielt. Das Mergen lief trotz neuem Ansatz immer chaotisch ab.  

\subsection{Abschließende Einschätzung des Team-Managers}
{\small Autor: Alex Hofmann}

Deutliche Leistungssteigerung schon jetzt zu sehen. Aufteilung der User-Stories direkt nach dem Planning hat die Arbeitsstruktur und -ablauf während des Sprints auf jeden Fall positiv beeinflusst.



\newpage

%%%%%% weitere Sprints analog


\section{Dokumentation}

\subsection{Handbuch}
{\small Autor: xxx}

XXX

\subsection{Installationsanleitung}
{\small Autor: xxx}

XXX

\subsection{Software-Lizenz}
{\small Autor: xxx}

XXX


\section{Projektabschluss}

\subsection{Protokoll der Abnahme und Inbetriebnahme beim Kunden}
{\small Autor: xxx}

XXX

\subsection{Präsentation auf der Messe}
{\small Autor: xxx}

Poster, Bericht

\subsection{Abschließende Einschätzung durch Product-Owner}
{\small Autor: xxx}

XXX

\subsection{Abschließende Einschätzung durch Software-Architekt}
{\small Autor: xxx}

XXX

\subsection{Abschließende Einschätzung durch Team-Manager}
{\small Autor: xxx}

XXX

\end{document}
