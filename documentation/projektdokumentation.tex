\documentclass[twoside]{report}

% ------
% Umlaute
\usepackage{ifluatex,ifxetex}
\ifluatex
  \usepackage{fontspec}
\else
  \ifxetex
    \usepackage{fontspec}
  \else
    \usepackage{selinput}
    \SelectInputMappings{
      adieresis={ä},
      germandbls={ß},
    }
    \usepackage[T1]{fontenc}
    %\usepackage{textcomp}% optional
    %\usepackage{lmodern}
  \fi
\fi

% ------
% Paper auf Deutsch
\usepackage[ngerman]{babel}



% ------
% Page layout
\usepackage[hmarginratio=1:1,top=32mm,columnsep=20pt]{geometry}
\usepackage[font=it]{caption}
\usepackage{paralist}
%\usepackage{multicol}


% ------
% Abstract
\usepackage{abstract}
	\renewcommand{\abstractnamefont}{\normalfont\bfseries}
	\renewcommand{\abstracttextfont}{\normalfont\small\itshape}


% ------
% Titling (section/subsection)
\usepackage{titlesec}
\renewcommand\thesection{\Roman{section}}
\titleformat{\section}[block]{\Large\scshape\bfseries}{\thesection.}{1em}{}
\setcounter{secnumdepth}{3}

% ------
% Tabellen über Seitenumbrüche hinweg
\usepackage{longtable}

% ------
% Header/footer
\usepackage{fancyhdr}
	\pagestyle{fancy}
	\fancyhead{}
	\fancyfoot{}
	\fancyhead[C]{Projektdokumentation $\bullet$ PROJEKTNAME $\bullet$ SS17$+$WS17/18}
	\fancyfoot[RO,LE]{}


% ------
% Clickable URLs (optional)
% \usepackage{hyperref}

% ------
% Literaturverweise mit Bibtex einbinden
\usepackage[authoryear,sectionbib,round]{natbib}

% ------
% Bilder laden
\usepackage[pdftex]{graphicx}

% ------
% Maketitle metadata
\title{\vspace{-5mm}%
	\fontsize{24pt}{10pt}\selectfont
	\textbf{Projektdokumentation}
	}	
\author{%
        % alle Autoren hier listen
        % 
	\large
	\textsc{Autor I -- E-Mail} \\[2mm]
	\textsc{Autor II -- E-Mail} \\[2mm]
	\normalsize	HTWK Leipzig 
	}
\date{}



%%%%%%%%%%%%%%%%%%%%%%%%
\begin{document}


% -------
% Titel und Abstract über beide Spalten
%\twocolumn[
%\begin{@twocolumnfalse}

\maketitle
\thispagestyle{fancy}

\tableofcontents

%%%%
%%%% Die Struktur des Dokuments bitte nicht aendern!!!
%%%%

\section{Anforderungsspezifikation}

\subsection{Initiale Kundenvorgaben}
{\small Autor: xxx}

Maecenas sed ultricies felis. Sed imperdiet dictum arcu a egestas.
In sapien ante, ultricies quis pellentesque ut, fringilla id sem. Proin justo libero, dapibus consequat auctor at, euismod et erat. Sed ut ipsum erat, iaculis vehicula lorem. Cras non dolor id libero blandit ornare. Pellentesque luctus fermentum eros ut posuere. Suspendisse rutrum suscipit massa sit amet molestie. Donec suscipit lacinia diam, eu posuere libero rutrum sed. Nam blandit lorem sit amet dolor vestibulum in lacinia purus varius. Ut tortor massa, rhoncus ut auctor eget, vestibulum ut justo.


\subsection{Produktvision}
{\small Autor: Alex Hofmann}
\\

\noindent Product Vision Board: \\
\begin{tabular}{|p{50mm}|p{50mm}|p{50mm}|}
  \hline
  \textbf{Target Group}                                                  & \textbf{Needs}                                                                                                                        & \textbf{Product}                                                                                                                                                                                                 \\
  \hline
  -Maschinenbau-Studenten \newline Maschinenbau-Profs \newline -Lehrende & Vgl. zu händisch: \newline einheitlicher, schneller \newline -plattformunabhängig \newline -Open Source \newline -Einfach zu bedienen & -Webanwendung \newline -Als Graph \newline $\rightarrow$ quasi als Baukasten \newline $\rightarrow$ Kantengewichtung, Bausteine wählbar \newline -Import/Export von Modellen \newline Normalisierung des Graphen \\
  \hline
\end{tabular}
\\

\noindent Die Webanwendung VarG ist entwickelt für Lehrende und Lernende aus dem Maschinenbau Bachelorstudiengang.
Diese erleichtert die einheitliche Erstellung, Bearbeitung, Optimierung sowie Im- bzw. Exportierung von sogenannten Variantenfolgegraphen. Darunter ist eine graphische Übersicht zu verstehen, die die möglichen Varianten eines Produktionsprozesses für ein Werkstück darstellt.



% Das hier ist ein Absatz, der die Grafik in Abbildung~\ref{fig:bild1} detailliert erläutert, erklärt und interpretiert.

% \begin{figure}[b]
%   \centering
%   \includegraphics[width=4.5cm]{bspbild1.png}
%   \caption{Beispiel für ein einspaltiges Bild}
%   \label{fig:bild1}
% \end{figure}


\subsection{Liste der funktionalen Anforderungen}
{\small Autor: Erik Heldt}

\begin{itemize}
  \item Erstellen von Zuständen mit Namen \& Kürzel
  \item Erstellen von Arbeitsschritten mit Namen \& Kürzel zwischen je 2 Zuständen
  \item Zuweisen von (Rüst-)Zeitkosten, (Rüst-)Geldkosten \& Losgröße zu Arbeitsschritt
  \item Anzeigen des günstigsten Weges im Graph, berechnet nach der angegebenen Kostenart
  \item Lokaler Export als Bilddatei oder importierbarer JSON \& Lokaler Import als JSON
  \item Hochladen in online gehostete Datenbank \& Laden aus online gehosteter Datenbank
  \item Login-Management für Zugriffskontrolle auf Anwendung
  \item Rollen-Management (Student, Professor) für Zugriffsrechte auf Datenbank
\end{itemize}

%
% soll der Inhalt dieser Subsection in einer separaten Datei
% (z.B. listefunktional.tex) liegen, dann kann dies mit dem
% folgenden Kommando geschehen.
%
% \input{listefunktional}

\subsection{Liste der nicht-funktionalen Anforderungen}
{\small Autor: Erik Heldt}

\begin{itemize}
  \item Schnelle Einarbeitung in die Anwendungsumgebung
  \item Einfacher \& intuitiver Umgang mit den Programmkomponenten und -funktionen
  \item Stabiler \& konsistenter Programmablauf, keine Abstürze oder Verluste von Dateien
  \item Kompatibilität mit so vielen modernen Browsern wie möglich
  \item Sicherheit \& korrekte Funktionalität des Login-Algorithmus und des DB-Rollenmanagements
  \item Datenschutz bei Login-Sessions einhalten
\end{itemize}

\subsection{Weitere Zuarbeiten zum Produktvisions-Workshop}
{\small Autor: Erik Heldt}

Für den Produktvisions-Workshop wurden 4 Dokumente erstellt, welche unterschiedliche Aspekte des Anwendungsentwurfs behandeln.

Die erste Ausarbeitung zeigt Ideen zur Darstellung der GUI, die Zweite ist ein Epic bzw. eine Zusammenfassung vieler User-Stories zu allgemeinen Anforderungen an die Funktionalität. Das dritte Dokument geht genauer auf spezifische Kernfunktionen ein und das vierte umfasst die Datenmodellierung des Programms.

\subsubsection{Zuarbeit von Linus Herterich, Jonas Gwozdz, Julius Hohlfeld}
INCLUDE GUI

\subsubsection{Zuarbeit von Erik Heldt, Alaa Aldin Karkoutli}
INCLUDE Epic

\subsubsection{Zuarbeit von Lennart Buchmann, Nils Buxel, Matthias Berger}
INCLUDE Kernfunktionalität

\subsubsection{Zuarbeit von Tim Henning, David Koch}
INCLUDE Datenmodell

\subsection{Liste der Kundengespräche mit Ergebnissen}
{\small Autor: xxx}

XXX



\section{Architektur und Entwurf}

\subsection{Zuarbeiten der Teammitglieder}
{\small Autor: Erik Heldt}

Für die Technologierecherche informierte sich das Team über verschiedene Technologien, mit denen die Anwendung entwickelt werden kann. Außerdem fassten wir erste Ideen zur Graphenanordnung zusammen, legten Coding Conventions fest und erstellten einen interaktiven GUI-Prototyp mit Adobe XD. Die Ausarbeitungen wurden in den nachfolgenden Dokumenten festgehalten.

\subsubsection{Zuarbeit von Tim Henning}
INCLUDE DjangoPython

\subsubsection{Zuarbeit von Erik Heldt}
INCLUDE RailsLang,RailsKurz,Graphenanordnung

\subsubsection{Zuarbeit von David Koch}
INCLUDE JavaCanvas,RechercheDB

\subsubsection{Zuarbeit von Matthias Berger, Nils Buxel}
INCLUDE CodingGuidelines

\subsubsection{Zuarbeit von Nils Buxel}
INCLUDE CodingConventionsCSS

\subsubsection{Zuarbeit von Julius Hohlfeld}
INCLUDE Angular

\subsubsection{Zuarbeit von Lennart Buchmann, Alaa Aldin Karkoutli, Jonas Gwozdz, Linus Herterich}
INCLUDE OpenSource

\subsubsection{Zuarbeit von Linus Herterich, Jonas Gwozdz}
INCLUDE AdobeXD

\subsection{Entscheidungen des Technologieworkshops}
{\small Autor: Erik Heldt}

Nach ausgiebigen Recherchen über verschiedenste Programmiersprachen, Frameworks und Bibliotheken entschieden wir uns für eine Webanwendung auf Basis von HTML/CSS/JavaScript.

Wir haben uns weiterhin auf das JS-Framework Vue.js geeinigt, da es viele Vorteile für die Front-End-Entwicklung mit sich bringt und von den vielen untersuchten Frameworks am intuitivsten erschien. Außerdem haben wir nach einer JS-Bibliothek zur Graphdarstellung recherchiert und unter verschiedenen Kandidaten stach Cytoscape mit seinen vielen Funktionen zur Graphenerstellung und -editierung am meisten heraus, was wir somit auch in unsere Architektur integrierten.

Bei der Programmierumgebung waren wir uns schnell einig, dass Visual Studio Code am besten für unsere Ansprüche geeignet ist. Wir installierten die IDE zusammen mit dem Plugin ESLint zur Unterstützung der Einhaltung standardmäßiger Coding Conventions.

\subsection{Überblick über Architektur}
{\small Autor: Linus Herterich}

VarG ist eine Web-App nach dem Client-Server Modell, wobei der Großteil der Berechnungen per JavaScript auf dem clientseitigen Browser
durchgeführt werden.
\\Serverseitig wird eine Datenbank (inkl API-Schnittstelle) zum persistenten Speichern der erstellten Graphen angeboten.
\\
\\Die Architektur der Web-App basiert auf dem Javascript-Webframework ''Vue.js'',
mit dem Webanwendungen nach dem MVVM Muster (Model View ViewModel) realisiert werden können. Die gesamte App ist nach logischen
Sites (Seiten, bei denen sich die URL ändert) und Components (wiederverwendbare, abgeschlossene Software-Schnipsel) aufgebaut.
Jede Vue Component (.vue Dateien) enthält ein HTML-Template (GUI), sowie Daten, mit denen das Template befüllt wird. Zudem werden
Funktionen definiert, die entweder zu bestimmten Laufzeitbedingungen der App oder durch Events und Trigger aufgerufen werden.
Die Kommunikation zwischen Components wird über Vererbungen zu Eltern-/ Kind-Components realisiert.
\\Die Web-App besteht im Entwicklungszustand aus vielen hunderten Dateien, welche vom Framework verwaltet werden. Sobald
die App in den Produktionsstatus wechselt, muss das Projekt kompiliert werden. Dies übernimmt ebenfalls das Framework, welches
hierfür Technologien wie ''WebPack'' einsetzt. So bleiben lediglich wenige HTML, CSS und JavaScript Dateien übrig, die anschließend
auf einem Web-Server (z.B. Apache) zur Verfügung gestellt werden müssen.
\\
\\Um die Darstellung einheitlich zu halten, haben wir die UI-Bibliothek ''vuetify'' genutzt. Diese hält sich an den Industriestandard
''Material Design'' von Google. Damit konnten wir alle unsere im Vorfeld erstellten Design-Konzepte umsetzen. Um an den ''vuetify'' Elementen
weitere optische Anpassungen vorzunehmen haben wir die CSS-Language-Extension ''less'' verwendet. Mit dieser war es möglich übersichtliche und
einheitliche Style-Vorgaben die Design-Komponenten anzuwenden.
\\
\\Damit Alle Daten Component-Übergreifend auf einen gemeinsamen Datenstamm zugreifen können und die Daten auch nach einer Session persistent
gespeichert werden können, haben wir die vue.js-Erweiterung ''vuex'' eingesetzt. Diese bietet eine zentralisierte Speichermöglichkeit
für alle Daten, die übergreifend verwendet werden müssen (beispielsweise Log-In Daten oder der Zustand des Variantenfolgegraphen).
\\
\\Für die Darstellung des Graphen (Knoten + Kanten und deren Beschriftung) haben wir das JavaScript Framework ''cytoscape.js'' verwendet.
Das Framework hält alle Graph-Daten in einem JavaScript Objekt, auf das mit verschiedenen API-Funktionen zugegriffen werden kann.
Die Darstellung des Graphen wird über ein Canvas HTML Element realisiert, in welches cytoscape die angelegten Knoten und Kanten zeichnet.
Cytoscape.js bietet ebenfalls eine Hand voll Algorithmen zur analytischen Auswertung des Graphen. Da die Optimierung des Variantenfolgegraphen
allerdings zusätzlicher Bedingungen und Parametern unterliegt, wurde ein eigener Variantenfolgegraph-Optimierungsalgorithmus entwickelt.
\\
\\Bei der Wahl der serverseitigen Architektur haben wir eine REST-Konforme (Representational State Transfer)
Architektur eingesetzt, in dessen Mitte eine MySQL Datenbank zur Speicherung der cytoscape Objekte, sowie Authentifizierungsdaten
steht. Auf die Daten der Datenbank greift eine API-Schnittstelle zu, welche mit Node.js umgesetzt ist (weitere Details zur Schnittstelle:
siehe II.4 - Schnittstellen). Anfragen an die API werden mit dem ''axios'' Framework per ''Promise-based'' HTTP-Requests gestellt. Die HTTP-Requests
folgen einem klaren Schema, welches vom serverseitigen Node.js interpretiert und an die Datenbank weitergeleitet wird.
\\
\\Um die Web-App großflächig zu testen haben wir uns zum einen für das Framework ''cypress'' entschieden, welches Integrationstests anhand der
HTML-Elemente übernimmt. Cypress wertet aus, ob bestimmte Elemente unter bestimmten Bedingungen vorhanden sind, beziehungsweise spezielle
Eigenschaften aufweisen. Die Cypress Tests haben wir auch erfolgreich an die ''CI / CD Pipeline'' von GitLab angeschlossen, sodass nach jedem
push die Tests durchlaufen (Stichwort: Regressionstest).
\\Zum anderen haben wir das Framework "jest" eingesetzt, mit dem einzelne Funktionen auf ihre Richtigkeit überprüft werden konnten. Vorallem
für die Optimierungsalgorithmen sind isolierte Tests nötig gewesen.
\\
\\Um eine Client-Server Architektur zu simulieren haben wir ''Docker'' eingesetzt. Dieses Tool erlaubt es virtuelle Maschinen zu erstellen,
welche untereinander kommunizieren können. Für Entwicklungszwecke haben wir einen Docker-Container für eine MySQL Datenbank und
einen Node.js-Webserver (API Schnittstelle) erzeugt. Ein weiterer Docker-Container wurde eingesetzt, auf dem ''Adminer'' lief. Mit diesem
Tool ist es möglich die MySQL-Datenbank komfortabel und umfangreich anzupassen.


\subsection{Definierte Schnittstellen}
{\small Autor: Julius Hohlfeld}

VarGs Funktionalitäten erforden eine Datenbank um die erstellten Graphen speichern und wieder abrufen zu können.\\
Um den Zugriff auf die Datenbank zu kontrollieren benötigen wir eine definierte Schnittstelle (bzw. API) zwischen Client,
Webserver und Datenbank.\\
\\
Diese Schnittstelle ist RESTfull - d.h. sie folgt einigen der sog. REST-Constraints. Eine Übersicht inwiefern zu REST und welche
Bedeutung es für das Projekt hat, findet sich im GitLab Wiki unter 'API Dokumentation'.\\
Die Schnittstelle setzt sich wie folgt zusammen:
\begin{itemize}
  \item \textbf{Vue}
        \\\textit{Framework für Client + Axios Modul für asnychrone (promise-based) HTTP-Requests}
  \item  \textbf{Express}
        \\\textit{Serverseitiges Node-Module für Webserver: hört angemeldete Ports auf Requests ab, die dem URI-Modell entsprechen}
  \item \textbf{Node.js}
        \\\textit{Serverseitige Programmierung des Webservers mit mysqljs als Driver, um auf die Datenbank zuzugreifen}
  \item \textbf{DB}
        \\\textit{MySQL-Datenbank auf extra Server}
\end{itemize}

Diese Struktur (kurz VenDB) entspricht einer Anpassung des sog. MEAN-Stacks auf das VarG-Projekt (MongoDB, Express, Angular, Node.js).\\
Dabei erfolgt jeglicher Austausch der Graphdaten im JSON Format damit auf die cytoscape.js Funktion zum Laden des Graphen zugegriffen werden kann.\\
\\
\subsubsection{Client}

Der Client enthält Trigger durch Events, welche Requests an den Webserver senden. Z.B.: das Aufrufen des Datenbankfenster löst eine Anfrage aus, welche alle Graphen des aktuellen Nutzers abfragt.
Diese werden durch das Axios-Modul umgesetzt. Nachdem der Trigger ausgelöst wird, schickt der Client eine asynchrone Request. Diese wird vom Webserver verarbeitet, welcher dann eine Antwort schickt. Diese kann von Axios aufgefangen werden (axios."request"(url, {}).then(response => {}).catch(error => {})).

\subsubsection{Server}

Der durch Express und mysqljs programmierte Webserver definiert folgende mögliche Zugriffstellen auf die Datenbank:
\begin{itemize}
  \item \textbf{Get-Requests}
        \begin{itemize}
          \item \textbf{graph}
                \\\textit{Fragt alle Graphen aus der Datenbank ab - für Admin reserviert.}
          \item \textbf{graph/:id?}
                \\\textit{Fragt einen spezifischen Graphen (entsprechend der ID) ab.}
          \item \textbf{graph/meta}
                \\\textit{Fragt Metadaten z.B.: Namen, Id, Stückzahl usw. ab für die Graphen des Nutzers ab.}
        \end{itemize}
  \item  \textbf{Put-Requests}
        \begin{itemize}
          \item \textbf{graph/:id?}
                \\\textit{Client schickt Server eine Repräsentation des Graphen in Json um einen bereits existierenden Graphen (entsprechend der ID) zu überschreiben.}
        \end{itemize}
  \item \textbf{Post-Requests}
        \begin{itemize}
          \item \textbf{graph?}
                \\\textit{Client schickt Server eine Repräsentation des Graphen in Json um einen neuen Eintrag für den Nutzer zu erzeugen.}
        \end{itemize}
  \item \textbf{Delete-Request}
        \begin{itemize}
          \item \textbf{graph/:id?}
                \\\textit{Spezifizierter Graph (entprechend der ID) wird aus der Datenbank gelöscht.}
        \end{itemize}
\end{itemize}

Das '?' bedeutet, dass hier auf bestimmte URL Queries geachetet werden kann.
Das ist nützlich um z.B.: einen Nutzer nur auf seine eigenen Graphen zugreifen zu lassen.
Diese werden dann in die entsprechenden Queries umgewandelt.\\

\subsection{Liste der Architekturentscheidungen}
{\small Autor: xxx}

XXX (bewusste und unbewusste Entscheidungen mit zeitlicher Einordnung)



\section{Prozess- und Implementationsvorgaben}

\subsection{Definition of Done}
{\small Autor: Tim Henning}
\\
Im Allgemeinen wurde in dem Projekt die Definition von "doneness"\ nicht all zu umfangreich
gestaltet, da es für viele Teammitglieder eines der ersten Softwareprojekte war. So wurden als Defintion of Done folgende
Punkte für alle Userstories aufgestellt:
\begin{itemize}
  \item >50\% Testabdeckung
  \item Technische Kommentare im Code
  \item Einhaltung der festgelegten Code Konventionen
\end{itemize}
Das Team hatte an sich zu den meisten Zeitpunkten eine klare Vorstellung was einen "fertigen Entwurf"\ kennzeichnet und wurde so auch in den Reviews untereinander kommuniziert.
Dies wiederum führte zu einer klaren Transparenz im Team, was die Qualität des Produktes erhöhte und das Zusammenarbeiten erleichterte.
Größtenteils wurde sich
an die allgemeinen Akzeptanzkritieren gehalten und viele Backlog-Einträge als "done"\ erklärt. Zu fast jeder Komponente wurde
getestet und zu den Methoden der einzelnen Komponenten wurden erklärende sinnvolle Codekommentare geschrieben.
Außerdem wurde im Team umfangreich kommuniziert und die Kriterien angepasst, wenn die Fertigstellung einer Userstorie doch mal nicht gänzlich klar war.
So wurde es ermöglicht nach der Hälfte des Projektes, am Ende jedes Sprints einen fertigen Productionbuild dem Kunden zu liefern.



\subsection{Coding Style}
{\small Autor: xxx}

XXX

\subsection{Zu nutzende Werkzeuge}
{\small Autor: Linus Herterich}

Im folgenden werden die Werkzeuge erwähnt, mit denen wir die Software
entwickelt haben. Zudem wird darauf eingegangen, über welche Kanäle kommuniziert
wurde.


\subsubsection{Voraussetzungen}
Das Versionsmanagement-Tool ''GitLab'' sowie das Zeitmanagement-Tool ''YouTrack''
wurde zu Beginn des Projekts vorgeschrieben. Die Commits in ''GitLab'' werden jeweils mit
der ID des zugehörigen YouTrack-Tickets am Anfang des Commit-Titels versehen.
\\Damit das gesamte Team einheitliche Versionen der verwendeten Bibliotheken benutzt,
wird der Paketmanager ''npm'' verwendet. Mir diesem lassen sich Pakete (und deren Versionen) definieren,
welche für das Projekt benötigt werden.
\\ Damit am Projekt gearbeitet werden kann, muss sich somit jedes Teammitglied die LTS-
Version von Node.js (welches npm enthält) installieren.
\\ Sobald Node.js global installiert ist, kann im ''code'' Verzeichnis der Befehl
''npm install'' ausgeführt werden, um die benötigten Bibliothekten zu installieren.

\subsubsection{Compiler}
Achtung: Das Kompileren funktioniert erst, sobald die Bibliothekten mit dem Befehl
''npm install'' (im /code Verzeichnis) installiert wurden
\\
Um Änderungen des Projektes einzusehen, muss das Projekt kompiliert werden.
''Vue.js'' bringt bereits einen Echtzeit-Compiler mit, welcher reagiert, sobald Änderungen
an Dateien im ''code'' Verzeichnis gemacht wurden. Um diesen Compiler aufzurufen, muss der npm-Befehl ''npm run serve'' im ''code'' Verzeichnis
aufgerufen werden.
\\Um das Projekt nicht während der Entwicklung zu kompilieren, sondern für die Produktion freizugeben,
muss der Befehl ''npm run build'' im ''code'' Verzeichnis aufgerufen werden. Es werden
anschließend die kompilierten Dateien im Verzeichnis ''code/dist'' abgelegt.
Diese können anschließend auf einem Webserver (z.B. Apache HTTP Server) hochgeladen werden.

\subsubsection{Entwicklungsumgebung}
Für die Entwicklung der Software wird der freie Quelltext-Editor ''Visual Studio Code''
von Microsoft verwendet. Dieser ist plattformunabhängig und kann durh zahlreiche Erweiterungen
angepasst werden. Beispielsweise kann durch ein Plugin die ''Vue.js''-eigene Syntax
vervollständigt und hervorgehoben werden.
\\Weitere Einstellungsvorgaben bezüglich der Entwicklungsumgebung wurden nicht getroffen.
Es muss allerdings darauf geachtet werden, dass die Coding-Conventions durch automatische
Formatierungen eingehalten werden.

\subsubsection{CI / CD Pipeline}
In der CI / CD Pipeline unseres Versionsmanagement-Tools, die nach jedem Git-Push
ausgeführt wird, werden folgende Operationen durchgeführt:
\begin{itemize}
  \item Test, ob das Projekt kompiliert (inklusive Syntaxprüfung durch ES-Lint)
  \item Cypress Tests durchführen (siehe ''Überblick über Architektur'')
  \item LaTEX Doku kompilieren
\end{itemize}

Sollte einer der Punkte fehlschlagen, wird der Autor des Git-Push's per E-Mail
darüber informiert. Somit ist die wahrscheinlichkeit, dass bestehende Features durch
neue Entwicklungen längerfristig ''zerstört'' werden, möglichst gering.

\subsubsection{Docker}
Um die Client-Server Architektur des Projektes lokal zu simulieren, wird die
Container-Virtualisierungssoftware ''Docker'' verwendet.
Mit dieser haben wir einen Webserver simuliert, auf dem die Datenbank ausgeführt
und verwaltet wird (siehe ''Überblick über Architektur''). Die Container werden
im Projekt-Ordner ''docker'' definiert.


\subsubsection{Kommunikationstools}
Zu Beginn des Projekts wurde sich auf das kostenlose Kommunikationstool ''Slack'' geeinigt.
Mit diesem ist es möglich in verschiedenen Kanälen Text, Dateien und Medien auszutauschen.
Auch private Konversationen, sowie Kleingruppen-Chaträume sind in diesem Tool möglich.
Die Software kann sowohl als App installiert werden, als auch im Browser verwendet werden.
\\Da wir über die Weihnachtsferien einen Sprint durchgeführt haben, führten wir Mitte Dezember
das Tool ''Discord'' ein, mit dem es möglich ist sich in Echtzeit-Sprachchats zusammenzufinden.
Dazu ist es es möglich seinen Desktop zu teilen, womit sich das Tool bestens geeignet hat,
um räumlich getrennt über Code-Passagen oder neue Features zu sprechen.
\\Die Kombination beider Tools hat problemlos funktioniert und uns auch
während des Lockdowns in der ''Corona-Krise'' geholfen. Da wir die Tools bereits frühzeitig
eingesetzt haben, war kaum eine Um- bzw. Eingewöhnungszeit zu Beginn der
präsenzfreien Zeit notwendig.


\newpage

%%%%%%%%%%%%
%% Abschnitt mit den Sprints beginnt hier
%%%%%%%%%%%%

\section{Sprint 1}


\subsection{Ziel des Sprints}
{\small Autor: Erik Heldt}

Der erste Sprint des VarG-Projekts lief vom 05.12.2019 bis zum 16.12.2019. Ziel war es, eine fundamentale Struktur und grundlegende Funktionalitäten für die Anwendung zu entwickeln, auf denen man später weiter aufbauen kann. Währenddessen konnte man allgemeine Erfahrungen mit dem Ablauf eines Sprints machen.

\subsection{User-Stories des Sprint-Backlogs}
{\small Autor: Erik Heldt}

\textbf{Grundstruktur}
Die Anwendung sollte zu Beginn ein grundlegendes Fundament aufweisen, damit sich alle Teammitglieder vorstellen können, wie am Ende das Programm aussehen soll. Dazu gehörte zu Beginn das Design der Startseite mit dem VarGraph im Zentrum und der Einbindung von Cytoscape in die Programmstruktur.

\textbf{Datenstruktur für Knoten}
Es sollte mit Hilfe von Cytoscape herausgefunden werden, wie man Knoten im Programmcode hinzufügen und speichern kann. Dafür sollte dann eine Datei im Programm angelegt werden.

\textbf{Knoten zu bestehender Datenstruktur hinzufügen}
Die Anwendung sollte eine einfache Funktionalität zum Erstellen neuer Knoten aka Produktionsschritte erhalten, um sich mit den Cytoscape-Funktionen näher vertraut zu machen. Hier war erstmal noch keine graphische Darstellung in der GUI notwendig, es reichte per Console logs zu testen.

\textbf{Darstellung eines Graphen in Weboberfläche}
In der Anwendung sollte zunächst ein statischer Graph mit Hilfe einer Cytoscape-Datenstruktur sichtbar dargestellt werden, damit man sehen konnte, wie so ein „CytoGraph“ überhaupt aussieht. User-Interaktion war hier noch nicht notwendig.

\textbf{Kanten anlegen}
Zusätzlich zu Knoten sollten auch Kanten zwischen bestehenden Knoten hinzugefügt werden können. Diese Kanten sollten mit verschiedenen Attributen in der Cytoscape-Datenstruktur gespeichert werden.

\textbf{Berechnung verschiedener Eigenschaften}
Anhand der mit den Kanten gespeicherten Attribute sollte eine Funktionalität entwickelt werden, welche die Gesamtkosten (Auswahl von Geld oder Zeit) aller unterschiedlichen Pfade berechnen und anzeigen sollte. Dies war der erste Schritt in Richtung Optimierung, d.h. später sollte diese Funktionalität automatisch den günstigsten Pfad herausfinden und anzeigen.

\subsection{Liste der durchgeführten Meetings}
{\small Autor: Erik Heldt}

\begin{itemize}
	\item Planning - 05.12.2019
	\item Weekly Scrum 1 - 09.12.2019
	\item Weekly Scrum 2 - 12.12.2019
	\item Review - 16.12.2019
	\item Retrospektive - 19.12.2019
\end{itemize}

\subsection{Ergebnisse des Planning-Meetings}
{\small Autor: Erik Heldt}

Im Planning-Meeting erklärten die Projektmanager zu Beginn noch einmal kurz, wie ein Sprint im Allgemeinen abläuft und haben auf die Bedeutsamkeit der Coding Guidelines hingewiesen. Anschließend wurden die ersten User-Stories vom Project Owner vorgestellt und von den Bachelorstudenten per Finger-System in ihrer Komplexität eingeschätzt. Weiterhin wurde festgelegt, dass die Bachelorstudenten während des Sprints die User-Stories selbst in Tasks aufteilen und diese dann bearbeiten sollen.

\subsection{Aufgewendete Arbeitszeit pro Person$+$Arbeitspaket}
{\small Autor: xxx}

\begin{longtable}{|p{4cm}|l|l|l|l|l|}
        \hline
	Arbeitspaket & Person & Start & Ende & h & Artefakt\\
        \hline
	Vue.js "Getting Started" Tutorial durcharbeiten (für alle) & Buchmann, Lennart & 07.12.19 & 07.12.19 & 3 & Tutorial abgeschlossen\\ \hline
	Beispielgraph erstellen & Buxel, Nils & 09.12.19 & 09.12.19 & 1 & index.js\\ \hline
	Kürzesten Weg mit A*-Algorithm berechnen u anzeigen lassen & Buxel, Nils &16.12.19 & 16.12.19 & 1 & index.js\\ \hline
	Funktionen zu Buttons hinzufügen & Gwozdz, Jonas & 14.12.19 & 16.12.19 & 4 & MenuControls.vue\\ \hline
	Task: Einbindung in Vue-Dateistruktur & Heldt, Erik & 15.12.19 & 15.12.19 & 3 & MenuControls.vue, BasicData.js\\ \hline
	Graphenanordnung & Heldt, Erik & 05.12.19 & 05.12.19 & 3 & Graphenanordnung.pdf\\ \hline
	Vue.js "Getting Started" Tutorial durcharbeiten (für alle) & Heldt, Erik & 11.12.19 & 11.12.19 & 2 & Tutorial abgeschlossen\\ \hline
	Funktionen zu Buttons hinzufügen & Henning, Tim & 10.12.19 & 10.12.19 & 2 & MenuControls.vue\\ \hline
	Vue.js "Getting Started" Tutorial durcharbeiten (für alle) & Henning, Tim & 06.12.19 & 06.12.19 & 3 & Tutorial abgeschlossen\\ \hline
	Einbindung von Cytoscape in Vue & Herterich, Linus & 10.12.19 & 10.12.19 & 4 & index.js\\ \hline
	Buttons für Knoten und Kantenerstellung & Herterich, Linus & 13.12.19 & 13.12.19 & 3 & CreateControls.vue\\ \hline
	Knoten zu Graph hinzufügen & Herterich, Linus & 16.12.19 & 16.12.19 & 2,5 & index.js, CreateControls.vue\\ \hline
	Grundstruktur aufbauen & Herterich, Linus & 05.12.19 & 07.12.19 & 9,5 & Vue-Dateistruktur, sämtliche Startkomponenten\\ \hline
	Task: Basic Datenstruktur & Hohlfeld, Julius & 15.12.19 & 15.12.19 & 8 & BasicData.js, MenuControls.vue\\ \hline
      \end{longtable}

\subsection{Konkrete Code-Qualität im Sprint}
{\small Autor: Erik Heldt}

Zu Beginn wurde viel experimentiert und hauptsächlich sollte der Code erstmal ein funktionierendes Programm erzeugen, weswegen weniger auf die Qualität geachtet wurde. Trotzdem wurde sich größtenteils an die Coding Conventions gehalten und bereits einige Kommentare verfasst.

\subsection{Konkrete Test-Überdeckung im Sprint}
{\small Autor: Erik Heldt}

Da der erste Sprint größtenteils nur zur Erstellung einer grundlegenden Datenstruktur und zur Einarbeitung in JavaScript und den genutzten Frameworks bzw. Bibliotheken gedient hat, gab es noch keine Tests.

\subsection{Ergebnisse des Reviews}
{\small Autor: Erik Heldt}

Im ersten Review-Meeting stellten die Bachelorstudenten ihre Ergebnisse aus dem Sprint vor und die Manager gaben ihr Feedback dazu. Da sich die meisten Teammitglieder noch nicht richtig in Vue.js und Cytoscape einarbeiten konnten und teilweise große Schwierigkeiten mit den Frameworks hatten, gab es noch viele offene Aufgaben und nicht jeder hatte etwas vorzuzeigen.
Als erstes stellten Julius H. und Erik die Datenstruktur für die Knoten vor. Weiterhin zeigte Julius, wie ein Knoten in der Anwendung dargestellt wird und dass dieser durch ungeschickte Verschiebung und Skalierung aus der GUI verschwinden kann. Deshalb kamen Vorschläge, zukünftig den Zoom zu limitieren und das grundsätzliche Graph-Layout nochmal zu überarbeiten.
Um allen den Einstieg in die neuen Programmiersprachen und Bibliotheken etwas zu vereinfachen, stellte daraufhin Linus die Grundstruktur vor und erklärte noch einmal genau die einzelnen Elemente in der Dateistruktur. Weiterhin zeigte er, wie man ESLint-Fehler bei der Konsolenausgabe verhindern kann.
Danach wurde zwischen den Managern und den Bachelorstudenten noch die zukünftige Berechnung der kürzesten Wege und die unbearbeiteten User-Stories besprochen und dass diese in den nächsten Sprint mit einfließen werden.
Zum Schluss wurden noch ein paar allgemeine Fragen zum Testen und zu Git geklärt.

\subsection{Ergebnisse der Retrospektive}
{\small Autor: Erik Heldt}

In der Retrospektive konnte jedes Teammitglied vor an die Tafel gehen und verschiedene Aspekte des Sprints mit einem Strich in einer Tabelle bewerten.
Die Bewertung ging ausgeglichen aus. Die Gruppenleistung und das Gesamtergebnis waren gut, aber die Einzelleistungen der meisten Teammitglieder nicht. Viele Aufgaben blieben offen und wurden nicht erledigt, wozu in der Diskussion verschiedene Gründe angeführt wurden. Einerseits war es für die meisten schwer, sich selbst in die neue Programmierumgebung samt den Frameworks und Bibliotheken einzuarbeiten. Andererseits wussten viele nicht, was und wie viel sie machen sollten, was auf die nicht festgelegte Aufgabenzuteilung im Planning und die schlechte Kommunikation im Team während des Sprints zurückgeführt wurde. Letzteres Problem plante man damit zu lösen, in zukünftigen Plannings immer direkt Verantwortliche für bestimmte User-Stories festzulegen und entsprechende Tickets sofort im Anschluss zu erstellen und zuzuweisen.
Beim Thema der Daily Meetings ist man zu dem Schluss gekommen, dass diese wenn möglich immer persönlich bleiben sollten und nur in Ausnahmefällen online z.B. über Discord stattfinden sollten. Weiterhin wurde diskutiert, ob die Zeitspanne zwischen Donnerstag und Montag evtl. zu kurz ist, um schon weitreichende Ergebnisse zu erzielen, da am Wochenende einige Teammitglieder nicht programmieren können. Deshalb sollten die ersten Meetings beim nächsten Sprint stattdessen Montag und Donnerstag stattfinden.
Ein weiterer Themenpunkt war die Organisation im Git. Es wurde festgelegt, dass der Master-Branch während des Sprints unberührt bleiben sollte, da dieser immer lauffähig sein muss. Stattdessen sollte sich jeder seinen eigenen Branch erstellen und diesen nach Abschluss der eigenen Aufgaben auf den neuen Developer-Branch namens "targetbranch" mergen. Am Ende jedes Sprints würde dann der Developer-Branch mit dem Master-Branch gemerged werden.

\subsection{Abschließende Einschätzung des Product-Owners}
{\small Autor: xxx}

XXX

\subsection{Abschließende Einschätzung des Software-Architekten}
{\small Autor: xxx}

XXX

\subsection{Abschließende Einschätzung des Team-Managers}
{\small Autor: xxx}

XXX


\newpage

\section{Sprint 2}


\subsection{Ziel des Sprints}
{\small Autor: Linus Herterich}

Nachdem im ersten Sprint hauptsächlich die Grundstruktur sowie erste Datenstrukturen entworfen wurden,
war es nun wichtig, dass sich das gesamte Team im Sprint 2 mit der Projektstruktur (besonders mit dem Framework Vue)
auseinandersetzt und erste UserStories direkt am Code umsetzt. Zudem blieben einige Tickets noch vom letzten Sprint offen,
welche nun auch bearbeitet werden sollten.

\subsection{User-Stories des Sprint-Backlogs}
{\small Autor: Linus Herterich}

\begin{itemize}
  \item \textbf{Designumsetzung nach Adobe Preview}
        \\\textit{
          Als Benutzer der WebApplikation möchte ich ein ansehnliche und intuitive
          Oberflächengesstaltung haben, damit ich die Applikation gerne verwende.}
  \item \textbf{Authentifizierung eines Nutzers}
        \\\textit{
          Als Nutzer möchte ich mich in die Web Applikation einloggen können,
          damit nicht jeder meine erzeugten Graphen einsehen kann.}
  \item \textbf{Logische verknüpfung zwischen Knoten erstellen}
        \\ (wurde in Sprint 1 nicht abgeschlossen)
        \\\textit{
          Ein Nutzer muss eine Abfolge der Knoten definieren können,
          damit ersichtlich wird welcher Produktionsschritt auf den nächsten folgt}
  \item \textbf{Berechnung der Eingenschaften des Gesamtgraphs}
        \\ (wurde in Sprint 1 nicht abgeschlossen)
        \\\textit{
          Ein Nutzer der Webanwendung VarG muss die berechneten gesamt Eigenschaften
          jedes Zusammenhängendes Pfades ausgeben lassen können um eine Auswahl
          eines Pfades zu treffen.}
  \item \textbf{Datenstruktur Ausarbeiten \& Knoten zu einer vorhandenen Datenstruktur hinzufügen}
        \\ (wurde in Sprint 1 nicht abgeschlossen)
        \\\textit{
          Als Nutzer möchte ich Knoten zu der Datenstruktur hinzufügen können
          um die möglichen Produktionsschritte des Werkstücks überblicken zu können}

\end{itemize}

\subsection{Liste der durchgeführten Meetings}
{\small Autor: Linus Herterich}

\begin{itemize}
  \item 19.12.2019: Planning Meeting
  \item 23.12.2019: Daily Meeting (in Discord)
  \item 28.12.2019: Daily Meeting (in Discord)
  \item 05.01.2020: Review Meeting
  \item 06.01.2020: Retrospektive
\end{itemize}

\subsection{Ergebnisse des Planning-Meetings}
{\small Autor: Linus Herterich}

Neben der Aufgabenverteilung wurde im Planning darüber gesprochen, dass die Arbeitsaufteilung im letzten
Sprint nicht gut geklappt hat. Es wurde anschließend beschlossen im nächsten Sprint die User-Stories direkt
an Studenten zuzuweisen, damit jeder einen Teilbereich hat, den er bearbeiten muss.
\\ Desweiteren wurde eine Änderung im Git angekündigt. In Zukunft müsse der "Master"\--Branch während eines Sprints
immer gleich bleiben und Funktionalitäten werden auf einen "Developer"\--Branch gemerged. Am Ende des Sprints
wird dann der "Developer"\--Branch auf den "Master"\--Branch gemerged. wichtig ist, dass der "Master"\--Branch zu jedem
Zeitpunkt lauffähig ist.
\\ Für den folgenden Sprint wurde beschlossen, die Daily Meetings online (auf einem Discord Server) abzuhalten,
da viele Studenten über die Weihnachtsferien in der Heimat sind und somit ein persönliches wöchentliches treffen
nicht möglich wäre.

\subsection{Aufgewendete Arbeitszeit pro Person$+$Arbeitspaket}
{\small Autor: Linus Herterich}

\begin{longtable}{|p{4cm}|p{2cm}|p{1.2cm}|p{1.2cm}|p{0.7cm}|p{3.8cm}|}
  \hline
  Arbeitspaket                                                          & Person                & Start    & Ende     & h     & Artefakt                                                    \\
  \hline
  UI: Login                                                             & Berger, Matthias      & 22.12.19 & 22.12.19 & 3,5   & Login Funktionalität \& Design                              \\ \hline
  UI: Login                                                             & Buchmann, Lennart     & 22.12.19 & 22.12.19 & 6     & Login Funktionalität \& Design                              \\ \hline
  UI: Grapheneditor                                                     & Gwozdz, Jonas         & 23.12.19 & 04.01.20 & 9     & GraphHeader.vue, Toolbar.vue                                \\ \hline
  Task: Einbindung in Vue-Dateistruktur                                 & Heldt, Erik           & 19.12.19 & 19.12.19 & 0,25  & BasicData.js                                                \\ \hline
  Abrufbaren Knoten in Graph einfügen                                   & Heldt, Erik           & 23.12.19 & 26.12.19 & 3,5   & BasicData.js, TestDatabase.js                               \\ \hline
  Testdatenbank mit Speichern und Laden                                 & Heldt, Erik           & 27.12.19 & 27.12.19 & 3,5   & TestDatabase.js                                             \\ \hline
  Highlighting eines kürzesten Pfades nach Anwendung des A* Algorithmus & Henning, Tim          & 24.12.19 & 03.01.20 & 9     & OptimizeControls.vue, index.js -> Graph Highlighting        \\ \hline
  Protokoll: Meeting 19.12.19                                           & Herterich, Linus      & 19.12.19 & 19.12.19 & 1     & meeting\_19\_12\_19.pdf                                     \\ \hline
  UI: Login                                                             & Herterich, Linus      & 20.12.19 & 20.12.19 & 5     & LoginForm.vue, Login.vue                                    \\ \hline
  UI: Home                                                              & Herterich, Linus      & 23.12.19 & 23.12.19 & 7     & HomeMenu.vue (component), Home.vue (view), Menu.vue (view)  \\ \hline
  UI: Neuer Graph                                                       & Herterich, Linus      & 28.12.19 & 28.12.19 & 1,5   & NewGraph.vue (view), NewGraph.vue (component)               \\ \hline
  UI: Grapheneditor                                                     & Herterich, Linus      & 02.01.20 & 04.01.20 & 11,75 & Graph.vue (view), zahlreiche components                     \\ \hline
  Graph zu Datenstruktur hinzufügen                                     & Hohlfeld, Julius      & 21.12.19 & 23.12.19 & 4     & BasicData.js, TestDatabase.js                               \\ \hline
  Testdatenbank mit Speichern und Laden                                 & Hohlfeld, Julius      & 27.12.19 & 03.01.20 & 8     & BasicData.js, TestDatabase.js, index.js, JSonPersistence.js \\ \hline
  Mergen und Anpassen                                                   & Hohlfeld, Julius      & 04.01.20 & 04.01.20 & 2     & Bugs entfernt \& Mergekonflikte behoben                     \\ \hline
  UI: Datenbank-Import Fenster                                          & Karkoutli, Alaa Aldin & 31.01.20 & 04.01.20 & 12,5  & Database.vue (view), DatabaseForm.vue (component)           \\ \hline
  Kanten zu Graph hinzufügen                                            & Koch, David           & 23.12.20 & 04.01.20 & 10    & Änderungen an index.js, CreateControls.vue (component)      \\ \hline
\end{longtable}

\subsection{Konkrete Code-Qualität im Sprint}
{\small Autor: Linus Herterich}

Es wurde sich größtenteils an die Coding-Guidelines gehalten. An wichtigen Stellen sowie vor jeder Funktion wurden Kommentare
geschrieben. Die Trennung zwischen Views und Components sowie die Auslagerung der Style-Dateien wurde ebenfalls eingehalten.

\subsection{Konkrete Test-Überdeckung im Sprint}
{\small Autor: Linus Herterich}

Ein Student wurde beauftragt bis zum Ende des Sprints ein geeignetes Test-Framework zu finden.
Somit wurden während des Sprints noch keine Tests geschrieben.

\subsection{Ergebnisse des Reviews}
{\small Autor: Linus Herterich}

Es wurden fast alle UserStories umgesetzt. Somit war der zweite Sprint erfolgreich.
Alle Studenten konnten sich in das Projekt einarbeiten und haben die Strukturierung
größtenteils verstanden und eingehalten.
\\ Das User-Interface wurde nach der Designvorlage umgesetzt und die ersten Graphen-Funktionen
(Hinzufügen von Knoten und Kanten \& Optimieren) funktionieren bereits.
\\ Da noch nicht feststeht, wo die Software gehostet werden soll und wie die Datenbank-Funktionalität
umgesetzt werden soll, wurde zunächst eine lokale Speicherlösung als "Datenbank" verwendet. Somit konnten
die Speichern- und Laden-Funktionen erfolgreich implementiert werden.
\\ Die Login-Funktionalität ist derzeit nur sporadisch eingerichtet und wird finalisiert,
sobald feststeht, wie die Authentifizierung der Nutzer erfolgen soll (Anbindung an HTWK Login?).
\\ Leider ist immernoch kein geeignetes Testframework gefunden worden, mit dem sich sowohl Vue.js
als auch cytoscape (Graphen-Funktionalitäten) testen lassen.

\subsection{Ergebnisse der Retrospektive}
{\small Autor: Linus Herterich}

Das Happiness-Barometer für diesen Sprint ist sehr gut ausgefallen. Das liegt hauptsächlich an der guten Aufgabenverteilung
sowie an den großen Erfolgen, die diesen Sprint erzielt wurden.
\\ Kritisiert wurde die die Kommunikation gegen Ende des Sprints. Das finale Mergen aller Branches war zu hektisch und unsicher.
\\ Es wurde sich darauf geeinigt in Zukunft zwei Dailies pro Woche abzuhalten und das letzte Meeting eines Sprints zum gemeinsamen Mergen zu verwenden.

\subsection{Abschließende Einschätzung des Product-Owners}
{\small Autor: Manuel Eckert}

Aus den bei dem Planning-Meeting vorgestellten User-Stories ergaben sich drei Subteams. Diese teilten sich in die Bereiche Login, UI-Design und Graph-Funktionalitäten auf. Damit wurde das konkretere Aufteilen der User-Stories auf Subteams umgesetzt. \\
Dies hatte einen positiven Einfluss auf die Anzahl der erfolgreich abgeschlossen Aufgaben. \\
Während des Reviews wurden fehlende Code Kommentare und eine zu niedrige Testabdeckung benängelt.


\subsection{Abschließende Einschätzung des Software-Architekten}
{\small Autor: Julius Jolig}

In diesem Sprint wurden bereits mehr Kommentare im Code verfasst, aber hier ist noch Luft nach oben. Die Bachelorstudenten haben sich gut mit Vue.js und cytoscape vertraut gemacht und gute Ergebnisse erzielt. Das Mergen lief trotz neuem Ansatz immer chaotisch ab.  

\subsection{Abschließende Einschätzung des Team-Managers}
{\small Autor: Alex Hofmann}

Deutliche Leistungssteigerung schon jetzt zu sehen. Aufteilung der User-Stories direkt nach dem Planning hat die Arbeitsstruktur und -ablauf während des Sprints auf jeden Fall positiv beeinflusst.



\newpage

\section{Sprint 3}

%
\subsection{Ziel des Sprints}
{\small Autor: Lennart Buchmann}

Nach der Einarbeitung des gesamten Teams in die Grundstruktur der Software, sowie der Frameworks, lag das Hauptaugenmerk des 
dritten Sprints in der verstärkten Herausarbeitung der geplanten Kernfunktionalitäten der Anwendung. Größere Aufgabenbereiche wurden 
durch Zweier- und Dreierteams gelöst. Übriggebliebenes aus den vorherigen Sprints sollte beendet werden 

\subsection{User-Stories des Sprint-Backlogs}
{\small Autor: Lennart Buchmann}

\begin{itemize}

  \item \textbf{Funktionalität der Datenbank}
        \\\textit{Als Nutzer will ich meine gespeicherten Graphen ansehen können, um diese weiter bearbeiten zu können.}
\item \textbf{Kontext Menu über rechte Maustaste}
        \\\textit{ Als Nutzer möchte ich Knoten und Kanten über einen Rechtsklick zur einfacheren Benutzung erstellen können.}
  \item \textbf{Authentifizierung eines Nutzers}
        \\\textit{Als Nutzer möchte ich mich in die Web Applikation einloggen können,
        damit nicht jeder meine erzeugten Graphen einsehen kann.}
  \item \textbf{Darstellung von Knoten und Kanteneigenschaften am Objekt}
        \\\textit{Als Benutzer möchte ich über einen Rechtsklick auf einen Knoten/Kante die Eigenschaften dieser bearbeiten können.}
  \item \textbf{Optimierung des Graphs}
        \\\textit{Als Benutzer möchte ich gerne sofort sehen können, wie hoch meine Kosten für den kürzesten Pfad sind, damit ich mich möglichst schnell für einen entscheiden kann.}
\item \textbf{Speicherung Graph}
        \\\textit{Als Nutzer möchte ich einen Graphen jederzeit bearbeiten und speichern können, auch wenn dieser noch unfertig ist.}

\end{itemize}


\subsection{Liste der durchgeführten Meetings}
{\small Autor: Lennart Buchmann}

\begin{itemize}
  \item 06.01.2020: Planning Meeting
  \item 09.01.2020: Weekly Scrum
  \item 13.01.2020: Weekly Scrum
  \item 16.01.2020: Weekly Scrum
  \item 20.01.2020: Review \&  Retrospektive Meeting
\end{itemize}


\subsection{Ergebnisse des Planning-Meetings}
{\small Autor: Lennart Buchmann}

Der 3. Sprint ist der letzte Sprint im laufenden Semester und der letzte Sprint vor den anstehenden Prüfungen. Während des Planning-Meetings wurde von allen einheitlich besprochen, dass die
Arbeitslast von jedem höher ist als während der vergangen Sprints. Es wurde sich daraufhin geeinigt lieber realistische Ziele zu setzen, sodass der 3. Sprint auch mit höhere Belastung erfolgreich 
abgeschlossen werden kann. Nach Besprechung und Schätzung der Tickets, wurden alle Aufgaben in kleinere Gruppen aufgeteilt. Größere Aufgaben, die nach Schätzung im aktuellen Sprint nicht umsetzbar wären, wurden auf den verlängerten 4. Sprint verschoben. 


\subsection{Aufgewendete Arbeitszeit pro Person$+$Arbeitspaket}
{\small Autor: Lennart Buchmann}

\begin{longtable}{|p{4cm}|p{2cm}|p{1.2cm}|p{1.2cm}|p{0.7cm}|p{3.8cm}|}
  \hline
  Arbeitspaket                                                          & Person                & Start    & Ende     & h     & Artefakt                                                    \\ \hline
  Login                                                             & Berger, Matthias      & 13.01.20 & 17.01.20 & 18   & Login Funktionalität \& Design                              \\ \hline
  Login                                                             & Buchmann, Lennart     & 18.01.20 & 18.01.20 & 6     & Login Funktionalität \& Design                              \\ \hline
  Knotendarstellung nach Designvorlage        & Gwozdz, Jonas         & 15.01.20 & 20.01.20 & 4,5     & GraphHeader.vue, Toolbar.vue                                \\ \hline
  Speicherung Graph			        &  Heldt, Erik           & 06.01.20 & 19.01.20 & 18  & BasicData.js                                                \\ \hline
  Optimierung des Graphs 			        & Henning, Tim          & 09.01.20 & 18.01.20 & 9     & OptimizeControls.vue, index.js -> Graph Highlighting        \\ \hline
  Speicherung Graph                                      & Herterich, Linus      & 07.01.20 & 19.01.20 & 24,25     & meeting\_19\_12\_19.pdf                                     \\ \hline
  Graph zu Datenstruktur hinzufügen             & Hohlfeld, Julius      & 07.01.20 & 20.01.20 & 19     & BasicData.js, TestDatabase.js                               \\ \hline--
  Funktionalität Neuer Graph Button               & Karkoutli, Alaa Aldin & 12.01.20 & 15.01.20 & 7  & Database.vue (view), DatabaseForm.vue (component)           \\ \hline
  Kanten zu Graph hinzufügen                         & Koch, David           & 17.01.20 & 19.01.20 & 10    & Änderungen an index.js, CreateControls.vue (component)      \\ \hline
\end{longtable}

\subsection{Konkrete Code-Qualität im Sprint}
{\small Autor: Lennart Buchmann}



\subsection{Konkrete Test-Überdeckung im Sprint}
{\small Autor: Lennart Buchmann}

Eine konkrete Auseinandersetzung mit Tests beziehungsweise entsprechenden Test-Frameworks fand während des 2. Sprints statt. Momentan befinden sich alle Teammitglieder noch in der Einarbeitungsphase. Aufgrund des fortgeschrittenes Semesters und der anstehenden Prüfungen lagen die Prioritäten vorwiegend auf der Bearbeitung der User-Stories. 


\subsection{Ergebnisse des Reviews}
{\small Autor: Lennart Buchmann}

Das Ergebnis der Reviews war in anbetracht der fortgeschrittenen Semesters durchgehenden positiv. Alle Teammitglieder haben die Ihnen zugewiesenen Aufgaben innerhalb des Sprints erledigt. 
Es wurde des Weiteren besprochen, dass der verlängerte Sprint während der Semesterferien dazu genutzt werden sollte, um Bugs zu beheben und somit jedem die Gelegenheit zu geben, sich in die Testframeworks einzuarbeiten und Tests für den geschriebenen Code zu verfassen.


\subsection{Ergebnisse der Retrospektive}
{\small Autor:  Lennart Buchmann}

Während der Retrospektive wurde von allen die grundsätzliche gute Kommunikation innerhalb des Teams gelobt. Alle empfanden auch die Aufteilung in kleinere Zweier- und Dreierteams zur Bearbeitung von Aufgaben für sehr hilfreich.  Eine gleichbleibende hohe Motivation und Produktivität soll auch während des Semesterferiensprints beibehalten werden. Punkte, welche verbessert werden sollten, sind das pünktliche Mergen der einzelnen Branches vor Ende des Sprints, das Kommentieren des Codes und das Verfassen von Tests. 


\subsection{Abschließende Einschätzung des Product-Owners}
{\small Autor: xxx}

XXX

\subsection{Abschließende Einschätzung des Software-Architekten}
{\small Autor: xxx}

XXX

\subsection{Abschließende Einschätzung des Team-Managers}
{\small Autor: Alex Hofmann}

Weiterhin aufstrebende Arbeit vom Team. Auch die Kommunikation bei Problemen, Fragen und Anregungen geht in eine positive Richtung.



\newpage

\section{Sprint 4}

\subsection{Ziel des Sprints}
{\small Autor: Jonas Gwozdz}

Während der Semesterferien haben wir an Sprint 4 weitergearbeitet. Dieser dauerte vom 23.01.2020 bis zum  09.04.2020. Der Ablauf war dabei weitestgehend planmäßig, bis auf dass die Meetings zum Review und der Retrospektive wegen Corona ohne persönliches Treffen stattfinden mussten.
In der Vorlesungsfreien Zeit besprachen wir uns gelegentlich über den aktuellen Zwischenstand. Der größte Fortschritt am Projekt wurde während der letzten beiden Wochen erzielt.

\subsection{User-Stories des Sprint-Backlogs}
{\small Autor: Jonas Gwozdz}

\begin{itemize}
  \item \textbf{Tests für bereits geschriebenen Code}
        \\\textit{Als Benutzer möchte ich eine Software benutzen, die getestet ist, damit keine unerwarteten Probleme auftauchen.}
  \item \textbf{ Validierung der möglichen Eingaben }
        \\\textit{
          Als Nutzer möchte ich bei versehentlicher falscher Eingabe wenn möglich gewarnt werden, damit ich nichts falsches abspeichere.}
  \item \textbf{Bug: Validation bei gleichem Knoten-Namen}
  \item \textbf{Darstellung von Kanten/Attributen }
        \\\textit{
          Als Benutzer will ich alle Kanten/Knoten gleichzeitig sehen können(nicht übereinander), damit ich einen schnelleren Überblick über das gesamte Konstrukt bekomme.}
  \item \textbf{Bug: Mehrere Edges zwischen Knoten nicht möglich}
        \\\textit{
          Wenn man mehrere Kanten zwischen zwei Knoten anlegt, sind diese nicht sichtbar. Löscht man dann einen Knoten, an dem diese "unsichtbaren" knoten hängen, so stürzt cytoscape ab.}
  \item \textbf{Remodel von Component NewGraph}
\end{itemize}

\subsection{Liste der durchgeführten Meetings}
{\small Autor: Jonas Gwozdz}

\begin{itemize}
\item 23.01.2020: Planning
\item 05.03.2020: Weekly
\item 12.03.2020: Weekly
\item 06.04.2020: Review
\item 09.04.2020: Retro
\end{itemize}

\subsection{Ergebnisse des Planning-Meetings}
{\small Autor: Jonas Gwozdz}

Anwesend: Alex, Julius J., Julius H., Linus, Jonas, Erik, Lennart, Nils, Tim, David, Matthias, Manuel\\
\\
Innerhalb dieses Meetings haben wir die Schwerpunkte des Sprints festgelegt und über den Workload über die Vorlesungsfreie Zeit diskutiert und den Zeitaufwand der User-Stories abgeschätzt.\\


\textbf{oberste Priorität: Tests}\\
Da wird bis zum bisherigen Zeitpunkt keine Testumgebung gefunden haben, die sich auf unseren Cytoscape-Graphen anwenden lässt, und wir dadurch viel Nachholbedarf in Sachen Testen hatten, musste dieses Ticket am dringendsten abgearbeitet werden.\\

\textbf{Sprint über Semesterferien}\\
Wir haben uns im Planning darauf geeinigt, den Sprint über die Semesterferien mit weniger User-Stories als üblich auszulegen, da nicht alle Teammitglieder in dieser Zeit voll verfügbar waren, Grund dafür waren vor Allem die noch andauernden Prüfungen und die Anschließenden Ferien, die evtl. schon anderweitig verplant waren. Zudem haben wir uns darauf geeinigt, regelmäßig Absprache über den Fortschritt unserer Arbeit zu halten.\\

\textbf{Datenbanken}\\
Die Datenbankrecherche hat ergeben, dass für unsere Zwecke mySQL oder NodeJS am optimalsten wäre. Die Definition der Datenbankschnittstelle zwischen DB und Frontend muss ebenfalls noch erledigt werden. Zudem haben wir festgestellt, dass die Bisher entworfene Datenbankoberfläche optisch nicht zum Rest der Anwendung passt, und deshalb überarbeitet werden muss.\\

\textbf{Weitere Sprintziele:}
\begin{itemize}
\item Optimierung der Kostendarstellung
\item negative Zahleingaben abfangen
\item automatische Zoomfunktion bei Knoten- oder Kantenwahl
\item allgemeine Bugfixes
\end{itemize}


\subsection{Aufgewendete Arbeitszeit pro Person$+$Arbeitspaket}
{\small Autor: Jonas Gwozdz}

\begin{longtable}{|p{4cm}|p{2cm}|p{1.2cm}|p{1.2cm}|p{0.7cm}|p{3.8cm}|}
  \hline
  Arbeitspaket                                                          & Person                & Start    & Ende     & h     & Artefakt                                                    \\
  \hline
  Tests für bereits geschriebenen Code                                  & Heldt, Erik           & 04.03.20 & 04.03.20 & 2     & Tests für ModifyDataControls.vue                            \\ \hline
  Neue Strukturierung                                                   & Heldt, Erik           & 26.01.20 & 26.01.20 & 1     & Umstrukturierung des Projekts                               \\ \hline
  Header Buttons und Metadaten-Speicherung                              & Heldt, Erik           & 05.03.20 & 12.03.20 & 6,75  & GraphHeader.vue                                             \\ \hline
  Aufräumen der Branches im GitLab                                      & Heldt, Erik           & 29.03.20 & 29.03.20 & 1     & Organisatorische Aufgabe                               \\ \hline
  Entfernen veralteter Komponenten und Methoden                         & Heldt, Erik           & 31.03.20 & 31.03.20 & 2     & Organisatorische Aufgabe                                             \\ \hline
  Tests für Graphoptimierung                                            & Henning, Tim          & 04.04.20 & 40.40.20 & 12    & vargraph.spec.js        \\ \hline
  Tests für bereits geschriebenen Code                                  & Herterich, Linus      & 30.01.20 & 12.02.20 & 7,5   & /code/cypress/integration/...                                     \\ \hline
  Header Buttons und Metadaten-Speicherung                              & Herterich, Linus      & 28.03.20 & 31.03.20 & 2,25  & /vargraph/graph/... \& GraphHeader.vue                  \\ \hline
  Aufräumen der Branches im GitLab                                      & Herterich, Linus      & 30.03.20 & 30.03.20 & 1     & Organisatorische Aufgabe  \\ \hline
  Darstellung von Kanten/Attributen                                     & Herterich, Linus      & 03.04.20 & 03.04.20 & 2     & VarGraph.vue               \\ \hline
  Remodel von Component NewGraph                                        & Herterich, Linus      & 30.03.20 & 30.03.20 & 3     & /vargraph/graph/...                   \\ \hline
  Refactoring                                                           & Herterich, Linus      & 29.03.20 & 30.03.20 & 9     & /vargraph/graph/...                                 \\ \hline
  Validierung: Login                                                    & Herterich, Linus      & 31.03.20 & 30.03.20 & 1,5   & /components/login/LoginForm                                  \\ \hline
  Einheitliche Alerts                                                   & Herterich, Linus      & 31.03.20 & 31.03.20 & 3     & Dialogs.vue \\ \hline
  Validierung CreateControls \& DetailControls                          & Herterich, Linus      & 31.03.20 & 01.04.20 & 5,5   & CreateControls.vue \& DetailControls.vue               \\ \hline
  Bug: Mehrere Edges zwischen Knoten nicht möglich                      & Herterich, Linus      & 01.04.20 & 01.04.20 & 2     & /vargraph/graph/...                   \\ \hline
  Knoten dort erstellen, wo rechtsklick passiert                        & Herterich, Linus      & 01.04.20 & 01.04.20 & 1,5   & /vargraph/graph/...                               \\ \hline
  keybinds für Menüs                                                    & Herterich, Linus      & 02.04.20 & 02.04.20 & 1     &                                   \\ \hline
  Keine Knoten aufeinander schieben                                     & Herterich, Linus      & 02.04.20 & 02.04.20 & 3     & /vargraph/graph/...  \\ \hline
  Einstellungsmenü erstellen                                            & Herterich, Linus      & 03.40.20 & 05.04.20 & 5,5   &              \\ \hline
  Tests für bereits geschriebenen Code                                  & Hohlfeld, Julius      & 05.02.20 & 04.03.20 & 10    & ZoomControls.spec \& SaveMenu.spec \& NewGraphMenu.spec \& DownloadMenu.spec \\ \hline
  Dialogfenster für Speichern, Laden und Export                         & Hohlfeld, Julius      & 24.01.20 & 24.01.20 & 2     & Toolbar.vue \\ \hline
  Validierung der möglichen Eingaben                                    & Hohlfeld, Julius      & 06.04.20 & 06.04.20 & 2     & divers                             \\ \hline
  Refactoring                                                           & Hohlfeld, Julius      & 31.03.20 & 31.03.20 & 2     & /vargraph/graph/...                     \\ \hline
  Testing für Kanten hinzufügen                                         & Koch, David           & 22.03.20 & 02.04.20 & 5     & addEdges.spec      \\ \hline
\end{longtable}

\subsection{Konkrete Code-Qualität im Sprint}
{\small Autor: Jonas Gwozdz}

Die Codequatlität im allgemeinen wurde während des Sprints erheblich durch das Refactoring verbessert. Zudem wurden in nahezu  allen Dateien einleitende Kommentare geschrieben, um die zukünftige Identifizierung der gebrauchten Dateien schneller und übersichtlicher zu gestalten.

\subsection{Konkrete Test-Überdeckung im Sprint}
{\small Autor: Jonas Gwozdz}

Die geschriebenen Cypress-Tests decken bereits eine Vielzahl an Funktionalitäten des Programms ab. Dazu zählen die Buttons für die Database, den Download, das Ausloggen. Zudem wurde getestet: der Speicherdialog, die Zoomeinstellungen, der Header des Graphen, das Hinzufügen von Knoten und das Erstellen eines neuen Graphen.

\subsection{Ergebnisse des Reviews}
{\small Autor: Jonas Gwozdz}

Anwesend: David, Erik, Julius J., Julius H., Jonas, Linus, Manuel, Matthias, Tim\\

Im Rahmen des Reviews haben wir wie gewohnt die Ergebnisse des Sprint bewertet und Schwierigkeiten besprochen.\\

\textbf{generelle Schwierigkeit: Testen}\\
Um unsere Programm zu testen, entschieden wir uns für das Framework "Cypress" entschieden. dieses bietet End-to-End Testing an, welches allerdings nur Ausgaben des Programms auswerten kann, und deshalb sozusagen keinen Blick unter die Haube zulässt, und somit eventuell Fehler unentdeckt bleiben. \\

\textbf{David:}
\begin{itemize}
\item Tests für Knotenfunktionalität geschrieben
\item mit Kantentests begonnen
\end{itemize}

\textbf{Erik:}
\begin{itemize}
\item Data Controls durch Header Buttons ersetzt
\item Editierungsfenster entfernt
\item Header Buttons getestet
\end{itemize}

\textbf{Jonas:}
\begin{itemize}
\item Testübersicht erstellt
\item Möglichkeit zum Informationsaustausch über Lücken und Bugs in Tests bereitgestellt
\end{itemize}

\textbf{Julius H.:}
\begin{itemize}
\item Tests für Toolbar, Zoom-Controls, Buttons und Eingabereihenfolgen geschrieben
\end{itemize}

\textbf{Julius H, Erik, Linus:}
\begin{itemize}
\item Refactoring des Graphen, Bugfixing und Validierung von Eingaben
\end{itemize}

\textbf{Linus:}
\begin{itemize}
\item Dialogue-Popups erstellt
\item Kürzelgenerierung implementiert
\item Knotenüberlagerung unterbunden, Mindestabstand implementiert
\item Einstellungsmenü erstellt und Implementation begonnen
\item Recherche zu Datenbankfenster
\end{itemize}

\subsection{Ergebnisse der Retrospektive}
{\small Autor: Jonas Gwozdz}

Anwesend: Alex, Erik, Julius J., Julius H., Jonas, Linus, Matthias, Tim\\

Zu Beginn des Sprints gab es keine Fortschritte zu vermelden, da vorerst die Prüfungen zu überstehen waren. In den beiden Wochen vor Sprintende wurden allerdings die wichtigsten User-Stories und sogar etwas mehr abgearbeitet.\\

\begin{center}
\begin{tabular}{ |c|c| }
\hline
 Positiv & Negativ \\
\hline 
 -produktive Endphase & -anfangs keine Kommunikation \\
 -viel Motivation bei Einigen & - wenig Motivation bei Einigen\\
 & -vereinzelt Tests ohne Sinn\\
 & -ausgefallene Meetings\\
\hline     
\end{tabular}
\end{center}
 

\subsection{Abschließende Einschätzung des Product-Owners}
{\small Autor: xxx}

XXX

\subsection{Abschließende Einschätzung des Software-Architekten}
{\small Autor: xxx}

XXX

\subsection{Abschließende Einschätzung des Team-Managers}
{\small Autor: Alex Hofmann}

Aufgrund der vorlesungsfreien Zeit war mit erhöhter Inaktivität aufgrund von Prüfungen, Urlaub und sonstigen Auszeiten zu rechnen.
Das Team hat sich dennoch demokratisch für einen Sprint während dieser Zeit entschieden. Trotz aller Umstände wurde mit der Umsetzung der Testfälle die Zielvorgabe erreicht.



\newpage

%%%%%% weitere Sprints analog


\section{Dokumentation}

\subsection{Handbuch}
{\small Autor: xxx}

XXX

\subsection{Installationsanleitung}
{\small Autor: Erik Heldt}

VarG ist eine plattformunabhängige Webanwendung, das heißt man muss nichts lokal auf seinem PC installieren, um sie zu benutzen. Alles was man benötigt, ist ein moderner Web-Browser und eine Internetverbindung (Browser-Empfehlung: Google Chrome oder Firefox). Öffne den Browser und gib in der URL-Leiste www.TODO-sampledomain.de ein. Nun befindest du dich im Home-Menü von VarG und kannst loslegen!

TODO: Installation der Datenbank dokumentieren sobald diese auf den HTWK-Server umgezogen ist.

\subsection{Software-Lizenz}
{\small Autor: xxx}

XXX


\section{Projektabschluss}

\subsection{Protokoll der Abnahme und Inbetriebnahme beim Kunden}
{\small Autor: xxx}

XXX

\subsection{Präsentation auf der Messe}
{\small Autor: xxx}

Poster, Bericht

\subsection{Abschließende Einschätzung durch Product-Owner}
{\small Autor: xxx}

XXX

\subsection{Abschließende Einschätzung durch Software-Architekt}
{\small Autor: xxx}

XXX

\subsection{Abschließende Einschätzung durch Team-Manager}
{\small Autor: xxx}

XXX

\end{document}
