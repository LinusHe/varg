\documentclass[twoside]{report}

% ------
% Umlaute
\usepackage{ifluatex,ifxetex}
\ifluatex
  \usepackage{fontspec}
\else
  \ifxetex
    \usepackage{fontspec}
  \else
    \usepackage{selinput}
    \SelectInputMappings{
      adieresis={ä},
      germandbls={ß},
    }
    \usepackage[T1]{fontenc}
    %\usepackage{textcomp}% optional
    %\usepackage{lmodern}
  \fi
\fi

% ------
% Paper auf Deutsch
\usepackage[ngerman]{babel}



% ------
% Page layout
\usepackage[hmarginratio=1:1,top=32mm,columnsep=20pt]{geometry}
\usepackage[font=it]{caption}
\usepackage{paralist}
%\usepackage{multicol}


% ------
% Abstract
\usepackage{abstract}
	\renewcommand{\abstractnamefont}{\normalfont\bfseries}
	\renewcommand{\abstracttextfont}{\normalfont\small\itshape}


% ------
% Titling (section/subsection)
\usepackage{titlesec}
\renewcommand\thesection{\Roman{section}}
\titleformat{\section}[block]{\Large\scshape\bfseries}{\thesection.}{1em}{}
\setcounter{secnumdepth}{3}

% ------
% Tabellen über Seitenumbrüche hinweg
\usepackage{longtable}

% ------
% Header/footer
\usepackage{fancyhdr}
	\pagestyle{fancy}
	\fancyhead{}
	\fancyfoot{}
	\fancyhead[C]{Projektdokumentation $\bullet$ VarG $\bullet$ WS2019/20$+$SS2020}
	\fancyfoot[RO,LE]{}


% ------
% Clickable URLs (optional)
\usepackage{hyperref}

% ------
% Literaturverweise mit Bibtex einbinden
\usepackage[authoryear,sectionbib,round]{natbib}

% ------
% Bilder laden
\usepackage[pdftex]{graphicx}

% ------
% Maketitle metadata
\title{\vspace{-5mm}%
	\fontsize{24pt}{10pt}\selectfont
	\textbf{Projektdokumentation}
	}	
\author{%
        % alle Autoren hier listen
        % 
	\large
	\textsc{Linus Herterich -- linus.herterich@stud.htwk-leipzig.de} \\[2mm]
	\textsc{David Koch -- david.koch.1@stud.htwk-leipzig.de} \\[2mm]
	\textsc{Jonas Gwozdz -- jonas.gwozdz@stud.htwk-leipzig.de} \\[2mm]
	\textsc{Erik Heldt -- erik.heldt@stud.htwk-leipzig.de} \\[2mm]
	\textsc{Alaa Aldin Karkoutli -- alaa\_aldin.karkoutli@stud.htwk-leipzig.de} \\[2mm]
	\textsc{Julius Hohlfeld -- julius.hohlfeld@stud.htwk-leipzig.de} \\[2mm]
	\textsc{Tim Henning -- tim.henning@stud.htwk-leipzig.de} \\[2mm]
	\textsc{Julius Jolig -- julius.jolig@stud.htwk-leipzig.de} \\[2mm]
	\textsc{Lennart Buchmann -- lennart.buchmann@stud.htwk-leipzig.de} \\[2mm]
	\textsc{Matthias Berger -- matthias.berger@stud.htwk-leipzig.de} \\[2mm]
	\normalsize	HTWK Leipzig 
	}
\date{}



%%%%%%%%%%%%%%%%%%%%%%%%
\begin{document}


% -------
% Titel und Abstract über beide Spalten
%\twocolumn[
%\begin{@twocolumnfalse}

\maketitle
\thispagestyle{fancy}

\tableofcontents

%%%%
%%%% Die Struktur des Dokuments bitte nicht aendern!!!
%%%%

\section{Anforderungsspezifikation}

\subsection{Initiale Kundenvorgaben}
{\small Autor: Jonas Gwozdz, Korrektur: Linus Herterich, Tim Henning}

\vspace{\baselineskip}
\noindent Die Vorgaben unseres Kunden, Prof. Gürtler, ließen uns viele Freiheiten in der Gestaltung des Programms. \par

\noindent Die gegebenen Vorgaben legten das Folgende fest: \par
\begin{itemize}
  \item Für die Herstellung eines Werkstücks gibt es in der Regel mehrere Alternativen.
  Ausgehend von einem Rohteil folgen verschiedene   Bearbeitungsschritte, die zum Fertigteil führen. 
  Die einzelnen Bearbeitungsschritte sollen mit einer Bearbeitungszeit und Kosten belegt werden. \par

  \item Im Tool soll ein horizontaler Graph erstellt werden können, auf dem die Teile als Knoten und 
  die Bearbeitungsschritte zwischen ihnen als Kanten dargestellt sind. Im Nachhinein soll der optimale 
  Fertigungsprozess nach Kosten oder Zeit ausgewertet werden. Alle hierfür benötigten Daten sollen im Tool eingegeben werden können.\par

  \item Rüstkosten und -zeit fallen pro Charge (festlegbar per Losgröße, mindestens aber ein mal) bei jedem Bearbeitungsschritt an. 
  Zusätzlich werden pro Stück für jeden Bearbeitungsschritt die benötigten Kosten und Zeit mit einbezogen. \par

  \item Das Tool soll bei Vorgabe all dieser Größen die günstigsten Varianten (nach Kosten, Zeit oder einer Kombination der beiden) 
  berechnen und die nächstbesten Varianten ausgeben.
\end{itemize}

\subsection{Produktvision}
{\small Autor: Alex Hofmann}
\\

\noindent Product Vision Board: \\
\begin{tabular}{|p{50mm}|p{50mm}|p{50mm}|}
  \hline
  \textbf{Target Group}                                                  & \textbf{Needs}                                                                                                                        & \textbf{Product}                                                                                                                                                                                                 \\
  \hline
  -Maschinenbau-Studenten \newline -Maschinenbau-Profs \newline -Lehrende & Vgl. zu händisch: \newline einheitlicher, schneller \newline -plattformunabhängig \newline -Open Source \newline -Einfach zu bedienen & -Webanwendung \newline -Als Graph \newline $\rightarrow$ quasi als Baukasten \newline $\rightarrow$ Kantengewichtung, Bausteine wählbar \newline -Import/Export von Modellen \newline Normalisierung des Graphen \\
  \hline
\end{tabular}
\\

\noindent Die Webanwendung VarG wird entwickelt für Lehrende und Lernende aus dem Maschinenbau Bachelorstudiengang.
Diese erleichtert die einheitliche Erstellung, Bearbeitung, Optimierung sowie Im- bzw. Exportierung von sogenannten Variantenfolgegraphen, kurz VarGraphs. Darunter ist eine graphische Übersicht zu verstehen, die die möglichen Varianten eines Produktionsprozesses für ein Werkstück darstellt.
\\
\\\textbf{Später überarbeitete Produktvision bzw. neue Projektbeschreibung:}
\\Die plattformunabhängige Open-Source Webanwendung ''VarG'' soll es Lehrenden und Lernenden aus dem Studiengang Maschinenbau ermöglichen, einfach und schnell Variantenfolgegraphen, kurz VarGraphs, zur Herstellung von Werkstücken zu visualisieren und nach verschiedenen Kriterien die günstigsten Wege berechnen und anzeigen zu lassen. Dafür stehen ihnen viele Features für Aufbau, Funktionsweise, Design, Export und Import zur Verfügung.


% Das hier ist ein Absatz, der die Grafik in Abbildung~\ref{fig:bild1} detailliert erläutert, erklärt und interpretiert.

% \begin{figure}[b]
%   \centering
%   \includegraphics[width=4.5cm]{bspbild1.png}
%   \caption{Beispiel für ein einspaltiges Bild}
%   \label{fig:bild1}
% \end{figure}


\subsection{Liste der funktionalen Anforderungen}
{\small Autor: Erik Heldt}

\begin{itemize}
  \item Erstellen von Zuständen (Teilen) mit Namen \& Kürzel
  \item Erstellen von Bearbeitungsschritten mit Namen \& Kürzel zwischen je 2 Zuständen
  \item Zuweisen von Rüstzeit, Geldkosten \& Losgröße zu Bearbeitungsschritten
  \item Anzeigen des günstigsten Weges im Graph, berechnet nach Zeit oder Kosten
  \item Lokaler Export als Bilddatei oder importierbarer JSON \& Lokaler Import als JSON
  \item Hochladen in online gehostete Datenbank \& Laden aus online gehosteter Datenbank
  \item Login-Management für Zugriffskontrolle auf Anwendung
  \item Rollen-Management (Student, Professor) für Zugriffsrechte auf Datenbank
\end{itemize}

%
% soll der Inhalt dieser Subsection in einer separaten Datei
% (z.B. listefunktional.tex) liegen, dann kann dies mit dem
% folgenden Kommando geschehen.
%
% \input{listefunktional}

\subsection{Liste der nicht-funktionalen Anforderungen}
{\small Autor: Erik Heldt}

\begin{itemize}
  \item Schnelle Einarbeitung in die Anwendungsumgebung
  \item Einfacher \& intuitiver Umgang mit den Programmkomponenten und -funktionen
  \item Stabiler \& konsistenter Programmablauf, keine Abstürze oder Verluste von Dateien
  \item Kompatibilität mit so vielen modernen Browsern wie möglich
  \item Sicherheit \& korrekte Funktionalität des Login-Algorithmus und des DB-Rollenmanagements
  \item Datenschutz bei Login-Sessions einhalten
\end{itemize}

\subsection{Weitere Zuarbeiten zum Produktvisions-Workshop}
{\small Autor: Erik Heldt}

Für den Produktvisions-Workshop wurden 4 Dokumente erstellt, welche unterschiedliche Aspekte des Anwendungsentwurfs behandeln:
\begin{itemize}
  \item Ideen zur Darstellung der GUI inklusive eines interaktiven GUI-Prototyps auf Adobe XD
  \item Epic bzw. eine Zusammenfassung vieler User-Stories zu allgemeinen Anforderungen an die Funktionalität
  \item Formulierung der Kernfunktionen des Programms
  \item Datenmodellierung des Programms
\end{itemize}
Die nachfolgende Liste an Zuarbeiten sind klickbare Verweise auf die jeweiligen Dokumente im ''zuarbeiten''-Ordner.

\subsubsection{Zuarbeit von Linus Herterich, Jonas Gwozdz, Julius Hohlfeld}
\href[pdfnewwindow=true]{file:zuarbeiten/Produktvision_-_VarG_GUI.pdf}{VarG GUI}

\subsubsection{Zuarbeit von Erik Heldt, Alaa Aldin Karkoutli}
\href[pdfnewwindow=true]{file:zuarbeiten/Erste_User-Story.pdf}{Erstes Epic}

\subsubsection{Zuarbeit von Lennart Buchmann, Nils Buxel, Matthias Berger}
\href[pdfnewwindow=true]{file:zuarbeiten/Produktvision_Kernfunktionalität.pdf}{Kernfunktionalität}

\subsubsection{Zuarbeit von Tim Henning, David Koch}
\href[pdfnewwindow=true]{file:zuarbeiten/Produktvision_-_Datenmodell.pdf}{Datenmodell}

\subsection{Liste der Kundengespräche mit Ergebnissen}
{\small Autor: Manuel Eckert}

\begin{tabular}[t]{|p{2cm}|p{4,5cm}|p{8cm}|}
  \hline Datum    &  Grund  &  Ergebnis\\ \hline
  30.10.2019 & Kickoff &  Überblick über die Anforderungen,\newline angenähertes Verständnis für das Produkt, \newline  Kennenlernen des Kunden,\\ \hline
  27.11.2019 & Produktvision & Abgleich Produktvision, \newline Weiteres Verständnis für das Produkt \\ \hline
  21.01.2020 & Zwischenstand & Auslieferung des Ersten Produktinkrements, \newline Feedback zum ausgelieferten Produkt \\ \hline
  08.07.2020 & Abnahme \& Retrospektive & Auslieferung des Produktes, \newline Rückblick und Feedback auf den Projektverlauf \\ \hline
  \end{tabular}
\\

Aufgrund der besonderen Situation während der letzten Monaten wurde die Kommunikation im Sommersemester 2020 zum Kunden hauptsächlich über E-Mails geführt. Zur verbesserten Feedbackgebung wurde das Projekt nach jedem Sprint auf einen Webserver gespielt. Somit war es dem Kunden möglich genauere Anforderungen und Wünsche zu formulieren

\section{Architektur und Entwurf}

\subsection{Zuarbeiten der Teammitglieder}
{\small Autor: Erik Heldt}

Für die Technologierecherche informierte sich das Team über verschiedene Technologien, mit denen die Anwendung entwickelt werden kann. Außerdem fassten wir erste Ideen zur Graphenanordnung zusammen und legten Coding Conventions fest. Die Ausarbeitungen wurden in den nachfolgenden Dokumenten festgehalten.
\\
\\Die nachfolgende Liste an Zuarbeiten sind klickbare Verweise auf die jeweiligen Dokumente im ''zuarbeiten''-Ordner.

\subsubsection{Zuarbeit von Tim Henning}
\href[pdfnewwindow=true]{file:zuarbeiten/Django_und_Python.pdf}{Django und Python}

\subsubsection{Zuarbeit von Erik Heldt}
\href[pdfnewwindow=true]{file:zuarbeiten/Ruby_on_Rails.pdf}{Ruby on Rails}
\\\href[pdfnewwindow=true]{file:zuarbeiten/Ruby_on_Rails_Kurz.pdf}{Ruby on Rails (kurz)}
\\\href[pdfnewwindow=true]{file:zuarbeiten/Graphenanordnung.pdf}{Graphenanordnung}

\subsubsection{Zuarbeit von David Koch}
\href[pdfnewwindow=true]{file:zuarbeiten/Java_Canvas.pdf}{Java Canvas}
\\\href[pdfnewwindow=true]{file:zuarbeiten/Recherche_Datenbanken.pdf}{Datenbanken}

\subsubsection{Zuarbeit von Matthias Berger, Nils Buxel}
\href[pdfnewwindow=true]{file:zuarbeiten/CodingGuidelines.pdf}{Coding Guidelines}

\subsubsection{Zuarbeit von Nils Buxel}
\href[pdfnewwindow=true]{file:zuarbeiten/BestPractise_CodingConventions_CSS.pdf}{Coding Conventions CSS}

\subsubsection{Zuarbeit von Julius Hohlfeld}
\href[pdfnewwindow=true]{file:zuarbeiten/AngularRecherche.pdf}{Angular}

\subsubsection{Zuarbeit von Lennart Buchmann, Alaa Aldin Karkoutli, Jonas Gwozdz, Linus Herterich}
\href[pdfnewwindow=true]{file:zuarbeiten/TechnologierechercheOpenSource.pdf}{Bibliotheken zur Graphenerstellung}

\subsection{Entscheidungen des Technologieworkshops}
{\small Autor: Erik Heldt}

Nach ausgiebigen Recherchen über verschiedenste Programmiersprachen, Frameworks und Bibliotheken entschieden wir uns für eine Webanwendung auf Basis von HTML/CSS/JavaScript.

Wir haben uns weiterhin auf das JS-Framework Vue.js geeinigt, da es viele Vorteile für die Front-End-Entwicklung mit sich bringt und von den vielen untersuchten Frameworks am intuitivsten erschien. Außerdem haben wir nach einer JS-Bibliothek zur Graphdarstellung recherchiert und unter verschiedenen Kandidaten stach Cytoscape mit seinen vielen Funktionen zur Graphenerstellung und -editierung am meisten heraus, was wir somit auch in unsere Architektur integrierten.

Bei der Programmierumgebung waren wir uns schnell einig, dass Visual Studio Code am besten für unsere Ansprüche geeignet ist. Wir installierten die IDE zusammen mit dem Plugin ESLint zur Unterstützung der Einhaltung standardmäßiger Coding Conventions.

\subsection{Überblick über Architektur}
{\small Autor: Linus Herterich}

VarG ist eine Web-App nach dem Client-Server Modell, wobei der Großteil der Berechnungen per JavaScript auf dem clientseitigen Browser
durchgeführt werden.
\\Serverseitig wird eine Datenbank (inkl. API-Schnittstelle) zum persistenten Speichern der erstellten Graphen angeboten.
\\
\\Die Architektur der Web-App basiert auf dem JavaScript-Webframework ''Vue.js'',
mit dem Webanwendungen nach dem MVVM Muster (Model View ViewModel) realisiert werden können. Die gesamte App ist nach logischen
Sites (Seiten, bei denen sich die URL ändert) und Components (wiederverwendbare, abgeschlossene Software-Schnipsel) aufgebaut.
Jede Vue Component (.vue Dateien) enthält ein HTML-Template (GUI), sowie Daten, mit denen das Template befüllt wird. Zudem werden
Funktionen definiert, die entweder zu bestimmten Laufzeitbedingungen der App oder durch Events und Trigger aufgerufen werden.
Die Kommunikation zwischen Components wird über Vererbungen zu Eltern-/ Kind-Components realisiert.
\\Die Web-App besteht im Entwicklungszustand aus vielen hunderten Dateien, welche vom Framework verwaltet werden. Sobald
die App in den Produktionsstatus wechselt, muss das Projekt kompiliert werden. Dies übernimmt ebenfalls das Framework, welches
hierfür Technologien wie ''WebPack'' einsetzt. So bleiben lediglich wenige HTML, CSS und JavaScript Dateien übrig, die anschließend
auf einem Web-Server (z.B. Apache) zur Verfügung gestellt werden müssen.
\\
\\Um die Darstellung einheitlich zu halten, haben wir die UI-Bibliothek 'Vuetify'' genutzt. Diese hält sich an den Industriestandard
''Material Design'' von Google. Damit konnten wir alle unsere im Vorfeld erstellten Design-Konzepte umsetzen. Um an den ''Vuetify'' Elementen
weitere optische Anpassungen vorzunehmen haben wir die CSS-Language-Extension ''less'' verwendet. Mit dieser ist es möglich, übersichtliche und
einheitliche Style-Vorgaben auf die Design-Komponenten anzuwenden.
\\
\\Damit alle Daten komponentenübergreifend auf einen gemeinsamen Datenstamm zugreifen können und die Daten auch nach einer Session persistent
gespeichert werden können, haben wir die Vue.js-Erweiterung ''Vuex'' eingesetzt. Diese bietet eine zentralisierte Speichermöglichkeit
für alle Daten, die übergreifend verwendet werden müssen (beispielsweise Log-In Daten oder der Zustand des VarGraphs).
\\
\\Für die Darstellung des Graphen (Knoten + Kanten und deren Beschriftung) haben wir die JavaScript Bibliothek ''Cytoscape.js'' verwendet.
Die Bibliothek hält alle Graph-Daten in einem JavaScript Objekt, auf das mit verschiedenen API-Funktionen zugegriffen werden kann.
Die Darstellung des Graphen wird über ein Canvas HTML Element realisiert, in welches Cytoscape die angelegten Knoten und Kanten zeichnet.
Cytoscape bietet ebenfalls eine Hand voll Algorithmen zur analytischen Auswertung des Graphen. Da die Optimierung des VarGraphs
allerdings zusätzlichen Bedingungen und Parametern unterliegt, wurde ein eigener VarGraph-Optimierungsalgorithmus entwickelt.
\\
\\Bei der Wahl der serverseitigen Architektur haben wir eine REST-konforme (Representational State Transfer)
Architektur eingesetzt, an dessen Ende eine MySQL Datenbank zur Speicherung der Cytoscape Objekte, sowie Authentifizierungsdaten
steht. Auf die Daten der Datenbank greift eine API-Schnittstelle zu, welche mit Node.js umgesetzt ist (weitere Details zur Schnittstelle:
siehe II.4 - Schnittstellen). Anfragen an die API werden mit dem ''axios'' Framework per ''Promise-based'' HTTP-Requests gestellt. Die HTTP-Requests
folgen einem klaren Schema, welches vom serverseitigen Node.js interpretiert und an die Datenbank weitergeleitet wird.
\\
\\Um die Web-App großflächig zu testen haben wir uns zum einen für das Framework ''Cypress'' entschieden, welches Integrationstests anhand der
HTML-Elemente übernimmt. Cypress wertet aus, ob bestimmte Elemente unter bestimmten Bedingungen vorhanden sind beziehungsweise spezielle
Eigenschaften aufweisen. Die Cypress Tests haben wir auch erfolgreich an die ''CI / CD Pipeline'' von GitLab angeschlossen, sodass nach jedem
push die Tests durchlaufen (Stichwort: Regressionstest).
\\Zum anderen haben wir das Framework ''jest'' für Unit-Tests eingesetzt, mit dem einzelne Funktionen auf ihre Richtigkeit überprüft werden können. Vor allem
für die Optimierungsalgorithmen sind isolierte Tests nötig gewesen.
\\
\\Um eine Client-Server Architektur zu simulieren haben wir ''Docker'' eingesetzt. Dieses Tool erlaubt es, virtuelle Maschinen zu erstellen,
welche untereinander kommunizieren können. Für Entwicklungszwecke haben wir einen Docker-Container für eine MySQL Datenbank und
einen Node.js-Webserver (API Schnittstelle) erzeugt. Ein weiterer Docker-Container wurde eingesetzt, auf dem ''Adminer'' läuft. Mit diesem
Tool ist es möglich, die MySQL-Datenbank komfortabel anzuzeigen und SQL-Zugriffe auszuführen.


\subsection{Definierte Schnittstellen}
{\small Autor: Julius Hohlfeld}

VarGs Funktionalitäten erfordern eine Datenbank, um die erstellten Graphen speichern und wieder abrufen zu können.\\
Um den Zugriff auf die Datenbank zu kontrollieren benötigen wir eine definierte Schnittstelle (bzw. API) zwischen Client,
Webserver und Datenbank.\\
\\
Diese Schnittstelle ist RESTful - d.h. sie folgt einigen der sog. REST-Constraints. Eine Übersicht zu REST und dessen Bedeutung für das Projekt findet sich im GitLab Wiki unter ''API Dokumentation''.\\
Die Schnittstelle setzt sich wie folgt zusammen:
\begin{itemize}
  \item \textbf{Vue}
        \\\textit{Framework für Client + Axios-Module für asnychrone (promise-based) HTTP-Requests}
  \item  \textbf{Express}
        \\\textit{Serverseitiges Node-Module für Webserver: hört angemeldete Ports auf Requests ab, die dem URI-Modell entsprechen}
  \item \textbf{Node.js}
        \\\textit{Serverseitige Programmierung des Webservers mit mysqljs als Driver, um auf die Datenbank zuzugreifen}
  \item \textbf{DB}
        \\\textit{MySQL-Datenbank auf extra Server}
\end{itemize}

Diese Struktur (kurz VenDB) entspricht einer Anpassung des sog. MEAN-Stacks auf das VarG-Projekt (MongoDB, Express, Angular, Node.js).\\
Dabei erfolgt jeglicher Austausch der Graphdaten im JSON-Format, damit auf die Cytoscape-Funktion zum Laden des Graphen zugegriffen werden kann.\\
\\
\subsubsection{Client}

Der Client enthält Trigger durch Events, welche Requests an den Webserver senden. Z.B.: das Aufrufen des Datenbankfenster löst eine Anfrage aus, welche alle Graphen des aktuellen Nutzers abfragt.
Diese werden durch das Axios-Modul umgesetzt. Nachdem der Trigger ausgelöst wird, schickt der Client eine asynchrone Request. Diese wird vom Webserver verarbeitet, welcher dann eine Antwort schickt. Diese kann von Axios aufgefangen werden (axios."request"(url, {}).then(response => {}).catch(error => {})).

\subsubsection{Server}

Der durch Express und mysqljs programmierte Webserver definiert folgende mögliche Zugriffstellen auf die Datenbank:
\begin{itemize}
  \item \textbf{Get-Requests}
        \begin{itemize}
          \item \textbf{graph}
                \\\textit{Fragt alle Graphen aus der Datenbank ab - für Admin reserviert.}
          \item \textbf{graph/:id?}
                \\\textit{Fragt einen spezifischen Graphen (entsprechend der ID) ab.}
          \item \textbf{graph/meta}
                \\\textit{Fragt Metadaten z.B.: Namen, Id, Stückzahl usw. ab für die Graphen des Nutzers ab.}
        \end{itemize}
  \item  \textbf{Put-Requests}
        \begin{itemize}
          \item \textbf{graph/:id?}
                \\\textit{Client schickt Server eine Repräsentation des Graphen in Json um einen bereits existierenden Graphen (entsprechend der ID) zu überschreiben.}
        \end{itemize}
  \item \textbf{Post-Requests}
        \begin{itemize}
          \item \textbf{graph?}
                \\\textit{Client schickt Server eine Repräsentation des Graphen in Json um einen neuen Eintrag für den Nutzer zu erzeugen.}
        \end{itemize}
  \item \textbf{Delete-Request}
        \begin{itemize}
          \item \textbf{graph/:id?}
                \\\textit{Spezifizierter Graph (entprechend der ID) wird aus der Datenbank gelöscht.}
        \end{itemize}
\end{itemize}

Das '?' bedeutet, dass hier auf bestimmte URL Queries geachetet werden kann.
Das ist nützlich um z.B.: einen Nutzer nur auf seine eigenen Graphen zugreifen zu lassen.
Diese werden dann in die entsprechenden Queries umgewandelt.\\

\subsection{Liste der Architekturentscheidungen}
{\small Autor: Alaa Aldin Karkoutli}\newline

\textbf{JavaScript} ist die grundlegende Programmiersprache, auf der diese App basiert ist. Des Weiteren entschieden wir uns für die folgenden Architekturen:
\begin{description}
    \item [I) Vue.js]: wurde als das JavaScript-Webframework der Web-App eingesetzt. 
    \item [II) Vuetify]: ist die UI-Bibliothek, für die uns entschieden haben.  
    \item [III) CSS-Language-Extenion]: wurde eingesetzt, um die optischen Dinge anzupassen.
    \item [IV) Vuex]: ist eine 'Vue.js-Erweiterung', die als Speicher der Daten benutzt wird.
    \item [V) Cytoscape.js]: ist die JavaScript-Bibliothek, für die uns entschieden haben, um die Graphen zu erstellen.
    \item [VI) Cypress]: ist ein Framework zum Testen der HTML-Elemente.
    \item [VII) jest]: ist das Framework zum Testen der Richtigkeit einzelner Funktionen.
    \item [VIII) REST-Konforme] (Representational State Transfer): ist die Architektur zur Kommunikation der DB und Authentifizierungsdaten mit der App.
    \item [IX) Node.js]: ist die ausgewählte Schnittstelle, um auf die Daten der DB zuzugreifen.
    \item [X) Axios 'Promise-based' HTTP-Request]: ist das eingesetzte Framework, um die Abfragen zu stellen. 
    \item [XI) Docker]: ist die Client-Server Architektur, für die uns entschieden haben.
    \item [XII) MySQL]: ist das DB-System, für das uns im Docker-Container entschieden haben.  
\end{description}

\section{Prozess- und Implementationsvorgaben}

\subsection{Definition of Done}
{\small Autor: Tim Henning}

Im Allgemeinen wurde in dem Projekt die Definition von 'doneness' nicht allzu umfangreich
festgelegt, da es für viele Teammitglieder eines der ersten Softwareprojekte war. So wurden als 'Definition of Done' folgende
Punkte für alle User-Stories aufgestellt:
\begin{itemize}
  \item >50\% Testabdeckung
  \item Technische Kommentare im Code
  \item Einhaltung der festgelegten Code-Konventionen
\end{itemize}
Das Team hatte zu jeder Zeit eine gute Vorstellung, was einen 'fertigen Entwurf' kennzeichnet und dies wurde so auch in den Reviews untereinander kommuniziert.
Das wiederum führte zu einem klaren Verständnis der wichtigsten Produktanforderungen, was die Qualität des Produktes erhöhte und das Zusammenarbeiten erleichterte.\\
Größtenteils wurde sich an die allgemeinen Akzeptanzkriterien gehalten und viele Backlog-Einträge als 'done' erklärt. Fast jede Komponente wurde
getestet und zu deren Methoden hilfreiche Codekommentare geschrieben.
Außerdem wurde im Team umfangreich kommuniziert und die Kriterien einer User-Story angepasst, wenn die Fertigstellung eines Arbeitspakets doch mal nicht gänzlich klar war.\\
So war es möglich, bereits nach der Hälfte der Projektzeit dem Kunden regelmäßig stabil laufende Production-Builds auf aktuellstem Stand der Anwendung bereitzustellen.

\newpage
\subsection{Coding Style}
{\small Autor: Jonas Gwozdz}

Beim Schreiben unseres Programmcodes haben wir uns an folgende Coding-Conventions gehalten:

\begin{itemize}
	\item Zeilenlänge: maximal 80 Zeichen\par
	\item Kommentare und Dokumentation\par

\begin{itemize}
	\item Kommentare auf Englisch\par
	\item Klassen und Methoden in kurzen, prägnanten Sätzen beschreiben\par
	\item Unnötige Kommentare vermeiden\par
	\item Kommentare aktuell halten\par
\end{itemize}

	\item Einrückung und Zeilenumbrüche\par

\begin{itemize}
	\item 2 Leerzeichen statt Tabulator\par
	\item '\{' hinter Methodendeklaration\par
	\item '\}' in neuer Zeile auf gleiche Einrückungsebene\par
	\item Optionale Zeilenumbrüche für Übersichtlichkeit\par
	\item Nur ein Import pro Zeile\par
\end{itemize}

	\item Leerzeichen\par

\begin{itemize}
	\item Vor und nach binären Operationen\par

\begin{itemize}
	\item Ausnahme nur im Fall von Verdeutlichung unterschiedlicher Prioritäten\par
\end{itemize}

	\item Keine Leerzeichen nach und vor Klammern\par
	\item Keine Leerzeichen vor Kommata und Semikolon\par
	\item Leerzeichen nach Kommata\par
	\item Keine Leerzeichen am Zeilenende\par
\end{itemize}

	\item Konsistentes Benennungsschema\par

\begin{itemize}
	\item Deskriptive Namen verwenden\par
	\item mixedCase für Variablen\par
	\item GROßSCHREIBUNG für Konstanten\par
	\item Keine Umlaute\par
	\item Reservierte Schlüsselwörter beachten\par
	\item Immer auf Englisch\par
	\item Bezeichner von Boolean-Werten beschreiben Zustand, der wahr oder falsch sein kann\par
	\item Hilfsvariablen möglichst gleich benennen\par
	\item Übergabe von Attributen an Konstruktoren\par

\begin{itemize}
	\item 'length' als Attribut, '\_length' als Argument\par
\end{itemize}

\end{itemize}

	\item Textcodierung UTF-8
\end{itemize}\par

\textbf{Best Practice}\par

\begin{itemize}
	\item Allgemeines\par

\begin{itemize}
	\item Kein 'language' Tag verwenden\par
	\item Wiederholungen vermeiden\par
	\item Dopplungen vermeiden\par
\end{itemize}

\end{itemize}

\begin{itemize}
	\item Variablen und Objekte\par

\begin{itemize}
	\item Keine globalen Variablen
	\item Lokale Variablen, auch Zahlenvariablen zu Beginn deklarieren und initialisieren\par
	\item Deklarationen mehrerer Variablen können zusammengefasst werden\par
	\item Datentyp wird über die Initialisierung zugewiesen\par
	\item Kapselung mittels Namespace\par
	\item Keine Deklaration mittels 'new $ \ldots $ ()'\par
\end{itemize}

\end{itemize}

\begin{itemize}
	\item Funktionen\par

\begin{itemize}
	\item Vergleiche mittels '==='
	\item Unter keinen Umständen 'eval()' benutzen\par
	\item Keine 'with' Statements\par
	\item Keine 'for ($ \ldots $  in $ \ldots $ )' Loops\par
	\item Jeder 'switch' hat einen 'default' Case\par
	\item Vorsicht bei Verwendung von 'typeof()'\par
	\item Nicht erhaltene Argumente gelten als 'undefined'
\end{itemize}\par

\end{itemize}


\subsection{Zu nutzende Werkzeuge}
{\small Autor: Linus Herterich}

Im Folgenden werden die Werkzeuge erwähnt, mit denen wir die Software
entwickelt haben. Zudem wird darauf eingegangen, über welche Kanäle kommuniziert
wurde.


\subsubsection{Voraussetzungen}
Das Versionsmanagement-Tool 'GitLab' sowie das Zeitmanagement-Tool 'YouTrack'
wurden zu Beginn des Projekts vorgeschrieben. Die Commits in 'GitLab' werden jeweils mit
der ID des zugehörigen YouTrack-Tickets am Anfang des Commit-Titels versehen.
\\Damit das gesamte Team einheitliche Versionen der verwendeten Bibliotheken benutzt,
wird der Paketmanager 'npm' verwendet. Mir diesem lassen sich Pakete (und deren Versionen) definieren,
welche für das Projekt benötigt werden.
\\ Damit am Projekt gearbeitet werden kann, muss sich somit jedes Teammitglied die LTS-
Version von Node.js (welches npm enthält) installieren.
\\ Sobald Node.js global installiert ist, kann im ''code'' Verzeichnis der Befehl
'npm install' ausgeführt werden, um die benötigten Bibliotheken zu installieren.

\subsubsection{Compiler}
Achtung: Das Kompilieren funktioniert erst, sobald die Bibliotheken mit dem Befehl
'npm install' (im /code Verzeichnis) installiert wurden.
\\
Um Änderungen des Projektes einzusehen, muss das Projekt kompiliert werden.
'Vue.js' bringt bereits einen Echtzeit-Compiler mit, welcher reagiert, sobald Änderungen
an Dateien im '/code' Verzeichnis vorgenommen wurden. Um diesen Compiler aufzurufen, muss der npm-Befehl \mbox{'npm run serve'} im '/code' Verzeichnis
aufgerufen werden.
\\Um das Projekt nicht während der Entwicklung zu kompilieren, sondern für die Produktion freizugeben,
muss der Befehl 'npm run build' im '/code' Verzeichnis aufgerufen werden. Es werden
anschließend die kompilierten Dateien im Verzeichnis 'code/dist' abgelegt.
Diese können dann auf einem Webserver (z.B. Apache HTTP Server) hochgeladen werden.

\subsubsection{Entwicklungsumgebung}
Für die Entwicklung der Software wird der freie Quelltext-Editor 'Visual Studio Code'
von Microsoft verwendet. Dieser ist plattformunabhängig und kann durch zahlreiche Erweiterungen
angepasst werden. Beispielsweise kann durch das Plugin 'Vetur' die Vue.js-eigene Syntax
vervollständigt und hervorgehoben werden.
\\Weitere Einstellungsvorgaben bezüglich der Entwicklungsumgebung wurden nicht getroffen.
Es muss allerdings darauf geachtet werden, dass die Coding-Conventions durch automatische
Formatierungen eingehalten werden.

\subsubsection{CI / CD Pipeline}
In der CI / CD Pipeline unseres Versionsmanagement-Tools, die nach jedem Git-Push
ausgeführt wird, werden folgende Operationen durchgeführt:
\begin{itemize}
  \item Test, ob das Projekt kompiliert (inklusive Syntaxprüfung durch ES-Lint)
  \item Cypress Tests durchführen (siehe 'Überblick über Architektur')
  \item LaTeX Doku kompilieren
\end{itemize}

Sollte einer der Punkte fehlschlagen, wird der Autor des Git-Push's per E-Mail
darüber informiert. Somit ist die Wahrscheinlichkeit, dass bestehende Features durch
neue Entwicklungen längerfristig ''zerstört'' werden, möglichst gering.

\subsubsection{Docker}
Um die Client-Server Architektur des Projektes lokal zu simulieren, wird die
Container-Virtualisierungssoftware 'Docker' verwendet.
Mit dieser haben wir einen Webserver simuliert, auf dem die Datenbank ausgeführt
und verwaltet wird (siehe 'Überblick über Architektur'). Die Container werden
im Projekt-Ordner 'docker' definiert.


\subsubsection{Kommunikationstools}
Zu Beginn des Projekts wurde sich auf das kostenlose Kommunikationstool 'Slack' geeinigt.
Mit diesem ist es möglich, in verschiedenen Kanälen Nachrichten, Dateien und Medien auszutauschen.
Auch private Konversationen sowie Kleingruppen-Chaträume sind in diesem Tool möglich.
Die Software kann sowohl als App installiert, als auch im Browser verwendet werden.
\\Da wir über die Weihnachtsferien einen Sprint durchgeführt haben, führten wir Mitte Dezember
das Tool 'Discord' ein, mit dem es möglich ist, sich in Echtzeit-Sprachchats zusammenzufinden.
Außerdem ist es möglich, seinen Desktop zu teilen, sodass sich dieses Tool bestens eignet,
um räumlich getrennt über Code-Passagen oder neue Features zu sprechen.
\\Die Kombination beider Tools hat problemlos funktioniert und uns auch
während des Lockdowns in der ''Corona-Krise'' geholfen. Da wir die Tools bereits frühzeitig
eingesetzt haben, war kaum eine Um- bzw. Eingewöhnungszeit zu Beginn der
präsenzfreien Zeit notwendig.


\newpage

%%%%%%%%%%%%
%% Abschnitt mit den Sprints beginnt hier
%%%%%%%%%%%%

\section{Sprint 1}


\subsection{Ziel des Sprints}
{\small Autor: Erik Heldt}

Der erste Sprint des VarG-Projekts lief vom 05.12.2019 bis zum 16.12.2019. Ziel war es, eine fundamentale Struktur und grundlegende Funktionalitäten für die Anwendung zu entwickeln, auf denen man später weiter aufbauen kann. Währenddessen konnte man allgemeine Erfahrungen mit dem Ablauf eines Sprints machen.

\subsection{User-Stories des Sprint-Backlogs}
{\small Autor: Erik Heldt}

\textbf{Grundstruktur}
Die Anwendung sollte zu Beginn ein grundlegendes Fundament aufweisen, damit sich alle Teammitglieder vorstellen können, wie am Ende das Programm aussehen soll. Dazu gehörte zu Beginn das Design der Startseite mit dem VarGraph im Zentrum und der Einbindung von Cytoscape in die Programmstruktur.

\textbf{Datenstruktur für Knoten}
Es sollte mit Hilfe von Cytoscape herausgefunden werden, wie man Knoten im Programmcode hinzufügen und speichern kann. Dafür sollte dann eine Datei im Programm angelegt werden.

\textbf{Knoten zu bestehender Datenstruktur hinzufügen}
Die Anwendung sollte eine einfache Funktionalität zum Erstellen neuer Knoten aka Produktionsschritte erhalten, um sich mit den Cytoscape-Funktionen näher vertraut zu machen. Hier war erstmal noch keine graphische Darstellung in der GUI notwendig, es reichte per Console logs zu testen.

\textbf{Darstellung eines Graphen in Weboberfläche}
In der Anwendung sollte zunächst ein statischer Graph mit Hilfe einer Cytoscape-Datenstruktur sichtbar dargestellt werden, damit man sehen konnte, wie so ein „CytoGraph“ überhaupt aussieht. User-Interaktion war hier noch nicht notwendig.

\textbf{Kanten anlegen}
Zusätzlich zu Knoten sollten auch Kanten zwischen bestehenden Knoten hinzugefügt werden können. Diese Kanten sollten mit verschiedenen Attributen in der Cytoscape-Datenstruktur gespeichert werden.

\textbf{Berechnung verschiedener Eigenschaften}
Anhand der mit den Kanten gespeicherten Attribute sollte eine Funktionalität entwickelt werden, welche die Gesamtkosten (Auswahl von Geld oder Zeit) aller unterschiedlichen Pfade berechnen und anzeigen sollte. Dies war der erste Schritt in Richtung Optimierung, d.h. später sollte diese Funktionalität automatisch den günstigsten Pfad herausfinden und anzeigen.

\subsection{Liste der durchgeführten Meetings}
{\small Autor: Erik Heldt}

\begin{itemize}
	\item Planning - 05.12.2019
	\item Weekly Scrum 1 - 09.12.2019
	\item Weekly Scrum 2 - 12.12.2019
	\item Review - 16.12.2019
	\item Retrospektive - 19.12.2019
\end{itemize}

\subsection{Ergebnisse des Planning-Meetings}
{\small Autor: Erik Heldt}

Im Planning-Meeting erklärten die Projektmanager zu Beginn noch einmal kurz, wie ein Sprint im Allgemeinen abläuft und haben auf die Bedeutsamkeit der Coding Guidelines hingewiesen. Anschließend wurden die ersten User-Stories vom Project Owner vorgestellt und von den Bachelorstudenten per Finger-System in ihrer Komplexität eingeschätzt. Weiterhin wurde festgelegt, dass die Bachelorstudenten während des Sprints die User-Stories selbst in Tasks aufteilen und diese dann bearbeiten sollen.

\subsection{Aufgewendete Arbeitszeit pro Person$+$Arbeitspaket}
{\small Autor: xxx}

\begin{longtable}{|p{4cm}|l|l|l|l|l|}
        \hline
	Arbeitspaket & Person & Start & Ende & h & Artefakt\\
        \hline
	Vue.js "Getting Started" Tutorial durcharbeiten (für alle) & Buchmann, Lennart & 07.12.19 & 07.12.19 & 3 & Tutorial abgeschlossen\\ \hline
	Beispielgraph erstellen & Buxel, Nils & 09.12.19 & 09.12.19 & 1 & index.js\\ \hline
	Kürzesten Weg mit A*-Algorithm berechnen u anzeigen lassen & Buxel, Nils &16.12.19 & 16.12.19 & 1 & index.js\\ \hline
	Funktionen zu Buttons hinzufügen & Gwozdz, Jonas & 14.12.19 & 16.12.19 & 4 & MenuControls.vue\\ \hline
	Task: Einbindung in Vue-Dateistruktur & Heldt, Erik & 15.12.19 & 15.12.19 & 3 & MenuControls.vue, BasicData.js\\ \hline
	Graphenanordnung & Heldt, Erik & 05.12.19 & 05.12.19 & 3 & Graphenanordnung.pdf\\ \hline
	Vue.js "Getting Started" Tutorial durcharbeiten (für alle) & Heldt, Erik & 11.12.19 & 11.12.19 & 2 & Tutorial abgeschlossen\\ \hline
	Funktionen zu Buttons hinzufügen & Henning, Tim & 10.12.19 & 10.12.19 & 2 & MenuControls.vue\\ \hline
	Vue.js "Getting Started" Tutorial durcharbeiten (für alle) & Henning, Tim & 06.12.19 & 06.12.19 & 3 & Tutorial abgeschlossen\\ \hline
	Einbindung von Cytoscape in Vue & Herterich, Linus & 10.12.19 & 10.12.19 & 4 & index.js\\ \hline
	Buttons für Knoten und Kantenerstellung & Herterich, Linus & 13.12.19 & 13.12.19 & 3 & CreateControls.vue\\ \hline
	Knoten zu Graph hinzufügen & Herterich, Linus & 16.12.19 & 16.12.19 & 2,5 & index.js, CreateControls.vue\\ \hline
	Grundstruktur aufbauen & Herterich, Linus & 05.12.19 & 07.12.19 & 9,5 & Vue-Dateistruktur, sämtliche Startkomponenten\\ \hline
	Task: Basic Datenstruktur & Hohlfeld, Julius & 15.12.19 & 15.12.19 & 8 & BasicData.js, MenuControls.vue\\ \hline
      \end{longtable}

\subsection{Konkrete Code-Qualität im Sprint}
{\small Autor: Erik Heldt}

Zu Beginn wurde viel experimentiert und hauptsächlich sollte der Code erstmal ein funktionierendes Programm erzeugen, weswegen weniger auf die Qualität geachtet wurde. Trotzdem wurde sich größtenteils an die Coding Conventions gehalten und bereits einige Kommentare verfasst.

\subsection{Konkrete Test-Überdeckung im Sprint}
{\small Autor: Erik Heldt}

Da der erste Sprint größtenteils nur zur Erstellung einer grundlegenden Datenstruktur und zur Einarbeitung in JavaScript und den genutzten Frameworks bzw. Bibliotheken gedient hat, gab es noch keine Tests.

\subsection{Ergebnisse des Reviews}
{\small Autor: Erik Heldt}

Im ersten Review-Meeting stellten die Bachelorstudenten ihre Ergebnisse aus dem Sprint vor und die Manager gaben ihr Feedback dazu. Da sich die meisten Teammitglieder noch nicht richtig in Vue.js und Cytoscape einarbeiten konnten und teilweise große Schwierigkeiten mit den Frameworks hatten, gab es noch viele offene Aufgaben und nicht jeder hatte etwas vorzuzeigen.
Als erstes stellten Julius H. und Erik die Datenstruktur für die Knoten vor. Weiterhin zeigte Julius, wie ein Knoten in der Anwendung dargestellt wird und dass dieser durch ungeschickte Verschiebung und Skalierung aus der GUI verschwinden kann. Deshalb kamen Vorschläge, zukünftig den Zoom zu limitieren und das grundsätzliche Graph-Layout nochmal zu überarbeiten.
Um allen den Einstieg in die neuen Programmiersprachen und Bibliotheken etwas zu vereinfachen, stellte daraufhin Linus die Grundstruktur vor und erklärte noch einmal genau die einzelnen Elemente in der Dateistruktur. Weiterhin zeigte er, wie man ESLint-Fehler bei der Konsolenausgabe verhindern kann.
Danach wurde zwischen den Managern und den Bachelorstudenten noch die zukünftige Berechnung der kürzesten Wege und die unbearbeiteten User-Stories besprochen und dass diese in den nächsten Sprint mit einfließen werden.
Zum Schluss wurden noch ein paar allgemeine Fragen zum Testen und zu Git geklärt.

\subsection{Ergebnisse der Retrospektive}
{\small Autor: Erik Heldt}

In der Retrospektive konnte jedes Teammitglied vor an die Tafel gehen und verschiedene Aspekte des Sprints mit einem Strich in einer Tabelle bewerten.
Die Bewertung ging ausgeglichen aus. Die Gruppenleistung und das Gesamtergebnis waren gut, aber die Einzelleistungen der meisten Teammitglieder nicht. Viele Aufgaben blieben offen und wurden nicht erledigt, wozu in der Diskussion verschiedene Gründe angeführt wurden. Einerseits war es für die meisten schwer, sich selbst in die neue Programmierumgebung samt den Frameworks und Bibliotheken einzuarbeiten. Andererseits wussten viele nicht, was und wie viel sie machen sollten, was auf die nicht festgelegte Aufgabenzuteilung im Planning und die schlechte Kommunikation im Team während des Sprints zurückgeführt wurde. Letzteres Problem plante man damit zu lösen, in zukünftigen Plannings immer direkt Verantwortliche für bestimmte User-Stories festzulegen und entsprechende Tickets sofort im Anschluss zu erstellen und zuzuweisen.
Beim Thema der Daily Meetings ist man zu dem Schluss gekommen, dass diese wenn möglich immer persönlich bleiben sollten und nur in Ausnahmefällen online z.B. über Discord stattfinden sollten. Weiterhin wurde diskutiert, ob die Zeitspanne zwischen Donnerstag und Montag evtl. zu kurz ist, um schon weitreichende Ergebnisse zu erzielen, da am Wochenende einige Teammitglieder nicht programmieren können. Deshalb sollten die ersten Meetings beim nächsten Sprint stattdessen Montag und Donnerstag stattfinden.
Ein weiterer Themenpunkt war die Organisation im Git. Es wurde festgelegt, dass der Master-Branch während des Sprints unberührt bleiben sollte, da dieser immer lauffähig sein muss. Stattdessen sollte sich jeder seinen eigenen Branch erstellen und diesen nach Abschluss der eigenen Aufgaben auf den neuen Developer-Branch namens "targetbranch" mergen. Am Ende jedes Sprints würde dann der Developer-Branch mit dem Master-Branch gemerged werden.

\subsection{Abschließende Einschätzung des Product-Owners}
{\small Autor: xxx}

XXX

\subsection{Abschließende Einschätzung des Software-Architekten}
{\small Autor: xxx}

XXX

\subsection{Abschließende Einschätzung des Team-Managers}
{\small Autor: xxx}

XXX


\newpage

\section{Sprint 2}


\subsection{Ziel des Sprints}
{\small Autor: Linus Herterich}

Nachdem im ersten Sprint hauptsächlich die Grundstruktur sowie erste Datenstrukturen entworfen wurden,
war es nun wichtig, dass sich das gesamte Team im Sprint 2 mit der Projektstruktur (besonders mit dem Framework Vue)
auseinandersetzt und erste UserStories direkt am Code umsetzt. Zudem blieben einige Tickets noch vom letzten Sprint offen,
welche nun auch bearbeitet werden sollten.

\subsection{User-Stories des Sprint-Backlogs}
{\small Autor: Linus Herterich}

\begin{itemize}
  \item \textbf{Designumsetzung nach Adobe Preview}
        \\\textit{
          Als Benutzer der WebApplikation möchte ich ein ansehnliche und intuitive
          Oberflächengesstaltung haben, damit ich die Applikation gerne verwende.}
  \item \textbf{Authentifizierung eines Nutzers}
        \\\textit{
          Als Nutzer möchte ich mich in die Web Applikation einloggen können,
          damit nicht jeder meine erzeugten Graphen einsehen kann.}
  \item \textbf{Logische verknüpfung zwischen Knoten erstellen}
        \\ (wurde in Sprint 1 nicht abgeschlossen)
        \\\textit{
          Ein Nutzer muss eine Abfolge der Knoten definieren können,
          damit ersichtlich wird welcher Produktionsschritt auf den nächsten folgt}
  \item \textbf{Berechnung der Eingenschaften des Gesamtgraphs}
        \\ (wurde in Sprint 1 nicht abgeschlossen)
        \\\textit{
          Ein Nutzer der Webanwendung VarG muss die berechneten gesamt Eigenschaften
          jedes Zusammenhängendes Pfades ausgeben lassen können um eine Auswahl
          eines Pfades zu treffen.}
  \item \textbf{Datenstruktur Ausarbeiten \& Knoten zu einer vorhandenen Datenstruktur hinzufügen}
        \\ (wurde in Sprint 1 nicht abgeschlossen)
        \\\textit{
          Als Nutzer möchte ich Knoten zu der Datenstruktur hinzufügen können
          um die möglichen Produktionsschritte des Werkstücks überblicken zu können}

\end{itemize}

\subsection{Liste der durchgeführten Meetings}
{\small Autor: Linus Herterich}

\begin{itemize}
  \item 19.12.2019: Planning Meeting
  \item 23.12.2019: Daily Meeting (in Discord)
  \item 28.12.2019: Daily Meeting (in Discord)
  \item 05.01.2020: Review Meeting
  \item 06.01.2020: Retrospektive
\end{itemize}

\subsection{Ergebnisse des Planning-Meetings}
{\small Autor: Linus Herterich}

Neben der Aufgabenverteilung wurde im Planning darüber gesprochen, dass die Arbeitsaufteilung im letzten
Sprint nicht gut geklappt hat. Es wurde anschließend beschlossen im nächsten Sprint die User-Stories direkt
an Studenten zuzuweisen, damit jeder einen Teilbereich hat, den er bearbeiten muss.
\\ Desweiteren wurde eine Änderung im Git angekündigt. In Zukunft müsse der "Master"\--Branch während eines Sprints
immer gleich bleiben und Funktionalitäten werden auf einen "Developer"\--Branch gemerged. Am Ende des Sprints
wird dann der "Developer"\--Branch auf den "Master"\--Branch gemerged. wichtig ist, dass der "Master"\--Branch zu jedem
Zeitpunkt lauffähig ist.
\\ Für den folgenden Sprint wurde beschlossen, die Daily Meetings online (auf einem Discord Server) abzuhalten,
da viele Studenten über die Weihnachtsferien in der Heimat sind und somit ein persönliches wöchentliches treffen
nicht möglich wäre.

\subsection{Aufgewendete Arbeitszeit pro Person$+$Arbeitspaket}
{\small Autor: Linus Herterich}

\begin{longtable}{|p{4cm}|p{2cm}|p{1.2cm}|p{1.2cm}|p{0.7cm}|p{3.8cm}|}
  \hline
  Arbeitspaket                                                          & Person                & Start    & Ende     & h     & Artefakt                                                    \\
  \hline
  UI: Login                                                             & Berger, Matthias      & 22.12.19 & 22.12.19 & 3,5   & Login Funktionalität \& Design                              \\ \hline
  UI: Login                                                             & Buchmann, Lennart     & 22.12.19 & 22.12.19 & 6     & Login Funktionalität \& Design                              \\ \hline
  UI: Grapheneditor                                                     & Gwozdz, Jonas         & 23.12.19 & 04.01.20 & 9     & GraphHeader.vue, Toolbar.vue                                \\ \hline
  Task: Einbindung in Vue-Dateistruktur                                 & Heldt, Erik           & 19.12.19 & 19.12.19 & 0,25  & BasicData.js                                                \\ \hline
  Abrufbaren Knoten in Graph einfügen                                   & Heldt, Erik           & 23.12.19 & 26.12.19 & 3,5   & BasicData.js, TestDatabase.js                               \\ \hline
  Testdatenbank mit Speichern und Laden                                 & Heldt, Erik           & 27.12.19 & 27.12.19 & 3,5   & TestDatabase.js                                             \\ \hline
  Highlighting eines kürzesten Pfades nach Anwendung des A* Algorithmus & Henning, Tim          & 24.12.19 & 03.01.20 & 9     & OptimizeControls.vue, index.js -> Graph Highlighting        \\ \hline
  Protokoll: Meeting 19.12.19                                           & Herterich, Linus      & 19.12.19 & 19.12.19 & 1     & meeting\_19\_12\_19.pdf                                     \\ \hline
  UI: Login                                                             & Herterich, Linus      & 20.12.19 & 20.12.19 & 5     & LoginForm.vue, Login.vue                                    \\ \hline
  UI: Home                                                              & Herterich, Linus      & 23.12.19 & 23.12.19 & 7     & HomeMenu.vue (component), Home.vue (view), Menu.vue (view)  \\ \hline
  UI: Neuer Graph                                                       & Herterich, Linus      & 28.12.19 & 28.12.19 & 1,5   & NewGraph.vue (view), NewGraph.vue (component)               \\ \hline
  UI: Grapheneditor                                                     & Herterich, Linus      & 02.01.20 & 04.01.20 & 11,75 & Graph.vue (view), zahlreiche components                     \\ \hline
  Graph zu Datenstruktur hinzufügen                                     & Hohlfeld, Julius      & 21.12.19 & 23.12.19 & 4     & BasicData.js, TestDatabase.js                               \\ \hline
  Testdatenbank mit Speichern und Laden                                 & Hohlfeld, Julius      & 27.12.19 & 03.01.20 & 8     & BasicData.js, TestDatabase.js, index.js, JSonPersistence.js \\ \hline
  Mergen und Anpassen                                                   & Hohlfeld, Julius      & 04.01.20 & 04.01.20 & 2     & Bugs entfernt \& Mergekonflikte behoben                     \\ \hline
  UI: Datenbank-Import Fenster                                          & Karkoutli, Alaa Aldin & 31.01.20 & 04.01.20 & 12,5  & Database.vue (view), DatabaseForm.vue (component)           \\ \hline
  Kanten zu Graph hinzufügen                                            & Koch, David           & 23.12.20 & 04.01.20 & 10    & Änderungen an index.js, CreateControls.vue (component)      \\ \hline
\end{longtable}

\subsection{Konkrete Code-Qualität im Sprint}
{\small Autor: Linus Herterich}

Es wurde sich größtenteils an die Coding-Guidelines gehalten. An wichtigen Stellen sowie vor jeder Funktion wurden Kommentare
geschrieben. Die Trennung zwischen Views und Components sowie die Auslagerung der Style-Dateien wurde ebenfalls eingehalten.

\subsection{Konkrete Test-Überdeckung im Sprint}
{\small Autor: Linus Herterich}

Ein Student wurde beauftragt bis zum Ende des Sprints ein geeignetes Test-Framework zu finden.
Somit wurden während des Sprints noch keine Tests geschrieben.

\subsection{Ergebnisse des Reviews}
{\small Autor: Linus Herterich}

Es wurden fast alle UserStories umgesetzt. Somit war der zweite Sprint erfolgreich.
Alle Studenten konnten sich in das Projekt einarbeiten und haben die Strukturierung
größtenteils verstanden und eingehalten.
\\ Das User-Interface wurde nach der Designvorlage umgesetzt und die ersten Graphen-Funktionen
(Hinzufügen von Knoten und Kanten \& Optimieren) funktionieren bereits.
\\ Da noch nicht feststeht, wo die Software gehostet werden soll und wie die Datenbank-Funktionalität
umgesetzt werden soll, wurde zunächst eine lokale Speicherlösung als "Datenbank" verwendet. Somit konnten
die Speichern- und Laden-Funktionen erfolgreich implementiert werden.
\\ Die Login-Funktionalität ist derzeit nur sporadisch eingerichtet und wird finalisiert,
sobald feststeht, wie die Authentifizierung der Nutzer erfolgen soll (Anbindung an HTWK Login?).
\\ Leider ist immernoch kein geeignetes Testframework gefunden worden, mit dem sich sowohl Vue.js
als auch cytoscape (Graphen-Funktionalitäten) testen lassen.

\subsection{Ergebnisse der Retrospektive}
{\small Autor: Linus Herterich}

Das Happiness-Barometer für diesen Sprint ist sehr gut ausgefallen. Das liegt hauptsächlich an der guten Aufgabenverteilung
sowie an den großen Erfolgen, die diesen Sprint erzielt wurden.
\\ Kritisiert wurde die die Kommunikation gegen Ende des Sprints. Das finale Mergen aller Branches war zu hektisch und unsicher.
\\ Es wurde sich darauf geeinigt in Zukunft zwei Dailies pro Woche abzuhalten und das letzte Meeting eines Sprints zum gemeinsamen Mergen zu verwenden.

\subsection{Abschließende Einschätzung des Product-Owners}
{\small Autor: Manuel Eckert}

Aus den bei dem Planning-Meeting vorgestellten User-Stories ergaben sich drei Subteams. Diese teilten sich in die Bereiche Login, UI-Design und Graph-Funktionalitäten auf. Damit wurde das konkretere Aufteilen der User-Stories auf Subteams umgesetzt. \\
Dies hatte einen positiven Einfluss auf die Anzahl der erfolgreich abgeschlossen Aufgaben. \\
Während des Reviews wurden fehlende Code Kommentare und eine zu niedrige Testabdeckung benängelt.


\subsection{Abschließende Einschätzung des Software-Architekten}
{\small Autor: Julius Jolig}

In diesem Sprint wurden bereits mehr Kommentare im Code verfasst, aber hier ist noch Luft nach oben. Die Bachelorstudenten haben sich gut mit Vue.js und cytoscape vertraut gemacht und gute Ergebnisse erzielt. Das Mergen lief trotz neuem Ansatz immer chaotisch ab.  

\subsection{Abschließende Einschätzung des Team-Managers}
{\small Autor: Alex Hofmann}

Deutliche Leistungssteigerung schon jetzt zu sehen. Aufteilung der User-Stories direkt nach dem Planning hat die Arbeitsstruktur und -ablauf während des Sprints auf jeden Fall positiv beeinflusst.



\newpage

\section{Sprint 3}


\subsection{Ziel des Sprints}
{\small Autor: Lennart Buchmann}

Nach der Einarbeitung des gesamten Teams in die Grundstruktur der Software, sowie der Frameworks, lag das Hauptaugenmerk des 
dritten Sprints in der verstärkten Herausarbeitung der geplanten Kernfunktionalitäten der Anwendung. Größere Aufgabenbereiche wurden 
durch Zweier- und Dreierteams gelöst. Übriggebliebenes aus den vorherigen Sprints sollte beendet werden 

\subsection{User-Stories des Sprint-Backlogs}
{\small Autor: Lennart Buchmann}

\begin{itemize}

  \item \textbf{Funktionalität der Datenbank}
        \\\textit{Als Nutzer will ich meine gespeicherten Graphen ansehen können, um diese weiter bearbeiten zu können.}
\item \textbf{Kontext Menu über rechte Maustaste}
        \\\textit{ Als Nutzer möchte ich Knoten und Kanten über einen Rechtsklick zur einfacheren Benutzung erstellen können.}
  \item \textbf{Authentifizierung eines Nutzers}
        \\\textit{Als Nutzer möchte ich mich in die Web Applikation einloggen können,
        damit nicht jeder meine erzeugten Graphen einsehen kann.}
  \item \textbf{Darstellung von Knoten und Kanteneigenschaften am Objekt}
        \\\textit{Als Benutzer möchte ich über einen Rechtsklick auf einen Knoten/Kante die Eigenschaften dieser bearbeiten können.}
  \item \textbf{Optimierung des Graphs}
        \\\textit{Als Benutzer möchte ich gerne sofort sehen können, wie hoch meine Kosten für den kürzesten Pfad sind, damit ich mich möglichst schnell für einen entscheiden kann.}
\item \textbf{Speicherung Graph}
        \\\textit{Als Nutzer möchte ich einen Graphen jederzeit bearbeiten und speichern können, auch wenn dieser noch unfertig ist.}

\end{itemize}


\subsection{Liste der durchgeführten Meetings}
{\small Autor: Lennart Buchmann}

\begin{itemize}
  \item 06.01.2020: Planning Meeting
  \item 09.01.2020: Weekly Scrum
  \item 13.01.2020: Weekly Scrum
  \item 16.01.2020: Weekly Scrum
  \item 20.01.2020: Review \&  Retrospektive Meeting
\end{itemize}


\subsection{Ergebnisse des Planning-Meetings}
{\small Autor: Lennart Buchmann}

Der 3. Sprint ist der letzte Sprint im laufenden Semester und der letzte Sprint vor den anstehenden Prüfungen. Während des Planning-Meetings wurde von allen einheitlich besprochen, dass die
Arbeitslast von jedem höher ist als während der vergangen Sprints. Es wurde sich daraufhin geeinigt lieber realistische Ziele zu setzen, sodass der 3. Sprint auch mit höhere Belastung erfolgreich 
abgeschlossen werden kann. Nach Besprechung und Schätzung der Tickets, wurden alle Aufgaben in kleinere Gruppen aufgeteilt. Größere Aufgaben, die nach Schätzung im aktuellen Sprint nicht umsetzbar wären, wurden auf den verlängerten 4. Sprint verschoben. 


\subsection{Aufgewendete Arbeitszeit pro Person$+$Arbeitspaket}
{\small Autor: Lennart Buchmann}

\begin{longtable}{|p{4cm}|p{2cm}|p{1.2cm}|p{1.2cm}|p{0.7cm}|p{3.8cm}|}
  \hline
  Arbeitspaket                                                          & Person                & Start    & Ende     & h     & Artefakt                                                    \\ \hline
  Login                                                             & Berger, Matthias      & 13.01.20 & 17.01.20 & 18   & Login Funktionalität \& Design                              \\ \hline
  Login                                                             & Buchmann, Lennart     & 18.01.20 & 18.01.20 & 6     & Login Funktionalität \& Design                              \\ \hline
  Knotendarstellung nach Designvorlage        & Gwozdz, Jonas         & 15.01.20 & 20.01.20 & 4,5     & GraphHeader.vue, Toolbar.vue                                \\ \hline
  Speicherung Graph			        &  Heldt, Erik           & 06.01.20 & 19.01.20 & 18  & BasicData.js                                                \\ \hline
  Optimierung des Graphs 			        & Henning, Tim          & 09.01.20 & 18.01.20 & 9     & OptimizeControls.vue, index.js -> Graph Highlighting        \\ \hline
  Speicherung Graph                                      & Herterich, Linus      & 07.01.20 & 19.01.20 & 24,25     & meeting\_19\_12\_19.pdf                                     \\ \hline
  Graph zu Datenstruktur hinzufügen             & Hohlfeld, Julius      & 07.01.20 & 20.01.20 & 19     & BasicData.js, TestDatabase.js                               \\ \hline--
  Funktionalität Neuer Graph Button               & Karkoutli, Alaa Aldin & 12.01.20 & 15.01.20 & 7  & Database.vue (view), DatabaseForm.vue (component)           \\ \hline
  Kanten zu Graph hinzufügen                         & Koch, David           & 17.01.20 & 19.01.20 & 10    & Änderungen an index.js, CreateControls.vue (component)      \\ \hline
\end{longtable}

\subsection{Konkrete Code-Qualität im Sprint}
{\small Autor: Lennart Buchmann}



\subsection{Konkrete Test-Überdeckung im Sprint}
{\small Autor: Lennart Buchmann}

Eine konkrete Auseinandersetzung mit Tests beziehungsweise entsprechenden Test-Frameworks fand während des 2. Sprints statt. Momentan befinden sich alle Teammitglieder noch in der Einarbeitungsphase. Aufgrund des fortgeschrittenes Semesters und der anstehenden Prüfungen lagen die Prioritäten vorwiegend auf der Bearbeitung der User-Stories. 


\subsection{Ergebnisse des Reviews}
{\small Autor: Lennart Buchmann}

Das Ergebnis der Reviews war in anbetracht der fortgeschrittenen Semesters durchgehenden positiv. Alle Teammitglieder haben die Ihnen zugewiesenen Aufgaben innerhalb des Sprints erledigt. 
Es wurde des Weiteren besprochen, dass der verlängerte Sprint während der Semesterferien dazu genutzt werden sollte, um Bugs zu beheben und somit jedem die Gelegenheit zu geben, sich in die Testframeworks einzuarbeiten und Tests für den geschriebenen Code zu verfassen.


\subsection{Ergebnisse der Retrospektive}
{\small Autor:  Lennart Buchmann}

Während der Retrospektive wurde von allen die grundsätzliche gute Kommunikation innerhalb des Teams gelobt. Alle empfanden auch die Aufteilung in kleinere Zweier- und Dreierteams zur Bearbeitung von Aufgaben für sehr hilfreich.  Eine gleichbleibende hohe Motivation und Produktivität soll auch während des Semesterferiensprints beibehalten werden. Punkte, welche verbessert werden sollten, sind das pünktliche Mergen der einzelnen Branches vor Ende des Sprints, das Kommentieren des Codes und das Verfassen von Tests. 


\subsection{Abschließende Einschätzung des Product-Owners}
{\small Autor: xxx}

XXX

\subsection{Abschließende Einschätzung des Software-Architekten}
{\small Autor: xxx}

XXX

\subsection{Abschließende Einschätzung des Team-Managers}
{\small Autor: Alex Hofmann}

Weiterhin aufstrebende Arbeit vom Team. Auch die Kommunikation bei Problemen, Fragen und Anregungen geht in eine positive Richtung.



\newpage

\section{Sprint 4}

\subsection{Ziel des Sprints}
{\small Autor: Jonas Gwozdz}

Während der Semesterferien haben wir an Sprint 4 weitergearbeitet. Dieser dauerte vom 23.01.2020 bis zum  09.04.2020. Der Ablauf war dabei weitestgehend planmäßig, bis auf dass die Meetings zum Review und der Retrospektive wegen Corona ohne persönliches Treffen stattfinden mussten.
In der Vorlesungsfreien Zeit besprachen wir uns gelegentlich über den aktuellen Zwischenstand. Der größte Fortschritt am Projekt wurde während der letzten beiden Wochen erzielt.

\subsection{User-Stories des Sprint-Backlogs}
{\small Autor: Jonas Gwozdz}

\begin{itemize}
  \item \textbf{Tests für bereits geschriebenen Code}
        \\\textit{Als Benutzer möchte ich eine Software benutzen, die getestet ist, damit keine unerwarteten Probleme auftauchen.}
  \item \textbf{ Validierung der möglichen Eingaben }
        \\\textit{
          Als Nutzer möchte ich bei versehentlicher falscher Eingabe wenn möglich gewarnt werden, damit ich nichts falsches abspeichere.}
  \item \textbf{Bug: Validation bei gleichem Knoten-Namen}
  \item \textbf{Darstellung von Kanten/Attributen }
        \\\textit{
          Als Benutzer will ich alle Kanten/Knoten gleichzeitig sehen können(nicht übereinander), damit ich einen schnelleren Überblick über das gesamte Konstrukt bekomme.}
  \item \textbf{Bug: Mehrere Edges zwischen Knoten nicht möglich}
        \\\textit{
          Wenn man mehrere Kanten zwischen zwei Knoten anlegt, sind diese nicht sichtbar. Löscht man dann einen Knoten, an dem diese "unsichtbaren" knoten hängen, so stürzt cytoscape ab.}
  \item \textbf{Remodel von Component NewGraph}
\end{itemize}

\subsection{Liste der durchgeführten Meetings}
{\small Autor: Jonas Gwozdz}

\begin{itemize}
\item 23.01.2020: Planning
\item 05.03.2020: Weekly
\item 12.03.2020: Weekly
\item 06.04.2020: Review
\item 09.04.2020: Retro
\end{itemize}

\subsection{Ergebnisse des Planning-Meetings}
{\small Autor: Jonas Gwozdz}

Anwesend: Alex, Julius J., Julius H., Linus, Jonas, Erik, Lennart, Nils, Tim, David, Matthias, Manuel\\
\\
Innerhalb dieses Meetings haben wir die Schwerpunkte des Sprints festgelegt und über den Workload über die Vorlesungsfreie Zeit diskutiert und den Zeitaufwand der User-Stories abgeschätzt.\\


\textbf{oberste Priorität: Tests}\\
Da wird bis zum bisherigen Zeitpunkt keine Testumgebung gefunden haben, die sich auf unseren Cytoscape-Graphen anwenden lässt, und wir dadurch viel Nachholbedarf in Sachen Testen hatten, musste dieses Ticket am dringendsten abgearbeitet werden.\\

\textbf{Sprint über Semesterferien}\\
Wir haben uns im Planning darauf geeinigt, den Sprint über die Semesterferien mit weniger User-Stories als üblich auszulegen, da nicht alle Teammitglieder in dieser Zeit voll verfügbar waren, Grund dafür waren vor Allem die noch andauernden Prüfungen und die Anschließenden Ferien, die evtl. schon anderweitig verplant waren. Zudem haben wir uns darauf geeinigt, regelmäßig Absprache über den Fortschritt unserer Arbeit zu halten.\\

\textbf{Datenbanken}\\
Die Datenbankrecherche hat ergeben, dass für unsere Zwecke mySQL oder NodeJS am optimalsten wäre. Die Definition der Datenbankschnittstelle zwischen DB und Frontend muss ebenfalls noch erledigt werden. Zudem haben wir festgestellt, dass die Bisher entworfene Datenbankoberfläche optisch nicht zum Rest der Anwendung passt, und deshalb überarbeitet werden muss.\\

\textbf{Weitere Sprintziele:}
\begin{itemize}
\item Optimierung der Kostendarstellung
\item negative Zahleingaben abfangen
\item automatische Zoomfunktion bei Knoten- oder Kantenwahl
\item allgemeine Bugfixes
\end{itemize}


\subsection{Aufgewendete Arbeitszeit pro Person$+$Arbeitspaket}
{\small Autor: Jonas Gwozdz}

\begin{longtable}{|p{4cm}|p{2cm}|p{1.2cm}|p{1.2cm}|p{0.7cm}|p{3.8cm}|}
  \hline
  Arbeitspaket                                                          & Person                & Start    & Ende     & h     & Artefakt                                                    \\
  \hline
  Tests für bereits geschriebenen Code                                  & Heldt, Erik           & 04.03.20 & 04.03.20 & 2     & Tests für ModifyDataControls.vue                            \\ \hline
  Neue Strukturierung                                                   & Heldt, Erik           & 26.01.20 & 26.01.20 & 1     & Umstrukturierung des Projekts                               \\ \hline
  Header Buttons und Metadaten-Speicherung                              & Heldt, Erik           & 05.03.20 & 12.03.20 & 6,75  & GraphHeader.vue                                             \\ \hline
  Aufräumen der Branches im GitLab                                      & Heldt, Erik           & 29.03.20 & 29.03.20 & 1     & Organisatorische Aufgabe                               \\ \hline
  Entfernen veralteter Komponenten und Methoden                         & Heldt, Erik           & 31.03.20 & 31.03.20 & 2     & Organisatorische Aufgabe                                             \\ \hline
  Tests für Graphoptimierung                                            & Henning, Tim          & 04.04.20 & 40.40.20 & 12    & vargraph.spec.js        \\ \hline
  Tests für bereits geschriebenen Code                                  & Herterich, Linus      & 30.01.20 & 12.02.20 & 7,5   & /code/cypress/integration/...                                     \\ \hline
  Header Buttons und Metadaten-Speicherung                              & Herterich, Linus      & 28.03.20 & 31.03.20 & 2,25  & /vargraph/graph/... \& GraphHeader.vue                  \\ \hline
  Aufräumen der Branches im GitLab                                      & Herterich, Linus      & 30.03.20 & 30.03.20 & 1     & Organisatorische Aufgabe  \\ \hline
  Darstellung von Kanten/Attributen                                     & Herterich, Linus      & 03.04.20 & 03.04.20 & 2     & VarGraph.vue               \\ \hline
  Remodel von Component NewGraph                                        & Herterich, Linus      & 30.03.20 & 30.03.20 & 3     & /vargraph/graph/...                   \\ \hline
  Refactoring                                                           & Herterich, Linus      & 29.03.20 & 30.03.20 & 9     & /vargraph/graph/...                                 \\ \hline
  Validierung: Login                                                    & Herterich, Linus      & 31.03.20 & 30.03.20 & 1,5   & /components/login/LoginForm                                  \\ \hline
  Einheitliche Alerts                                                   & Herterich, Linus      & 31.03.20 & 31.03.20 & 3     & Dialogs.vue \\ \hline
  Validierung CreateControls \& DetailControls                          & Herterich, Linus      & 31.03.20 & 01.04.20 & 5,5   & CreateControls.vue \& DetailControls.vue               \\ \hline
  Bug: Mehrere Edges zwischen Knoten nicht möglich                      & Herterich, Linus      & 01.04.20 & 01.04.20 & 2     & /vargraph/graph/...                   \\ \hline
  Knoten dort erstellen, wo rechtsklick passiert                        & Herterich, Linus      & 01.04.20 & 01.04.20 & 1,5   & /vargraph/graph/...                               \\ \hline
  keybinds für Menüs                                                    & Herterich, Linus      & 02.04.20 & 02.04.20 & 1     &                                   \\ \hline
  Keine Knoten aufeinander schieben                                     & Herterich, Linus      & 02.04.20 & 02.04.20 & 3     & /vargraph/graph/...  \\ \hline
  Einstellungsmenü erstellen                                            & Herterich, Linus      & 03.40.20 & 05.04.20 & 5,5   &              \\ \hline
  Tests für bereits geschriebenen Code                                  & Hohlfeld, Julius      & 05.02.20 & 04.03.20 & 10    & ZoomControls.spec \& SaveMenu.spec \& NewGraphMenu.spec \& DownloadMenu.spec \\ \hline
  Dialogfenster für Speichern, Laden und Export                         & Hohlfeld, Julius      & 24.01.20 & 24.01.20 & 2     & Toolbar.vue \\ \hline
  Validierung der möglichen Eingaben                                    & Hohlfeld, Julius      & 06.04.20 & 06.04.20 & 2     & divers                             \\ \hline
  Refactoring                                                           & Hohlfeld, Julius      & 31.03.20 & 31.03.20 & 2     & /vargraph/graph/...                     \\ \hline
  Testing für Kanten hinzufügen                                         & Koch, David           & 22.03.20 & 02.04.20 & 5     & addEdges.spec      \\ \hline
\end{longtable}

\subsection{Konkrete Code-Qualität im Sprint}
{\small Autor: Jonas Gwozdz}

Die Codequatlität im allgemeinen wurde während des Sprints erheblich durch das Refactoring verbessert. Zudem wurden in nahezu  allen Dateien einleitende Kommentare geschrieben, um die zukünftige Identifizierung der gebrauchten Dateien schneller und übersichtlicher zu gestalten.

\subsection{Konkrete Test-Überdeckung im Sprint}
{\small Autor: Jonas Gwozdz}

Die geschriebenen Cypress-Tests decken bereits eine Vielzahl an Funktionalitäten des Programms ab. Dazu zählen die Buttons für die Database, den Download, das Ausloggen. Zudem wurde getestet: der Speicherdialog, die Zoomeinstellungen, der Header des Graphen, das Hinzufügen von Knoten und das Erstellen eines neuen Graphen.

\subsection{Ergebnisse des Reviews}
{\small Autor: Jonas Gwozdz}

Anwesend: David, Erik, Julius J., Julius H., Jonas, Linus, Manuel, Matthias, Tim\\

Im Rahmen des Reviews haben wir wie gewohnt die Ergebnisse des Sprint bewertet und Schwierigkeiten besprochen.\\

\textbf{generelle Schwierigkeit: Testen}\\
Um unsere Programm zu testen, entschieden wir uns für das Framework "Cypress" entschieden. dieses bietet End-to-End Testing an, welches allerdings nur Ausgaben des Programms auswerten kann, und deshalb sozusagen keinen Blick unter die Haube zulässt, und somit eventuell Fehler unentdeckt bleiben. \\

\textbf{David:}
\begin{itemize}
\item Tests für Knotenfunktionalität geschrieben
\item mit Kantentests begonnen
\end{itemize}

\textbf{Erik:}
\begin{itemize}
\item Data Controls durch Header Buttons ersetzt
\item Editierungsfenster entfernt
\item Header Buttons getestet
\end{itemize}

\textbf{Jonas:}
\begin{itemize}
\item Testübersicht erstellt
\item Möglichkeit zum Informationsaustausch über Lücken und Bugs in Tests bereitgestellt
\end{itemize}

\textbf{Julius H.:}
\begin{itemize}
\item Tests für Toolbar, Zoom-Controls, Buttons und Eingabereihenfolgen geschrieben
\end{itemize}

\textbf{Julius H, Erik, Linus:}
\begin{itemize}
\item Refactoring des Graphen, Bugfixing und Validierung von Eingaben
\end{itemize}

\textbf{Linus:}
\begin{itemize}
\item Dialogue-Popups erstellt
\item Kürzelgenerierung implementiert
\item Knotenüberlagerung unterbunden, Mindestabstand implementiert
\item Einstellungsmenü erstellt und Implementation begonnen
\item Recherche zu Datenbankfenster
\end{itemize}

\subsection{Ergebnisse der Retrospektive}
{\small Autor: Jonas Gwozdz}

Anwesend: Alex, Erik, Julius J., Julius H., Jonas, Linus, Matthias, Tim\\

Zu Beginn des Sprints gab es keine Fortschritte zu vermelden, da vorerst die Prüfungen zu überstehen waren. In den beiden Wochen vor Sprintende wurden allerdings die wichtigsten User-Stories und sogar etwas mehr abgearbeitet.\\

\begin{center}
\begin{tabular}{ |c|c| }
\hline
 Positiv & Negativ \\
\hline 
 -produktive Endphase & -anfangs keine Kommunikation \\
 -viel Motivation bei Einigen & - wenig Motivation bei Einigen\\
 & -vereinzelt Tests ohne Sinn\\
 & -ausgefallene Meetings\\
\hline     
\end{tabular}
\end{center}
 

\subsection{Abschließende Einschätzung des Product-Owners}
{\small Autor: xxx}

XXX

\subsection{Abschließende Einschätzung des Software-Architekten}
{\small Autor: xxx}

XXX

\subsection{Abschließende Einschätzung des Team-Managers}
{\small Autor: Alex Hofmann}

Aufgrund der vorlesungsfreien Zeit war mit erhöhter Inaktivität aufgrund von Prüfungen, Urlaub und sonstigen Auszeiten zu rechnen.
Das Team hat sich dennoch demokratisch für einen Sprint während dieser Zeit entschieden. Trotz aller Umstände wurde mit der Umsetzung der Testfälle die Zielvorgabe erreicht.



\newpage

\section{Sprint 5}

\subsection{Ziel des Sprints}
{\small Autor: Tim Henning}

Der vierte Sprint des VarG-Projektes lief vom 13.04.20 bis zum 23.04.20.
Ziel des Sprints war zum einem, dass die nicht vollendeten Aufgaben aus Sprint 4 nachgeholt werden, und das sich um die Schnittstelle zwischen Frontend und Backend gekümmert wird. Weiterhin wurde geäußert viel Recherche zum Thema Datenbanken, Shibboleth Anbindung und bereitstellen eines Servers des IT-Servicezentrums der HTWK, zu betreiben. Außerdem sollten zum vorhandenen Optimierungsalgorithmus noch einige Besserungen vorgenommen werden.

\subsection{User-Stories des Sprint-Backlogs}
{\small Autor: Tim Henning}

\textbf{Datenbank, Initiale Aufgaben zur Bereitstellung}
Als Nutzer möchte ich gerne auf eine, mit dem Rest der App, konsistente Oberfläche zugreifen können, damit ich mich einfacher zurecht finde. Zudem möchte ich gerne einen Überblick über die vorhandenen Elemente (Bearbeitungsmaschinen) anzeigen lassen und in meinen Graphen übernehmen können, damit ich die Eigenschaften dieses Elements nicht jedes mal neu heraussuchen muss.

\textbf{Entwurf der Schnittstelle zwischen Backend und Frontend}
Als Nutzer möchte ich Daten aus der Datenbank abrufen/anzeigen lassen können, damit der Graph schneller erstellt werden kann.

\textbf{Login}
Nach dem Login, in die Applikation sollen meine Anmeldedaten gespeichert werden, damit ich mich beim erneuten laden der Seite nicht neu einloggen muss.

\textbf{Optimierung}
Als Benutzer möchte ich optimale Wege des erstellten Graphen anzeigen lassen können, damit ich eine bessere Auswahl zwischen den einzelnen Bearbeitungsschritten treffen kann.


\subsection{Liste der durchgeführten Meetings}
{\small Autor: Tim Henning}

\begin{itemize}
	\item Planning - 13.04.2020
	\item Weekly Scrum 1 - 16.04.2020
	\item Weekly Scrum 2 - 20.04.2020
	\item Review - 23.04.2020
	\item Retrospektive - 23.04.2020
\end{itemize}

\subsection{Ergebnisse des Planning-Meetings}
{\small Autor: Tim Henning}

Anwesend: Jonas G., Erik H. Linus H., Lennart B., Tim H., David K., Matthias B., Alaa Aldin K., Manuel E., Julius J., Alex H.\\
\\

Neben der Aufgabenverteilung wurden noch einige zusätzliche Punkte besprochen, die nicht in den User Stories aufgetaucht sind. So zum Beispiel sollte nach jedem Sprint ein Production Build angelegt werden, der auf einem Server liegt, damit der Kunde regelmäßig das Produkt testen kann. Weiterhin wurde gefordert die neue Testumgebung Cypress in die Git-Pipeline einzubinden. Außerdem sollte am IT-Servicezentrum nachgefragt werden, ob es möglich ist eine Shibboleth Anbindung zu bekommen und ob die HTWK einen Server bereitstelle, auf dem der Production Build später gehostet werden kann.

\subsection{Aufgewendete Arbeitszeit pro Person$+$Arbeitspaket}
{\small Autor: Tim Henning}

\begin{longtable}{|p{4cm}|p{2cm}|p{1.2cm}|p{1.2cm}|p{0.7cm}|p{3.8cm}|}
        \hline
	Arbeitspaket & Person & Start & Ende & h & Artefakt\\
        \hline
	UI: Login & Beger, Matthias & 13.04.20 & 13.04.20 & 2,5 & Recherche, Konzeption\\ \hline
	UI:Login & Buchmann, Lennart & 23.04.20 & 23.04.20 & 5 & Recherche, Konzeption\\ \hline
UI: Datenbank; Initiale Aufgaben zur Bereitstellung & Gwozdz, Jonas  & 13.04.20 & 23.04.20 & 15 & Datenbankfenster Redesign, Responsiveness der Datenbankseite, Button Platzierungen \\ \hline
 Task: Sprint 4 Dokumentation & Gwozdz, Jonas  & 13.04.20  & 13.04.20 & 5 & Sprint4.tex \\ \hline
UI: Entwurf der SChnittstelle Backend <-> Frontend & Heldt, Erik  & 18.04.20 & 18.04.20  & 1,5 & SaveMenu.vue, TestDataBase.js \\ \hline
Task: Recherche Zusammenspiel Vue + Datenbank & Heldt, Erik  & 15.04.20  & 16.04.20  & 2 & Installation Axios, HTTP Requests \\ \hline
Task: Button UI/UX Änderungen und Validierung bei Erstellung von Kanten & Heldt, Erik  & 17.04.20 &23.04.20 & 7 & CreateControl.vue, DetailControls.vue\\ \hline
Task: Gesamkosten und /-zeit einschließlich der Produktanzahl & Henning, Tim  & 15.04.20  & 16.04.20  & 4 & optimization.js \\ \hline
Task: Alten Optimierungsalgorithmus umbauen & Henning,  Tim  & 17.04.20 & 23.04.20  & 11 & optimization.js\\ \hline
Task: Sprint 5 Dokumentation & Henning,Tim  & 23.04.20  & 23.04.20  & 3 & Sprint5.tex \\ \hline
UI: Datenbank; Initiale Aufgaben zur Bereitstellung &  Herterich, Linus  & 15.04.20  & 15.04.20 &  2,5 & Datenbankseite nun als Component \\ \hline
Sprint 2 Dokumentation  &  Herterich, Linus& 13.04.20 & 13.04.20 & 3,5 & Sprint2.tex \\ \hline
Task: Erstellung Production Build auf Server & Herterich, Linus  & 14.04.20  & 14.04.20 & 3 & läuft auf varg.nfl-server.de \\ \hline
Task: Cypress Test in die Gitlab Pipeline & Herterich, Linus  & 16.04.20 & 16.04.20 & 4,5 & .gitlab-ci.yml\\ \hline
Task: Kaputte Tests reparieren & Herterich, Linus  & 17.04.20  & 17.04.20  & 2 &  code/cypress/integration/.. \\ \hline
Task:Graph aus Hauptmenü importieren & Herterich, Linus  & 17.04.20  & 17.04.20  & 2 & Importieren aus Hauptmenü umgesetzt  \\ \hline
Task: Redesign Graphen Seite(Navigation Drawer) & Herterich, Linus  & 16.04.20  & 23.04.20 & 11 & 2 neue Designkonzepte \\ \hline
UI: Entwurf der Schnittstelle Backend <-> Frontend & Hohlfeld, Julius  & 13.04.20  & 22.04.20 & 6,5 & Dokumentation der API-Recherche und erste Entwürfe, API Dokumentation im Git Wiki \\ \hline
Task: Auswahl von Endzustand ohne Startzustand & Karkoutli, Alaa Aldin  &  17.04.20 & 22.04.20  & 14 & ausgewählte Startzustände aus Liste der Endzustände entfernt, OptimizeControls.vue  \\ \hline
Task: Auslagern der Optimize Controlls & Koch, David & 16.04.20   & 16.04.20 &  2 & OptimizeControls.vue \\ \hline
Task: Neuer Optimierungsalgorithmus & Koch, David & 17.04.20 & 23.04.20 & 10 & Beginn eines neuen Algorithmus \\ \hline

      \end{longtable}

\subsection{Konkrete Code-Qualität im Sprint}
{\small Autor: Tim Henning}

Die Codequalität hat sich zum vorherigen Sprint nicht entscheidend geändert. Durch den Umbau des Optimierungsalgorithmus hat man nun aber eine etwas höhere Speicherplatz- und Laufzeitkomplexität. Dies soll im nächsten Sprint angegangen und verbessert werden. Durch das Redesign ist die Website im allgemeinen ästhetischer geworden.


\subsection{Konkrete Test-Überdeckung im Sprint}
{\small Autor: Tim Henning}

Durch das Hinzufügen der Cypress Tests in die Pipeline des Git-Repository ist nun eine relativ gutes Feedback für den jeweiligen Entwickler und Tester vorhanden. Dieser bekommt nach durchführen der Pipeline eine E-mail, falls der Test fehlschlägt. Für den Optimierungsalgorithmus hingegen fehlen noch ein paar Tests.

\subsection{Ergebnisse des Reviews}
{\small Autor: Tim Henning}

Anwesend: Jonas G., Erik H. Linus H., Lennart B., Tim H., David K., Alaa Aldin K., Manuel E., Julius J., Alex H.\\


Im Review hat wie gehabt, jeder seine erledigten und angefangen Aufgaben vorgestellt und bewertet. So wurde bei der Optimierung die Stückzahl in den Algorithmus integriert, die Endzustände ohne Startzustände werden nun angezeigt und es wurde parallel an zwei neuen Algorithmen gearbeitet, die es ermöglichen die k-besten Pfade auszugeben, und nicht nur den optimalsten Pfad. Dabei wurde einer fertig gestellt, der die Pfade in der Konsole ausgeben kann. Dieser hat aber eine recht hohe Laufzeit- und Speicherplatzkomplexität. Daher wurde ein weitere Algorithmus angefangen, welcher im nächsten Sprint weiterentwickelt und angepasst wird. Zur Userstory der Datenbank und den Initalen Aufgaben zur Bereitstellung wurden erste HTTP Requests angefangen und ausprobiert sowie Axios installiert. Da aber die Datenbank nocht nicht konkret fest stand und noch kein Server von der HTWK zur Verfügung war, wurde sich primär um Bugfixing, Testing und Valiedierungen von Eingaben gekümmert. Die Buttons werden nun nach Windows Standard rechts unten angezeit und sind im Text-only Stil. Desweiteren wurde der Datenbankscreen angepasst und hat nun eine übersichtlichere Darstellung der Elemente, die später einmal aus der Datenbank geladen werden. Zur Userstory der Schnittstelle zwischen Frontend und Backend wurde viel Recherche betrieben. Dabei wurde ein Dokument erstellt, welches alle wichtigen und relevanten Informationen zum Thema API zusammen trägt. Dieses ist im Git- Wiki zu finden. Im Login Team wurde sich damit beschäftigt ein Rollenmanagement einzuführen und die Anbindung an das Shibboleth zu bekommen. Dies wird im nächsten Sprint weitergeführt. Ebenfalls wurde bei dem IT-Servicentrum der HTWK ein Server bestellt mit folgenden Spezifikationen:
\begin{itemize}
	\item 64 Bit, Debian
	\item 4GB Ram, 30GB Festplatte
	\item Anzahl der CPU's: 1
	\item Name der VM: Varg
	\item Netz: DMZ-VM-Fak
	\item Verwendungszweck: Softwareprojekt
	\item Verantwortlicher Prof.: Prof. Dr. Martin Gürtler
	\item Bemerkungen: Anfragen ob ITSZ Apache ausrollt\newline
\end{itemize}


Als letzter Punkt wurde im Sprint ein neues Design angefangen. Dort wurden auch schon die meisten Funktionen und Menüs implementiert und zum Ende des nächsten Sprints fertig gestellt. Das Projekt wird zum testen für den Kunden auf dem privaten Server eines Teammitgliedes gehostet.


\subsection{Ergebnisse der Retrospektive}
{\small Autor: Tim Henning}

Anwesend: Jonas G., Erik H. Linus H., Lennart B., Tim H., David K., Alaa Aldin K., Manuel E., Julius J., Alex H.\\
\\

Die Retrospektive fand in diesem Sprint online nach dem KALM Prinzip (Keep, Add, Less, More) statt und es wurden wie gewohnt Punkte die das Team ändern muss, aber auch welche die positiv waren und beibehalten werden sollen, angesprochen. So wurde die zahlreiche Teilnahme an den Meetings, sowie die Motivation in diesem Sprint als sehr positiv gewertet. Was im nächsten Sprint hinzu kommen sollte wäre u.a. eine weitere Person für das Team welches sich um das Zusammenspiel zwischen Frontend und Backend kümmert. Auch sollen die Testdokumentationen im Wiki ergänzt und ausgefüllt werden, um nach zu vollziehen welche Components bereits getestet wurden. Desweitern war ein wichtiger Punkt die zeitliche Absprache über das mergen der Branches und das aufräumen im Git Repository. Als Anmerkung unter dem Punkt "Less" , wurde zum einen das hinzufügen neuer Features genannt. Das Team will sich in den nächsten Sprints um Robustheit und Testing des vorhanden Codes kümmern und nicht all zu viele neue Features hinzufügen. Außerdem wurde noch angemerkt das die einzelnen Mitglieder YouTrack konsequenter nutzen sollen, um eine bessere Übersicht über den Workflow zu bekommen. Zum Schluss wurde noch erwähnt das der Sprint sehr positiv bewertet wurde, da viele Ziele erreicht wurden und viele neue Erkenntnisse zustande kamen, sowie das sich viele Teammitglieder an dem Sprint beteiligt haben.

\subsection{Abschließende Einschätzung des Product-Owners}
{\small Autor: Manuel Eckert}

Der Ablauf in den einzelnen Meetings läuft extrem reibungslos. Alle Teammitglieder fühlen sich mit dem Produkt identifiziert. Dies merkte man sehr in der Beteiligung und dem aufgewendeten Arbeitseinsatz während des Sprintes. Dies bedeutete auch eine hohe Anzahl an abgeschlossenen User-Stories. Durch das parallele Arbeiten an Front- und Backend wurde eine gute Produktivität erreicht. Die Schnittstelle zwischen Front- und Backend wurde ebenfalls konzipiert. \\
In diesem Sprint wurde das Produkt auch auf einen Webserver aufgespielt, dass der Kunde die Möglichkeit hat, sich länger mit dem Produkt auseinander zu setzten und damit auch ein besseres und detaillierteres Feedback geben kann. \\
Damit wurde dieser lange Sprint, der über die Prüfungszeit und Semesterferien ging, positiv abgeschlossen.

\subsection{Abschließende Einschätzung des Software-Architekten}
{\small Autor: Julius Jolig}

In diesem Sprint wurde eine CI Pipeline implementiert, wodurch fehlerhafte branches direkt nach dem pushen ins GitLab entdeckt werden können. Auch Test werden in der Pipeline ausgeführt. Allerdings fehlt noch ein Test für den Optimierungsalgorithmus. Das Mergen am Ende des Sprint lief im Vergleich zum letzten Sprint sehr gut ab.

\subsection{Abschließende Einschätzung des Team-Managers}
{\small Autor: Alex Hofmann}

Mit Beginn des neuen Semesters und der damit verbundenen Wiederaufnahme der (Online-) Präsenzveranstaltungen nahm auch die Teilnahme am Projekt wieder zu. Bis auf die beiden Aussteiger haben alle Teammitglieder mitgewirkt. Diese Motivation gilt es auch in den kommenden Wochen aufrecht zu erhalten.


\newpage

\section{Sprint 6}

\subsection{Ziel des Sprints}
{\small Autor: David Koch}

Vom 27.04. bis 07.05. arbeiteten wir an Sprint 6. Auf Grund von Corona fanden auch hier alle Treffen in Form von Online-Konferenzen statt. Die Zwischenstandsberichte bei den einzelnen Treffen wirkten zwar wenig erfolgversprechend, dennoch wurden fast alle Aufgaben bearbeitet und die Ergebnisse ins Projekt eingebunden.
Die Meetings verliefen problemfrei, die Aufgaben wurden beim Planning gut verteilt und die Zwischenstandsmeetings halfen dabei, kleinere Probleme schnell zu lösen.

\subsection{User-Stories des Sprint-Backlogs}
{\small Autor: David Koch}

\begin{itemize}
  \item \textbf{ Optimierung des Graphen }
        \\\textit{
          Der Optimierungsalgorithmus soll überarbeitet werden. Erstens sollen die Start- und Endzustände automatisch ausgewählt werden. Zweitens soll eine neue Größe an den Kanten -- die Losgröße, die die Wichtung der Rüstkosten beeinflusst -- eingebunden werden. Drittens soll der Algorithmus nicht nur den besten, sondern auch den zweit-, dritt-, usw.- besten Pfad ausgeben. Das Interface soll an die neuen Ein- und Ausgabemöglichkeiten angepasst werden.}
  \item \textbf{ Erstellen von Bearbeitungsschritten durch Klicken }
        \\\textit{
          Knoten und Kanten sollen durch wenige Klicks schnell erstellt werden können. Das Hinzufügen der Eigenschaften wird anschließend durchgeführt und die Validierungen werden dementsprechend von der Erstellung auf den Optimierungsalgorithmus verschoben.}
  \item \textbf{ Sammelticket für Feedback vom Kunden }
        \\\textit{
          Die 'Knoten' sollen als 'Teile' bezeichnet werden und die 'Kanten' als 'Bearbeitungsschritte'. Der Name der Bearbeitungsschritte soll besser erkennbar sein.}
  \item \textbf{ Design }
        \\\textit{
          Das Design soll auf die neuste Version angepasst werden.}
  \item \textbf{ Login }
        \\\textit{
          Alle Buttons auf der Login-Seite, die zum Testen verwendet wurden, sollen entfernt werden. Die Credentials sollen gespeichert werden und der Graph soll bei einem Neuladen der Seite nicht verloren gehen. Außerdem wird die Anbindung an Shibboleth weiter bearbeitet.}
  \item \textbf{ Backend Datenbank }
        \\\textit{
          Es soll eine Datenbank in Docker aufgesetzt werden, um erste Tests auf einer echten Datenbank durchführen und dafür eine Schnittstelle implementieren zu können. Sobald der angeforderte HTWK-Server bereitsteht, soll die Datenbank darauf aufgesetzt und parallel dazu eine Dokumentation geschrieben werden. Des Weiteren sollen Testdaten für die Datenbank erstellt werden.}

\end{itemize}

\subsection{Liste der durchgeführten Meetings}
{\small Autor: David Koch}

\begin{itemize}
\item 27.04.2020: Planning
\item 30.04.2020: Weekly
\item 04.05.2020: Weekly
\item 07.05.2020: Review, Retro
\end{itemize}

\subsection{Ergebnisse des Planning-Meetings}
{\small Autor: David Koch}

Anwesend: Alex, Julius J., Julius H., Linus, Jonas, Erik, Lennart, Nils, Tim, David, Matthias, Manuel\\
\\
Innerhalb dieses Meetings haben wir die Schwerpunkte des Sprints festgelegt, den Zeitaufwand der User-Stories abgeschätzt und die daraus entstehenden Aufgaben verteilt.\\


\textbf{Design}\\
Auf Grund der wandelnden Wünsche des Kunden wurde das Interface des Öfteren redesigned. Da sich das Projekt dem Ende nähert, soll in diesem Sprint die letzte Änderung am Interface vorgenommen werden.\\

\textbf{Warten auf Antwort der HTWK}\\
Die Datenbank auf einem HTWK-Server einzurichten sowie die Anbindung des Logins an Shibboleth sind abhängig von Antworten des ITSZ der HTWK und können daher eventuell noch nicht bearbeitet werden.\\

\textbf{Weitere Sprintziele:}
\begin{itemize}
\item Bugtickets bearbeiten
\item Losgröße als neue Kanteneigenschaft hinzufügen
\item neu hinzugefügten Code testen
\end{itemize}


\subsection{Aufgewendete Arbeitszeit pro Person$+$Arbeitspaket}
{\small Autor: David Koch}

\begin{longtable}{|p{4cm}|p{2cm}|p{1.2cm}|p{1.2cm}|p{0.7cm}|p{3.8cm}|}
  \hline
  Arbeitspaket                                                          & Person                & Start    & Ende     & h     & Artefakt \\
  \hline
  Login                                                                 & Berger, Matthias      & 28.04.20 & 28.04.20 & 4     & Grundlagenrechte \\
  \hline
  Login                                                                 & Berger, Matthias      & 07.05.20 & 07.05.20 & 5     & Konzeption und Umsetzung von Ladebildschirm, Timeout \& Weiterbildung \\
  \hline
  Login                                                                 & Buchmann, Lennart     & 07.05.20 & 07.05.20 & 5     & Konzeption und Umsetzung von Ladebildschirm, Timeout \& Weiterbildung \\
  \hline
  Login                                                                 & Buchmann, Lennart     & 02.05.20 & 02.05.20 & 4     & Grundlagenrechte \\
  \hline
  Login                                                                 & Buchmann, Lennart     & 04.05.20 & 04.05.20 & 5     & Login Persistenz \\
  \hline
  Optimierung des Graphen                                               & Gwozdz, Jonas         & 30.04.20 & 30.04.20 & 1,5   & Design erstellen \\
  \hline
  Sammelticket für Feedback vom Kunden                                  & Gwozdz, Jonas         & 29.04.20 & 29.04.20 & 1,5   & Beschriftung ändern \\
  \hline
  BUG TRACKER                                                           & Gwozdz, Jonas         & 04.05.20 & 04.05.20 & 5     & Bug: Node-Overlapping ab 3 Knoten \\
  \hline
  Backend Datenbnk                                                      & Heldt, Erik           & 28.04.20 & 05.05.20 & 2,75  & Schreiben einer echten Datenbank auf Docker \\
  \hline
  Backend Datenbank                                                     & Heldt, Erik           & 06.05.20 & 06.05.20 & 0,5   & HTTP-Request (Client-Side) \\
  \hline
  Design                                                                & Heldt, Erik           & 06.05.20 & 06.05.20 & 3     & Ausführliche Tests für GraphInfo.vue \\
  \hline
  Optimierung des Graphen                                               & Henning, Tim          & 05.05.20 & 05.05.20 & 2     & Initialzustände automatisch auswählen \\
  \hline
  Optimierung des Graphen                                               & Henning, Tim          & 06.05.20 & 06.05.20 & 1     & Ausgabe der Gesamtkosten/-zeit auf dem User-Interface \\
  \hline
  Optimierung des Graphen                                               & Herterich, Linus      & 30.04.20 & 30.04.20 & 2     & Settings $\rightarrow$ Optimierung persistent speichern \\
  \hline
  Design                                                                & Herterich, Linus      & 30.04.20 & 06.05.20 & 5,5   & Redesign Optimierungsansicht \\
  \hline
  Design                                                                & Herterich, Linus      & 06.05.20 & 06.05.20 & 1     & Graph-Editor expandierbar \\
  \hline
  Design                                                                & Herterich, Linus      & 04.05.20 & 04.05.20 & 1     & Tests zum neuen Design \\
  \hline
  Design                                                                & Herterich, Linus      & 03.05.20 & 04.05.20 & 1,5   & Save Menu $\rightarrow$ Test fixen \\
  \hline
  Design                                                                & Herterich, Linus      & 04.05.20 & 04.05.20 & 1     & NewGraph Menu $\rightarrow$ Test fixen \\
  \hline
  Design                                                                & Herterich, Linus      & 04.05.20 & 04.05.20 & 1,5   & Überschreiben Dialog für Save-Menu \\
  \hline
  Design                                                                & Herterich, Linus      & 27.04.20 & 30.04.20 & 1,5   & Avatar-Menü $\rightarrow$ Ausloggen \& Einstellungen \\
  \hline
  Design                                                                & Herterich, Linus      & 27.04.20 & 27.04.20 & 2     & Merge auf Targetbranch \\
  \hline
  Design                                                                & Herterich, Linus      & 30.04.20 & 04.05.20 & 2,5   & Components aufräumen \\
  \hline
  Design                                                                & Herterich, Linus      & 27.04.20 & 27.04.20 & 0,5   & Login fixen \\
  \hline
  Design                                                                & Herterich, Linus      & 27.04.20 & 27.04.20 & 1     & Tests fixen \\
  \hline
  Design                                                                & Herterich, Linus      & 07.05.20 & 07.05.20 & 0,5   & Altes Design in 'removed code' \\
  \hline
  Backend Datenbank                                                     & Hohlfeld, Julius      & 05.05.20 & 06.05.20 & 7,5   & API-Parser \\
  \hline
  Backend Datenbank                                                     & Hohlfeld, Julius      & 03.05.20 & 07.05.20 & 4     & Node.js Programmierung (Server-Side) \\
  \hline
  Backend Datenbank                                                     & Hohlfeld, Julius      & 28.04.20 & 29.04.20 & 14,5  & Set-Up Docker MySQL \\
  \hline
  Backend Datenbank                                                     & Hohlfeld, Julius      & 03.05.20 & 03.05.20 & 3     & Umgehung des MySQL Authentifizierungsprotokoll \\
  \hline
  Backend Datenbank                                                     & Hohlfeld, Julius      & 03.05.20 & 05.05.20 & 5     & Anbindung von Docker zu JS \\
  \hline
  Login                                                                 & Karkoutli, Alaa Aldin & 07.05.20 & 07.05.20 & 5     & Konzeption und Umsetzung von Ladebildschirm, Timeout \& Weiterbildung \\
  \hline
  Login                                                                 & Karkoutli, Alaa Aldin & 04.05.20 & 04.05.20 & 4     & Grundlagenrechte \\
  \hline
  Optimierung des Graphen                                               & Koch, David           & 30.04.20 & 06.05.20 & 12    & Ausgabe der Gesamtkosten/-zeit auf dem User-Interface \\
  \hline
\end{longtable}

\subsection{Konkrete Test-Überdeckung im Sprint}
{\small Autor: David}

Die Testabdeckung war deutlich besser als bei vorherigen Sprints. Auch wenn nicht zu allen bearbeiteten Aufgaben Tests angelegt wurden, lag dies lediglich an mangelnder Zeit, wenn sich eine User-Story als aufwendiger herausstellte als erwartet.

\subsection{Ergebnisse des Reviews}
{\small Autor: David Koch}

Anwesend: Alaa Aldin, Lennart, David, Erik, Julius J., Julius H., Jonas, Linus, Manuel, Tim\\

Im Rahmen des Reviews haben wir wie gewohnt die Ergebnisse des Sprints bewertet und Schwierigkeiten besprochen.\\

\textbf{Problem: Warten auf das ITSZ der HTWK}\\
Die derzeit wichtigsten ausstehenden Aufgaben betreffen vor allem die Themen Server und Datenbank. Da die Anwendung samt Datenbank und Shibboleth-Login am Ende auf einem Server der HTWK laufen soll, können wir dies ohne Antwort des ITSZ bisher nur eingeschränkt bearbeiten. \\

\textbf{Design:}
\begin{itemize}
\item neues Design wurde eingebunden
\item davon sind einige (wenige) Teile noch nicht funktional 
\end{itemize}

\textbf{Optimierung des Graphen:}
\begin{itemize}
\item Gesamtkosten und -zeit werden im neuen UI ausgegeben, alternative Pfade allerdings noch nicht
\item automatische Auswahl von Start- und Endknoten wurde fertiggestellt, jedoch noch nicht mit implementiert
\item Losgröße wurde noch nicht eingebaut
\end{itemize}

\textbf{Erstellung von Bearbeitungsschritten durch Klicken:}
\begin{itemize}
\item Validierung der Kanten von Erstellen auf Bearbeiten umgelegt
\item vor der Optimierung wird überprüft, ob alle Eigenschaften gegeben sind
\end{itemize}

\textbf{Sammelticket für Feedback vom Kunden:}
\begin{itemize}
\item Tests für Toolbar, Zoom-Controls, Buttons und Eingabereihenfolgen geschrieben
\end{itemize}

\textbf{Julius H., Erik, Linus:}
\begin{itemize}
\item Knoten und Kanten heißen jetzt Teil und Bearbeitungsschritt
\item Automatisches Verschieben bei überlappenden Knoten wurde überarbeitet
\end{itemize}

\textbf{Login:}
\begin{itemize}
\item Login-Persistenz ist eingebaut
\item Ablaufender Zeitstempel ist eingebaut, wird aber bisher bei zu wenigen Aktionen refreshed
\item zum Testen benötigte Buttons wurden entfernt
\item Graph geht beim Neuladen der Seite nicht mehr verloren
\item Login-Credentials können gespeichert werden
\item erste Erfahrung sammeln im Umgang mit Shibboleth
\item Anbindung an Shibboleth noch nicht möglich (fehlende Antwort des ITSZ)
\end{itemize}

\textbf{Backend Datenbank:}
\begin{itemize}
\item Docker Datenbank aufgesetzt und angebunden (noch nicht auf HTWK-Server)
\item Grundlegende Struktur für API-Entwicklung implementiert
\item Testdaten in Datenbank gespeichert, erste SQL-Befehle auf der Datenbank getestet
\item Dokumentation vorhanden
\end{itemize}

\subsection{Ergebnisse der Retrospektive}
{\small Autor: David Koch}

Anwesend: Alaa Aldin, Lennart, David, Erik, Julius J., Julius H., Jonas, Linus, Manuel, Tim\\

Nach den zwischenzeitlich schlechten Einschätzungen bezüglich des Erfolgs dieses Sprints gab es doch eine positive Überraschung, wie viele der User-Stories schon umgesetzt werden konnten. Die Retrospektive wird wie im letzten Sprint nach dem KALM-Prinzip durchgeführt.\\

\textbf{Keep:}
\begin{itemize}
\item Produktivität/Motivation
\item Zusammenarbeit 
\item Zeiten buchen
\item Übersicht im YouTrack
\item Absprache beim Mergen
\item kommentierter Code
\end{itemize}

\textbf{Add:}
\begin{itemize}
\item Fortschritte mitteilen
\item Dark Mode
\item Antworten des ITSZ
\end{itemize}

\textbf{Less:}
\begin{itemize}
\item keine Anmerkungen
\end{itemize}

\textbf{More:}
\begin{itemize}
\item Tests
\item Fehlerhafte Tests fixen
\item Bugfixing
\item Branches löschen
\end{itemize}


\subsection{Abschließende Einschätzung des Product-Owners}
{\small Autor: Manuel Eckert}

Kurz vor Beginn des Sprints konnten wir nochmals Feedback von unserem Kunden erhalten. Daraufhin haben sich kurzfristig einige Änderungen für diesen Sprint ergeben, wie beispielsweise bei der Optimierung des erzeugten Graphen oder der Erstellung von einzelnen Knoten-/Kantenelementen.\\
Weiterhin wurde ein moderneres und ansprechenderes UI-Design implementiert. \\
Zum Ende des Sprints wurde ein zufriedenstellendes Ergebnis erreicht und fast alle Aufgaben wurden zu einer guten Zufriedenheit komplettiert. 
Leider gibt es immer noch in einigen Modulen nicht genügend Tests um die geforderte Definition of Done zu erreichen. \\
Das ITSZ war für uns immer noch nicht telefonisch oder via E-Mail zu erreichen.

\subsection{Abschließende Einschätzung des Software-Architekten}
{\small Autor: Julius Jolig}

Die noch ausstehende Antwort des ITSZ der HTWK verzögert die Arbeit am Projekt. Mit diesem Sprint konnte eine höhere Testabdeckung erzielt werden. 

\subsection{Abschließende Einschätzung des Team-Managers}
{\small Autor: Alex Hofmann}

Der Sprint schien nach Stand des letzten Scrum Meetings eher negativ auszufallen, jedoch wurden einige User-Stories durch großes Engagement des Teams dennoch in der verbleibenden Zeit finalisiert.




\newpage

\section{Sprint 7}

\subsection{Ziel des Sprints}
{\small Autor: Julius Hohlfeld}

Ziel des Sprints war es die wichtige Features wie die Datenbank und Optimierung weiterzuentwickeln und Fortschritte in Design und Usability zu machen.\\

\subsection{User-Stories des Sprint-Backlogs}
{\small Autor: Julius Hohlfeld}

\begin{itemize}
  \item \textbf{Bugs fixen}
        \\\textit{Als Benutzer möchte ich eine Software benutzen, in welcher keine unerwarteten Probleme auftauchen.}
  \item \textbf{ Optimierung - Losgröße }
        \\\textit{
        Als Nutzer möchte ich Losgrößen der Bearbeitungsschritte einstellen und optimieren. Die Optimierung soll automatisch Knoten für Start- und Endzustand auswählen.}
  \item \textbf{Drag und Drop Erstellung für Kanten }
        \\\textit{
          Als Benutzer möchte ich schnell und effizient Kanten erstellen können.}
  \item \textbf{Design - Dark Mode}
        \\\textit{
          Als User will ich persönliche Präferenz über das Aussehen (konkret Dark Mode) der Applikation treffen.}
  \item \textbf{Login}
        \\\textit{
          Als Nutzer will mich mit dem HTWK-Login einloggen (Shibboleth) und erwarte eine persistente Erfahrung während ich eingeloggt bin.}
  \item \textbf{Backend-Datenbank}
        \\\textit{
          Als Benutzer möchte ich Graphen über die Anwendung und eine API auf einer Datenbank speichern und von dort aus runterladen.}
\end{itemize}

\subsection{Liste der durchgeführten Meetings}
{\small Autor: Julius Hohlfeld}

\begin{itemize}
\item 11.05.2020: Planning
\item 15.05.2020: Weekly
\item 18.05.2020: Weekly
\item 22.05.2020: Review \& Retro
\end{itemize}

\subsection{Ergebnisse des Planning-Meetings}
{\small Autor: Julius Hohlfeld}

Anwesend: Alaa Aldin, David, Erik, Jonas, Julius H., Lennart, Linus, Matthias, Tim, Alex, Manuel, Julius J.\\
\\
Im Planning haben wir die Ziele für die unterschiedlichen Arbeitsbereiches des Projekts festgelegt und sie in ihrer Schwierigkeit bewertet.\\


\textbf{Backend Datenbanken}\\
Es wurde über die Überführung des Datenbankprototyps in Docker auf die HTWK-Server gesprochen. Allerdings hängt dies zur Zeit noch von der Bereitstellung von der HTWK ab.
Um das Projekt auch auf echten Servern zu testen, wurde über Alternativen dikutiert. Als Ziel wurde gesetzt weiter die API auszubauen, besonders POST- und DELETE Request sollen
umgesetzt werden. Diese Ziele wurden mit einer 6 bewertet.\\

\textbf{Login}\\
Es wurde fesges,tellt dass man auch bei Shibole noch auf eine Antwort der HTWK-IT wartet.
Es galt als Ziel die Persistenz des Graphen in einer Sitzung zu erreichen, welches mit 6 bewertet wurde.\\

\textbf{Optimierung des Graphen}\\
Auf Hinweis des Kunden sollen Losgrößen für Kanten implementiert und bei der Optimierung miteingerechnet werden. Auch eine automatische Auswahl der Start- und Endknoten, falls der User
es selber noch nicht ausgewählt hat, soll erzeitl werden. Die Gruppe schätzte diese Vorhaben mit einer 5 ab.\\

\textbf{Design}\\
Auf Wunsch des Teams wurde ein Dark Mode für diesen Sprint als Ziel gesetzt. Diese Aufgabe wurde mit einer 4 bewertet.\\

\textbf{Erstellung der Kanten durch Drag\&Drop}\\
Um die Usability der Anwendung zu erhöhen, wurde vorgeschlagen zwischen verschiedenen Knoten Kanten über Drag\&Drop zu erstellen. Es wurde noch diskutiert inwiefern dies mit der Optimierung
und Validierung der Kanten kollidiert, da diese User spezifisierte Eingaben benötigen. Das Team stimmte ab und schätzte die Schwierigkeit auf 5.\\

\textbf{Bugs fixen}\\
Über die letzen Sprints hatten sich Bugs gesammelt und einige Teammitglieder hatten sich extra auf Bughunt begeben um Probleme zu finden. Daraus ergab sich eine Liste folgender Bugs, die es zu fixen galt:

\begin{itemize}
\item Selection-Felder bei Optimierungs-Einstellungen werden nicht initial geladen
\item Falsche Anzeige der Gesamtkosten und Zeit nach Importierungen
\item Datenbank GUI wechselt nicht automatisch die Seite wenn mehr Elemente angezeigt werden
\item Start und Endzustandsanzeige
\item Fehlerhafte Verschiebung bei Manchen Knotenkonstellationen
\item Nach Laden des Graphen falscher Produktname und Produktanzahl
\item Zu viele Dialogs übereinander
\item VarG-Dialog verschwindet bei Überlappung zu schnell
\item Startknoten
\item Valdierung bei Stückzahleingabe
\item Import wird auf Richtigkeit getestet
\item Slider für Anzahl der Optimierungswege größer als Wege generell
\end{itemize}

\subsection{Aufgewendete Arbeitszeit pro Person$+$Arbeitspaket}
{\small Autor: Julius Hohlfeld}

\begin{longtable}{|p{4cm}|p{2cm}|p{1.2cm}|p{1.2cm}|p{0.7cm}|p{3.8cm}|}
  \hline
  Arbeitspaket                                                          & Person                & Start    & Ende     & h     & Artefakt                                                    \\
  \hline
  Bug: Nach laden des Graphen falscher Produktname und Produktanzahl    & Berger, Matthias      & 17.05.20 & 17.05.20 & 1     & GraphInfo.vue                                               \\ \hline
  Rollenmanagement                                                      & Berger, Matthias      & 17.05.20 & 17.05.20 & 1     & store/store.js                                              \\ \hline
  Persistenz des Graphen                                                & Berger, Matthias      & 16.05.20 & 21.05.20 & 12,5  & store/store.js                                              \\ \hline
  Überarbeitung der Weiterleitung                                       & Berger, Matthias      & 16.05.20 & 16.05.20 & 1     & router/index.js                                             \\ \hline

  Bug: Nach laden des Graphen falscher Produktname und Produktanzahl    & Buchmann, Lennart     & 17.05.20 & 17.05.20 & 1     & GraphInfo.vue                                               \\ \hline
  Rollenmanagement                                                      & Buchmann, Lennart     & 17.05.20 & 17.05.20 & 1     & store/store.js                                              \\ \hline
  Persistenz des Graphen                                                & Buchmann, Lennart     & 16.05.20 & 21.05.20 & 11,5  & store/store.js                                              \\ \hline
  Überarbeitung der Weiterleitung                                       & Buchmann, Lennart     & 16.05.20 & 16.05.20 & 1     & router/index.js                                             \\ \hline

  Tests für Einstellungen                                               & Gwozdz, Jonas         & 14.05.20 & 14.05.20 & 0,5   & settings\_spec.js                                            \\ \hline
  Testen der Applikation                                                & Gwozdz, Jonas         & 12.05.20 & 12.05.20 & 0,5   & Testen der Applikation                                      \\ \hline
  Dark Mode                                                             & Gwozdz, Jonas         & 13.05.20 & 21.05.20 & 18    & Darkmode.vue                                                \\ \hline
  Farbschema erstellen                                                  & Gwozdz, Jonas         & 13.05.20 & 13.05.20 & 0,75  & darkmode.less                                               \\ \hline
  "varg" zu "VarG" ändern                                               & Gwozdz, Jonas         & 14.05.20 & 14.05.20 & 0,33  & Gesamtes Projekt                                            \\ \hline
  
  Bug: Importieren funktioniert bei veränderten Knotenpositionen nicht  & Heldt, Erik           & 12.05.20 & 12.05.20 & 0,14  & removed Code                                                \\ \hline
  Bug: MenuControls-Fenster ist nach unten verschoben                   & Heldt, Erik           & 12.05.20 & 12.05.20 & 0,14  & removed Code                                                \\ \hline
  Validierung bei Stückzahleingabe                                      & Heldt, Erik           & 12.05.20 & 12.05.20 & 1     & GraphInfo.vue                                               \\ \hline
  Axios-Requests für Post, Get usw.                                     & Heldt, Erik           & 13.05.20 & 20.05.20 & 8,25  & DatabaseForm.vue                                            \\ \hline
  Ersetzen der TestDatabase durch MySQL Datenbank                       & Heldt, Erik           & 13.05.20 & 20.05.20 & 6,25  & DatabaseForm.vue                                            \\ \hline
  Testdaten für die Datenbank                                           & Heldt, Erik           & 19.05.20 & 19.05.20 & 0,25  & dump.sql                                                    \\ \hline
  Kodierungsfehler beheben                                              & Heldt, Erik           & 19.05.20 & 19.05.20 & 1,5   & api.js                                                      \\ \hline

  Initialzustände auswählen                                             & Henning, Tim          & 12.05.20 & 18.05.20 & 4     & optimization.js                                             \\ \hline
  Bug: Startknoten                                                      & Henning, Tim          & 12.05.20 & 12.05.20 & 2     & Graph.vue                                                   \\ \hline
  Tests für Optimierung                                                 & Henning, Tim          & 18.05.20 & 18.05.20 & 3     & optimize.spec.js                                            \\ \hline

  Optimierung des Graphen                                               & Herterich, Linus      & 21.05.20 & 21.05.20 & 0,5   & optimization.js                                             \\ \hline
  Erstellung von Bearbeitungsschritte durch Klicken                     & Herterich, Linus      & 21.05.20 & 21.05.20 & 0,5   & RightClickMenu.vue                                          \\ \hline
  Validierung umstellen                                                 & Herterich, Linus      & 18.05.20 & 18.05.20 & 0,5   & DetailControls.vue                                          \\ \hline
  Tests für Einstellungen                                               & Herterich, Linus      & 14.05.20 & 14.05.20 & 0,75  & settings\_spec.js                                            \\ \hline
  VarG-Dialog verschwindet bei Überlappung zu schnell                   & Herterich, Linus      & 12.05.20 & 12.05.20 & 3     & Dialogs.vue                                                 \\ \hline
  Dark-Mode                                                             & Herterich, Linus      & 17.05.20 & 19.05.20 & 2,5   & Darkmode.vue                                                \\ \hline
  Konzeption Kantenerstellung                                           & Herterich, Linus      & 14.05.20 & 14.05.20 & 0,5   & konzeptuelle Aufgabe                                        \\ \hline
  Drag\&Drop Funktionalität                                              & Herterich, Linus      & 14.05.20 & 19.05.20 & 8     & edges.js                                                    \\ \hline
  Rechtsklick-Menü: Kante "von"/"nach" überarbeiten                     & Herterich, Linus      & 21.05.20 & 21.05.20 & 0,5   & RightClickMenu.vue                                          \\ \hline
  Tests: Kantenerstellung                                               & Herterich, Linus      & 21.05.20 & 21.05.20 & 0,5   & quickEdges\_spec.js                                          \\ \hline
  Warnungs-Dialog Farbe anpassen                                        & Herterich, Linus      & 18.05.20 & 18.05.20 & 0,5   & Dialogs.vue                                                 \\ \hline
  zweit- bis x-besten Graphen ausgeben                                  & Herterich, Linus      & 19.05.20 & 20.05.20 & 3,75  & OptimizeControls.Vue                                        \\ \hline
  Bug: Losgröße - Detail Menü                                           & Herterich, Linus      & 19.05.20 & 19.05.20 & 0,5   & DetailControls.vue                                          \\ \hline
  Validierung Losgröße                                                  & Herterich, Linus      & 19.05.20 & 19.05.20 & 0,5   & DetailControls.vue                                          \\ \hline
  Zu viele Dialogs übereinander                                         & Herterich, Linus      & 21.05.20 & 21.05.20 & 1     & Dialogs.vue                                                 \\ \hline
  Neue Labels                                                           & Herterich, Linus      & 21.05.20 & 21.05.20 & 0,75  & edges.js                                                    \\ \hline

  API-Parser                                                            & Hohlfeld, Julius      & 12.05.20 & 14.05.20 & 0,66  & api.js                                                      \\ \hline         
  Import wird nicht auf Richtigkeit getestet                            & Hohlfeld, Julius      & 13.05.20 & 19.05.20 & 9,25  & JsonPersistence.js                                          \\ \hline
  Express-Post                                                          & Hohlfeld, Julius      & 14.05.20 & 14.05.20 & 2,25  & api.js                                                      \\ \hline
  Express-Delete                                                        & Hohlfeld, Julius      & 12.05.20 & 12.05.20 & 0,08  & api.js                                                      \\ \hline
  Express-Put                                                           & Hohlfeld, Julius      & 12.05.20 & 12.05.20 & 0,5   & api.js                                                      \\ \hline
  Sprint Doku 7                                                         & Hohlfeld, Julius      & 11.05.20 & 22.05.20 & 3,5   & sprint7.tex                                                 \\ \hline
  Kodierungsfehler beheben                                              & Hohlfeld, Julius      & 19.05.20 & 20.05.20 & 3,5   & api.js                                                      \\ \hline
  
  Persistenz des Graphen                                                & Karkoutli, Alaa Aldin & 16.05.20 & 21.05.20 & 13,5  & store/store.js                                              \\ \hline
  Überarbeitung der Weiterleitung                                       & Karkoutli, Alaa Aldin & 16.05.20 & 16.05.20 & 1     & router/index.js                                             \\ \hline

  Losgröße einbinden                                                    & Koch, David           & 14.05.20 & 15.05.20 & 3     & edges.js                                                    \\ \hline
  Tests für Optimierung                                                 & Koch, David           & 18.05.20 & 18.05.20 & 2     & optimize.spec.js                                            \\ \hline
  zweit bis x-besten Graphen ausgeben                                   & Koch, David           & 19.05.20 & 20.05.20 & 7     & OptimizeControls.Vue                                        \\ \hline
  \\ \hline
\end{longtable}

\subsection{Konkrete Code-Qualität im Sprint}
{\small Autor: Julius Hohlfeld}

Die Qualtität des Codes war weiterhin gut. Das Team wurde allerdings vom Software-Architekten darauf hingewiesen, die Kommentare verständlicher zu schreiben.\\

\subsection{Konkrete Test-Überdeckung im Sprint}
{\small Autor: Julius Hohlfeld}

Sämtliche neuen Änderungen wurden auch mit Tests kontrolliert. Es kam zur Diskussion auch Tests für die API zu schreiben, welche noch nicht getestet wird.\\

\subsection{Ergebnisse des Reviews}
{\small Autor: Julius Hohlfeld}

Anwesend: Alaa Aldin, Erik, Jonas, Julius H., Lennart, David, Manuel, Alex, Julius J. \\

Im Review stellte jeder Anwesende seine Arbeit vor. Einige fehlten, wurden aber durch eng zusammenarbeitende Teammitglieder vertreten. \\
Die Ergebnisse und Erkenntnisse wurden ausgewertet.\\

\textbf{HTWK-IT}\\
Mehrfach kam das Thema der fehlenden Kommunikation auf Seiten der HTWK-IT für unser projektkritischen Anfragen zur Sprache. Es wurden neue Kommunikationsversuche besprochen.
\\

\textbf{David und Tim:}
\begin{itemize}
\item Losgröße integriert
\item Pfad-Optimierung beachtet Losgröße + Ausgabe
\item Initialzustände beim Optimieren
\item Optimierungstests
\end{itemize}

\textbf{Erik:}
\begin{itemize}
\item Stückzahl: keine Kommas und führende Nullen
\item Laden, Löschen, Hochladen von Graphen in DB
\item Hashkey für Speicherung
\end{itemize}

\textbf{Jonas:}
\begin{itemize}
\item Test für Kantenerstellung
\item Darkmode (eigene Komponente)
\end{itemize}

\textbf{Julius H.:}
\begin{itemize}
\item API-Anpassungen
\item JSON Validation
\item Kodierungsfehler behoben
\end{itemize}

\textbf{Linus:}
\begin{itemize}
\item Pfad-Ausgabe
\item Drag\&Drop + Tests
\item Benachrichtigungs Quality of Life
\item Einstellungsmenü Tests
\end{itemize}

\textbf{Alaa Aldin, Lennart und Matthias:}
\begin{itemize}
\item Graph hat Persistenz
\item Rollenmanagement
\item Weiterleitung
\end{itemize}

\textbf{Bugs}\\
Es wurden Bugs angesprochen, welche im Verlauf des Sprints oder beim mergen auf den Targetbranch entdeckt wurden.\\

\subsection{Ergebnisse der Retrospektive}
{\small Autor: Julius Hohlfeld}

Anwesend: Alaa Aldin, Erik, Jonas, Julius H., Lennart, David, Manuel, Alex, Julius J. \\

Das Team war zufrieden mit der erreichten Leistung und hat eine positve Ansicht zum Sprint. Angesichts des baldigem Ende des Softwareprojekts wurde auch die Verwendung der restlichen Zeit kurz diskutiert.\\

\begin{center}
\begin{tabular}{ |c|c|c|c| }
\hline
 Keep & Add & Less & More\\
\hline 
 -Zusammenhalt & -Server & -neue instabile Features (weil wenig Sprints übrig) & -Druck bei HTWK für Serverplatz \\
 -Kommunikation & - Tests & & Tests\\
 & & & -Bugfixes\\
 & & & -Code Kommentare\\
 & & & -Quality assurance\\
\hline     
\end{tabular}
\end{center}

\subsection{Abschließende Einschätzung des Product-Owners}
{\small Autor: Manuel Eckert}

In den vorherigen Sprints wurde größerer Fokus auf die Ausarbeitung des Frontends gelegt. Somit konnte in diesem und den folgenden Sprints der Schwerpunkt auf das Backend gelegt werden. Dies wurde zusätzlich durch eine fehlende Rückmeldung des ITSZ begünstigt. \\
Inzwischen haben die Entwickler eine sehr gut Kompetenz entwickelt, den Umfang der einzelnen Aufgaben einzuschätzen. Somit konnten wir fast alle im Planning-Meeting besprochenen User-Storys erfolgreich abschließen. \\
Leider mussten wir Aufgrund der fortgeschrittenen Zeit und der schlechten Kommunikation zum ITSZ, eine Authentifizierung mittels Shibboleth verwerfen.   

\subsection{Abschließende Einschätzung des Software-Architekten}
{\small Autor: Julius Jolig}

Die noch immer ausstehende Antwort des ITSZ der HTWK verzögert die Arbeit am Projekt. Hier muss eine Lösung gefunden werden. Die Anzahl der Kommentare im Code ist soweit gut, allerdings müssen diese klarer formuliert werden. 

\subsection{Abschließende Einschätzung des Team-Managers}
{\small Autor: Alex Hofmann}

Weiterhin trotz aller Umstände eine sehr fokussierte und lobenswerte Arbeit des Teams, gerade auch im Vergleich zu anderen Teams, wie man von deren Masterstudenten hört.



\newpage

\section{Sprint 8}

\subsection{Ziel des Sprints}
{\small Autor: Alaa Aldin Karkoutli}

Ziel des Sprints war es, das Software zu dokumentieren und die noch stehenden Bugs zu beheben.\\

\subsection{User-Stories des Sprint-Backlogs}
{\small Autor: Alaa Aldin Karkoutli}

\begin{itemize}
  \item \textbf{Bugs fixen}
        \\\textit{Als Benutzer möchte ich eine Software benutzen, in welcher keine unerwarteten Probleme auftauchen.}
  \item \textbf{ Optimierung }
        \\\textit{
        Als Nutzer möchte ich wissen, nach welchen Kriterien die Graphen optemiert sind .}
  \item \textbf{Dokumentation }
        \\\textit{
          Als Benutzer möchte mit einem Software arbeiten, das eine vollständige Dokumentation hat .}
  \item \textbf{Persistenz der eingetragenen Daten}
        \\\textit{
          Als User will ich Kanten/Knoten Eigenschaften automatisch zu speichern, wenn sie bearbeitet werden.}
  \item \textbf{Login}
        \\\textit{
          Als Nutzer möchte ich mehr Zugriff auf der DatenBank haben wenn ich Admin bin(Rollenmangment) und Login noch bissel optemiert sehen.}
  \item \textbf{Backend-Datenbank}
        \\\textit{
          Als Benutzer möchte ich Test-Graphen in der Datenbank finden und die DatenBank vor Gefahr schützen .}
\end{itemize}

\subsection{Liste der durchgeführten Meetings}
{\small Autor: Alaa Aldin Karkoutli}

\begin{itemize}
\item 25.05.2020: Planning
\item 28.05.2020: Weekly
\item 01.06.2020: Weekly
\item 04.06.2020: Review \& Retro
\end{itemize}

\subsection{Ergebnisse des Planning-Meetings}
{\small Autor: Alaa Aldin Karkoutli}

Anwesend: Alaa Aldin, David, Erik, Jonas, Julius H., Lennart, Linus, Matthias, Tim, Alex, Manuel, Julius J.\\
\\
Im Planning haben wir die wichtigsten Punkte festgesetzt, die in der letzten Sprints Priorität haben, und die Schwierigkeit von denen bewertet.\\


\textbf{Backend Datenbanken}\\
Die Datenbank soll auf einem Server aufgesetzt werden, wenn Server bereitgestellt ist.
Die generierte Test-Daten können in die Datenbank eingespeist werden. Als Ziel wurde gesetzt die Bugs zu beheben und alles nebenbei zu dokumentieren. 
Diese Ziele wurden wie im letzten Sprint mit einer 6 bewertet.\\

\textbf{Login}\\
Weil es keine Antwort von IT-HTWK gab, konnte das Software an Schibboleth nicht verbunden werden.
Es galt als Ziel Bugs zu behebn, Login zu überarbeiten und einen einfachen Zugriff auf die Datenbank vorzubereiten, in Absprache mit Backend-Team.
Diese Ziele wurden mit einer 7 bewertet.\\

\textbf{Optimierung}\\
Es ist wichtig anzuzeigen, nach welchen Kriterien der Graph Optimiert ist.
Dieses Ziel wurde mit 4 bewertet.\\

\textbf{Persistenz der eingetragen Daten}\\
Die Eigenschaften der Kanten/Knoten sollen automatisch gespeichert, wenn der Benutzer irgendeine Änderung auf diese Eigenschaften macht.
Die Losgröße soll auch ohne Scrollbar dargestellt werden.
Diese Ziele wurden mit 4 bewertet, aber nach hinten eingestellt, da Doku Priorität hat. \\

\textbf{Dokumentation}\\
Das Software soll ausführlich dokumentiert werden. Tests, Code-Kommentare und 
Technische Ausarbeitungen müssen noch gemacht werden. Die Gedanken, die nicht gemacht werden konnten (Schibboleth, Datenbank..), müssen
auch in der Dokumentation stehen. 
Das Team stimmte ab und schätzte die Schwierigkeit auf 7.\\

\textbf{Bugs fixen}\\
Möglichst alle Bugs zu beheben ist seit den letzten Sprints in Bearbeitung. Das Ziel ist es,
ein fehlerfreies Software abzugeben, indem alles ausprobieren und die Bugs lösen.

\subsection{Aufgewendete Arbeitszeit pro Person$+$Arbeitspaket}
{\small Autor: Alaa Aldin Karkoutli}

\begin{longtable}{|p{4cm}|p{2cm}|p{1.2cm}|p{1.2cm}|p{0.7cm}|p{3.8cm}|}
  \hline
  Arbeitspaket                                                          & Person                & Start    & Ende     & h     & Artefakt                                                    \\
  \hline
  Überarbeitung Login                                                   & Berger, Matthias      & 31.05.20 & 31.05.20 & 4     & LoginForm.vue $+$ store/store.js                                              \\ \hline
  

  Überarbeitung Login                                                   & Buchmann, Lennart     & 31.05.20 & 31.05.20 & 4     & LoginForm.vue $+$ store/store.js                                              \\ \hline
  Überarbeitung Login                                                   & Buchmann, Lennart     & 04.06.20 & 04.06.20 & 1     & LoginForm.vue $+$ store/store.js                                             \\ \hline


  Bug: Fehlerhafte Verschiebung bei Manchen Knotenkonstellationen       & Gwozdz, Jonas         & 27.05.20 & 27.05.20 & 2,15   & position.js $+$ src/vargraph/graph/edges.js                                            \\ \hline
  Bug: Fehlerhafte Verschiebung bei Manchen Knotenkonstellationen       & Gwozdz, Jonas         & 02.06.20 & 02.06.20 & 2,30   & position.js $+$ src/vargraph/graph/nodes.js                                      \\ \hline
  Verschieben der Knoten beim anlegen                                   & Gwozdz, Jonas         & 02.06.20 & 02.06.20 & 0,30   & src/vargraph/graph/nodes.js                                                \\ \hline
  Test für Darkmode                                                     & Gwozdz, Jonas         & 03.06.20 & 03.06.20 & 1,40   & darkmode\_spec.js                                               \\ \hline
  I.1 Initiale Kundenvorgaben                                           & Gwozdz, Jonas         & 03.06.20 & 03.06.20 & 0,40   & Dokumentation                                            \\ \hline
  III.2 Coding Style                                                    & Gwozdz, Jonas         & 03.06.20 & 03.06.20 & 1,10   & Dokumentation                                            \\ \hline
  
  
  Dokumentation                                                         & Heldt, Erik           & 28.05.20 & 28.05.20 & 3,45  & documentation/projektdokumentation.tex                                                \\ \hline
  Anzeigen des Graphen als Bild in der GUI                              & Heldt, Erik           & 01.06.20 & 03.06.20 & 3,30  & DatabaseForm.vue                                              \\ \hline
  Ersetzen der TestDatabase durch MySQL DB                              & Heldt, Erik           & 03.06.20 & 03.06.20 & 0,30  & removedcode/ExportDatabase                                              \\ \hline
  Überarbeitung Login                                                   & Heldt, Erik           & 04.06.20 & 04.06.20 & 1     & verschiedene Stellen                                            \\ \hline
  DB Button in Home Menu nicht korrekt angebunden                       & Heldt, Erik           & 03.06.20 & 03.06.20 & 1     & NewGraph.vue                                            \\ \hline
  Testen der Applikation                                                & Heldt, Erik           & 27.05.20 & 27.05.20 & 0,30  &                                                    \\ \hline
  Konsistenz bei der Positionierung ähnlicher Komponenten               & Heldt, Erik           & 01.06.20 & 03.06.20 & 0,50  & verschiedene Stellen                                      \\ \hline

  III.1 Definition of Done                                              & Henning, Tim          & 03.06.20 & 03.06.20 & 1     & projektdokumentation.tex                                             \\ \hline
  Bug: Start und Endzustandsanzeige                                     & Henning, Tim          & 29.05.20 & 29.05.20 & 2     & vargraph/graph/optimizations.js                                                   \\ \hline
  Selection-Felder bei Optimierungs-Einstellungen                       & Henning, Tim          & 31.05.20 & 31.05.20 & 4     & GraphInfo.vue $+$ vargraph/graph/optimizations.js                                          \\ \hline

  II.3 Überblick über Architektur                                       & Herterich, Linus      & 03.06.20 & 04.06.20 & 1,45  & projektdokumentation.tex                                             \\ \hline
  Testen der Applikation                                                & Herterich, Linus      & 03.06.20 & 03.06.20 & 0,5   &                                           \\ \hline
  Drag-n-Drop Button verschwindet nach drüberhovern nicht               & Herterich, Linus      & 28.05.20 & 03.06.20 & 2,10  & verschiedene Stellen                                          \\ \hline
  
  Bug: Falsche Anzeige der Gesamtkosten und Zeit nach Importieren       & Hohlfeld, Julius      & 25.05.20 & 28.05.20 & 1     & vargraph/JSonPersistence.js                                                    \\ \hline         
  Datenbank GUI wechselt nicht automatisch die Seite                    & Hohlfeld, Julius      & 27.05.20 & 27.05.20 & 1     & verschiedene Stellen                                          \\ \hline
  Schutz vor SQL-Injections                                             & Hohlfeld, Julius      & 03.06.20 & 03.06.20 & 1,30  & api.js                                                      \\ \hline
  II.4 Definierte Schnittstellen                                        & Hohlfeld, Julius      & 03.06.20 & 04.06.20 & 3,30  & projektdokumentation.tex                                                      \\ \hline
  Backend Test Recherche                                                & Hohlfeld, Julius      & 03.06.20 & 03.06.20 & 0,5   & Recherche                                                    \\ \hline

  Bugs:Persistenz des Graphen                                           & Karkoutli, Alaa Aldin & 02.06.20 & 02.06.20 & 3     & router/index.js                                              \\ \hline
  Sprint-Doku                                                           & Karkoutli, Alaa Aldin & 01.06.20 & 05.06.20 & 4     & documentation/Sprint\_8.tex                                            \\ \hline
  Graph wird nicht heruntergeladen,"Neuer Graph"->"Speichern"           & Karkoutli, Alaa Aldin & 01.06.20 & 01.06.20 & 1     & ExportDownload.vue $+$ HomeMenu.vue                                          \\ \hline
  Überarbeitung Login                                                   & Karkoutli, Alaa Aldin & 31.05.20 & 31.05.20 & 4     & LoginForm.vue $+$ store/store.js                                          \\ \hline
  
  
  Selection-Felder bei Optimierungs-Einstellungen                       & Koch, David           & 31.05.20 & 31.05.20 & 2     & GraphInfo.vue $+$ vargraph/graph/optimizations.js                                                   \\ \hline
  X.1 Handbuch                                                          & Koch, David           & 04.06.20 & 04.06.20 & 2     & projektdokumentation.tex                                         
\\ \hline
\end{longtable}

\subsection{Konkrete Code-Qualität im Sprint}
{\small Autor: Alaa Aldin Karkoutli}

Die Codequatlität war weiterhin gut.

\subsection{Ergebnisse des Reviews}
{\small Autor: Alaa Aldin Karkoutli}
        
Anwesend: Alaa Aldin, Erik, Jonas, Julius H., Lennart, David, Manuel, Alex, Julius J., Matthias, Linus, Tim \\

Im Review stellte jeder Anwesende seine Arbeit vor.\\
Die Ergebnisse und Erkenntnisse wurden ausgewertet.\\


\textbf{David:}
\begin{itemize}
\item Handbuch dokumentiert
\item Optimierung nicht viel gemacht
\end{itemize}

\textbf{Erik:}
\begin{itemize}
\item Dokumentationen geschrieben
\item Datenbank Preview-Bilder anzeigen
\item Testdatabase.js entfernt
\end{itemize}

\textbf{Jonas:}
\begin{itemize}
\item Springen der Knoten gefixt
\item Bechreibung in der Kanten lesen
\item Tests für Darkmode
\item Dokumentation geschrieben
\end{itemize}

\textbf{Julius H.:}
\begin{itemize}
\item Dokumentation geschrieben.
\item Sicher vor SQL-Injection
\item Bugs (Importiern und Datenbank) gefixet
\end{itemize}

\textbf{Linus:}
\begin{itemize}
\item Bug gefixt (Darg und Drop Knoten)
\item Dokumentation geschrieben
\end{itemize}

\textbf{Alaa Aldin, Lennart und Matthias (Login-Team):}
\begin{itemize}
\item Login-Überarbeitung vorbereitet (Code auskommentiert)
\item Rollenmanagement 
\end{itemize}

\textbf{Alaa Aldin:}
\begin{itemize}
\item Bugs(NeuerGraph -> Speichern, Persistenz der Graphen, Logout) gefixt
\item Knoten und Kanten in der gespeicherten Positionen laden
\item Änderungen auf importierten Graphen in der LocalStorage speichern
\end{itemize}

\textbf{Bugs}\\
Bugs sollen im Laufe des kommenden Sprints gefixt werden.\\

\subsection{Ergebnisse der Retrospektive}
{\small Autor: Alaa Aldin Karkoutli}

Anwesend: Alaa Aldin, Erik, Jonas, Julius H., Lennart, David, Manuel, Alex, Julius J. \\

Das Team war zufrieden mit der erreichten Leistung und hat eine positve Ansicht zum Sprint. Angesichts des baldigem Ende des Softwareprojekts wurde auch die Verwendung der restlichen Zeit kurz diskutiert.\\

\begin{center}
\begin{tabular}{ |c|c|c|c| }
\hline
 Keep & Add & Less & More\\
\hline 
-Teamwork & -Motivation & -unkompilierbare Doku & -Code Kommentare \\
-Bugfixing & -Sprint & & -Branches aufräumen \\
 & -funktionierender Login & & -Infos über Produktions-Version \\
 & -Console Errors entfernen & & \\
\hline     
\end{tabular}
\end{center}

\subsection{Abschließende Einschätzung des Product-Owners}
{\small Autor: Manuel Eckert}

Da dies der vorletzte Sprint war, wurde Wert darauf gelegt, möglichst keine große neue Funktionalitäten zu entwickeln, sonder möglichst die bestehenden zu finalisieren. Weiterhin galt es Bugs die sich beim Entwickeln ergeben haben zu beseitigen. Als letzten großen Punkt wurde die Projektdokumentation in diesem Sprint bearbeitet. \\
Da oft noch keine ausführliche Dokumentation zu den einzelnen Sprints geschrieben wurde, hat dies auch einen großen Teil des Sprints eingenommen. Trotz allem konnten wir alle User-Stories erfolgreich beendigen. \\
Da wir bis zu diesem Zeitpunkt vom ITSZ immer noch keine Nachricht bekommen haben, ob beziehungsweise wann wir einen Serverplatz bekommen können haben wir uns nach anderen Möglichkeiten umgesehen. Glücklicherweise hat sich der StudierendenRat sofort bereit erklärt uns zu helfen und ein Teil ihres Servers, welcher sich auch im HTWK-Netz befindet, für uns bereitzustellen.

\subsection{Abschließende Einschätzung des Software-Architekten}
{\small Autor: xxx}

XXX

\subsection{Abschließende Einschätzung des Team-Managers}
{\small Autor: Alex Hofmann}

Es ist vereinzelt an der Motivation zu spüren, dass sich das Semester und somit auch das Softwareprojekt dem Ende zuneigt.
Dies liegt vermutlich auch daran, dass aufgrund von nur noch einem verbleibenden Sprint einige Komponenten, woran die Bachelor jetzt schon mehrere Wochen dran gearbeitet haben, auf der Strecke bleiben werden.


\newpage

\section{Sprint 9}


\subsection{Ziel des Sprints}
{\small Autor: Matthias Berger}

Das Hauptaugenmerk im 9. Sprint wurde auf die Behebung noch vorhandener Bug, sowie die Projektdokumentation gelegt. Lediglich die Umsetzung der Loginfunktion im Backend wurde als größeres Ziel verfolgt.

\subsection{User-Stories des Sprint-Backlogs}
{\small Autor: Matthias Berger}

\begin{itemize}
\item \textbf{Ausarbeitung des Latex Dokuments, Code Kommentare, Technische Ausarbeitung}

\item \textbf{Persistenz der eingetragenen Daten }
      \\\textit{Wenn Kanten/Knoten Eigenschaften bearbeitet werden, automatische Speicherung (Wenn Fenster geschlossen wird, bzw aus dem Eigenschaftenfenster geklickt wird)
      \\Darstellung der Losgröße ohne Scrollbar (oder bei leerem Feld dort hin springen zweite Seite)}

\item \textbf{Umzug Datenbank auf Fachschaftsserver}
      \\\textit{Datenbank soll auf den Server des FSR umgezogen werden
      \\VarG von dort aus hosten lassen}
       
\item \textbf{Optimierung}
      \\\textit{Klarer machen, nach welchem Kriterium( Zeit/Kosten) Optimiert wurde ( Auf Oberfläche und in Einstellungen)}
      
\item \textbf{BUG TRACKER}

\item \textbf{Login}
      \\\textit{Überarbeitung Login (nicht in Klartext lesbar, ... )}
      
\item \textbf{Dokumentation}

\item \textbf{Abschlusspräsentation Messe vorbereiten }
\end{itemize}

\subsection{Liste der durchgeführten Meetings}
{\small Autor: Matthias Berger}
\begin{itemize}
  \item 08.06.2020: Planning Meeting
  \item 11.06.2020: Weekly Scrum
  \item 15.06.2020: Weekly Scrum
  \item 18.06.2020: Review \& Retro
\end{itemize}


\subsection{Ergebnisse des Planning-Meetings}
{\small Autor: Matthias Berger}

Anwesend: Jonas Gwozdz, Erik Heldt, Linus Herterich, Julius Hohlfeld, Lennart Buchmann, Tim Henning, Matthias Berger, Alaa Aldin Karkoutli, Manuel Eckert, Julius Jolig, Alex Hofmann\\

Zunächst wurde sich darauf geeinigt keine weiteren großen Projekte in Angriff zu nehmen. Stattdessen sollen ggf. Kommentare im Quelltext hinterlassen werden, die evtl. verfolgte Lösungsansätze dokumentieren sollen. Einzig großes Projekt war die Überarbeitung des Logins, sodass Nutzername und Passwort nicht weiter im Quelltext auslesbar sind. Ursprünglich sollte dies durch die Anbindung an Shibboleth ersetzt werden. Stattdessen musste nun innerhalb kürzester Zeit eine Verifikation im eigenen Backend umgesetzt werden. Die Hauptpriorität sollte auf der Dokumentation des Projektes, der Präsentation für die Messe und weiteren Bugfixes liegen.

\subsection{Aufgewendete Arbeitszeit pro Person$+$Arbeitspaket}
{\small Autor: xxx}

\begin{longtable}{|p{4cm}|l|l|l|l|l|}
        \hline
        Arbeitspaket & Person & Start & Ende & h & Artefakt\\
        \hline
        Dummyklassen & Musterstudi & 3.5.09 & 12.5.09 & 14 & Klasse.java\\ \hline
        AP XYZ &  &  &  & & \\ \hline
      \end{longtable}

\subsection{Konkrete Code-Qualität im Sprint}
{\small Autor: Matthias Berger}

Eines der Hauptziele im Sprint war die Kommentierung des Quelltextes zu überarbeiten. Es ist also davon auszugehen, dass sich die Qualität des Codes im laufe des Sprintes verbessert hat.

\subsection{Konkrete Test-Überdeckung im Sprint}
{\small Autor: Matthias Berger}

Da während des Sprints keine neuen Funktionalitäten implementiert wurden, sondern lediglich Bugfixing und Überarbeitung  im Vordergrund standen war es auch nicht nötig weitere Tests zu formulieren.

\subsection{Ergebnisse des Reviews}
{\small Autor: Matthias Berger}

Anwesend: Jonas Gwozdz, Erik Heldt, Linus Herterich, Julius Hohlfeld, Lennart Buchmann, Tim Henning, David Koch, Matthias Berger, Manuel Eckert, Julius Jolig, Alex Hofmann\\

Im Rahmen des Reviews wurde festgehalten, dass der Großteil der der geplanten  Sprintziele umgesetzt werden konnte. Um letzte kleinere Ausbesserungen, den vollständigen Umzug auf den vom FSR bereitgestellten Server und die Vorbereitung einer Präsentation für die geplante Messe zu ermöglichen wurde sich darauf geeinigt den Sprint zu verlängern. Neue Ziele wurden nicht formuliert.

\subsection{Ergebnisse der Retrospektive}
{\small Autor: Matthias Berger}

Anwesend: Jonas Gwozdz, Erik Heldt, Linus Herterich, Julius Hohlfeld, Lennart Buchmann, Tim Henning, David Koch, Matthias Berger, Manuel Eckert, Julius Jolig, Alex Hofmann\\

Da es sich um den letzten offiziellen Sprint handelte wurde in der Retrospektive auf das KALM-Schema verzichtet. Stattdessen wurde der Verlauf des gesamten Projektes bewertet. Die Meinungen hierzu fielen durchweg positiv aus. So wurde  positiv bewertet, dass sich im laufe des Projektes trotz der anfänglichen Schwierigkeiten in Organisation und Kommunikation zwischen den Teammitgliedern ein Teamgeist und ein Produkt, mit dem sich alle Beteiligten identifizieren können entwickelt hat. Auch die Kommunikation innerhalb und zwischen den Einzelteams und die Brücke, die zwischen den beiden Studiengängen der Teammitglieder geschlagen wurde wurde positiv hervorgehoben.

\subsection{Abschließende Einschätzung des Product-Owners}
{\small Autor: Manuel Eckert}

In diesem Sprint konnten wir endlich das entwickelte Backend auf einen Server im Netz der HTWK spielen. Dies war noch der letzte Meilenstein zu einer erfolgreichen Auslieferung des Produktes. \\
Da dies dann auch der letzte Sprint des SoftwareProjektes war, wurden nur noch kleinere User-Stories verteilt. Somit konnten sich die Entwickler auf diese Aufgaben konzentrieren und es wurde sichergestellt, dass keine Funktionalitäten Aufgrund von Zeitmangel verworfen werden mussten. \\
Das ein Projektabschluss trotz allem doch sehr schnell anstrengend werden kann haben wir alle gemerkt, da nicht immer alles ganz nach Plan läuft und in letzter Sekunde doch noch einmal Probleme auftauchen. \\
Nach dem Sprint war das gesamte Team sehr zufrieden, mit dem was wir trotzdem noch in diesem Sprint abgearbeitet haben. \\ 

\subsection{Abschließende Einschätzung des Software-Architekten}
{\small Autor: xxx}

XXX

\subsection{Abschließende Einschätzung des Team-Managers}
{\small Autor: xxx}

XXX

\newpage

%%%%%% weitere Sprints analog


\section{Dokumentation}

\subsection{Handbuch}
{\small Autor: David Koch}
\begin{itemize}
  \item \textbf{ Login }
    \\\
      Um sich einloggen zu können, benötigt man einen Account. Diesen legt man an, indem man sich im Login-Fenster mit einem neuen Benutzernamen und Passwort einloggt. Der angegebene Name wird als neuer Nutzer (mit Rolle: 'Student') angelegt und das dazugehörige Passwort gespeichert. Beides kann nachträglich in den Benutzereinstellungen (siehe 'Einstellungen') geändert werden.
      \\\textbf{Hinweis:} Wenn beim Versuch, einen neuen Account zu erstellen, der Fehler 'Ungültige Login-Daten' auftritt, bedeutet dies, dass der eingegebene Benutzername bereits existiert. In diesem Fall muss ein anderer Name gewählt werden.\\
      Ein Ändern der Rolle (von 'Student' auf 'Admin') ist nur durch direkten Zugriff auf die Datenbank möglich. (Momentan wird eine programmeigene User-Datenbank benutzt, diese könnte später durch eine Anbindung an Shibboleth ersetzt werden.)
  \item \textbf{ Neuen Graphen erstellen }
    \\\
      Über den Menüpunkt 'Neuen Graphen erstellen' im Startfenster gelangt man in das 'Neues Produkt'-Fenster. Hier können Name und Stückzahl des neuen Graphen festgelegt werden. Über den 'Starten'  Button wird der neue Graph erstellt. Im Graphenfenster gelangt man über den 'Neuer Graph' Button zurück ins Startmenü. Über die Stift-Icons neben dem Produktnamen und der Stückzahl können diese nachträglich bearbeitet werden.
  \item \textbf{ Erstellen und Bearbeiten von Zuständen }
    \\\
      Das 'Neues Teil' Menü öffnet sich entweder über Rechtsklick innerhalb des Graphenfensters, gefolgt von einem Linksklick auf 'Neues Teil' oder durch einen Linksklick auf den Plus-Button in der rechten unteren Ecke, gefolgt von einem Linksklick auf 'Neues Teil'. Hier kann neben Name, Kürzel und Farbe des Zustands auch ein dazugehöriges Icon mittels URL eingebunden werden. Alle Angaben außer dem Icon sind Pflichtangaben, sodass der 'erstellen' Button erst betätigt werden kann, wenn alle diese Felder ausgefüllt sind.\\
      Mit einem Linksklick auf einen bereits erstellten Zustand öffnet sich das 'Teil bearbeiten' Menü. Dieses ist aufgebaut wie das 'Neues Teil' Menü. Hier können alle Eigenschaften des ausgewählten Zustandes bearbeitet werden.
  \item \textbf{ Erstellen und Bearbeiten von Bearbeitungsschritten }
    \\\
      Das 'Neuer Bearbeitungsschritt'-Menü öffnet sich entweder über Rechtsklick innerhalb des Graphenfensters, gefolgt von einem Linksklick auf 'Neuer Bearbeitungsschritt',oder durch einen Linksklick auf den Plus-Button in der rechten unteren Ecke, gefolgt von einem Linksklick auf 'Neuer Bearbeitungsschritt'.Im ersten Teil können Name, Kürzel sowie Start- und Endzustand gewählt werden. Im zweiten Teil definiert man Losgröße, Zeit- und Geldkosten sowie Zeit und Geldrüstkosten des Bearbeitungsschrittes. Alle Felder sind Pflichtfelder, es müssen also alle Felder ausgefüllt sein, um den 'erstellen'-Button betätigen zu können.\\
      Des Weiteren kann man Bearbeitungsschritte auch mittels drag-and-drop erstellen. Dafür bewegt man den Mauszeiger über den Zustand, der als Startzustand dienen soll, klickt auf das 'Neuer Bearbeitungsschritt'-Icon und zieht die entstehende Linie zum gewünschten Endzustand. Anschließend öffnet sich der zweite Teil des 'Neuer Bearbeitungsschritt'-Menüs, in dem dann die übrigen Eigenschaften  nachträglich eingefügt werden müssen.\\
      Mit einem Linksklick auf einen bereits erstellten Bearbeitungsschritt öffnet sich das 'Bearbeitungsschritt bearbeiten'-Menü. Dieses ist aufgebaut wie das 'Neuer Bearbeitungsschritt'-Menü. Es können hier alle Eigenschaften des angeklickten Bearbeitungsschritt geändert werden.
  \item \textbf{ Graph optimieren }
    \\\
      Um den besten Weg von einem der Startzustände zum Endzustand zu finden, klickt man auf 'Graph optimieren' unter 'Gesamtkosten' (um nach Kosten zu optimieren) oder unter 'Gesamtzeit' (um nach Zeit zu optimieren). Die Start- und Endzustände werden in diesem Fall automatisch gewählt. Nach Betätigen des Buttons verschwindet dieser und wird durch die Gesamtkosten bzw. die Gesamtzeit des besten Weges im Graphen ersetzt. Darüber hinaus wird dieser Weg Orange markiert. Abhängig davon, ob nach Kosten oder Zeit optimiert wurde wird das entsprechende Icon sowie das Wort ('Gesamtkosten' oder 'Gesamtzeit') ebenfalls Orange markiert.\\
      Möchte man die Parameter der Optimierung eigenhändig bearbeiten, lässt sich das entsprechende Menü öffnen, indem man entweder auf eines der Zahnräder (neben 'Gesamtkosten' oder 'Gesamtzeit') klickt oder auf 'Einstellungen' am oberen Rand klickt und auf den Reiter 'Optimierung' wechselt. In diesem Menü lassen sich die Start- und Endzustände auswählen, ob nach Kosten oder Zeit optimiert werden soll sowie die Anzahl der alternativen Wege, die ausgegeben werden sollen. (Diese Wege sind unter den Einstellungen aufgelistet) Auch in diesem Menü kann man die Optimierung starten, indem man auf 'Optimierung starten' (unten im Einstellungsmenü) klickt. Sind Start- und Endzustände nicht ausgewählt, werden diese automatisch bestimmt, die Optimierungsart (Zeit/Kosten) muss allerdings ausgewählt sein. (Alternativ dient der Button 'anwenden' zum übernehmen der Einstellungen und 'schließen' zum verwerfen der Einstellungen. In beiden Fällen schließt sich das Menü.)\\
      Nach klicken auf den'Optimierung starten'-Button (im Einstellungsmenü) wird dieser ersetzt durch die besten Wege im Graphen (begrenzt durch die gewünschte Anzahl der Alternativwege). Durch Klicken auf einen dieser, erhält man Einblick in die einzelnen Zustände und Bearbeitungsschritte des Weges, durch Klicken auf den Kreis links vom angezeigten Weg, wird dieser anstelle des besten Weges orange markiert.\\
      Sobald etwas am Graph geändert wird, erscheinen die 'Graph optimieren' bzw. 'Optimierung starten' Buttons wieder. Zuvor lässt sich der Graph durch Klicken auf das Wiederholen-Icon (neben 'Gesammtzeit' oder 'Gesamtkosten' im Graphenfenster) erneut optimieren.
  \item \textbf{ Einstellungen }
    \\\
      Mit einem Linksklick auf 'Einstellungen' am oberen Rand öffnet sich das Einstellungsfenster. Im Reiter 'Graph' lassen sich Einstellungen bezüglich der Darstellung des Graphs, wie zum Beispiel die Einheiten oder die angezeigten Details der Verknüpfungen (Bearbeitungsschritte), vornehmen. Außerdem lässt sich hier ein Raster aktivieren, auf dem die Knoten dann gebunden sind.\\
      Unter dem Reiter 'Benutzer' lassen sich Benutzername und Passwort ändern sowie der Account löschen. Beim Ändern des Benutzernamens wird in der Datenbank der Autor jedes mit diesem Account erstellten Graphen ebenfalls geändert. Beim Löschen des Accounts werden auch alle mit diesem Account erstellten Graphen gelöscht. Zum Löschen des Accounts oder Ändern des Passworts wird das Passwort benötigt. Unter dem Reiter 'Hilfe' kann dieses 
buch aufgerufen werden.
  \item \textbf{ Offline Speicher }
    \\\
      Über den 'Export'-Button am oberen Rand lässt sich der Graph in den Formaten .json, .png oder .jpg exportieren und lokal speichern. Über den 'Import'-Button lässt sich ein in .json exportierter Graph wieder ins Programm laden.
  \item \textbf{ Online Speicher }
    \\\
      Über den Button 'Datenbank' am oberen Rand öffnet sich das Datenbank-Fenster. Hier können erstellte Graphen online gespeichert bzw. wieder geladen werden. Dabei können Accounts mit der Rolle 'Student' lediglich auf ihre eigenen Graphen zugreifen. Accounts mit der Rolle 'Admin' haben Zugriff auf alle Graphen.
\end{itemize}


\subsection{Installationsanleitung}
\subsubsection{Clientseitige Installation für Anwender}
{\small Autor: Erik Heldt}

VarG ist eine plattformunabhängige Webanwendung, das heißt man muss nichts lokal auf seinem PC installieren, um sie zu benutzen. Alles was man benötigt, ist ein moderner Web-Browser und eine Internetverbindung (Browser-Empfehlung: Google Chrome oder Firefox). Öffne den Browser und gib in der URL-Leiste \url{https://sam.imn.htwk-leipzig.de} ein. Nun befindest du dich im Home-Menü von VarG und kannst loslegen!

\subsubsection{Clientseitige Installation für Entwickler}
{\small Autor: Erik Heldt}

Um die Anwendung im Entwicklungszustand ausführen zu können, musst du Node.js und npm auf deinem PC installieren. Wie das geht erfährst du hier: \url{https://www.npmjs.com/get-npm}.
\\Node.js ist eine JavaScript-Entwicklungsumgebung, die benötigt wird, um die Anwendung samt der genutzten Frameworks und Bibliotheken kompilieren und ausführen zu können. Node Package Manager, oder kurz npm, wird mit Node.js mitgeliefert und verwaltet alle installierten Pakete, die beim Bauen des Programms verwendet werden.

Weiterhin wird das Versionsmanagement-Tool Git benötigt. Den Download dafür gibt es hier: \url{https://git-scm.com/downloads} bzw. für Windows-Nutzer wird die Git-Bash empfohlen: \url{https://gitforwindows.org}.
\\
\\Sind diese Tools nun installiert, musst du dir das VarG-Repository von GitLab auf deinen PC herunterladen bzw. ''klonen''. Um vollständigen Zugriff auf dieses Repository zu haben, musst du im GitLab dem Projekt zugeordnet sein. Besitzt du also die entsprechenden Rechte, navigiere im Terminal in einen Ordner auf deinem Rechner, in dem du das Projekt speichern willst, und gib dort den Befehl \begin{verbatim}git clone https://gitlab.imn.htwk-leipzig.de/weicker/varg.git\end{verbatim} ein. Warte, bis das Herunterladen abgeschlossen ist, und öffne dann den neu erschienenen Ordner ''varg'' in einem Code-Editor (empfohlen wird Visual Studio Code).
\\
\\Es sollten dort mehrere Ordner zu sehen sein, unter anderem der ''code''-Ordner. Darin ist der gesamte Quellcode des Projekts enthalten. Du solltest dich also zur Ausführung des Programms immer in diesem Ordner aufhalten. Um nun den Code zu kompilieren und als Entwicklungsversion auszuführen, folge bitte diesem Tutorial: \href[pdfnewwindow=true]{file:zuarbeiten/InstallationVarg.pdf}{VarG-Installation von Linus Herterich} (klickbarer Link, Anleitung nur für VSCode).\newpage

\subsubsection{Serverseitige Installation}
{\small Autor: Linus Herterich}
\\
\textbf{Login-Daten}
\\Im folgenden werden die Login Daten für den aktuellen Server, welcher unter der Adresse \url{https://sam.imn.htwk-leipzig.de} 
erreichbar ist, genannt. Die Installationsanleitung ist aber auch so formuliert, dass sie auf anderen Servern 
nachgestellt werden kann.
\\SSH-Befehl (für z.B. Git-Bash), um auf den Server per Remote zuzugreifen (nur im HTWK Netz oder per HTWK-VPN möglich):
\begin{verbatim}
  ssh root@sam.imn.htwk-leipzig.de
\end{verbatim}
Login-Daten für SSH:
\begin{verbatim}
  User: 'root' | Passwort: 'zuwinket3771{Harne'
\end{verbatim}
Login-Daten für die MySQL Datenbank:
\begin{verbatim}
  User: 'root' | Passwort: 'l_GD6P67+V' | Datenbank: 'vargdb'
\end{verbatim}
\textbf{Webserver und SSL}
\\Für die serverseitige Installation des Projekts wird ein SSL-zertifizierter Webserver benötigt.
In unserem Fall wurde Apache 2.4 verwendet, um das Frontend live zu schalten. Es sind aber auch andere
gängige HTTP Webserver möglich. Wenn andere Webservertechnologien verwendet werden, muss folgende
Anleitung beachtet werden, damit das Vue-Routing funktioniert: \url{https://router.vuejs.org/guide/essentials/history-mode.html}.
Für die Installation eines SSL-Zertifikats wurde der certbot ( \url{https://certbot.eff.org/} ) verwendet. 
\\ACHTUNG: Das SSL-Zertifikat muss im Sommer 2021 verlängert werden, damit die Webanwendung weiter problemlos funktioniert 
(Anleitung hierfür findet sich auf der certbot Webseite).
\\Am Ende der Datei \begin{verbatim}
  varg/docker/node.js/api.js
\end{verbatim}
müssen die SSL '.perm' Zertifikate verlinkt werden, damit die API auch auf die SSL-Zertifikate zugreifen kann
und somit eine verschlüsselte Kommunikation zwischen Frontend und Backend stattfinden kann.
Derzeit sind bereits die richtigen Pfade eingetragen.
\\\\
\textbf{Klonen des Projekts}
\\Sind die Grundvoraussetzungen gegebenen, kann das Projekt auf dem Webserver geklont werden.
Hierzu muss sich zunächst auf dem Server eingeloggt werden (Anmeldedaten siehe oben). Als nächstes muss mit den Befehl
\begin{verbatim}
  cd /var/www/html
\end{verbatim} 
in das Standard Webserver-Verzeichnis von Apache navigiert werden. Dort wird anschließend das Projekt mit dem Befehl
\begin{verbatim}
  git clone https://gitlab.imn.htwk-leipzig.de/weicker/varg.git
\end{verbatim} 
geklont. Nun müssen GitLab-Nutzername und -Passwort eingegeben werden, damit das Projekt in den Ordner '/varg' geklont werden kann.
Will man dies umgehen, kann auch ein SSH-Schlüssel generiert werden und bei einem GitLab Account hinterlegt werden (weitere Details:
\url{https://docs.gitlab.com/ee/ssh/})
\\Nun wurde ein Ordner mit folgendem Pfad angelegt, indem die Projekt-Daten sind:
\begin{verbatim}
  /var/www/html/varg
\end{verbatim}
Navigiert man per 'cd' in den Ordner, so kann dort das Projekt aktualisiert (mit 'git pull') oder andere Branches ausgewählt werden (mit 'git checkout ...').
\\\\
\textbf{Unterschiede zwischen Entwicklungsversion und Produktionsversion}
\\Im Code sind einige Stellen zu ändern, damit das Projekt auf dem Server lauffähig ist. Am besten wird hierfür ein neuer Branch erstellt, welcher zwar auf dem 
aktuellen Projektstand ist, aber die Server-Änderungen beinhaltet. Dieser Branch wird anschließend auf dem Server gepullt (wir nannten diesen Branch immer 'prod').
\\Bei den Änderungen handelt es sich zum einen um IP-Adressen, die zum API-Docker-Container zeigen, aber auf der Live Version zur Live-API-Adresse zeigen müssen. 
\begin{itemize}
\item Mit dem 'Suchen und Ersetzen'-Tool im IDE (in VSCode: View -> Search) müssen all diese IP-Adressen ausgetauscht werden, die für die HTTP-Requests verwendet werden:
\begin{verbatim}
  'http://192.168.99.101:1110' 
  (Kann auch abweichen. Bitte selbst in den Dateien (bspw. SettingsAccount.vue)
   überprüfen, an welche Domain alle Axios-Requests gehen)
\end{verbatim}
in allen Dateien zur Live-API Adresse ändern: 
\begin{verbatim}
  'https://sam.imn.htwk-leipzig.de:7070'
\end{verbatim}
Wichtig: Die Endungen (z.B. .../VarG/graph/meta) dürfen nicht verändert werden und müssen an der Adresse angehängt bleiben (außer man ändert auch die Express-Routen in der API).

\item Des Weiteren müssen die Datenbank-Zugangsdaten in folgender Datei geändert werden:
\begin{verbatim}
  /var/www/html/varg/docker/node.js/api.js
\end{verbatim}
Die Zugangsdaten sind am Anfang in einer Konstanten (namens config) abgespeichert und müssen auf folgende Daten geändert werden:
\begin{verbatim}
  host: 'localhost',
  user: 'root',
  password: 'l_GD6P67+V',
  database: 'vargdb'
\end{verbatim}
\item Weiterhin muss in api.js folgende Zeile geändert werden:
\begin{verbatim}
  res.header("Access-Control-Allow-Origin", "http://localhost:8080");
\end{verbatim}
zu
\begin{verbatim}
  res.header("Access-Control-Allow-Origin", "https://sam.imn.htwk-leipzig.de");
\end{verbatim}\newpage
\item Außerdem muss am Ende der Datei der Code-Block
\begin{verbatim}
  api.listen(8080, () => {
    console.log('API listens to 8080');
  });
\end{verbatim}
durch folgenden Code ersetzt werden:
\begin{verbatim}
  https
    .createServer(
      {
        key: fs.readFileSync("/etc/letsencrypt/live/sam.imn.htwk-leipzig.de/privkey.pem"),
        cert: fs.readFileSync("/etc/letsencrypt/live/sam.imn.htwk-leipzig.de/cert.pem"),
      },
      api
    )
    .listen(7070);
\end{verbatim}
Damit dies funktioniert müssen eventuell am Anfang der Datei folgende Module importiert werden:
\begin{verbatim}
  const express = require("express");
  const mysql_driver = require("mysql");
  const fs = require("fs");
  const https = require("https");
\end{verbatim}
\end{itemize}
Wie bereits erwähnt sollten all diese Änderungen auf einem parallelen Branch durchgeführt und anschließend gepusht werden.
Im Anschluss kann das Projekt mit den Live-Änderungen auf dem Server gepullt werden.
\\\\
\textbf{Installation des Produktions-Frontend}
\\Sobald alle nötigen Produktions-Anpassungen durchgeführt wurden und das Projekt im richtigen Ordner geklont wurde, 
kann mit npm das Projekt kompiliert werden. Zunächst muss in folgenden Ordner navigiert werden:
\begin{verbatim}
  cd /var/www/html/varg/code
\end{verbatim}
anschließend wird folgender Befehl ausgeführt (Voraussetzung ist eine LTS-Version von npm auf dem Web-Server):
\begin{verbatim}
  npm install
\end{verbatim}
und im Anschluss:
\begin{verbatim}
  npm run build
\end{verbatim}\newpage
\noindent Nun wurden die kompilierten Dateien in den Ordner
\begin{verbatim}
  /var/www/html/varg/code/dist
\end{verbatim}
abgelegt. Apache (oder jede andere HTML Server Technologie) muss so konfiguriert werden, 
dass der DocumentRoot in diesen Ordner zeigt (auf dem aktuellen Server ist das bereits eingestellt). Wird mit Apache gearbeitet, muss zudem sichergestellt sein, dass 'mod rewrite' aktiviert ist
(siehe \url{https://wiki.ubuntuusers.de/Apache/mod_rewrite/} ). Sonst wird die '.htaccess' Datei im '/dist' Ordner nicht richtig gelesen und 
die Navigation zwischen einzelnen Seiten funktioniert nicht richtig.
\\Ist alles problemlos abgelaufen, sollte nun das Frontend unter der URL \url{https://sam.imn.htwk-leipzig.de} erreichbar sein.
\\\\
\textbf{Installation MySQL}
\\Auf dem Webserver muss eine aktuelle Version von MySQL laufen. Es kann folgende Anleitung zur Installation verwendet werden:
\url{https://wiki.ubuntuusers.de/MySQL/} . Es ist darauf zu achten, dass als root Passwort \begin{verbatim}'l_GD6P67+V'\end{verbatim} gewählt wird. 
Ansonsten muss ein anderes Passwort in der 'api.js' Datei (siehe oben) eingetragen werden.
\\Sobald MySQL installiert ist, kann über das 'Adminer'-Tool, welches beim Frontend unter der URL \url{https://sam.imn.htwk-leipzig.de/adminer.php} 
erreichbar ist, eine Datenbank mit dem Namen 'vargdb' angelegt werden. Um die initialen Tabellen (evtl. mit Beispieldaten) anzulegen, kann im 
Anschluss folgende Datei in die Datenbank per Adminer importiert werden:
\begin{verbatim}
  varg/docker/mysql/dump.sql
\end{verbatim} 
Nun ist die Datenbank eingerichtet und kann über die API angesteuert werden.
\\\\Falls der aktuelle Server weitergenutzt wird, ist bereits MySQL mit der aktuellen Datenbank installiert. Um auf die MySQL Datenbank per
Terminal zuzugreifen, muss sich per SSH eingeloggt werden und anschließend folgender Befehl eingegeben werden:
\begin{verbatim}
  mysql -uroot -p
\end{verbatim}
Nun wird das MySQL passwort gefordert. Das aktuelle Passwort ist:
\begin{verbatim}
  l_GD6P67+V
\end{verbatim}
\textbf{API-Server starten}
\\Der API-Server basiert auf der 'Node.js' Technologie. Damit die API funktioniert, muss Node.js auf dem Server installiert sein.
Die API kann dann mit dem Befehl:
\begin{verbatim}
  node /var/www/html/varg/docker/node.js/api.js
\end{verbatim}
gestartet werden. Man sieht nun den Log der API. Bei jeder Anfrage wird nun eine Zeile ausgegeben. Wenn es Probleme mit der Verbindung
zur MySQL Datenbank gibt, werden diese hier angezeigt.
\\\\
\textbf{API-Server dauerhaft laufen lassen}
\\Ein Problem ist, dass die API nur läuft, solange auch das Terminal geöffnet ist, in dem der Befehl aufgerufen wurde.
Da aber die API immer laufen muss, kann die Technologie 'forever' verwendet werden (\url{https://www.npmjs.com/package/forever}).\\
Mit folgendem Befehl kann die API dauerhaft gestartet werden:
\begin{verbatim}
  forever start -o out.log -e err.log /var/www/html/varg/docker/node.js/api.js
\end{verbatim}
Für 'out.log' und 'err.log' können auch andere Namen oder Pfade verwendet werden. Es handelt sich hierbei um den Output bzw. Error-Meldungen,
die ansonsten über das Terminal ausgegeben worden wären.
\\Mit folgendem Befehl kann nun angezeigt werden, ob die API (noch) mit forever läuft:
\begin{verbatim}
  forever list
\end{verbatim}
Die dort angegebene Liste beinhaltet neben dem Namen der ausgeführten API-Datei auch eine ID. Diese kann verwendet werden, um mit folgendem
Befehl die API zu stoppen:
\begin{verbatim}
  forever stop [ID]
\end{verbatim}
\textbf{Projekt Live (Produktions-Version) aktualisieren}
\\Wenn neue Features auf dem Server installiert werden sollen, so muss nach der Entwicklung zunächst der neuentwickelte Branch auf einem neuen (Produktions-)Branch
wie oben beschrieben angepasst werden (IPs austauschen etc.). Dieser geänderte Branch wird anschließend per 'git pull'
im Verzeichnis '/var/www/html/varg/' auf dem Web-Server heruntergeladen.
\\Als nächstes kann die neue Version kompiliert werden (siehe Installation Produktions-Frontend).
\\Wenn Änderungen an der API-Logik vorgenommen wurden, so muss die API mit dem ''forever stop'' Befehl gestoppt und anschließend neu gestartet werden.
\\Falls an der MySQL Datenbank etwas grundlegendes geändert wurde, so kann die Datenbank mit dem Adminer Tool (\url{https://sam.imn.htwk-leipzig.de/adminer.php})
bearbeitet werden.
\\\\\textbf{Vergessenes Passwort eines Users ändern}
\\Wir verwenden eine programmeigene User-Datenbank zur Verwaltung des Logins und der damit verbundenen User-Accounts (Tabelle 'userreg' in der MySQL DB). Da diese sehr simpel implementiert ist und keine E-Mail zur Anmeldung erforderlich ist, konnten wir keine Frontend-Funktionalität für 'Passwort vergessen' einbauen (Passwort ändern ist natürlich möglich, aber dafür wird das derzeitige Passwort benötigt). Wenn ein User sein Passwort vergessen sollte, muss er sich an einen Backend-Admin mit Zugriff auf 'Adminer' wenden. Im Folgenden wird beschrieben wie der Admin das Passwort für den User manuell ändern kann:
\begin{enumerate}
  \item Vergewissern, dass es sich wirklich um den Account des Users handelt und er nicht versucht, sich Zugriff auf einen anderen Account zu verschaffen
  \item In Adminer einloggen (Login-Daten siehe oben)
  \item Links auf ''SQL-Kommando'' klicken
  \item In dem Fenster folgenden Befehl eingeben:
           \begin{verbatim}
  SET @userName = "...";
  SET @password = "...";
  UPDATE userreg
    SET password = AES_ENCRYPT(@password, UNHEX(SHA2(@password, 512)))
    WHERE userName = @userName;
           \end{verbatim}
  \item In der 1. Zeile die drei Punkte ersetzen durch den (exakten!) Namen des Users, dessen Passwort geändert werden soll (kann aus der userreg-Tabelle kopiert werden)
  \item In der 2. Zeile die drei Punkte ersetzen durch ein neues temporäres Passwort, welches der User für seinen nächsten Login benutzen kann
  \item Alles andere so lassen und auf 'Ausführen' klicken - wenn es geklappt hat sollte mindestens eine grüne Nachricht mit dem Text 'Abfrage ausgeführt, 1 Datensatz betroffen.' erscheinen
  \item Dem User mitteilen, dass er nach dem ersten Login mit dem neuen Passwort bitte sofort in seine Account-Einstellungen gehen und das temporäre Passwort in sein eigenes ändern soll
\end{enumerate}

\subsection{Software-Lizenz}
{\small Autor: Linus Herterich}

Im Folgenden werden die verwendeten Bibliotkeken und deren Lizenz aufgelistet:
\begin{itemize}
  \item Vue.js -- MIT License: Copyright (c) 2013-present Yuxi Evan You
  \item vuetify -- MIT License: Copyright (c) 2016-2020 John Jeremy Leider
  \item cytoscape -- MIT License: Copyright (c) 2016-2020, The Cytoscape Consortium
  \item cytoscape-node-html-label -- MIT License: Copyright (c) 2017 Kalugin Sergey
  \item cypress -- MIT Licence: Copyright (c) 2015 Cypress.io, LLC
  \item jest -- MIT Licence: Copyright (c) Facebook, Inc. and its affiliates
  \item axios -- MIT License: Copyright (c) 2014-present Matt Zabriskie
  \item darkmode.js -- MIT License: Copyright (c) 2018 Nickolas
  \item file-saver.js -- MIT License: Copyright (c) 2016 Eli Grey
  \item file-saver.js -- MIT License: Copyright (c) 2016 Eli Grey
  \item Node.js -- MIT License
  \item express.js -- MIT License: Copyright (c) 2017 StrongLoop, IBM, and other expressjs.com contributors
  \item MySQL -- GPLv2 License: Copyright (c) 2020, Oracle Corporation and/or its affiliates
  \item Docker -- Apache License 2.0: Copyright (c) 2020 Docker Inc.
  \item Adminer -- Apache License 2.0 or GPLv2
  \item LaTeX -- LaTeX Project Public License (LPPL)
  \item https.js -- MIT License
  \item forever -- MIT License
  \item Apache-HTTP-Server -- Apache License 2.0: Copyright (c) 1997-2020 The Apache Software Foundation
\end{itemize}

Da ausschließlich die MIT Lizenz verwendet wurde, werden wir auch die Software ''VarG'' unter der
MIT-Lizenz veröffentlichen.
\\ \\
VarG-Lizenz:
\\
\\ Copyright (c) 2020 HTWK-Leipzig
\\ Permission is hereby granted, free of charge, to any person obtaining a copy of this software and associated documentation files (the ''Software''), to deal in the Software without restriction, including without limitation the rights to use, copy, modify, merge, publish, distribute, sublicense, and/or sell copies of the Software, and to permit persons to whom the Software is furnished to do so, subject to the following conditions:
\\ The above copyright notice and this permission notice shall be included in all copies or substantial portions of the Software.
\\THE SOFTWARE IS PROVIDED ''AS IS'', WITHOUT WARRANTY OF ANY KIND, EXPRESS OR IMPLIED, INCLUDING BUT NOT LIMITED TO THE WARRANTIES OF MERCHANTABILITY, FITNESS FOR A PARTICULAR PURPOSE AND NONINFRINGEMENT. IN NO EVENT SHALL THE AUTHORS OR COPYRIGHT HOLDERS BE LIABLE FOR ANY CLAIM, DAMAGES OR OTHER LIABILITY, WHETHER IN AN ACTION OF CONTRACT, TORT OR OTHERWISE, ARISING FROM, OUT OF OR IN CONNECTION WITH THE SOFTWARE OR THE USE OR OTHER DEALINGS IN THE SOFTWARE.


\section{Projektabschluss}

\subsection{Protokoll der Abnahme und Inbetriebnahme beim Kunden}
{\small Autor: Julius Jolig}

Am 08.07.20 konnte das Projekt VarG an den Kunden Prof. Martin Gürtler übergeben werden. Dabei wurde die Dokumentation und die benötigten Zugangsdaten zur Bedingung und Wartung der Software überreicht. Während der Übergabe wurden alle in der Anwendung zur Verfügung stehenden Komponenten des Frontend und Backends, sowie der UI vorgeführt und erläutert. Anschließend folgte ein Gesprächsaustausch über die Weiterführung des Projektes. Dabei wurde auch auf das Hosting der Anwendung beim FSR eingegangen. Abschließend erhielten wir ein gesamtheitliches Feedback von unserem Kunden. 

\subsection{Präsentation auf der Messe}
{\small Autor: Erik Heldt}
\\Die Messe fand in diesem Jahr wegen Corona etwas anders als gewohnt als große Online-Konferenz statt.
\\Wir haben eine Präsentation vorbereitet (Link: \href[pdfnewwindow=true]{file:zuarbeiten/VarG_Messepräsentation.pdf}{VarG Messepräsentation}),
um einen groben Überblick über die Features und die Architektur unseres Projekts zu geben. Außerdem haben wir die Anwendung auf dem HTWK-Server öffentlich zugänglich gemacht
und einen Messe-Login erstellt, um unseren Gästen eine Live-Demo bereitzustellen, mit der sie selbst während der Vorstellung interagieren konnten.
\\\\Unsere Präsentation verlief sehr gut. Wir haben viel Lob sowohl für das Design als auch die vielen Funktionalitäten erhalten.
Ein Gast hat uns sogar gefragt, ob wir ein existierendes Projekt übernommen haben, weil es so ein umfangreiches Programm ist.
Es gab auch noch weitere spezifische Fragen zu verschiedenen Funktionalitäten, wie z.B. dem Login, der Datenbank, der Optimierung und dem Server-Hosting,
die wir alle so gut wie möglich beantwortet haben.
\\\\Zusammengefasst war die Messe also ein großer Erfolg für VarG.

\subsection{Abschließende Einschätzung durch Product-Owner}
{\small Autor: Manuel Eckert}

Aus der Sicht des PO war das Projekt "VarG "\ ein voller Erfolg! Unser Kunde war mit dem Produkt sehr zufrieden und begeistert über die Anzahl der umgesetzten Features. \\
Alles in Allem war das Softwareprojekt, für das gesamte Team, ein Lernprozess. Anfangs waren die Planning- und Review-Treffen sehr träge und einseitig. Alle Teammitglieder mussten sich erst in Ihre Rolle einfinden. Eine merkbare Änderung wurde aber schon im zweiten Sprint festgestellt. Dies hat uns einen großen Schub für das Projekt gegeben. So hat man mit Beendigung eines jeden Sprints, eine deutliche Steigerung in der Produktivität und Qualität der Ticketbearbeitung gespürt. In den letzten Sprints vor Projektabschluss haben die Entwickler viel schneller verstanden, was die einzelnen User-Stories umfassen und was es braucht, um diese erfolgreich abzuschließen. Dies hat meine Arbeit als PO sehr vereinfacht und ich konnte meine Energie auf andere Arbeitsbereiche im Projekt konzentrieren. \\
Rückblickend war es exzellent, dass sich einzelne Subteams innerhalb des Entwicklerteams geformt haben, welche sich auf einzelne Bereiche innerhalb des Projektes fokussierten. Der Austausch zwischen den einzelnen Teams war aber trotzdem stets durch die Daily Treffen gegeben. Damit gab es Experten in den unterschiedlichen Bereichen, welches sich sehr positiv auf des Projekt ausgewirkt hat. \\ 
Die Kommunikation mit unserem Kunden verlief meist sehr gut. Anfangs war es nicht einfach ein gemeinsames und gleiches Verständnis für die Anforderungen zu entwickeln. Durch genauere und stärkere Auseinandersetzung mit dem Thema ist es uns allerdings gelungen die Anforderungen des Kunden zu treffen. Eine große Hilfe war dabei auch, dass wir, zum Start des SS2020, das Produktinkrement nach jedem Sprint auf einen Webserver gespielt haben. Somit hatte der Kunde jederzeit die Möglichkeit das Produktinkrement zu testen. Dies ermöglichte dem Kunden genaueres Feedback zu geben, was sich dann auch in der Qualität und Präzision des Produktes wiederspiegelt. Über die gesamte Laufzeit des Projektes ist festzuhalten, dass sich die Anforderungen des Kunden entwickelten. \\
Das Projekt war ein großer Zugewinn für jeden einzelnen von uns. Eine so intensive und freie Auseinandersetzung mit einem Produkt ermöglicht es, wichtige neue Fähigkeiten für das Arbeiten als Teil eines Teams zu entwickeln. Jeder in seiner Rolle konnte einen erheblichen Mehrwert aus diesem Projekt ziehen. \\
 

\subsection{Abschließende Einschätzung durch Softwarearchitekt}
{\small Autor: Julius Jolig}

Nach einem verhaltenem Start in das Projekt zeigten sich die Teammitglied engagiert und lieferten gute Ergebnissen ab. Die anfänglichen Probleme wie Unsicherheiten bei der Verwendung des Frameworks oder beim Mergen konnten mit der Zeit behoben werden. \\
Ab der Mitte des Projektes war das Team gut eingespielt. Das Paarprogrammieren hat hierzu auch einen großen Beitrag geleistet. Dadurch konnte auch die Qualität des Codes verbessert werden. Auch das unterschiedliche Niveau der Programmierfähigkeiten, dass zu Projektbegin noch sehr durchwachsen war, hat sich im Verlaufe der Zeit angeglichen, da die Teammitglieder voneinander lernen konnten. \\
Ingesamt war es ein gelungenes Projekt, dass eine Software hervorgebracht hat, die sich definitiv sehen lassen kann. 


\subsection{Abschließende Einschätzung durch Team-Manager}
{\small Autor: Alex Hofmann}

Das Team zeigte rückblickend eine engagierte und couragierte Leistung über das gesamte Projekt hinweg, so dass auch der Ausfall zweier Teammitglieder locker kompensiert werden konnte. \\
Der Start verlief noch etwas holprig, da eine gewisse Eingewöhnung sowohl in den Scrum-Prozess als auch auf technischer Seite nötig war, danach lief es aber umso besser. \\
Durch die Aufteilung des Teams in kleinere Sub-Teams hatte jedes Mitglied über das gesamte Projekt hinweg seinen eigenen klar definierten Arbeitsbereich. Diese Teamstruktur hat dem Projekt im Arbeitsablauf in meinen Augen einen deutlichen Aufschwung gegeben. \\
Auch das präsenzfreie Semester stellte keine Hürde dar, ganz im Gegenteil. Der Einsatz des Tools Discord belebte Dank der Möglichkeit des Screensharings die Meetings und brachte neue Möglichkeiten in Sachen Pair-Programming. \\
Auch das Klima innerhalb des Teams war stets angenehm, so dass es nie zu Konflikten untereinander kam bzw. Kritik konstruktiv diskutiert wurde. \\
Es hat Spaß gemacht, mit Team VarG zusammenzuarbeiten und auch das Resultat kann sich definitiv zeigen lassen!

\end{document}
