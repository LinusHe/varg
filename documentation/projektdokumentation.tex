\documentclass[twoside]{report}

% ------
% Umlaute
\usepackage{ifluatex,ifxetex}
\ifluatex
  \usepackage{fontspec}
\else
  \ifxetex
    \usepackage{fontspec}
  \else
    \usepackage{selinput}
    \SelectInputMappings{
      adieresis={ä},
      germandbls={ß},
    }
    \usepackage[T1]{fontenc}
    %\usepackage{textcomp}% optional
    %\usepackage{lmodern}
  \fi
\fi

% ------
% Paper auf Deutsch
\usepackage[ngerman]{babel}



% ------
% Page layout
\usepackage[hmarginratio=1:1,top=32mm,columnsep=20pt]{geometry}
\usepackage[font=it]{caption}
\usepackage{paralist}
%\usepackage{multicol}


% ------
% Abstract
\usepackage{abstract}
	\renewcommand{\abstractnamefont}{\normalfont\bfseries}
	\renewcommand{\abstracttextfont}{\normalfont\small\itshape}


% ------
% Titling (section/subsection)
\usepackage{titlesec}
\renewcommand\thesection{\Roman{section}}
\titleformat{\section}[block]{\Large\scshape\bfseries}{\thesection.}{1em}{}
\setcounter{secnumdepth}{3}

% ------
% Tabellen über Seitenumbrüche hinweg
\usepackage{longtable}

% ------
% Header/footer
\usepackage{fancyhdr}
	\pagestyle{fancy}
	\fancyhead{}
	\fancyfoot{}
	\fancyhead[C]{Projektdokumentation $\bullet$ VarG $\bullet$ WS19/20$+$SS20}
	\fancyfoot[RO,LE]{}


% ------
% Clickable URLs (optional)
% \usepackage{hyperref}

% ------
% Literaturverweise mit Bibtex einbinden
\usepackage[authoryear,sectionbib,round]{natbib}

% ------
% Bilder laden
\usepackage[pdftex]{graphicx}

% ------
% Maketitle metadata
\title{\vspace{-5mm}%
	\fontsize{24pt}{10pt}\selectfont
	\textbf{Projektdokumentation}
	}	
\author{%
        % alle Autoren hier listen
        % 
	\large
	\textsc{Autor I -- E-Mail} \\[2mm]
	\textsc{Autor II -- E-Mail} \\[2mm]
	\normalsize	HTWK Leipzig 
	}
\date{}



%%%%%%%%%%%%%%%%%%%%%%%%
\begin{document}


% -------
% Titel und Abstract über beide Spalten
%\twocolumn[
%\begin{@twocolumnfalse}

\maketitle
\thispagestyle{fancy}

\tableofcontents

%%%%
%%%% Die Struktur des Dokuments bitte nicht aendern!!!
%%%%

\section{Anforderungsspezifikation}

\subsection{Initiale Kundenvorgaben}
{\small Autor: xxx}

Maecenas sed ultricies felis. Sed imperdiet dictum arcu a egestas.
In sapien ante, ultricies quis pellentesque ut, fringilla id sem. Proin justo libero, dapibus consequat auctor at, euismod et erat. Sed ut ipsum erat, iaculis vehicula lorem. Cras non dolor id libero blandit ornare. Pellentesque luctus fermentum eros ut posuere. Suspendisse rutrum suscipit massa sit amet molestie. Donec suscipit lacinia diam, eu posuere libero rutrum sed. Nam blandit lorem sit amet dolor vestibulum in lacinia purus varius. Ut tortor massa, rhoncus ut auctor eget, vestibulum ut justo.


\subsection{Produktvision}
{\small Autor: Alex Hofmann}
\\

\noindent Product Vision Board: \\
\begin{tabular}{|p{50mm}|p{50mm}|p{50mm}|}
  \hline
  \textbf{Target Group}                                                  & \textbf{Needs}                                                                                                                        & \textbf{Product}                                                                                                                                                                                                 \\
  \hline
  -Maschinenbau-Studenten \newline Maschinenbau-Profs \newline -Lehrende & Vgl. zu händisch: \newline einheitlicher, schneller \newline -plattformunabhängig \newline -Open Source \newline -Einfach zu bedienen & -Webanwendung \newline -Als Graph \newline $\rightarrow$ quasi als Baukasten \newline $\rightarrow$ Kantengewichtung, Bausteine wählbar \newline -Import/Export von Modellen \newline Normalisierung des Graphen \\
  \hline
\end{tabular}
\\

\noindent Die Webanwendung VarG ist entwickelt für Lehrende und Lernende aus dem Maschinenbau Bachelorstudiengang.
Diese erleichtert die einheitliche Erstellung, Bearbeitung, Optimierung sowie Im- bzw. Exportierung von sogenannten Variantenfolgegraphen. Darunter ist eine graphische Übersicht zu verstehen, die die möglichen Varianten eines Produktionsprozesses für ein Werkstück darstellt.



% Das hier ist ein Absatz, der die Grafik in Abbildung~\ref{fig:bild1} detailliert erläutert, erklärt und interpretiert.

% \begin{figure}[b]
%   \centering
%   \includegraphics[width=4.5cm]{bspbild1.png}
%   \caption{Beispiel für ein einspaltiges Bild}
%   \label{fig:bild1}
% \end{figure}


\subsection{Liste der funktionalen Anforderungen}

XXX

%
% soll der Inhalt dieser Subsection in einer separaten Datei
% (z.B. listefunktional.tex) liegen, dann kann dies mit dem
% folgenden Kommando geschehen.
%
% \input{listefunktional}

\subsection{Liste der nicht-funktionalen Anforderungen}
{\small Autor: xxx}

XXX

\subsection{Weitere Zuarbeiten zum Produktvisions-Workshop}

XXX

\subsubsection{Zuarbeit von Autor X}
XXX
\subsubsection{Zuarbeit von Autor Y}
XXX

\subsection{Liste der Kundengespräche mit Ergebnissen}
{\small Autor: xxx}

XXX



\section{Architektur und Entwurf}

\subsection{Zuarbeiten der Teammitglieder}

XXX

\subsection{Entscheidungen des Technologieworkshops}
{\small Autor: xxx}

XXX

\subsection{Überblick über Architektur}
{\small Autor: xxx}

XXX

\subsection{Definierte Schnittstellen}
{\small Autor: xxx}

XXX

\subsection{Liste der Architekturentscheidungen}
{\small Autor: xxx}

XXX (bewusste und unbewusste Entscheidungen mit zeitlicher Einordnung)



\section{Prozess- und Implementationsvorgaben}

\subsection{Definition of Done}
{\small Autor: xxx}

XXX

\subsection{Coding Style}
{\small Autor: xxx}

XXX

\subsection{Zu nutzende Werkzeuge}
{\small Autor: xxx}

XXX

\newpage

%%%%%%%%%%%%
%% Abschnitt mit den Sprints beginnt hier
%%%%%%%%%%%%

\section{Sprint 1}


\subsection{Ziel des Sprints}
{\small Autor: Erik Heldt}

Der erste Sprint des VarG-Projekts lief vom 05.12.2019 bis zum 16.12.2019. Ziel war es, eine fundamentale Struktur und grundlegende Funktionalitäten für die Anwendung zu entwickeln, auf denen man später weiter aufbauen kann. Währenddessen konnte man allgemeine Erfahrungen mit dem Ablauf eines Sprints machen.

\subsection{User-Stories des Sprint-Backlogs}
{\small Autor: Erik Heldt}

\textbf{Grundstruktur}
Die Anwendung sollte zu Beginn ein grundlegendes Fundament aufweisen, damit sich alle Teammitglieder vorstellen können, wie am Ende das Programm aussehen soll. Dazu gehörte zu Beginn das Design der Startseite mit dem VarGraph im Zentrum und der Einbindung von Cytoscape in die Programmstruktur.

\textbf{Datenstruktur für Knoten}
Es sollte mit Hilfe von Cytoscape herausgefunden werden, wie man Knoten im Programmcode hinzufügen und speichern kann. Dafür sollte dann eine Datei im Programm angelegt werden.

\textbf{Knoten zu bestehender Datenstruktur hinzufügen}
Die Anwendung sollte eine einfache Funktionalität zum Erstellen neuer Knoten aka Produktionsschritte erhalten, um sich mit den Cytoscape-Funktionen näher vertraut zu machen. Hier war erstmal noch keine graphische Darstellung in der GUI notwendig, es reichte per Console logs zu testen.

\textbf{Darstellung eines Graphen in Weboberfläche}
In der Anwendung sollte zunächst ein statischer Graph mit Hilfe einer Cytoscape-Datenstruktur sichtbar dargestellt werden, damit man sehen konnte, wie so ein „CytoGraph“ überhaupt aussieht. User-Interaktion war hier noch nicht notwendig.

\textbf{Kanten anlegen}
Zusätzlich zu Knoten sollten auch Kanten zwischen bestehenden Knoten hinzugefügt werden können. Diese Kanten sollten mit verschiedenen Attributen in der Cytoscape-Datenstruktur gespeichert werden.

\textbf{Berechnung verschiedener Eigenschaften}
Anhand der mit den Kanten gespeicherten Attribute sollte eine Funktionalität entwickelt werden, welche die Gesamtkosten (Auswahl von Geld oder Zeit) aller unterschiedlichen Pfade berechnen und anzeigen sollte. Dies war der erste Schritt in Richtung Optimierung, d.h. später sollte diese Funktionalität automatisch den günstigsten Pfad herausfinden und anzeigen.

\subsection{Liste der durchgeführten Meetings}
{\small Autor: Erik Heldt}

\begin{itemize}
	\item Planning - 05.12.2019
	\item Weekly Scrum 1 - 09.12.2019
	\item Weekly Scrum 2 - 12.12.2019
	\item Review - 16.12.2019
	\item Retrospektive - 19.12.2019
\end{itemize}

\subsection{Ergebnisse des Planning-Meetings}
{\small Autor: Erik Heldt}

Im Planning-Meeting erklärten die Projektmanager zu Beginn noch einmal kurz, wie ein Sprint im Allgemeinen abläuft und haben auf die Bedeutsamkeit der Coding Guidelines hingewiesen. Anschließend wurden die ersten User-Stories vom Project Owner vorgestellt und von den Bachelorstudenten per Finger-System in ihrer Komplexität eingeschätzt. Weiterhin wurde festgelegt, dass die Bachelorstudenten während des Sprints die User-Stories selbst in Tasks aufteilen und diese dann bearbeiten sollen.

\subsection{Aufgewendete Arbeitszeit pro Person$+$Arbeitspaket}
{\small Autor: xxx}

\begin{longtable}{|p{4cm}|l|l|l|l|l|}
        \hline
	Arbeitspaket & Person & Start & Ende & h & Artefakt\\
        \hline
	Vue.js "Getting Started" Tutorial durcharbeiten (für alle) & Buchmann, Lennart & 07.12.19 & 07.12.19 & 3 & Tutorial abgeschlossen\\ \hline
	Beispielgraph erstellen & Buxel, Nils & 09.12.19 & 09.12.19 & 1 & index.js\\ \hline
	Kürzesten Weg mit A*-Algorithm berechnen u anzeigen lassen & Buxel, Nils &16.12.19 & 16.12.19 & 1 & index.js\\ \hline
	Funktionen zu Buttons hinzufügen & Gwozdz, Jonas & 14.12.19 & 16.12.19 & 4 & MenuControls.vue\\ \hline
	Task: Einbindung in Vue-Dateistruktur & Heldt, Erik & 15.12.19 & 15.12.19 & 3 & MenuControls.vue, BasicData.js\\ \hline
	Graphenanordnung & Heldt, Erik & 05.12.19 & 05.12.19 & 3 & Graphenanordnung.pdf\\ \hline
	Vue.js "Getting Started" Tutorial durcharbeiten (für alle) & Heldt, Erik & 11.12.19 & 11.12.19 & 2 & Tutorial abgeschlossen\\ \hline
	Funktionen zu Buttons hinzufügen & Henning, Tim & 10.12.19 & 10.12.19 & 2 & MenuControls.vue\\ \hline
	Vue.js "Getting Started" Tutorial durcharbeiten (für alle) & Henning, Tim & 06.12.19 & 06.12.19 & 3 & Tutorial abgeschlossen\\ \hline
	Einbindung von Cytoscape in Vue & Herterich, Linus & 10.12.19 & 10.12.19 & 4 & index.js\\ \hline
	Buttons für Knoten und Kantenerstellung & Herterich, Linus & 13.12.19 & 13.12.19 & 3 & CreateControls.vue\\ \hline
	Knoten zu Graph hinzufügen & Herterich, Linus & 16.12.19 & 16.12.19 & 2,5 & index.js, CreateControls.vue\\ \hline
	Grundstruktur aufbauen & Herterich, Linus & 05.12.19 & 07.12.19 & 9,5 & Vue-Dateistruktur, sämtliche Startkomponenten\\ \hline
	Task: Basic Datenstruktur & Hohlfeld, Julius & 15.12.19 & 15.12.19 & 8 & BasicData.js, MenuControls.vue\\ \hline
      \end{longtable}

\subsection{Konkrete Code-Qualität im Sprint}
{\small Autor: Erik Heldt}

Zu Beginn wurde viel experimentiert und hauptsächlich sollte der Code erstmal ein funktionierendes Programm erzeugen, weswegen weniger auf die Qualität geachtet wurde. Trotzdem wurde sich größtenteils an die Coding Conventions gehalten und bereits einige Kommentare verfasst.

\subsection{Konkrete Test-Überdeckung im Sprint}
{\small Autor: Erik Heldt}

Da der erste Sprint größtenteils nur zur Erstellung einer grundlegenden Datenstruktur und zur Einarbeitung in JavaScript und den genutzten Frameworks bzw. Bibliotheken gedient hat, gab es noch keine Tests.

\subsection{Ergebnisse des Reviews}
{\small Autor: Erik Heldt}

Im ersten Review-Meeting stellten die Bachelorstudenten ihre Ergebnisse aus dem Sprint vor und die Manager gaben ihr Feedback dazu. Da sich die meisten Teammitglieder noch nicht richtig in Vue.js und Cytoscape einarbeiten konnten und teilweise große Schwierigkeiten mit den Frameworks hatten, gab es noch viele offene Aufgaben und nicht jeder hatte etwas vorzuzeigen.
Als erstes stellten Julius H. und Erik die Datenstruktur für die Knoten vor. Weiterhin zeigte Julius, wie ein Knoten in der Anwendung dargestellt wird und dass dieser durch ungeschickte Verschiebung und Skalierung aus der GUI verschwinden kann. Deshalb kamen Vorschläge, zukünftig den Zoom zu limitieren und das grundsätzliche Graph-Layout nochmal zu überarbeiten.
Um allen den Einstieg in die neuen Programmiersprachen und Bibliotheken etwas zu vereinfachen, stellte daraufhin Linus die Grundstruktur vor und erklärte noch einmal genau die einzelnen Elemente in der Dateistruktur. Weiterhin zeigte er, wie man ESLint-Fehler bei der Konsolenausgabe verhindern kann.
Danach wurde zwischen den Managern und den Bachelorstudenten noch die zukünftige Berechnung der kürzesten Wege und die unbearbeiteten User-Stories besprochen und dass diese in den nächsten Sprint mit einfließen werden.
Zum Schluss wurden noch ein paar allgemeine Fragen zum Testen und zu Git geklärt.

\subsection{Ergebnisse der Retrospektive}
{\small Autor: Erik Heldt}

In der Retrospektive konnte jedes Teammitglied vor an die Tafel gehen und verschiedene Aspekte des Sprints mit einem Strich in einer Tabelle bewerten.
Die Bewertung ging ausgeglichen aus. Die Gruppenleistung und das Gesamtergebnis waren gut, aber die Einzelleistungen der meisten Teammitglieder nicht. Viele Aufgaben blieben offen und wurden nicht erledigt, wozu in der Diskussion verschiedene Gründe angeführt wurden. Einerseits war es für die meisten schwer, sich selbst in die neue Programmierumgebung samt den Frameworks und Bibliotheken einzuarbeiten. Andererseits wussten viele nicht, was und wie viel sie machen sollten, was auf die nicht festgelegte Aufgabenzuteilung im Planning und die schlechte Kommunikation im Team während des Sprints zurückgeführt wurde. Letzteres Problem plante man damit zu lösen, in zukünftigen Plannings immer direkt Verantwortliche für bestimmte User-Stories festzulegen und entsprechende Tickets sofort im Anschluss zu erstellen und zuzuweisen.
Beim Thema der Daily Meetings ist man zu dem Schluss gekommen, dass diese wenn möglich immer persönlich bleiben sollten und nur in Ausnahmefällen online z.B. über Discord stattfinden sollten. Weiterhin wurde diskutiert, ob die Zeitspanne zwischen Donnerstag und Montag evtl. zu kurz ist, um schon weitreichende Ergebnisse zu erzielen, da am Wochenende einige Teammitglieder nicht programmieren können. Deshalb sollten die ersten Meetings beim nächsten Sprint stattdessen Montag und Donnerstag stattfinden.
Ein weiterer Themenpunkt war die Organisation im Git. Es wurde festgelegt, dass der Master-Branch während des Sprints unberührt bleiben sollte, da dieser immer lauffähig sein muss. Stattdessen sollte sich jeder seinen eigenen Branch erstellen und diesen nach Abschluss der eigenen Aufgaben auf den neuen Developer-Branch namens "targetbranch" mergen. Am Ende jedes Sprints würde dann der Developer-Branch mit dem Master-Branch gemerged werden.

\subsection{Abschließende Einschätzung des Product-Owners}
{\small Autor: xxx}

XXX

\subsection{Abschließende Einschätzung des Software-Architekten}
{\small Autor: xxx}

XXX

\subsection{Abschließende Einschätzung des Team-Managers}
{\small Autor: xxx}

XXX


\newpage

\section{Sprint 2}


\subsection{Ziel des Sprints}
{\small Autor: Linus Herterich}

Nachdem im ersten Sprint hauptsächlich die Grundstruktur sowie erste Datenstrukturen entworfen wurden,
war es nun wichtig, dass sich das gesamte Team im Sprint 2 mit der Projektstruktur (besonders mit dem Framework Vue)
auseinandersetzt und erste UserStories direkt am Code umsetzt. Zudem blieben einige Tickets noch vom letzten Sprint offen,
welche nun auch bearbeitet werden sollten.

\subsection{User-Stories des Sprint-Backlogs}
{\small Autor: Linus Herterich}

\begin{itemize}
  \item \textbf{Designumsetzung nach Adobe Preview}
        \\\textit{
          Als Benutzer der WebApplikation möchte ich ein ansehnliche und intuitive
          Oberflächengesstaltung haben, damit ich die Applikation gerne verwende.}
  \item \textbf{Authentifizierung eines Nutzers}
        \\\textit{
          Als Nutzer möchte ich mich in die Web Applikation einloggen können,
          damit nicht jeder meine erzeugten Graphen einsehen kann.}
  \item \textbf{Logische verknüpfung zwischen Knoten erstellen}
        \\ (wurde in Sprint 1 nicht abgeschlossen)
        \\\textit{
          Ein Nutzer muss eine Abfolge der Knoten definieren können,
          damit ersichtlich wird welcher Produktionsschritt auf den nächsten folgt}
  \item \textbf{Berechnung der Eingenschaften des Gesamtgraphs}
        \\ (wurde in Sprint 1 nicht abgeschlossen)
        \\\textit{
          Ein Nutzer der Webanwendung VarG muss die berechneten gesamt Eigenschaften
          jedes Zusammenhängendes Pfades ausgeben lassen können um eine Auswahl
          eines Pfades zu treffen.}
  \item \textbf{Datenstruktur Ausarbeiten \& Knoten zu einer vorhandenen Datenstruktur hinzufügen}
        \\ (wurde in Sprint 1 nicht abgeschlossen)
        \\\textit{
          Als Nutzer möchte ich Knoten zu der Datenstruktur hinzufügen können
          um die möglichen Produktionsschritte des Werkstücks überblicken zu können}

\end{itemize}

\subsection{Liste der durchgeführten Meetings}
{\small Autor: Linus Herterich}

\begin{itemize}
  \item 19.12.2019: Planning Meeting
  \item 23.12.2019: Daily Meeting (in Discord)
  \item 28.12.2019: Daily Meeting (in Discord)
  \item 05.01.2020: Review Meeting
  \item 06.01.2020: Retrospektive
\end{itemize}

\subsection{Ergebnisse des Planning-Meetings}
{\small Autor: Linus Herterich}

Neben der Aufgabenverteilung wurde im Planning darüber gesprochen, dass die Arbeitsaufteilung im letzten
Sprint nicht gut geklappt hat. Es wurde anschließend beschlossen im nächsten Sprint die User-Stories direkt
an Studenten zuzuweisen, damit jeder einen Teilbereich hat, den er bearbeiten muss.
\\ Desweiteren wurde eine Änderung im Git angekündigt. In Zukunft müsse der "Master"\--Branch während eines Sprints
immer gleich bleiben und Funktionalitäten werden auf einen "Developer"\--Branch gemerged. Am Ende des Sprints
wird dann der "Developer"\--Branch auf den "Master"\--Branch gemerged. wichtig ist, dass der "Master"\--Branch zu jedem
Zeitpunkt lauffähig ist.
\\ Für den folgenden Sprint wurde beschlossen, die Daily Meetings online (auf einem Discord Server) abzuhalten,
da viele Studenten über die Weihnachtsferien in der Heimat sind und somit ein persönliches wöchentliches treffen
nicht möglich wäre.

\subsection{Aufgewendete Arbeitszeit pro Person$+$Arbeitspaket}
{\small Autor: Linus Herterich}

\begin{longtable}{|p{4cm}|p{2cm}|p{1.2cm}|p{1.2cm}|p{0.7cm}|p{3.8cm}|}
  \hline
  Arbeitspaket                                                          & Person                & Start    & Ende     & h     & Artefakt                                                    \\
  \hline
  UI: Login                                                             & Berger, Matthias      & 22.12.19 & 22.12.19 & 3,5   & Login Funktionalität \& Design                              \\ \hline
  UI: Login                                                             & Buchmann, Lennart     & 22.12.19 & 22.12.19 & 6     & Login Funktionalität \& Design                              \\ \hline
  UI: Grapheneditor                                                     & Gwozdz, Jonas         & 23.12.19 & 04.01.20 & 9     & GraphHeader.vue, Toolbar.vue                                \\ \hline
  Task: Einbindung in Vue-Dateistruktur                                 & Heldt, Erik           & 19.12.19 & 19.12.19 & 0,25  & BasicData.js                                                \\ \hline
  Abrufbaren Knoten in Graph einfügen                                   & Heldt, Erik           & 23.12.19 & 26.12.19 & 3,5   & BasicData.js, TestDatabase.js                               \\ \hline
  Testdatenbank mit Speichern und Laden                                 & Heldt, Erik           & 27.12.19 & 27.12.19 & 3,5   & TestDatabase.js                                             \\ \hline
  Highlighting eines kürzesten Pfades nach Anwendung des A* Algorithmus & Henning, Tim          & 24.12.19 & 03.01.20 & 9     & OptimizeControls.vue, index.js -> Graph Highlighting        \\ \hline
  Protokoll: Meeting 19.12.19                                           & Herterich, Linus      & 19.12.19 & 19.12.19 & 1     & meeting\_19\_12\_19.pdf                                     \\ \hline
  UI: Login                                                             & Herterich, Linus      & 20.12.19 & 20.12.19 & 5     & LoginForm.vue, Login.vue                                    \\ \hline
  UI: Home                                                              & Herterich, Linus      & 23.12.19 & 23.12.19 & 7     & HomeMenu.vue (component), Home.vue (view), Menu.vue (view)  \\ \hline
  UI: Neuer Graph                                                       & Herterich, Linus      & 28.12.19 & 28.12.19 & 1,5   & NewGraph.vue (view), NewGraph.vue (component)               \\ \hline
  UI: Grapheneditor                                                     & Herterich, Linus      & 02.01.20 & 04.01.20 & 11,75 & Graph.vue (view), zahlreiche components                     \\ \hline
  Graph zu Datenstruktur hinzufügen                                     & Hohlfeld, Julius      & 21.12.19 & 23.12.19 & 4     & BasicData.js, TestDatabase.js                               \\ \hline
  Testdatenbank mit Speichern und Laden                                 & Hohlfeld, Julius      & 27.12.19 & 03.01.20 & 8     & BasicData.js, TestDatabase.js, index.js, JSonPersistence.js \\ \hline
  Mergen und Anpassen                                                   & Hohlfeld, Julius      & 04.01.20 & 04.01.20 & 2     & Bugs entfernt \& Mergekonflikte behoben                     \\ \hline
  UI: Datenbank-Import Fenster                                          & Karkoutli, Alaa Aldin & 31.01.20 & 04.01.20 & 12,5  & Database.vue (view), DatabaseForm.vue (component)           \\ \hline
  Kanten zu Graph hinzufügen                                            & Koch, David           & 23.12.20 & 04.01.20 & 10    & Änderungen an index.js, CreateControls.vue (component)      \\ \hline
\end{longtable}

\subsection{Konkrete Code-Qualität im Sprint}
{\small Autor: Linus Herterich}

Es wurde sich größtenteils an die Coding-Guidelines gehalten. An wichtigen Stellen sowie vor jeder Funktion wurden Kommentare
geschrieben. Die Trennung zwischen Views und Components sowie die Auslagerung der Style-Dateien wurde ebenfalls eingehalten.

\subsection{Konkrete Test-Überdeckung im Sprint}
{\small Autor: Linus Herterich}

Ein Student wurde beauftragt bis zum Ende des Sprints ein geeignetes Test-Framework zu finden.
Somit wurden während des Sprints noch keine Tests geschrieben.

\subsection{Ergebnisse des Reviews}
{\small Autor: Linus Herterich}

Es wurden fast alle UserStories umgesetzt. Somit war der zweite Sprint erfolgreich.
Alle Studenten konnten sich in das Projekt einarbeiten und haben die Strukturierung
größtenteils verstanden und eingehalten.
\\ Das User-Interface wurde nach der Designvorlage umgesetzt und die ersten Graphen-Funktionen
(Hinzufügen von Knoten und Kanten \& Optimieren) funktionieren bereits.
\\ Da noch nicht feststeht, wo die Software gehostet werden soll und wie die Datenbank-Funktionalität
umgesetzt werden soll, wurde zunächst eine lokale Speicherlösung als "Datenbank" verwendet. Somit konnten
die Speichern- und Laden-Funktionen erfolgreich implementiert werden.
\\ Die Login-Funktionalität ist derzeit nur sporadisch eingerichtet und wird finalisiert,
sobald feststeht, wie die Authentifizierung der Nutzer erfolgen soll (Anbindung an HTWK Login?).
\\ Leider ist immernoch kein geeignetes Testframework gefunden worden, mit dem sich sowohl Vue.js
als auch cytoscape (Graphen-Funktionalitäten) testen lassen.

\subsection{Ergebnisse der Retrospektive}
{\small Autor: Linus Herterich}

Das Happiness-Barometer für diesen Sprint ist sehr gut ausgefallen. Das liegt hauptsächlich an der guten Aufgabenverteilung
sowie an den großen Erfolgen, die diesen Sprint erzielt wurden.
\\ Kritisiert wurde die die Kommunikation gegen Ende des Sprints. Das finale Mergen aller Branches war zu hektisch und unsicher.
\\ Es wurde sich darauf geeinigt in Zukunft zwei Dailies pro Woche abzuhalten und das letzte Meeting eines Sprints zum gemeinsamen Mergen zu verwenden.

\subsection{Abschließende Einschätzung des Product-Owners}
{\small Autor: Manuel Eckert}

Aus den bei dem Planning-Meeting vorgestellten User-Stories ergaben sich drei Subteams. Diese teilten sich in die Bereiche Login, UI-Design und Graph-Funktionalitäten auf. Damit wurde das konkretere Aufteilen der User-Stories auf Subteams umgesetzt. \\
Dies hatte einen positiven Einfluss auf die Anzahl der erfolgreich abgeschlossen Aufgaben. \\
Während des Reviews wurden fehlende Code Kommentare und eine zu niedrige Testabdeckung benängelt.


\subsection{Abschließende Einschätzung des Software-Architekten}
{\small Autor: Julius Jolig}

In diesem Sprint wurden bereits mehr Kommentare im Code verfasst, aber hier ist noch Luft nach oben. Die Bachelorstudenten haben sich gut mit Vue.js und cytoscape vertraut gemacht und gute Ergebnisse erzielt. Das Mergen lief trotz neuem Ansatz immer chaotisch ab.  

\subsection{Abschließende Einschätzung des Team-Managers}
{\small Autor: Alex Hofmann}

Deutliche Leistungssteigerung schon jetzt zu sehen. Aufteilung der User-Stories direkt nach dem Planning hat die Arbeitsstruktur und -ablauf während des Sprints auf jeden Fall positiv beeinflusst.



\newpage

\section{Sprint 3}

%
\subsection{Ziel des Sprints}
{\small Autor: Lennart Buchmann}

Nach der Einarbeitung des gesamten Teams in die Grundstruktur der Software, sowie der Frameworks, lag das Hauptaugenmerk des 
dritten Sprints in der verstärkten Herausarbeitung der geplanten Kernfunktionalitäten der Anwendung. Größere Aufgabenbereiche wurden 
durch Zweier- und Dreierteams gelöst. Übriggebliebenes aus den vorherigen Sprints sollte beendet werden 

\subsection{User-Stories des Sprint-Backlogs}
{\small Autor: Lennart Buchmann}

\begin{itemize}

  \item \textbf{Funktionalität der Datenbank}
        \\\textit{Als Nutzer will ich meine gespeicherten Graphen ansehen können, um diese weiter bearbeiten zu können.}
\item \textbf{Kontext Menu über rechte Maustaste}
        \\\textit{ Als Nutzer möchte ich Knoten und Kanten über einen Rechtsklick zur einfacheren Benutzung erstellen können.}
  \item \textbf{Authentifizierung eines Nutzers}
        \\\textit{Als Nutzer möchte ich mich in die Web Applikation einloggen können,
        damit nicht jeder meine erzeugten Graphen einsehen kann.}
  \item \textbf{Darstellung von Knoten und Kanteneigenschaften am Objekt}
        \\\textit{Als Benutzer möchte ich über einen Rechtsklick auf einen Knoten/Kante die Eigenschaften dieser bearbeiten können.}
  \item \textbf{Optimierung des Graphs}
        \\\textit{Als Benutzer möchte ich gerne sofort sehen können, wie hoch meine Kosten für den kürzesten Pfad sind, damit ich mich möglichst schnell für einen entscheiden kann.}
\item \textbf{Speicherung Graph}
        \\\textit{Als Nutzer möchte ich einen Graphen jederzeit bearbeiten und speichern können, auch wenn dieser noch unfertig ist.}

\end{itemize}


\subsection{Liste der durchgeführten Meetings}
{\small Autor: Lennart Buchmann}

\begin{itemize}
  \item 06.01.2020: Planning Meeting
  \item 09.01.2020: Weekly Scrum
  \item 13.01.2020: Weekly Scrum
  \item 16.01.2020: Weekly Scrum
  \item 20.01.2020: Review \&  Retrospektive Meeting
\end{itemize}


\subsection{Ergebnisse des Planning-Meetings}
{\small Autor: Lennart Buchmann}

Der 3. Sprint ist der letzte Sprint im laufenden Semester und der letzte Sprint vor den anstehenden Prüfungen. Während des Planning-Meetings wurde von allen einheitlich besprochen, dass die
Arbeitslast von jedem höher ist als während der vergangen Sprints. Es wurde sich daraufhin geeinigt lieber realistische Ziele zu setzen, sodass der 3. Sprint auch mit höhere Belastung erfolgreich 
abgeschlossen werden kann. Nach Besprechung und Schätzung der Tickets, wurden alle Aufgaben in kleinere Gruppen aufgeteilt. Größere Aufgaben, die nach Schätzung im aktuellen Sprint nicht umsetzbar wären, wurden auf den verlängerten 4. Sprint verschoben. 


\subsection{Aufgewendete Arbeitszeit pro Person$+$Arbeitspaket}
{\small Autor: Lennart Buchmann}

\begin{longtable}{|p{4cm}|p{2cm}|p{1.2cm}|p{1.2cm}|p{0.7cm}|p{3.8cm}|}
  \hline
  Arbeitspaket                                                          & Person                & Start    & Ende     & h     & Artefakt                                                    \\ \hline
  Login                                                             & Berger, Matthias      & 13.01.20 & 17.01.20 & 18   & Login Funktionalität \& Design                              \\ \hline
  Login                                                             & Buchmann, Lennart     & 18.01.20 & 18.01.20 & 6     & Login Funktionalität \& Design                              \\ \hline
  Knotendarstellung nach Designvorlage        & Gwozdz, Jonas         & 15.01.20 & 20.01.20 & 4,5     & GraphHeader.vue, Toolbar.vue                                \\ \hline
  Speicherung Graph			        &  Heldt, Erik           & 06.01.20 & 19.01.20 & 18  & BasicData.js                                                \\ \hline
  Optimierung des Graphs 			        & Henning, Tim          & 09.01.20 & 18.01.20 & 9     & OptimizeControls.vue, index.js -> Graph Highlighting        \\ \hline
  Speicherung Graph                                      & Herterich, Linus      & 07.01.20 & 19.01.20 & 24,25     & meeting\_19\_12\_19.pdf                                     \\ \hline
  Graph zu Datenstruktur hinzufügen             & Hohlfeld, Julius      & 07.01.20 & 20.01.20 & 19     & BasicData.js, TestDatabase.js                               \\ \hline--
  Funktionalität Neuer Graph Button               & Karkoutli, Alaa Aldin & 12.01.20 & 15.01.20 & 7  & Database.vue (view), DatabaseForm.vue (component)           \\ \hline
  Kanten zu Graph hinzufügen                         & Koch, David           & 17.01.20 & 19.01.20 & 10    & Änderungen an index.js, CreateControls.vue (component)      \\ \hline
\end{longtable}

\subsection{Konkrete Code-Qualität im Sprint}
{\small Autor: Lennart Buchmann}



\subsection{Konkrete Test-Überdeckung im Sprint}
{\small Autor: Lennart Buchmann}

Eine konkrete Auseinandersetzung mit Tests beziehungsweise entsprechenden Test-Frameworks fand während des 2. Sprints statt. Momentan befinden sich alle Teammitglieder noch in der Einarbeitungsphase. Aufgrund des fortgeschrittenes Semesters und der anstehenden Prüfungen lagen die Prioritäten vorwiegend auf der Bearbeitung der User-Stories. 


\subsection{Ergebnisse des Reviews}
{\small Autor: Lennart Buchmann}

Das Ergebnis der Reviews war in anbetracht der fortgeschrittenen Semesters durchgehenden positiv. Alle Teammitglieder haben die Ihnen zugewiesenen Aufgaben innerhalb des Sprints erledigt. 
Es wurde des Weiteren besprochen, dass der verlängerte Sprint während der Semesterferien dazu genutzt werden sollte, um Bugs zu beheben und somit jedem die Gelegenheit zu geben, sich in die Testframeworks einzuarbeiten und Tests für den geschriebenen Code zu verfassen.


\subsection{Ergebnisse der Retrospektive}
{\small Autor:  Lennart Buchmann}

Während der Retrospektive wurde von allen die grundsätzliche gute Kommunikation innerhalb des Teams gelobt. Alle empfanden auch die Aufteilung in kleinere Zweier- und Dreierteams zur Bearbeitung von Aufgaben für sehr hilfreich.  Eine gleichbleibende hohe Motivation und Produktivität soll auch während des Semesterferiensprints beibehalten werden. Punkte, welche verbessert werden sollten, sind das pünktliche Mergen der einzelnen Branches vor Ende des Sprints, das Kommentieren des Codes und das Verfassen von Tests. 


\subsection{Abschließende Einschätzung des Product-Owners}
{\small Autor: xxx}

XXX

\subsection{Abschließende Einschätzung des Software-Architekten}
{\small Autor: xxx}

XXX

\subsection{Abschließende Einschätzung des Team-Managers}
{\small Autor: Alex Hofmann}

Weiterhin aufstrebende Arbeit vom Team. Auch die Kommunikation bei Problemen, Fragen und Anregungen geht in eine positive Richtung.



\newpage

\section{Sprint 4}

\subsection{Ziel des Sprints}
{\small Autor: Jonas Gwozdz}

Während der Semesterferien haben wir an Sprint 4 weitergearbeitet. Dieser dauerte vom 23.01.2020 bis zum  09.04.2020. Der Ablauf war dabei weitestgehend planmäßig, bis auf dass die Meetings zum Review und der Retrospektive wegen Corona ohne persönliches Treffen stattfinden mussten.
In der Vorlesungsfreien Zeit besprachen wir uns gelegentlich über den aktuellen Zwischenstand. Der größte Fortschritt am Projekt wurde während der letzten beiden Wochen erzielt.

\subsection{User-Stories des Sprint-Backlogs}
{\small Autor: Jonas Gwozdz}

\begin{itemize}
  \item \textbf{Tests für bereits geschriebenen Code}
        \\\textit{Als Benutzer möchte ich eine Software benutzen, die getestet ist, damit keine unerwarteten Probleme auftauchen.}
  \item \textbf{ Validierung der möglichen Eingaben }
        \\\textit{
          Als Nutzer möchte ich bei versehentlicher falscher Eingabe wenn möglich gewarnt werden, damit ich nichts falsches abspeichere.}
  \item \textbf{Bug: Validation bei gleichem Knoten-Namen}
  \item \textbf{Darstellung von Kanten/Attributen }
        \\\textit{
          Als Benutzer will ich alle Kanten/Knoten gleichzeitig sehen können(nicht übereinander), damit ich einen schnelleren Überblick über das gesamte Konstrukt bekomme.}
  \item \textbf{Bug: Mehrere Edges zwischen Knoten nicht möglich}
        \\\textit{
          Wenn man mehrere Kanten zwischen zwei Knoten anlegt, sind diese nicht sichtbar. Löscht man dann einen Knoten, an dem diese "unsichtbaren" knoten hängen, so stürzt cytoscape ab.}
  \item \textbf{Remodel von Component NewGraph}
\end{itemize}

\subsection{Liste der durchgeführten Meetings}
{\small Autor: Jonas Gwozdz}

\begin{itemize}
\item 23.01.2020: Planning
\item 05.03.2020: Weekly
\item 12.03.2020: Weekly
\item 06.04.2020: Review
\item 09.04.2020: Retro
\end{itemize}

\subsection{Ergebnisse des Planning-Meetings}
{\small Autor: Jonas Gwozdz}

Anwesend: Alex, Julius J., Julius H., Linus, Jonas, Erik, Lennart, Nils, Tim, David, Matthias, Manuel\\
\\
Innerhalb dieses Meetings haben wir die Schwerpunkte des Sprints festgelegt und über den Workload über die Vorlesungsfreie Zeit diskutiert und den Zeitaufwand der User-Stories abgeschätzt.\\


\textbf{oberste Priorität: Tests}\\
Da wird bis zum bisherigen Zeitpunkt keine Testumgebung gefunden haben, die sich auf unseren Cytoscape-Graphen anwenden lässt, und wir dadurch viel Nachholbedarf in Sachen Testen hatten, musste dieses Ticket am dringendsten abgearbeitet werden.\\

\textbf{Sprint über Semesterferien}\\
Wir haben uns im Planning darauf geeinigt, den Sprint über die Semesterferien mit weniger User-Stories als üblich auszulegen, da nicht alle Teammitglieder in dieser Zeit voll verfügbar waren, Grund dafür waren vor Allem die noch andauernden Prüfungen und die Anschließenden Ferien, die evtl. schon anderweitig verplant waren. Zudem haben wir uns darauf geeinigt, regelmäßig Absprache über den Fortschritt unserer Arbeit zu halten.\\

\textbf{Datenbanken}\\
Die Datenbankrecherche hat ergeben, dass für unsere Zwecke mySQL oder NodeJS am optimalsten wäre. Die Definition der Datenbankschnittstelle zwischen DB und Frontend muss ebenfalls noch erledigt werden. Zudem haben wir festgestellt, dass die Bisher entworfene Datenbankoberfläche optisch nicht zum Rest der Anwendung passt, und deshalb überarbeitet werden muss.\\

\textbf{Weitere Sprintziele:}
\begin{itemize}
\item Optimierung der Kostendarstellung
\item negative Zahleingaben abfangen
\item automatische Zoomfunktion bei Knoten- oder Kantenwahl
\item allgemeine Bugfixes
\end{itemize}


\subsection{Aufgewendete Arbeitszeit pro Person$+$Arbeitspaket}
{\small Autor: Jonas Gwozdz}

\begin{longtable}{|p{4cm}|p{2cm}|p{1.2cm}|p{1.2cm}|p{0.7cm}|p{3.8cm}|}
  \hline
  Arbeitspaket                                                          & Person                & Start    & Ende     & h     & Artefakt                                                    \\
  \hline
  Tests für bereits geschriebenen Code                                  & Heldt, Erik           & 04.03.20 & 04.03.20 & 2     & Tests für ModifyDataControls.vue                            \\ \hline
  Neue Strukturierung                                                   & Heldt, Erik           & 26.01.20 & 26.01.20 & 1     & Umstrukturierung des Projekts                               \\ \hline
  Header Buttons und Metadaten-Speicherung                              & Heldt, Erik           & 05.03.20 & 12.03.20 & 6,75  & GraphHeader.vue                                             \\ \hline
  Aufräumen der Branches im GitLab                                      & Heldt, Erik           & 29.03.20 & 29.03.20 & 1     & Organisatorische Aufgabe                               \\ \hline
  Entfernen veralteter Komponenten und Methoden                         & Heldt, Erik           & 31.03.20 & 31.03.20 & 2     & Organisatorische Aufgabe                                             \\ \hline
  Tests für Graphoptimierung                                            & Henning, Tim          & 04.04.20 & 40.40.20 & 12    & vargraph.spec.js        \\ \hline
  Tests für bereits geschriebenen Code                                  & Herterich, Linus      & 30.01.20 & 12.02.20 & 7,5   & /code/cypress/integration/...                                     \\ \hline
  Header Buttons und Metadaten-Speicherung                              & Herterich, Linus      & 28.03.20 & 31.03.20 & 2,25  & /vargraph/graph/... \& GraphHeader.vue                  \\ \hline
  Aufräumen der Branches im GitLab                                      & Herterich, Linus      & 30.03.20 & 30.03.20 & 1     & Organisatorische Aufgabe  \\ \hline
  Darstellung von Kanten/Attributen                                     & Herterich, Linus      & 03.04.20 & 03.04.20 & 2     & VarGraph.vue               \\ \hline
  Remodel von Component NewGraph                                        & Herterich, Linus      & 30.03.20 & 30.03.20 & 3     & /vargraph/graph/...                   \\ \hline
  Refactoring                                                           & Herterich, Linus      & 29.03.20 & 30.03.20 & 9     & /vargraph/graph/...                                 \\ \hline
  Validierung: Login                                                    & Herterich, Linus      & 31.03.20 & 30.03.20 & 1,5   & /components/login/LoginForm                                  \\ \hline
  Einheitliche Alerts                                                   & Herterich, Linus      & 31.03.20 & 31.03.20 & 3     & Dialogs.vue \\ \hline
  Validierung CreateControls \& DetailControls                          & Herterich, Linus      & 31.03.20 & 01.04.20 & 5,5   & CreateControls.vue \& DetailControls.vue               \\ \hline
  Bug: Mehrere Edges zwischen Knoten nicht möglich                      & Herterich, Linus      & 01.04.20 & 01.04.20 & 2     & /vargraph/graph/...                   \\ \hline
  Knoten dort erstellen, wo rechtsklick passiert                        & Herterich, Linus      & 01.04.20 & 01.04.20 & 1,5   & /vargraph/graph/...                               \\ \hline
  keybinds für Menüs                                                    & Herterich, Linus      & 02.04.20 & 02.04.20 & 1     &                                   \\ \hline
  Keine Knoten aufeinander schieben                                     & Herterich, Linus      & 02.04.20 & 02.04.20 & 3     & /vargraph/graph/...  \\ \hline
  Einstellungsmenü erstellen                                            & Herterich, Linus      & 03.40.20 & 05.04.20 & 5,5   &              \\ \hline
  Tests für bereits geschriebenen Code                                  & Hohlfeld, Julius      & 05.02.20 & 04.03.20 & 10    & ZoomControls.spec \& SaveMenu.spec \& NewGraphMenu.spec \& DownloadMenu.spec \\ \hline
  Dialogfenster für Speichern, Laden und Export                         & Hohlfeld, Julius      & 24.01.20 & 24.01.20 & 2     & Toolbar.vue \\ \hline
  Validierung der möglichen Eingaben                                    & Hohlfeld, Julius      & 06.04.20 & 06.04.20 & 2     & divers                             \\ \hline
  Refactoring                                                           & Hohlfeld, Julius      & 31.03.20 & 31.03.20 & 2     & /vargraph/graph/...                     \\ \hline
  Testing für Kanten hinzufügen                                         & Koch, David           & 22.03.20 & 02.04.20 & 5     & addEdges.spec      \\ \hline
\end{longtable}

\subsection{Konkrete Code-Qualität im Sprint}
{\small Autor: Jonas Gwozdz}

Die Codequatlität im allgemeinen wurde während des Sprints erheblich durch das Refactoring verbessert. Zudem wurden in nahezu  allen Dateien einleitende Kommentare geschrieben, um die zukünftige Identifizierung der gebrauchten Dateien schneller und übersichtlicher zu gestalten.

\subsection{Konkrete Test-Überdeckung im Sprint}
{\small Autor: Jonas Gwozdz}

Die geschriebenen Cypress-Tests decken bereits eine Vielzahl an Funktionalitäten des Programms ab. Dazu zählen die Buttons für die Database, den Download, das Ausloggen. Zudem wurde getestet: der Speicherdialog, die Zoomeinstellungen, der Header des Graphen, das Hinzufügen von Knoten und das Erstellen eines neuen Graphen.

\subsection{Ergebnisse des Reviews}
{\small Autor: Jonas Gwozdz}

Anwesend: David, Erik, Julius J., Julius H., Jonas, Linus, Manuel, Matthias, Tim\\

Im Rahmen des Reviews haben wir wie gewohnt die Ergebnisse des Sprint bewertet und Schwierigkeiten besprochen.\\

\textbf{generelle Schwierigkeit: Testen}\\
Um unsere Programm zu testen, entschieden wir uns für das Framework "Cypress" entschieden. dieses bietet End-to-End Testing an, welches allerdings nur Ausgaben des Programms auswerten kann, und deshalb sozusagen keinen Blick unter die Haube zulässt, und somit eventuell Fehler unentdeckt bleiben. \\

\textbf{David:}
\begin{itemize}
\item Tests für Knotenfunktionalität geschrieben
\item mit Kantentests begonnen
\end{itemize}

\textbf{Erik:}
\begin{itemize}
\item Data Controls durch Header Buttons ersetzt
\item Editierungsfenster entfernt
\item Header Buttons getestet
\end{itemize}

\textbf{Jonas:}
\begin{itemize}
\item Testübersicht erstellt
\item Möglichkeit zum Informationsaustausch über Lücken und Bugs in Tests bereitgestellt
\end{itemize}

\textbf{Julius H.:}
\begin{itemize}
\item Tests für Toolbar, Zoom-Controls, Buttons und Eingabereihenfolgen geschrieben
\end{itemize}

\textbf{Julius H, Erik, Linus:}
\begin{itemize}
\item Refactoring des Graphen, Bugfixing und Validierung von Eingaben
\end{itemize}

\textbf{Linus:}
\begin{itemize}
\item Dialogue-Popups erstellt
\item Kürzelgenerierung implementiert
\item Knotenüberlagerung unterbunden, Mindestabstand implementiert
\item Einstellungsmenü erstellt und Implementation begonnen
\item Recherche zu Datenbankfenster
\end{itemize}

\subsection{Ergebnisse der Retrospektive}
{\small Autor: Jonas Gwozdz}

Anwesend: Alex, Erik, Julius J., Julius H., Jonas, Linus, Matthias, Tim\\

Zu Beginn des Sprints gab es keine Fortschritte zu vermelden, da vorerst die Prüfungen zu überstehen waren. In den beiden Wochen vor Sprintende wurden allerdings die wichtigsten User-Stories und sogar etwas mehr abgearbeitet.\\

\begin{center}
\begin{tabular}{ |c|c| }
\hline
 Positiv & Negativ \\
\hline 
 -produktive Endphase & -anfangs keine Kommunikation \\
 -viel Motivation bei Einigen & - wenig Motivation bei Einigen\\
 & -vereinzelt Tests ohne Sinn\\
 & -ausgefallene Meetings\\
\hline     
\end{tabular}
\end{center}
 

\subsection{Abschließende Einschätzung des Product-Owners}
{\small Autor: xxx}

XXX

\subsection{Abschließende Einschätzung des Software-Architekten}
{\small Autor: xxx}

XXX

\subsection{Abschließende Einschätzung des Team-Managers}
{\small Autor: Alex Hofmann}

Aufgrund der vorlesungsfreien Zeit war mit erhöhter Inaktivität aufgrund von Prüfungen, Urlaub und sonstigen Auszeiten zu rechnen.
Das Team hat sich dennoch demokratisch für einen Sprint während dieser Zeit entschieden. Trotz aller Umstände wurde mit der Umsetzung der Testfälle die Zielvorgabe erreicht.



\newpage
\section{Sprint 5}

\subsection{Ziel des Sprints}
{\small Autor: Tim Henning}

Der vierte Sprint des VarG-Projektes lief vom 13.04.20 bis zum 23.04.20.
Ziel des Sprints war zum einem, dass die nicht vollendeten Aufgaben aus Sprint 4 nachgeholt werden, und das sich um die Schnittstelle zwischen Frontend und Backend gekümmert wird. Weiterhin wurde geäußert viel Recherche zum Thema Datenbanken, Shibboleth Anbindung und bereitstellen eines Servers des IT-Servicezentrums der HTWK, zu betreiben. Außerdem sollten zum vorhandenen Optimierungsalgorithmus noch einige Besserungen vorgenommen werden.

\subsection{User-Stories des Sprint-Backlogs}
{\small Autor: Tim Henning}

\textbf{Datenbank, Initiale Aufgaben zur Bereitstellung}
Als Nutzer möchte ich gerne auf eine, mit dem Rest der App, konsistente Oberfläche zugreifen können, damit ich mich einfacher zurecht finde. Zudem möchte ich gerne einen Überblick über die vorhandenen Elemente (Bearbeitungsmaschinen) anzeigen lassen und in meinen Graphen übernehmen können, damit ich die Eigenschaften dieses Elements nicht jedes mal neu heraussuchen muss.

\textbf{Entwurf der Schnittstelle zwischen Backend und Frontend}
Als Nutzer möchte ich Daten aus der Datenbank abrufen/anzeigen lassen können, damit der Graph schneller erstellt werden kann.

\textbf{Login}
Nach dem Login, in die Applikation sollen meine Anmeldedaten gespeichert werden, damit ich mich beim erneuten laden der Seite nicht neu einloggen muss.

\textbf{Optimierung}
Als Benutzer möchte ich optimale Wege des erstellten Graphen anzeigen lassen können, damit ich eine bessere Auswahl zwischen den einzelnen Bearbeitungsschritten treffen kann.


\subsection{Liste der durchgeführten Meetings}
{\small Autor: Tim Henning}

\begin{itemize}
	\item Planning - 13.04.2020
	\item Weekly Scrum 1 - 16.04.2020
	\item Weekly Scrum 2 - 20.04.2020
	\item Review - 23.04.2020
	\item Retrospektive - 23.04.2020
\end{itemize}

\subsection{Ergebnisse des Planning-Meetings}
{\small Autor: Tim Henning}

Anwesend: Jonas G., Erik H. Linus H., Lennart B., Tim H., David K., Matthias B., Alaa Aldin K., Manuel E., Julius J., Alex H.\\
\\

Neben der Aufgabenverteilung wurden noch einige zusätzliche Punkte besprochen, die nicht in den User Stories aufgetaucht sind. So zum Beispiel sollte nach jedem Sprint ein Production Build angelegt werden, der auf einem Server liegt, damit der Kunde regelmäßig das Produkt testen kann. Weiterhin wurde gefordert die neue Testumgebung Cypress in die Git-Pipeline einzubinden. Außerdem sollte am IT-Servicezentrum nachgefragt werden, ob es möglich ist eine Shibboleth Anbindung zu bekommen und ob die HTWK einen Server bereitstelle, auf dem der Production Build später gehostet werden kann.

\subsection{Aufgewendete Arbeitszeit pro Person$+$Arbeitspaket}
{\small Autor: Tim Henning}

\begin{longtable}{|p{4cm}|p{2cm}|p{1.2cm}|p{1.2cm}|p{0.7cm}|p{3.8cm}|}
        \hline
	Arbeitspaket & Person & Start & Ende & h & Artefakt\\
        \hline
	UI: Login & Beger, Matthias & 13.04.20 & 13.04.20 & 2,5 & Recherche, Konzeption\\ \hline
	UI:Login & Buchmann, Lennart & 23.04.20 & 23.04.20 & 5 & Recherche, Konzeption\\ \hline
UI: Datenbank; Initiale Aufgaben zur Bereitstellung & Gwozdz, Jonas  & 13.04.20 & 23.04.20 & 15 & Datenbankfenster Redesign, Responsiveness der Datenbankseite, Button Platzierungen \\ \hline
 Task: Sprint 4 Dokumentation & Gwozdz, Jonas  & 13.04.20  & 13.04.20 & 5 & Sprint4.tex \\ \hline
UI: Entwurf der SChnittstelle Backend <-> Frontend & Heldt, Erik  & 18.04.20 & 18.04.20  & 1,5 & SaveMenu.vue, TestDataBase.js \\ \hline
Task: Recherche Zusammenspiel Vue + Datenbank & Heldt, Erik  & 15.04.20  & 16.04.20  & 2 & Installation Axios, HTTP Requests \\ \hline
Task: Button UI/UX Änderungen und Validierung bei Erstellung von Kanten & Heldt, Erik  & 17.04.20 &23.04.20 & 7 & CreateControl.vue, DetailControls.vue\\ \hline
Task: Gesamkosten und /-zeit einschließlich der Produktanzahl & Henning, Tim  & 15.04.20  & 16.04.20  & 4 & optimization.js \\ \hline
Task: Alten Optimierungsalgorithmus umbauen & Henning,  Tim  & 17.04.20 & 23.04.20  & 11 & optimization.js\\ \hline
Task: Sprint 5 Dokumentation & Henning,Tim  & 23.04.20  & 23.04.20  & 3 & Sprint5.tex \\ \hline
UI: Datenbank; Initiale Aufgaben zur Bereitstellung &  Herterich, Linus  & 15.04.20  & 15.04.20 &  2,5 & Datenbankseite nun als Component \\ \hline
Sprint 2 Dokumentation  &  Herterich, Linus& 13.04.20 & 13.04.20 & 3,5 & Sprint2.tex \\ \hline
Task: Erstellung Production Build auf Server & Herterich, Linus  & 14.04.20  & 14.04.20 & 3 & läuft auf varg.nfl-server.de \\ \hline
Task: Cypress Test in die Gitlab Pipeline & Herterich, Linus  & 16.04.20 & 16.04.20 & 4,5 & .gitlab-ci.yml\\ \hline
Task: Kaputte Tests reparieren & Herterich, Linus  & 17.04.20  & 17.04.20  & 2 &  code/cypress/integration/.. \\ \hline
Task:Graph aus Hauptmenü importieren & Herterich, Linus  & 17.04.20  & 17.04.20  & 2 & Importieren aus Hauptmenü umgesetzt  \\ \hline
Task: Redesign Graphen Seite(Navigation Drawer) & Herterich, Linus  & 16.04.20  & 23.04.20 & 11 & 2 neue Designkonzepte \\ \hline
UI: Entwurf der Schnittstelle Backend <-> Frontend & Hohlfeld, Julius  & 13.04.20  & 22.04.20 & 6,5 & Dokumentation der API-Recherche und erste Entwürfe, API Dokumentation im Git Wiki \\ \hline
Task: Auswahl von Endzustand ohne Startzustand & Karkoutli, Alaa Aldin  &  17.04.20 & 22.04.20  & 14 & ausgewählte Startzustände aus Liste der Endzustände entfernt, OptimizeControls.vue  \\ \hline
Task: Auslagern der Optimize Controlls & Koch, David & 16.04.20   & 16.04.20 &  2 & OptimizeControls.vue \\ \hline
Task: Neuer Optimierungsalgorithmus & Koch, David & 17.04.20 & 23.04.20 & 10 & Beginn eines neuen Algorithmus \\ \hline

      \end{longtable}

\subsection{Konkrete Code-Qualität im Sprint}
{\small Autor: Tim Henning}

Die Codequalität hat sich zum vorherigen Sprint nicht entscheidend geändert. Durch den Umbau des Optimierungsalgorithmus hat man nun aber eine etwas höhere Speicherplatz- und Laufzeitkomplexität. Dies soll im nächsten Sprint angegangen und verbessert werden. Durch das Redesign ist die Website im allgemeinen ästhetischer geworden.


\subsection{Konkrete Test-Überdeckung im Sprint}
{\small Autor: Tim Henning}

Durch das Hinzufügen der Cypress Tests in die Pipeline des Git-Repository ist nun eine relativ gutes Feedback für den jeweiligen Entwickler und Tester vorhanden. Dieser bekommt nach durchführen der Pipeline eine E-mail, falls der Test fehlschlägt. Für den Optimierungsalgorithmus hingegen fehlen noch ein paar Tests.

\subsection{Ergebnisse des Reviews}
{\small Autor: Tim Henning}

Anwesend: Jonas G., Erik H. Linus H., Lennart B., Tim H., David K., Alaa Aldin K., Manuel E., Julius J., Alex H.\\


Im Review hat wie gehabt, jeder seine erledigten und angefangen Aufgaben vorgestellt und bewertet. So wurde bei der Optimierung die Stückzahl in den Algorithmus integriert, die Endzustände ohne Startzustände werden nun angezeigt und es wurde parallel an zwei neuen Algorithmen gearbeitet, die es ermöglichen die k-besten Pfade auszugeben, und nicht nur den optimalsten Pfad. Dabei wurde einer fertig gestellt, der die Pfade in der Konsole ausgeben kann. Dieser hat aber eine recht hohe Laufzeit- und Speicherplatzkomplexität. Daher wurde ein weitere Algorithmus angefangen, welcher im nächsten Sprint weiterentwickelt und angepasst wird. Zur Userstory der Datenbank und den Initalen Aufgaben zur Bereitstellung wurden erste HTTP Requests angefangen und ausprobiert sowie Axios installiert. Da aber die Datenbank nocht nicht konkret fest stand und noch kein Server von der HTWK zur Verfügung war, wurde sich primär um Bugfixing, Testing und Valiedierungen von Eingaben gekümmert. Die Buttons werden nun nach Windows Standard rechts unten angezeit und sind im Text-only Stil. Desweiteren wurde der Datenbankscreen angepasst und hat nun eine übersichtlichere Darstellung der Elemente, die später einmal aus der Datenbank geladen werden. Zur Userstory der Schnittstelle zwischen Frontend und Backend wurde viel Recherche betrieben. Dabei wurde ein Dokument erstellt, welches alle wichtigen und relevanten Informationen zum Thema API zusammen trägt. Dieses ist im Git- Wiki zu finden. Im Login Team wurde sich damit beschäftigt ein Rollenmanagement einzuführen und die Anbindung an das Shibboleth zu bekommen. Dies wird im nächsten Sprint weitergeführt. Ebenfalls wurde bei dem IT-Servicentrum der HTWK ein Server bestellt mit folgenden Spezifikationen:
\begin{itemize}
	\item 64 Bit, Debian
	\item 4GB Ram, 30GB Festplatte
	\item Anzahl der CPU's: 1
	\item Name der VM: Varg
	\item Netz: DMZ-VM-Fak
	\item Verwendungszweck: Softwareprojekt
	\item Verantwortlicher Prof.: Prof. Dr. Martin Gürtler
	\item Bemerkungen: Anfragen ob ITSZ Apache ausrollt\newline
\end{itemize}


Als letzter Punkt wurde im Sprint ein neues Design angefangen. Dort wurden auch schon die meisten Funktionen und Menüs implementiert und zum Ende des nächsten Sprints fertig gestellt. Das Projekt wird zum testen für den Kunden auf dem privaten Server eines Teammitgliedes gehostet.


\subsection{Ergebnisse der Retrospektive}
{\small Autor: Tim Henning}

Anwesend: Jonas G., Erik H. Linus H., Lennart B., Tim H., David K., Alaa Aldin K., Manuel E., Julius J., Alex H.\\
\\

Die Retrospektive fand in diesem Sprint online nach dem KALM Prinzip (Keep, Add, Less, More) statt und es wurden wie gewohnt Punkte die das Team ändern muss, aber auch welche die positiv waren und beibehalten werden sollen, angesprochen. So wurde die zahlreiche Teilnahme an den Meetings, sowie die Motivation in diesem Sprint als sehr positiv gewertet. Was im nächsten Sprint hinzu kommen sollte wäre u.a. eine weitere Person für das Team welches sich um das Zusammenspiel zwischen Frontend und Backend kümmert. Auch sollen die Testdokumentationen im Wiki ergänzt und ausgefüllt werden, um nach zu vollziehen welche Components bereits getestet wurden. Desweitern war ein wichtiger Punkt die zeitliche Absprache über das mergen der Branches und das aufräumen im Git Repository. Als Anmerkung unter dem Punkt "Less" , wurde zum einen das hinzufügen neuer Features genannt. Das Team will sich in den nächsten Sprints um Robustheit und Testing des vorhanden Codes kümmern und nicht all zu viele neue Features hinzufügen. Außerdem wurde noch angemerkt das die einzelnen Mitglieder YouTrack konsequenter nutzen sollen, um eine bessere Übersicht über den Workflow zu bekommen. Zum Schluss wurde noch erwähnt das der Sprint sehr positiv bewertet wurde, da viele Ziele erreicht wurden und viele neue Erkenntnisse zustande kamen, sowie das sich viele Teammitglieder an dem Sprint beteiligt haben.

\subsection{Abschließende Einschätzung des Product-Owners}
{\small Autor: Manuel Eckert}

Der Ablauf in den einzelnen Meetings läuft extrem reibungslos. Alle Teammitglieder fühlen sich mit dem Produkt identifiziert. Dies merkte man sehr in der Beteiligung und dem aufgewendeten Arbeitseinsatz während des Sprintes. Dies bedeutete auch eine hohe Anzahl an abgeschlossenen User-Stories. Durch das parallele Arbeiten an Front- und Backend wurde eine gute Produktivität erreicht. Die Schnittstelle zwischen Front- und Backend wurde ebenfalls konzipiert. \\
In diesem Sprint wurde das Produkt auch auf einen Webserver aufgespielt, dass der Kunde die Möglichkeit hat, sich länger mit dem Produkt auseinander zu setzten und damit auch ein besseres und detaillierteres Feedback geben kann. \\
Damit wurde dieser lange Sprint, der über die Prüfungszeit und Semesterferien ging, positiv abgeschlossen.

\subsection{Abschließende Einschätzung des Software-Architekten}
{\small Autor: Julius Jolig}

In diesem Sprint wurde eine CI Pipeline implementiert, wodurch fehlerhafte branches direkt nach dem pushen ins GitLab entdeckt werden können. Auch Test werden in der Pipeline ausgeführt. Allerdings fehlt noch ein Test für den Optimierungsalgorithmus. Das Mergen am Ende des Sprint lief im Vergleich zum letzten Sprint sehr gut ab.

\subsection{Abschließende Einschätzung des Team-Managers}
{\small Autor: Alex Hofmann}

Mit Beginn des neuen Semesters und der damit verbundenen Wiederaufnahme der (Online-) Präsenzveranstaltungen nahm auch die Teilnahme am Projekt wieder zu. Bis auf die beiden Aussteiger haben alle Teammitglieder mitgewirkt. Diese Motivation gilt es auch in den kommenden Wochen aufrecht zu erhalten.


\newpage


%%%%%% weitere Sprints analog


\section{Dokumentation}

\subsection{Handbuch}
{\small Autor: David Koch}
\begin{itemize}
  \item \textbf{ Login }
    \\\
      Um sich einlogggen zu können, benötigt man einen Account. Diesen legt man an, indem man sich im Login-Screen mit einem neuen Benutzernamen und Passwort einloggt. Der angegebene Name wird als neuer Nutzer (mit Rolle: 'Student') angelegt und das dazugehörige Passwort gespeichert. Beides kann nachträglich in den Benutzereinstellungen (siehe 'Einstellungen') geändert werden. Ein ändern der Rolle (von 'Student' auf 'Admin') ist nur durch direkten zugriff auf die Datenbank möglich. (Momentan wird eine eigene Datenbank benutzt, diese kann später durch eine Anbindung an shiboleth ersetzt werden)
  \item \textbf{ Neuen Graphen erstellen }
    \\\
      Über den Menüpunkt 'Neuen Graphen erstellen' im Startfenster gelangt man in das 'Neues Produkt' Fenster. Hier kann Name und Stückzahl des neuen Graphen festgelegt werden. Über den 'Starten'  Button wird der neue Graph erstellt. Im Graphenfenster gelangt man über den 'Neuer Graph' Button zurück ins Startmenü. Über die Stift-Icons neben dem Produktnamen und der Stückzahl können diese nachträglich bearbeitet werden.
  \item \textbf{ Erstellen und Bearbeiten von Zuständen }
    \\\
      Das 'Neues Teil' Menü öffnet sich entweder über Rechtsklick innerhalb des Graphenfensters, gefolgt von einem Linksklick auf 'Neues Teil' oder durch einen Linksklick auf den Plus-Button in der rechten unteren Ecke, gefolgt von einem Linksklick auf 'Neues Teil'. Hier kann neben Name, Kürzel und Farbe des Zustands auch ein dazugehöriges Icon mittels URL gewählt werden. Alle Angaben außer dem Icon sind Pflichtangaben, sodass der 'erstellen' Button erst betätigt werden kann, wenn alle diese Felder beschrieben sind.
Mit einem Linksklick auf einen bereits erstellten Zustand öffnet sich das 'Teil bearbeiten' Menü. Dieses ist aufgebaut wie das 'Neues Teil' Menü. Hier können alle Eigenschaften des angeklickten Zustandes bearbeitet werden.
  \item \textbf{ Erstellen und Bearbeiten von Bearbeitungsschritten }
    \\\
      Das 'Neuer Bearbeitungsschritt' Menü öffnet sich entweder über Rechtsklick innerhalb des Graphenfensters, gefolgt von einem Linksklick auf 'Neuer Bearbeitungsschritt' ,oder durch einen Linksklick auf den Plus-Button in der rechten unteren Ecke, gefolgt von einem Linksklick auf 'Neuer Bearbeitungsschritt'. Im ersten Teil können Name, Kürzel sowie Start- und Endzustand gewählt werden. Im zweiten Teil definiert man Losgröße, Zeit- und Geldkosten sowie Zeit und Geldrüstkosten des Bearbeitungsschrittes. Alle Felder sind Pflichtfelder, es müssen also alle Felder beschrieben sein, um den 'erstellen' Button betätigen zu können. Des Weiteren kann man Bearbeitungsschritte auch mittels drag'n'drop erstellen. Dafür bewegt man den Mauszeiger über den Zustand, der als Startzustand dienen soll, klickt auf das neuer Bearbeitungsschritt-Icon und zieht die entstehende Linie zu dem Endzustand. Anschließend öffnet sich der zweite Teil des 'Neuer Bearbeitungsschritt' Menüs, in dem dann die übrigen Eigenschaften  nachträglich eingefügt werden müssen.
Mit einem Linksklick auf einen bereits erstellten Bearbeitungsschritt öffnet sich das 'Bearbeitungsschritt bearbeiten' Menü. Dieses ist aufgebaut wie das 'Neuer Bearbeitungsschritt' Menü. Hier können alle Eigenschaften des angeklickten Bearbeitungsschritt geändert werden.
  \item \textbf{ Graph optimieren }
    \\\
      Um den besten Weg von einem der Startzustände zum Endzustand zu finden, klickt man auf 'Graph optimieren' unter 'Gesamtkosten' (um nach Kosten zu optimieren) oder unter 'Gesamtzeit' (um nach Zeit zu optimieren), Start- und Endzustände werden in diesem Fall automatisch gewählt. Nach dem Betätigen des Buttons verschwindet dieser und wird ersetzt durch die Gesamtkosten bzw. die Gesamtzeit des besten Weges im Graphen, außerdem ist dieser Weg Orange markiert. Abhängig davon, ob nach Kosten oder Zeit optimiert wurde wird das entsprechende Icon sowie das Wort ('Gesamtkosten' oder 'Gesamtzeit') ebenfals Orange markiert.
Möchte man die Parameter der Optimierung eigenhändig bearbeiten, lässt sich das entsprechende Menü öffnen, indem man entweder auf eines der Zahnräder (neben 'Gesamtkosten' oder 'Gesamtzeit' ) klickt oder auf 'Einstellungen' am oberen Rand klickt und in den Tab 'Optimierung' wechselt. In diesem Menü lassen sich die Start- und Endzustände auswählen, sowie wonach optimiert werden soll (Kosten oder Zeit) und wie oft optimiert werden soll (3 mal optimieren heißt, dass die besten 3 Wege angezeigt werden, diese Wege sind unter den Einstellungen aufgelistet). Auch in diesem Menü kann man die Optimierung starten, indem man auf 'Optimierung starten' (unten im Einstellungsmenü) klickt, sind Start- und Endzustände nicht ausgewählt, werden diese wieder automatisch bestimmt, die Optimierungsart (Zeit/Kosten) muss allerdings ausgewählt sein (alternativ dient der Button 'anwenden' zum übernehmen der Einstellungen und 'schließen' zum verwerfen der Einstellungen. In beiden Fällen schließt das Menü).
Nach dem Betätigen des 'Optimierung starten' Buttons (im Einstellungsmenü) wird dieser ersetzt durch die besten Wege im Graphen (begrenzt durch Optimierungsanzahl). Durch klick auf einen dieser, erhält man einen Einblick in die einzelnen Zustände und Bearbeitungsschritte des Weges, durch klick auf den Kreis links vom angezeigten Weg, wird dieser anstatt dem besten Weg orange markiert.
Sobald etwas am Graph geändert wird, erscheinen die 'Graph optimieren' bzw. 'Optimierung starten' Buttons wieder. Zuvor lässt sich der Graph durch Klicken auf das Wiederholen-Icon (neben 'Gesammtzeit' oder 'Gesamtkosten' im Graphenfenster) erneut optimieren.
  \item \textbf{ Einstellungen }
    \\\
      Mit einem Linksklick auf 'Einstellungen' am oberen Rand öffnet sich das Einstellungenfenster. Im Tab 'Graph' lassen sich Einstellungen bezüglich der Darstellung des Graphs vornehmen, wie zum Beispiel die Einheiten oder die angezeigten Details der Verknüpfungen (Bearbeitungsschritte). Außerdem lässt sich hier ein "Raster" aktivieren, auf dem die Knoten dann gebunden sind. Im Tab 'Benutzer' lassen sich Benutzername und Passwort ändern sowie der Account löschen. Beim Ändern des Benutzernamen wird der 'Autor' jedes mit diesem Account erstellten Graphen ebenfalls geändert, beim Löschen des Accounts werden auch alle mit diesem Account erstellten Graphen gelöscht. Zum Löschen des Accounts oder Ändern des Passworts wird das Passwort benötigt. Der Tab 'Hilfe' ist noch nicht funktional.
  \item \textbf{ Offline Speicher }
    \\\
      Über den Button 'Export' am oberen Rand lässt sich der Graph in den Formaten .json, .png, .svp und .jpg exportieren und lokal speichern. Über den 'Import' Button lässt sich ein in .json exportierter Graph wieder im Programm öffnen.
  \item \textbf{ Online Speicher }
    \\\
      Über den Button 'Datenbank' am oberen Rand öffnet sich das Datenbank-Fenster. Hier können erstellte Graphen online gespeichert bzw. wieder geladen werden. Dabei können Accounts mit der Rolle 'Student' lediglich auf Ihre eigenen Graphen zugreifen. Accounts mit der Rolle 'Admin' haben Zugriff auf alle Graphen.
\end{itemize}


\subsection{Installationsanleitung}
{\small Autor: xxx}

XXX

\subsection{Software-Lizenz}
{\small Autor: xxx}

XXX


\section{Projektabschluss}

\subsection{Protokoll der Abnahme und Inbetriebnahme beim Kunden}
{\small Autor: xxx}

XXX

\subsection{Präsentation auf der Messe}
{\small Autor: xxx}

Poster, Bericht

\subsection{Abschließende Einschätzung durch Product-Owner}
{\small Autor: xxx}

XXX

\subsection{Abschließende Einschätzung durch Software-Architekt}
{\small Autor: xxx}

XXX

\subsection{Abschließende Einschätzung durch Team-Manager}
{\small Autor: xxx}

XXX

\end{document}
