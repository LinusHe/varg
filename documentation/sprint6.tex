\subsection{Ziel des Sprints}
{\small Autor: David Koch}

Vom 27.04. bis 07.05. arbeiteten wir an Sprint 6. Auf Grund von Corona fanden auch hier alle Treffen in Form von online-conference-calls statt. Die Zwischenstände bei den einzelnen Treffen wirkte zwar wenig erfolgversprechend, dennoch wurden fast alle Aufgaben bearbeitet und eingebunden.
Die Meetings verliefen problemfrei, die Aufgaben wurden beim Planning gut verteilt und die Zwischenstandsmeetings halfen, kleinere Probleme schnell zu lösen.

\subsection{User-Stories des Sprint-Backlogs}
{\small Autor: David Koch}

\begin{itemize}
  \item \textbf{ Optimierung des Graphen }
        \\\textit{
          Der Optimierungsalgorithmus soll überarbeitet werden, zum einen sollen die Start- und Endzustände automatisch ausgewählt werden sowie eine neue Größe der Kanten - die Losgröße, die die Wichtung der Rüstkosten beeinflusst - eingebunden werden. Zum anderen soll der Algorithmus nicht nur den besten sondern auch den zweit-, dritt-, usw -besten Pfad auszugeben, das Interface soll an die neuen Ein- und Ausgaben angepasst werden.}
  \item \textbf{ Erstellen von Bearbeitungsschritten durch klicken }
        \\\textit{
          Knoten und Kanten sollen durch wenige clicks schnell erstellen werden können. Das Hinzufügen der Eigenschaften wird anschließend durchgeführt, die Validierungen werden dem entsprechend von der Erstellung auf den Optimierungsalgorithmus verschoben.}
  \item \textbf{ Sammelticket für Feedback vom Kunden }
        \\\textit{
          Die "Knoten" sollen als "Teile" bezeichnet werden und die "Kanten" als "Bearbeitungsschritte". Der Name der Bearbeitungsschritte soll besser erkennbar sein.}
  \item \textbf{ Design }
        \\\textit{
          Das Design soll auf die neuste Version angepasst werden.}
  \item \textbf{ Login }
        \\\textit{
          Alle Buttons auf der Login-Seite, die zum Testen verwendet wurden, sollen entfernt werden. Die Credentials sollen gespeichert werden und der Graph soll bei einem reload der Seite nicht verloren gehen. Außerdem wird die Anbindung an Shiboleth weiter bearbeitet.}
  \item \textbf{ Backend Datenbank }
        \\\textit{
          Es soll eine Datenbank in Docker Container aufgesetzt werden, um erste Tests auf einer Datenbank durchführen zu können und damit bereits eine Schnittstelle implementiert werden kann. Sobald der angeforderte HTWK-Server bereitsteht, soll die Datenbank aufgesetzt und parallel dazu eine Dokumentation aufgesetzt werden. Des Weiteren sollen Testdaten für die Datenbank generiert werden.}

\end{itemize}

\subsection{Liste der durchgeführten Meetings}
{\small Autor: David Koch}

\begin{itemize}
\item 27.04.2020: Planning
\item 30.04.2020: Weekly
\item 04.05.2020: Weekly
\item 07.05.2020: Review, Retro
\end{itemize}

\subsection{Ergebnisse des Planning-Meetings}
{\small Autor: David Koch}

Anwesend: Alex, Julius J., Julius H., Linus, Jonas, Erik, Lennart, Nils, Tim, David, Matthias, Manuel\\
\\
Innerhalb dieses Meetings haben wir die Schwerpunkte des Sprints festgelegt, den Zeitaufwand der User-Stories abgeschätzt und die daraus entstehenden Aufgaben verteilt.\\


\textbf{Design}\\
Auf Grund der wandelnden Wünsche des Kunden wurde das Interface des Öfteren redesigned. Da sich das Projekt dem Ende nähert, soll in diesem Sprint die letzte Änderung am Interface vorgenommen werden.\\

\textbf{Warten auf Antwort der HTWK}\\
Die Datenbank auf einem HTWK-Server einzurichten sowie die Anbindung des Logins an Shiboleth sind abhängig von Antworten des ITSZ der HTWK und können daher eventuell noch nicht bearbeitet werden.

\textbf{Weitere Sprintziele:}
\begin{itemize}
\item Bugtickets bearbeiten
\item Losgröße als neue Kanteneigenschaft hinzufügen
\item neue hinzugefügten Code testen
\end{itemize}


\subsection{Aufgewendete Arbeitszeit pro Person$+$Arbeitspaket}
{\small Autor: David Koch}

\begin{longtable}{|p{4cm}|p{2cm}|p{1.2cm}|p{1.2cm}|p{0.7cm}|p{3.8cm}|}
  \hline
  Arbeitspaket                                                          & Person                & Start    & Ende     & h     & Artefakt \\
  \hline
  Login                                                                 & Berger, Matthias      & 28.04.20 & 28.04.20 & 4     & Grundlagenrechte \\
  \hline
  Login                                                                 & Berger, Matthias      & 07.05.20 & 07.05.20 & 5     & Konzeption und Umsetzung von Ladebildschirm, Timeout \& Weiterbildung \\
  \hline
  Login                                                                 & Buchmann, Lennart     & 07.05.20 & 07.05.20 & 5     & Konzeption und Umsetzung von Ladebildschirm, Timeout \& Weiterbildung \\
  \hline
  Login                                                                 & Buchmann, Lennart     & 02.05.20 & 02.05.20 & 4     & Grundlagenrechte \\
  \hline
  Login                                                                 & Buchmann, Lennart     & 04.05.20 & 04.05.20 & 5     & Login Persistenz \\
  \hline
  Optimierung des Graphen                                               & Gwozdz, Jonas         & 30.04.20 & 30.04.20 & 1,5   & Design erstellen \\
  \hline
  Sammelticket für Feedback vom Kunden                                  & Gwozdz, Jonas         & 29.04.20 & 29.04.20 & 1,5   & Beschriftung ändern \\
  \hline
  BUG TRACKER                                                           & Gwozdz, Jonas         & 04.05.20 & 04.05.20 & 5     & Bud: Node-Overlapping ab 3 Knoten \\
  \hline
  Backend Datenbnk                                                      & Heldt, Erik           & 28.04.20 & 05.05.20 & 2,75  & Schreiben einer echten Datenbank auf Docker \\
  \hline
  Backend Datenbank                                                     & Heldt, Erik           & 06.05.20 & 06.05.20 & 0,5   & HTTP-Request (Client-Side) \\
  \hline
  Design                                                                & Heldt, Erik           & 06.05.20 & 06.05.20 & 3     & Ausführliche Tests für GraphInfo.vue \\
  \hline
  Optimierung des Graphen                                               & Henning, Tim          & 05.05.20 & 05.05.20 & 2     & Initialzustände automatisch auswählen \\
  \hline
  Optimierung des Graphen                                               & Henning, Tim          & 06.05.20 & 06.05.20 & 1     & Ausgabe der Gesammtkosten/-zeit auf dem Userinterface \\
  \hline
  Optimierung des Graphen                                               & Herterich, Linus      & 30.04.20 & 30.04.20 & 2     & Settings -> Optimierung persistent speichern \\
  \hline
  Design                                                                & Herterich, Linus      & 30.04.20 & 06.05.20 & 5,5   & Redesign Optimierungsansicht \\
  \hline
  Design                                                                & Herterich, Linus      & 06.05.20 & 06.05.20 & 1     & Graph-Editor expandierbar \\
  \hline
  Design                                                                & Herterich, Linus      & 04.05.20 & 04.05.20 & 1     & Tests zum neuen Design \\
  \hline
  Design                                                                & Herterich, Linus      & 03.05.20 & 04.05.20 & 1,5   & Save Menu - Test fixen \\
  \hline
  Design                                                                & Herterich, Linus      & 04.05.20 & 04.05.20 & 1     & NewGraph Menu - Test fixen \\
  \hline
  Design                                                                & Herterich, Linus      & 04.05.20 & 04.05.20 & 1,5   & Überschreiben Dialog für Save-Menu \\
  \hline
  Design                                                                & Herterich, Linus      & 27.04.20 & 30.04.20 & 1,5   & Avatar-Menü $\rightarrow$ Ausloggen \& Einstellungen \\
  \hline
  Design                                                                & Herterich, Linus      & 27.04.20 & 27.04.20 & 2     & Merge auf Targetbranch \\
  \hline
  Design                                                                & Herterich, Linus      & 30.04.20 & 04.05.20 & 2,5   & Components aufräumen \\
  \hline
  Design                                                                & Herterich, Linus      & 27.04.20 & 27.04.20 & 0,5   & Login fixen \\
  \hline
  Design                                                                & Herterich, Linus      & 27.04.20 & 27.04.20 & 1     & Tests fixen \\
  \hline
  Design                                                                & Herterich, Linus      & 07.05.20 & 07.05.20 & 0,5   & Altes Design in "removed code" \\
  \hline
  Backend Datenbank                                                     & Hohlfeld, Julius      & 05.05.20 & 06.05.20 & 7,5   & API-Parser \\
  \hline
  Backend Datenbank                                                     & Hohlfeld, Julius      & 03.05.20 & 07.05.20 & 4     & Node.js Programmierung (Server-Side) \\
  \hline
  Backend Datenbank                                                     & Hohlfeld, Julius      & 28.04.20 & 29.04.20 & 14,5  & Set-Up Docker MySQL \\
  \hline
  Backend Datenbank                                                     & Hohlfeld, Julius      & 03.05.20 & 03.05.20 & 3     & Umgehung des MySQL Authentifizierungsprotokoll \\
  \hline
  Backend Datenbank                                                     & Hohlfeld, Julius      & 03.05.20 & 05.05.20 & 5     & Anbindung von Docker zu JS \\
  \hline
  Login                                                                 & Karkoutli, Alaa Aldin & 07.05.20 & 07.05.20 & 5     & Konzeption und Umsetzung von Ladebildschirm, Timeout \& Weiterbildung \\
  \hline
  Login                                                                 & Karkoutli, Alaa Aldin & 04.05.20 & 04.05.20 & 4     & Grundlagenrechte \\
  \hline
  Optimierung des Graphen                                               & Koch, David           & 30.04.20 & 06.05.20 & 12    & Ausgabe der Gesammtkosten/-zeit auf dem Userinterface \\
\end{longtable}

\subsection{Konkrete Test-Überdeckung im Sprint}
{\small Autor: David}

Die Testabdeckung ist deutlich besser als bei vorherigen Sprints. Auch wenn nicht zu allen bearbeiteten Aufgaben Tests angelegt wurden, lag dies lediglich an mangelnder Zeit, wenn sich eine Userstory als aufwendiger als erwartet herausstellte.

\subsection{Ergebnisse des Reviews}
{\small Autor: David Koch}

Anwesend: Alaa Aldin, Lennart, David, Erik, Julius J., Julius H., Jonas, Linus, Manuel, Tim\\

Im Rahmen des Reviews haben wir wie gewohnt die Ergebnisse des Sprint bewertet und Schwierigkeiten besprochen.\\

\textbf{Problem: Warten auf das ITSZ der HTWK}\\
Die derzeit wichtigsten ausstehenden Aufgaben betreffen vor allem die Themen Server und Datenbank. Da die Daten am Ende auf einem Server der HTWK und die Datenbank eine Anbindung zu Shiboleth haben soll, können wir dies ohne die bisher fehlenden Antworten des ITSZ bisher nur eingeschränkt bearbeiten. \\

\textbf{Design:}
\begin{itemize}
\item neues Design wurde eingebunden
\item einige (wenige) Teile sund noch nicht funktional 
\end{itemize}

\textbf{Optimierung des Graphen:}
\begin{itemize}
\item Gesamtkosten- und zeit werden im neuen UI ausgegeben, alternative Pfade allerdings noch nicht
\item automatische Auswahl von Start- und Endknoten wurde fertiggestellt, jedoch noch nicht mit eingefügt
\item Losgröße wurde noch nicht eingebaut
\end{itemize}

\textbf{Erstellung von Bearbeitungsschritten durch klciken:}
\begin{itemize}
\item Validierung der Kanten von Erstellen auf Bearbeiten umgelegt
\item vor der Optimierung wird überprüft, ob alle Eigenschaften gegeben sind
\end{itemize}

\textbf{Sammelticket für Feedback vom Kunden:}
\begin{itemize}
\item Tests für Toolbar, Zoom-Controls, Buttons und Eingabereihenfolgen geschrieben
\end{itemize}

\textbf{Julius H, Erik, Linus:}
\begin{itemize}
\item Knoten und Kanten heißen jetzt Teil und Bearbeitungsschritt
\item Automatisches Verschieben bei überlappenden Knoten wurde überarbeitet
\end{itemize}

\textbf{Login:}
\begin{itemize}
\item Login-Persistenz ist eingebaut
\item Ablaufender Zeitstempel ist eingebaut, wird aber bisher bei zu wenigen Aktionen refreshed
\item zum Testen benötigte Buttons wurden entfernt
\item Graph geht beim reload nicht mehr verloren
\item Login-Credentials können gespeichert werden
\item erste Erfahrung sammeln im Umgang mit Shiboleth
\item Anbindung an Shiboleth noch nicht möglich (fehlende Antwort des ITSZ)
\end{itemize}

\textbf{Backend Datenbank:}
\begin{itemize}
\item Docker Datenbank aufgesetzt und angebunden (noch nicht auf HTWK-Server)
\item Grundlegende Struktur für API-Entwicklung implementiert
\item Testdaten in Datenbank eingespei, erste SQL-Befehle auf der Datenbank getestet
\item Dokumentation vorhanden
\end{itemize}

\subsection{Ergebnisse der Retrospektive}
{\small Autor: David Koch}

Anwesend: Alaa Aldin, Lennart, David, Erik, Julius J., Julius H., Jonas, Linus, Manuel, Tim\\

Nach den zwischendurch schlechten Einschätzungen bezüglich des Erfolgs dieses Sprints gab es doch eine positive Überraschung, wie viel der Userstories schon umgesetzt werden konnte. Die Retrospektive wird wie im letzten Sprint nach dem KALM-Prinzip durchgeführt.


\textbf{Keep:}
\begin{itemize}
\item Produktivität/Motivation
\item Zusammenarbeit 
\item Zeiten buchen
\item Übersicht im YouTrack
\item Absprache beim Mergen
\item kommentierter Code
\end{itemize}

\textbf{Add:}
\begin{itemize}
\item Fortschritte mitteilen
\item Dark Mode
\item Antworten des ITSZ
\end{itemize}

\textbf{Less:}
\begin{itemize}
\item keine Anmerkungen
\end{itemize}

\textbf{More:}
\begin{itemize}
\item Tests
\item Fehlerhafte Tests fixen
\item Bugfixing
\item Branches löschen
\end{itemize}


\subsection{Abschließende Einschätzung des Product-Owners}
{\small Autor: Manuel Eckert}

Kurz vor Beginn des Sprints konnten wir nochmals Feedback vom Kunden erhalten. Daraufhin haben sich noch einige Änderungen für diesen Sprint ergeben wie Änderungen für die Optimierung, das Backend und die Erstellung von einzelnen Knoten-/Kantenelementen. Zudem wurde ein moderneres und ansprechenderes UI-Design implementiert. \\
Zum Ende des Sprints wurde ein passables Ergebniss erreicht und fast alle Aufgaben wurden zu einer guten Zufriedernheit fertig gestellt. 
Leider gibt es immer noch in einigen Modulen nicht genügend Tests um die geforderten DoDs zu erfüllen. \\
Das ITSZ ist für uns immer noch nicht telefonisch erreichbar und hat auf die von uns gesendeten E-Mauls nicht geantwortet. 

\subsection{Abschließende Einschätzung des Software-Architekten}
{\small Autor: xxx}

XXX

\subsection{Abschließende Einschätzung des Team-Managers}
{\small Autor: Alex Hofmann}

Der Sprint schien nach Stand des letzten Scrum Meetings eher negativ auszufallen, jedoch wurden einige User-Stories durch großes Engagement des Teams dennoch in der verbleibenden Zeit finalisiert.


