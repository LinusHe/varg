
\subsection{Ziel des Sprints}
{\small Autor: Lennart Buchmann}

Nach der Einarbeitung des gesamten Teams in die Grundstruktur der Software, sowie der Frameworks, lag das Hauptaugenmerk des 
dritten Sprints in der verstärkten Herausarbeitung der geplanten Kernfunktionalitäten der Anwendung. Größere Aufgabenbereiche wurden 
durch Zweier- und Dreierteams gelöst. Übriggebliebenes aus den vorherigen Sprints sollte beendet werden 

\subsection{User-Stories des Sprint-Backlogs}
{\small Autor: Lennart Buchmann}

\begin{itemize}

  \item \textbf{Funktionalität der Datenbank}
        \\\textit{Als Nutzer will ich meine gespeicherten Graphen ansehen können, um diese weiter bearbeiten zu können.}
\item \textbf{Kontext Menu über rechte Maustaste}
        \\\textit{ Als Nutzer möchte ich Knoten und Kanten über einen Rechtsklick zur einfacheren Benutzung erstellen können.}
  \item \textbf{Authentifizierung eines Nutzers}
        \\\textit{Als Nutzer möchte ich mich in die Web Applikation einloggen können,
        damit nicht jeder meine erzeugten Graphen einsehen kann.}
  \item \textbf{Darstellung von Knoten und Kanteneigenschaften am Objekt}
        \\\textit{Als Benutzer möchte ich über einen Rechtsklick auf einen Knoten/Kante die Eigenschaften dieser bearbeiten können.}
  \item \textbf{Optimierung des Graphs}
        \\\textit{Als Benutzer möchte ich gerne sofort sehen können, wie hoch meine Kosten für den kürzesten Pfad sind, damit ich mich möglichst schnell für einen entscheiden kann.}
\item \textbf{Speicherung Graph}
        \\\textit{Als Nutzer möchte ich einen Graphen jederzeit bearbeiten und speichern können, auch wenn dieser noch unfertig ist.}

\end{itemize}


\subsection{Liste der durchgeführten Meetings}
{\small Autor: Lennart Buchmann}

\begin{itemize}
  \item 06.01.2020: Planning Meeting
  \item 09.01.2020: Weekly Scrum
  \item 13.01.2020: Weekly Scrum
  \item 16.01.2020: Weekly Scrum
  \item 20.01.2020: Review \&  Retrospektive Meeting
\end{itemize}


\subsection{Ergebnisse des Planning-Meetings}
{\small Autor: Lennart Buchmann}

Der 3. Sprint ist der letzte Sprint im laufenden Semester und der letzte Sprint vor den anstehenden Prüfungen. Während des Planning-Meetings wurde von allen einheitlich besprochen, dass die
Arbeitslast von jedem höher ist als während der vergangen Sprints. Es wurde sich daraufhin geeinigt lieber realistische Ziele zu setzen, sodass der 3. Sprint auch mit höhere Belastung erfolgreich 
abgeschlossen werden kann. Nach Besprechung und Schätzung der Tickets, wurden alle Aufgaben in kleinere Gruppen aufgeteilt. Größere Aufgaben, die nach Schätzung im aktuellen Sprint nicht umsetzbar wären, wurden auf den verlängerten 4. Sprint verschoben. 


\subsection{Aufgewendete Arbeitszeit pro Person$+$Arbeitspaket}
{\small Autor: Lennart Buchmann}

\begin{longtable}{|p{4cm}|p{2cm}|p{1.2cm}|p{1.2cm}|p{0.7cm}|p{3.8cm}|}
  \hline
  Arbeitspaket                                                          & Person                & Start    & Ende     & h     & Artefakt                                                    \\ \hline
  Login                                                             & Berger, Matthias      & 13.01.20 & 17.01.20 & 18   & Login Funktionalität \& Design                              \\ \hline
  Login                                                             & Buchmann, Lennart     & 18.01.20 & 18.01.20 & 6     & Login Funktionalität \& Design                              \\ \hline
  Knotendarstellung nach Designvorlage        & Gwozdz, Jonas         & 15.01.20 & 20.01.20 & 4,5     & GraphHeader.vue, Toolbar.vue                                \\ \hline
  Speicherung Graph			        &  Heldt, Erik           & 06.01.20 & 19.01.20 & 18  & BasicData.js                                                \\ \hline
  Optimierung des Graphs 			        & Henning, Tim          & 09.01.20 & 18.01.20 & 9     & OptimizeControls.vue, index.js -> Graph Highlighting        \\ \hline
  Speicherung Graph                                      & Herterich, Linus      & 07.01.20 & 19.01.20 & 24,25     & meeting\_19\_12\_19.pdf                                     \\ \hline
  Graph zu Datenstruktur hinzufügen             & Hohlfeld, Julius      & 07.01.20 & 20.01.20 & 19     & BasicData.js, TestDatabase.js                               \\ \hline--
  Funktionalität Neuer Graph Button               & Karkoutli, Alaa Aldin & 12.01.20 & 15.01.20 & 7  & Database.vue (view), DatabaseForm.vue (component)           \\ \hline
  Kanten zu Graph hinzufügen                         & Koch, David           & 17.01.20 & 19.01.20 & 10    & Änderungen an index.js, CreateControls.vue (component)      \\ \hline
\end{longtable}

\subsection{Konkrete Code-Qualität im Sprint}
{\small Autor: Lennart Buchmann}



\subsection{Konkrete Test-Überdeckung im Sprint}
{\small Autor: Lennart Buchmann}

Eine konkrete Auseinandersetzung mit Tests beziehungsweise entsprechenden Test-Frameworks fand während des 2. Sprints statt. Momentan befinden sich alle Teammitglieder noch in der Einarbeitungsphase. Aufgrund des fortgeschrittenes Semesters und der anstehenden Prüfungen lagen die Prioritäten vorwiegend auf der Bearbeitung der User-Stories. 


\subsection{Ergebnisse des Reviews}
{\small Autor: Lennart Buchmann}

Das Ergebnis der Reviews war in anbetracht der fortgeschrittenen Semesters durchgehenden positiv. Alle Teammitglieder haben die Ihnen zugewiesenen Aufgaben innerhalb des Sprints erledigt. 
Es wurde des Weiteren besprochen, dass der verlängerte Sprint während der Semesterferien dazu genutzt werden sollte, um Bugs zu beheben und somit jedem die Gelegenheit zu geben, sich in die Testframeworks einzuarbeiten und Tests für den geschriebenen Code zu verfassen.


\subsection{Ergebnisse der Retrospektive}
{\small Autor:  Lennart Buchmann}

Während der Retrospektive wurde von allen die grundsätzliche gute Kommunikation innerhalb des Teams gelobt. Alle empfanden auch die Aufteilung in kleinere Zweier- und Dreierteams zur Bearbeitung von Aufgaben für sehr hilfreich.  Eine gleichbleibende hohe Motivation und Produktivität soll auch während des Semesterferiensprints beibehalten werden. Punkte, welche verbessert werden sollten, sind das pünktliche Mergen der einzelnen Branches vor Ende des Sprints, das Kommentieren des Codes und das Verfassen von Tests. 


\subsection{Abschließende Einschätzung des Product-Owners}
{\small Autor: xxx}

XXX

\subsection{Abschließende Einschätzung des Software-Architekten}
{\small Autor: xxx}

XXX

\subsection{Abschließende Einschätzung des Team-Managers}
{\small Autor: Alex Hofmann}

Weiterhin aufstrebende Arbeit vom Team. Auch die Kommunikation bei Problemen, Fragen und Anregungen geht in eine positive Richtung.

