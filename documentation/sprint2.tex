
\subsection{Ziel des Sprints}
{\small Autor: Linus Herterich}

Nachdem im ersten Sprint hauptsächlich die Grundstruktur sowie erste Datenstrukturen entworfen wurden,
war es nun wichtig, dass sich das gesamte Team im Sprint 2 mit der Projektstruktur (besonders mit dem Framework Vue)
auseinandersetzt und erste UserStories direkt am Code umsetzt. Zudem blieben einige Tickets noch vom letzten Sprint offen,
welche nun auch bearbeitet werden sollten.

\subsection{User-Stories des Sprint-Backlogs}
{\small Autor: Linus Herterich}

\begin{itemize}
  \item \textbf{Designumsetzung nach Adobe Preview}
        \\\textit{
          Als Benutzer der WebApplikation möchte ich ein ansehnliche und intuitive
          Oberflächengesstaltung haben, damit ich die Applikation gerne verwende.}
  \item \textbf{Authentifizierung eines Nutzers}
        \\\textit{
          Als Nutzer möchte ich mich in die Web Applikation einloggen können,
          damit nicht jeder meine erzeugten Graphen einsehen kann.}
  \item \textbf{Logische verknüpfung zwischen Knoten erstellen}
        \\ (wurde in Sprint 1 nicht abgeschlossen)
        \\\textit{
          Ein Nutzer muss eine Abfolge der Knoten definieren können,
          damit ersichtlich wird welcher Produktionsschritt auf den nächsten folgt}
  \item \textbf{Berechnung der Eingenschaften des Gesamtgraphs}
        \\ (wurde in Sprint 1 nicht abgeschlossen)
        \\\textit{
          Ein Nutzer der Webanwendung VarG muss die berechneten gesamt Eigenschaften
          jedes Zusammenhängendes Pfades ausgeben lassen können um eine Auswahl
          eines Pfades zu treffen.}
  \item \textbf{Datenstruktur Ausarbeiten \& Knoten zu einer vorhandenen Datenstruktur hinzufügen}
        \\ (wurde in Sprint 1 nicht abgeschlossen)
        \\\textit{
          Als Nutzer möchte ich Knoten zu der Datenstruktur hinzufügen können
          um die möglichen Produktionsschritte des Werkstücks überblicken zu können}

\end{itemize}

\subsection{Liste der durchgeführten Meetings}
{\small Autor: Linus Herterich}

\begin{itemize}
  \item 19.12.2019: Planning Meeting
  \item 23.12.2019: Daily Meeting (in Discord)
  \item 28.12.2019: Daily Meeting (in Discord)
  \item 05.01.2020: Review Meeting
  \item 06.01.2020: Retrospektive
\end{itemize}

\subsection{Ergebnisse des Planning-Meetings}
{\small Autor: Linus Herterich}

Neben der Aufgabenverteilung wurde im Planning darüber gesprochen, dass die Arbeitsaufteilung im letzten
Sprint nicht gut geklappt hat. Es wurde anschließend beschlossen im nächsten Sprint die User-Stories direkt
an Studenten zuzuweisen, damit jeder einen Teilbereich hat, den er bearbeiten muss.
\\ Desweiteren wurde eine Änderung im Git angekündigt. In Zukunft müsse der "Master"\--Branch während eines Sprints
immer gleich bleiben und Funktionalitäten werden auf einen "Developer"\--Branch gemerged. Am Ende des Sprints
wird dann der "Developer"\--Branch auf den "Master"\--Branch gemerged. wichtig ist, dass der "Master"\--Branch zu jedem
Zeitpunkt lauffähig ist.
\\ Für den folgenden Sprint wurde beschlossen, die Daily Meetings online (auf einem Discord Server) abzuhalten,
da viele Studenten über die Weihnachtsferien in der Heimat sind und somit ein persönliches wöchentliches treffen
nicht möglich wäre.

\subsection{Aufgewendete Arbeitszeit pro Person$+$Arbeitspaket}
{\small Autor: Linus Herterich}

\begin{longtable}{|p{4cm}|p{2cm}|p{1.2cm}|p{1.2cm}|p{0.7cm}|p{3.8cm}|}
  \hline
  Arbeitspaket                                                          & Person                & Start    & Ende     & h     & Artefakt                                                    \\
  \hline
  UI: Login                                                             & Berger, Matthias      & 22.12.19 & 22.12.19 & 3,5   & Login Funktionalität \& Design                              \\ \hline
  UI: Login                                                             & Buchmann, Lennart     & 22.12.19 & 22.12.19 & 6     & Login Funktionalität \& Design                              \\ \hline
  UI: Grapheneditor                                                     & Gwozdz, Jonas         & 23.12.19 & 04.01.20 & 9     & GraphHeader.vue, Toolbar.vue                                \\ \hline
  Task: Einbindung in Vue-Dateistruktur                                 & Heldt, Erik           & 19.12.19 & 19.12.19 & 0,25  & BasicData.js                                                \\ \hline
  Abrufbaren Knoten in Graph einfügen                                   & Heldt, Erik           & 23.12.19 & 26.12.19 & 3,5   & BasicData.js, TestDatabase.js                               \\ \hline
  Testdatenbank mit Speichern und Laden                                 & Heldt, Erik           & 27.12.19 & 27.12.19 & 3,5   & TestDatabase.js                                             \\ \hline
  Highlighting eines kürzesten Pfades nach Anwendung des A* Algorithmus & Henning, Tim          & 24.12.19 & 03.01.20 & 9     & OptimizeControls.vue, index.js -> Graph Highlighting        \\ \hline
  Protokoll: Meeting 19.12.19                                           & Herterich, Linus      & 19.12.19 & 19.12.19 & 1     & meeting\_19\_12\_19.pdf                                     \\ \hline
  UI: Login                                                             & Herterich, Linus      & 20.12.19 & 20.12.19 & 5     & LoginForm.vue, Login.vue                                    \\ \hline
  UI: Home                                                              & Herterich, Linus      & 23.12.19 & 23.12.19 & 7     & HomeMenu.vue (component), Home.vue (view), Menu.vue (view)  \\ \hline
  UI: Neuer Graph                                                       & Herterich, Linus      & 28.12.19 & 28.12.19 & 1,5   & NewGraph.vue (view), NewGraph.vue (component)               \\ \hline
  UI: Grapheneditor                                                     & Herterich, Linus      & 02.01.20 & 04.01.20 & 11,75 & Graph.vue (view), zahlreiche components                     \\ \hline
  Graph zu Datenstruktur hinzufügen                                     & Hohlfeld, Julius      & 21.12.19 & 23.12.19 & 4     & BasicData.js, TestDatabase.js                               \\ \hline
  Testdatenbank mit Speichern und Laden                                 & Hohlfeld, Julius      & 27.12.19 & 03.01.20 & 8     & BasicData.js, TestDatabase.js, index.js, JSonPersistence.js \\ \hline
  Mergen und Anpassen                                                   & Hohlfeld, Julius      & 04.01.20 & 04.01.20 & 2     & Bugs entfernt \& Mergekonflikte behoben                     \\ \hline
  UI: Datenbank-Import Fenster                                          & Karkoutli, Alaa Aldin & 31.01.20 & 04.01.20 & 12,5  & Database.vue (view), DatabaseForm.vue (component)           \\ \hline
  Kanten zu Graph hinzufügen                                            & Koch, David           & 23.12.20 & 04.01.20 & 10    & Änderungen an index.js, CreateControls.vue (component)      \\ \hline
\end{longtable}

\subsection{Konkrete Code-Qualität im Sprint}
{\small Autor: Linus Herterich}

Es wurde sich größtenteils an die Coding-Guidelines gehalten. An wichtigen Stellen sowie vor jeder Funktion wurden Kommentare
geschrieben. Die Trennung zwischen Views und Components sowie die Auslagerung der Style-Dateien wurde ebenfalls eingehalten.

\subsection{Konkrete Test-Überdeckung im Sprint}
{\small Autor: Linus Herterich}

Ein Student wurde beauftragt bis zum Ende des Sprints ein geeignetes Test-Framework zu finden.
Somit wurden während des Sprints noch keine Tests geschrieben.

\subsection{Ergebnisse des Reviews}
{\small Autor: Linus Herterich}

Es wurden fast alle UserStories umgesetzt. Somit war der zweite Sprint erfolgreich.
Alle Studenten konnten sich in das Projekt einarbeiten und haben die Strukturierung
größtenteils verstanden und eingehalten.
\\ Das User-Interface wurde nach der Designvorlage umgesetzt und die ersten Graphen-Funktionen
(Hinzufügen von Knoten und Kanten \& Optimieren) funktionieren bereits.
\\ Da noch nicht feststeht, wo die Software gehostet werden soll und wie die Datenbank-Funktionalität
umgesetzt werden soll, wurde zunächst eine lokale Speicherlösung als "Datenbank" verwendet. Somit konnten
die Speichern- und Laden-Funktionen erfolgreich implementiert werden.
\\ Die Login-Funktionalität ist derzeit nur sporadisch eingerichtet und wird finalisiert,
sobald feststeht, wie die Authentifizierung der Nutzer erfolgen soll (Anbindung an HTWK Login?).
\\ Leider ist immernoch kein geeignetes Testframework gefunden worden, mit dem sich sowohl Vue.js
als auch cytoscape (Graphen-Funktionalitäten) testen lassen.

\subsection{Ergebnisse der Retrospektive}
{\small Autor: Linus Herterich}

Das Happiness-Barometer für diesen Sprint ist sehr gut ausgefallen. Das liegt hauptsächlich an der guten Aufgabenverteilung
sowie an den großen Erfolgen, die diesen Sprint erzielt wurden.
\\ Kritisiert wurde die die Kommunikation gegen Ende des Sprints. Das finale Mergen aller Branches war zu hektisch und unsicher.
\\ Es wurde sich darauf geeinigt in Zukunft zwei Dailies pro Woche abzuhalten und das letzte Meeting eines Sprints zum gemeinsamen Mergen zu verwenden.

\subsection{Abschließende Einschätzung des Product-Owners}
{\small Autor: Manuel Eckert}

Aus den bei dem Planning-Meeting vorgestellten User-Stories ergaben sich drei Subteams. Diese teilten sich in die Bereiche Login, UI-Design und Graph-Funktionalitäten auf. Damit wurde das konkretere Aufteilen der User-Stories auf Subteams umgesetzt. \\
Dies hatte einen positiven Einfluss auf die Anzahl der erfolgreich abgeschlossen Aufgaben. \\
Während des Reviews wurden fehlende Code Kommentare und eine zu niedrige Testabdeckung benängelt.


\subsection{Abschließende Einschätzung des Software-Architekten}
{\small Autor: Julius Jolig}

In diesem Sprint wurden bereits mehr Kommentare im Code verfasst, aber hier ist noch Luft nach oben. Die Bachelorstudenten haben sich gut mit Vue.js und cytoscape vertraut gemacht und gute Ergebnisse erzielt. Das Mergen lief trotz neuem Ansatz immer chaotisch ab.  

\subsection{Abschließende Einschätzung des Team-Managers}
{\small Autor: Alex Hofmann}

Deutliche Leistungssteigerung schon jetzt zu sehen. Aufteilung der User-Stories direkt nach dem Planning hat die Arbeitsstruktur und -ablauf während des Sprints auf jeden Fall positiv beeinflusst.

