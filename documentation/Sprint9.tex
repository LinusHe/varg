
\subsection{Ziel des Sprints}
{\small Autor: Matthias Berger , Korrektur: Tim Henning}

Das Hauptaugenmerk im 9. Sprint wurde auf die Behebung noch vorhandener Bugs, sowie die Projektdokumentation gelegt. Lediglich die Umsetzung der Loginfunktion im Backend wurde als größeres Ziel verfolgt.

\subsection{User-Stories des Sprint-Backlogs}
{\small Autor: Matthias Berger}

\begin{itemize}
\item \textbf{Ausarbeitung des Latex Dokuments, Code Kommentare, Technische Ausarbeitung}

\item \textbf{Persistenz der eingetragenen Daten }
      \\\textit{Wenn Kanten/Knoten Eigenschaften bearbeitet werden, automatische Speicherung (Wenn Fenster geschlossen wird, bzw aus dem Eigenschaftenfenster geklickt wird)
      \\Darstellung der Losgröße ohne Scrollbar (oder bei leerem Feld dort hin springen zweite Seite)}

\item \textbf{Umzug Datenbank auf Fachschaftsserver}
      \\\textit{Datenbank soll auf den Server des FSR umgezogen werden
      \\VarG von dort aus hosten lassen}
       
\item \textbf{Optimierung}
      \\\textit{Klarer machen, nach welchem Kriterium( Zeit/Kosten) Optimiert wurde ( Auf Oberfläche und in Einstellungen)}
      
\item \textbf{BUG TRACKER}

\item \textbf{Login}
      \\\textit{Überarbeitung Login (nicht in Klartext lesbar, ... )}
      
\item \textbf{Dokumentation}

\item \textbf{Abschlusspräsentation Messe vorbereiten }
\end{itemize}

\subsection{Liste der durchgeführten Meetings}
{\small Autor: Matthias Berger}
\begin{itemize}
  \item 08.06.2020: Planning Meeting
  \item 11.06.2020: Weekly Scrum
  \item 15.06.2020: Weekly Scrum
  \item 18.06.2020: Review \& Retro
\end{itemize}


\subsection{Ergebnisse des Planning-Meetings}
{\small Autor: Matthias Berger}

Anwesend: Jonas Gwozdz, Erik Heldt, Linus Herterich, Julius Hohlfeld, Lennart Buchmann, Tim Henning, Matthias Berger, Alaa Aldin Karkoutli, Manuel Eckert, Julius Jolig, Alex Hofmann\\

Zunächst wurde sich darauf geeinigt keine weiteren großen Projekte in Angriff zu nehmen. Stattdessen sollen ggf. Kommentare im Quelltext hinterlassen werden, die evtl. verfolgte Lösungsansätze dokumentieren sollen. Einzig großes Projekt war die Überarbeitung des Logins, sodass Nutzername und Passwort nicht weiter im Quelltext auslesbar sind. Ursprünglich sollte dies durch die Anbindung an Shibboleth ersetzt werden. Stattdessen musste nun innerhalb kürzester Zeit eine Verifikation im eigenen Backend umgesetzt werden. Die Hauptpriorität sollte auf der Dokumentation des Projektes, der Präsentation für die Messe und weiteren Bugfixes liegen.

\subsection{Aufgewendete Arbeitszeit pro Person$+$Arbeitspaket}
{\small Autor: Matthias Berger}

\begin{longtable}{|p{4cm}|p{2cm}|p{1.2cm}|p{1.2cm}|p{0.7cm}|p{3.8cm}|}
      \hline
      Arbeitspaket                                                           & Person                & Start    & Ende     & h    & Artefakt                                                  \\ \hline
      Testen der Applikation                                                 & Berger, Matthias      & 08.06.20 & 08.06.20 & 0.5  & Testen der Applikation                                    \\ \hline
      Überarbeitung Login                                                    & Berger, Matthias      & 16.06.20 & 18.06.20 & 6.75 & LoginForm.vue, store.js                                   \\ \hline
      Umzug Datenbank auf Fachschaftsserver                                  & Berger, Matthias      & 11.06.20 & 11.06.20 & 1    & Organisation und Kommunikation mit FSR                    \\ \hline
      Testen der Applikation                                                 & Buchmann, Lennart     & 08.06.20 & 08.06.20 & 0.5  & Testen der Applikation                                    \\ \hline
      Überarbeitung Login                                                    & Buchmann, Lennart     & 16.06.20 & 18.06.20 & 6.75 & LoginForm.vue, store.js                                   \\ \hline
      Umzug Datenbank auf Fachschaftsserver                                  & Buchmann, Lennart     & 11.06.20 & 11.06.20 & 1    & Organisation und Kommunikation mit FSR                    \\ \hline
      Testen der Applikation                                                 & Gwozdz, Jonas         & 16.06.20 & 18.06.20 & 1.75 & Testen der Applikation                                    \\ \hline
      I.2 Produktvision                                                      & Gwozdz, Jonas         & 15.06.20 & 15.06.20 & 0.5  & projektdokumentation.tex                                  \\ \hline
      II.2 Entscheidungen des Technologieworkshops                           & Gwozdz, Jonas         & 15.06.20 & 15.06.20 & 0.25 & projektdokumentation.tex                                  \\ \hline
      III.2 Coding Style                                                     & Gwozdz, Jonas         & 09.06.20 & 09.06.20 & 1.5  & projektdokumentation.tex                                  \\ \hline
      Dokumentation in Latex-Dokument einfügen                               & Gwozdz, Jonas         & 15.06.20 & 15.06.20 & 0.5  & projektdokumentation.tex                                  \\ \hline
      X.2 Installationsanleitung                                             & Heldt, Erik           & 12.06.20 & 12.06.20 & 0.75 & projektdokumentation.tex                                  \\ \hline
      DB Button in Home Menu nicht korrekt angebunden                        & Heldt, Erik           & 17.06.20 & 17.06.20 & 2.75 & HomeMenu.vue                                              \\ \hline
      Überarbeitung Login                                                    & Heldt, Erik           & 16.06.20 & 17.06.20 & 1    & store.js                                                  \\ \hline
      Konsistenz bei der Positionierung ähnlicher Komponent                  & Heldt, Erik           & 17.06.20 & 18.06.20 & 0.75 & LoginForm.vue                                             \\ \hline
      Peer- Review                                                           & Heldt, Erik           & 11.06.20 & 11.06.20 & 1    & projektdokumentation.tex                                  \\ \hline
      Umstellung des DB primary keys von fileName hash auf fileName+userName & Heldt, Erik           & 17.06.20 & 17.06.20 & 0.75 & DatabaseForm.vue, ExportDatabase.vue, dump.sql, api.js    \\ \hline
      Code-Kommentare für DB GUI                                             & Heldt, Erik           & 18.06.20 & 18.06.20 & 1    & ExportDatabase.vue                                        \\ \hline
      Testen der Applikation                                                 & Henning, Tim          & 18.06.20 & 18.06.20 & 0.5  & Testen der Applikation                                    \\ \hline
      Mehrere Startzustände                                                  & Henning, Tim          & 11.06.20 & 11.06.20 & 1    & SettingsOptimize.vue, optimizations.js,                   \\ \hline
      Automatische Umrechnung der Zeit bei Optimierung                       & Henning, Tim          & 11.06.20 & 11.06.20 & 1.5  & GraphInfo.vue                                             \\ \hline
      Peer- Review                                                           & Henning, Tim          & 18.06.20 & 18.06.20 & 0.5  & projektdokumentation.tex                                  \\ \hline
      Graph optimieren funktioniert nicht(alternative Pfade)                 & Henning, Tim          & 11.06.20 & 11.06.20 & 5    & GraphInfo.vue, SettingsOptimize.vue, optimizations.js     \\ \hline
      Zeitformat in Alternativen pfaden                                      & Henning, Tim          & 18.06.20 & 18.06.20 & 1    & SettingsOptimize.vue                                      \\ \hline
      Umstellung Persistenz der Menüs                                        & Herterich, Linus      & 13.06.20 & 13.06.20 & 2    & SettingsMenu.less, CreateControls.vue, DetailControls.vue \\ \hline
      Umstellung Zweiseitiges Layout für Kanten-Menüs                        & Herterich, Linus      & 09.06.20 & 09.06.20 & 3    & DetailControls.vue, CreateControls.vue                    \\ \hline
      III.3 Zu nutzende Werkzeuge                                            & Herterich, Linus      & 08.06.20 & 08.06.20 & 2    & projektdokumentation.tex                                  \\ \hline
      X.3 Software-Lizenz                                                    & Herterich, Linus      & 08.06.20 & 08.06.20 & 0.75 & projektdokumentation.tex                                  \\ \hline
      Frontend auf Server                                                    & Herterich, Linus      & 11.06.20 & 12.06.20 & 4    & Frontend auf Server                                       \\ \hline
      Anpassungen am Projekt für Livegang                                    & Herterich, Linus      & 11.06.20 & 11.06.20 & 0.25 & Anpassungen am Projekt für Livegang                       \\ \hline
      HTTPS Zertifikat installieren                                          & Herterich, Linus      & 16.06.20 & 16.06.20 & 0.5  & HTTPS Zertifikat installieren                             \\ \hline
      Docker Container auf HTWK-Server                                       & Hohlfeld, Julius      & 14.06.20 & 14.06.20 & 2    & Docker Container auf HTWK-Server                          \\ \hline
      Put-Request/ Update funktioniert nicht                                 & Hohlfeld, Julius      & 16.06.20 & 16.06.20 & 2.5  & api.js                                                    \\ \hline
      II.5 Liste der Architekturentscheidungen                               & Karkoutli, Alaa Aldin & 15.06.20 & 15.06.20 & 2    & projektdokumentation.tex                                  \\ \hline
      Aktualisierte Produktname und Stückzahl in Store speichern             & Karkoutli, Alaa Aldin & 14.06.20 & 14.06.20 & 0.5  & GraphInfo.vue, ExportDownload.vue, HomeMenu.vue, store.js \\ \hline
      Auf Graph Seite Highlighten, nach was optimiert wird                   & Koch, David           & 10.06.20 & 17.06.20 & 4    & GraphInfo.vue, SettingsOptimize.vue, optimizations.js     \\ \hline
      Buttons im Graphfenster differenzieren                                 & Koch, David           & 10.06.20 & 16.06.20 & 2    & GraphInfo.vue, SettingsOptimize.vue                       \\ \hline
\end{longtable}


\subsection{Konkrete Code-Qualität im Sprint}
{\small Autor: Matthias Berger}

Eines der Hauptziele im Sprint war die Kommentierung des Quelltextes zu überarbeiten. Es ist also davon auszugehen, dass sich die Qualität des Codes im laufe des Sprintes verbessert hat.

\subsection{Konkrete Test-Überdeckung im Sprint}
{\small Autor: Matthias Berger}

Da während des Sprints keine neuen Funktionalitäten implementiert wurden, sondern lediglich Bugfixing und Überarbeitung  im Vordergrund standen war es auch nicht nötig weitere Tests zu formulieren.

\subsection{Ergebnisse des Reviews}
{\small Autor: Matthias Berger}

Anwesend: Jonas Gwozdz, Erik Heldt, Linus Herterich, Julius Hohlfeld, Lennart Buchmann, Tim Henning, David Koch, Matthias Berger, Manuel Eckert, Julius Jolig, Alex Hofmann\\

Im Rahmen des Reviews wurde festgehalten, dass der Großteil der geplanten  Sprintziele umgesetzt werden konnte. Um letzte kleinere Ausbesserungen, den vollständigen Umzug auf den vom FSR bereitgestellten Server und die Vorbereitung einer Präsentation für die geplante Messe zu ermöglichen, wurde sich darauf geeinigt den Sprint zu verlängern. Neue Ziele wurden nicht formuliert.

\subsection{Ergebnisse der Retrospektive}
{\small Autor: Matthias Berger}

Anwesend: Jonas Gwozdz, Erik Heldt, Linus Herterich, Julius Hohlfeld, Lennart Buchmann, Tim Henning, David Koch, Matthias Berger, Manuel Eckert, Julius Jolig, Alex Hofmann\\

Da es sich um den letzten offiziellen Sprint handelte wurde in der Retrospektive auf das KALM-Schema verzichtet. Stattdessen wurde der Verlauf des gesamten Projektes bewertet. Die Meinungen hierzu fielen durchweg positiv aus. So wurde  positiv bewertet, dass sich im laufe des Projektes trotz der anfänglichen Schwierigkeiten in Organisation und Kommunikation zwischen den Teammitgliedern ein Teamgeist und ein Produkt, mit dem sich alle Beteiligten identifizieren können entwickelt hat. Auch die Kommunikation innerhalb und zwischen den Einzelteams und die Brücke, die zwischen den beiden Studiengängen der Teammitglieder entstanden ist, wurde positiv hervorgehoben.

\subsection{Abschließende Einschätzung des Product-Owners}
{\small Autor: Manuel Eckert}

In diesem Sprint konnten wir endlich das entwickelte Backend auf einen Server im Netz der HTWK spielen. Dies war noch der letzte Meilenstein zu einer erfolgreichen Auslieferung des Produktes. \\
Da dies dann auch der letzte Sprint des Softwareprojektes war, wurden nur noch kleinere User-Stories verteilt. Somit konnten sich die Entwickler auf diese Aufgaben konzentrieren und es wurde sichergestellt, dass keine Funktionalitäten Aufgrund von Zeitmangel verworfen werden mussten. \\
Das ein Projektabschluss trotz allem doch sehr schnell anstrengend werden kann haben wir alle gemerkt, da nicht immer alles ganz nach Plan läuft und in letzter Sekunde doch noch einmal Probleme auftauchen. \\
Nach dem Sprint war das gesamte Team sehr zufrieden, mit dem was wir trotzdem noch in diesem Sprint abgearbeitet haben. \\ 

\subsection{Abschließende Einschätzung des Software-Architekten}
{\small Autor: xxx}

XXX

\subsection{Abschließende Einschätzung des Team-Managers}
{\small Autor: xxx}

XXX