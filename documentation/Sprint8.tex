\subsection{Ziel des Sprints}
{\small Autor: Alaa Aldin Karkoutli}

Ziel des Sprints war es, die Software zu dokumentieren und die noch stehenden Bugs zu beheben.\\

\subsection{User-Stories des Sprint-Backlogs}
{\small Autor: Alaa Aldin Karkoutli}

\begin{itemize}
  \item \textbf{Bugs fixen}
        \\\textit{Als Benutzer möchte ich eine Software benutzen, in welcher keine unerwarteten Probleme auftauchen.}
  \item \textbf{ Optimierung }
        \\\textit{
        Als Nutzer möchte ich wissen, nach welchen Kriterien die Graphen optemiert sind .}
  \item \textbf{Dokumentation }
        \\\textit{
          Als Benutzer möchte mit einer Software arbeiten, die eine vollständige Dokumentation hat .}
  \item \textbf{Persistenz der eingetragenen Daten}
        \\\textit{
          Als User will ich Kanten/Knoten Eigenschaften automatisch speichern, wenn sie bearbeitet werden.}
  \item \textbf{Login}
        \\\textit{
          Als Nutzer möchte ich mehr Zugriff auf die Datenbank haben wenn ich Admin bin(Rollenmangment) und den Login optemiert sehen.}
  \item \textbf{Backend-Datenbank}
        \\\textit{
          Als Benutzer möchte ich Test-Graphen in der Datenbank finden und die Datenbank vor Gefahr schützen.}
\end{itemize}

\subsection{Liste der durchgeführten Meetings}
{\small Autor: Alaa Aldin Karkoutli}

\begin{itemize}
\item 25.05.2020: Planning
\item 28.05.2020: Weekly
\item 01.06.2020: Weekly
\item 04.06.2020: Review \& Retro
\end{itemize}

\subsection{Ergebnisse des Planning-Meetings}
{\small Autor: Alaa Aldin Karkoutli}

Anwesend: Alaa Aldin, David, Erik, Jonas, Julius H., Lennart, Linus, Matthias, Tim, Alex, Manuel, Julius J.\\
\\
Im Planning haben wir die wichtigsten Punkte festgesetzt, die in den letzten Sprints Priorität haben, und die Schwierigkeit von denen bewertet.\\


\textbf{Backend Datenbanken}\\
Die Datenbank soll auf einem Server aufgesetzt werden, wenn der Server bereitgestellt ist.
Die generierte Test-Daten können in die Datenbank eingespeist werden. Als Ziel wurde gesetzt die Bugs zu beheben und alles nebenbei zu dokumentieren. 
Diese Ziele wurden wie im letzten Sprint mit einer 6 bewertet.\\

\textbf{Login}\\
Weil es keine Antwort von IT-HTWK gab, konnte die Software an Shibboleth nicht verbunden werden.
Es galt als Ziel Bugs zu beheben, den Login zu überarbeiten und einen einfachen Zugriff auf die Datenbank vorzubereiten, in Absprache mit Backend-Team.
Diese Ziele wurden mit einer 7 bewertet.\\

\textbf{Optimierung}\\
Es ist wichtig anzuzeigen, nach welchen Kriterien der Graph Optimiert.
Dieses Ziel wurde mit 4 bewertet.\\

\textbf{Persistenz der eingetragen Daten}\\
Die Eigenschaften der Kanten/Knoten sollen automatisch gespeichert, wenn der Benutzer irgendeine Änderung auf diese Eigenschaften macht.
Die Losgröße soll auch ohne Scrollbar dargestellt werden.
Diese Ziele wurden mit 4 bewertet, aber nach hinten eingestellt, da Doku Priorität hat. \\

\textbf{Dokumentation}\\
Die Software soll ausführlich dokumentiert werden. Tests, Code-Kommentare und 
Technische Ausarbeitungen müssen noch gemacht werden. Die Ideen, die nicht gemacht werden konnten (Shibboleth, Datenbank..), müssen
auch in der Dokumentation stehen. 
Das Team stimmte ab und schätzte die Schwierigkeit auf 7.\\

\textbf{Bugs fixen}\\
Möglichst alle Bugs zu beheben ist seit den letzten Sprints in Bearbeitung. Das Ziel ist es,
eine fehlerfreie Software abzugeben, indem alles ausprobieren und die Bugs lösen.

\subsection{Aufgewendete Arbeitszeit pro Person$+$Arbeitspaket}
{\small Autor: Alaa Aldin Karkoutli}

\begin{longtable}{|p{4cm}|p{2cm}|p{1.2cm}|p{1.2cm}|p{0.7cm}|p{6.3cm}|}
  \hline
  Arbeitspaket                                                          & Person                & Start    & Ende     & h     & Artefakt                                                    \\
  \hline
  Überarbeitung Login                                                   & Berger, Matthias      & 31.05.20 & 31.05.20 & 4     & LoginForm.vue $+$ store/store.js                                              \\ \hline
  

  Überarbeitung Login                                                   & Buchmann, Lennart     & 31.05.20 & 31.05.20 & 4     & LoginForm.vue $+$ store/store.js                                              \\ \hline
  Überarbeitung Login                                                   & Buchmann, Lennart     & 04.06.20 & 04.06.20 & 1     & LoginForm.vue $+$ store/store.js                                             \\ \hline


  Bug: Fehlerhafte Verschiebung bei Manchen Knotenkonstellationen       & Gwozdz, Jonas         & 27.05.20 & 27.05.20 & 2,15   & position.js $+$ src/vargraph/graph/edges.js                                            \\ \hline
  Bug: Fehlerhafte Verschiebung bei Manchen Knotenkonstellationen       & Gwozdz, Jonas         & 02.06.20 & 02.06.20 & 2,30   & position.js $+$ src/vargraph/graph/nodes.js                                      \\ \hline
  Verschieben der Knoten beim anlegen                                   & Gwozdz, Jonas         & 02.06.20 & 02.06.20 & 0,30   & src/vargraph/graph/nodes.js                                                \\ \hline
  Test für Darkmode                                                     & Gwozdz, Jonas         & 03.06.20 & 03.06.20 & 1,40   & darkmode\_spec.js                                               \\ \hline
  I.1 Initiale Kundenvorgaben                                           & Gwozdz, Jonas         & 03.06.20 & 03.06.20 & 0,40   & Dokumentation                                            \\ \hline
  III.2 Coding Style                                                    & Gwozdz, Jonas         & 03.06.20 & 03.06.20 & 1,10   & Dokumentation                                            \\ \hline
  
  
  Dokumentation                                                         & Heldt, Erik           & 28.05.20 & 28.05.20 & 3,45  & documentation/projektdokumentation.tex                                                \\ \hline
  Anzeigen des Graphen als Bild in der GUI                              & Heldt, Erik           & 01.06.20 & 03.06.20 & 3,30  & DatabaseForm.vue                                              \\ \hline
  Ersetzen der TestDatabase durch MySQL DB                              & Heldt, Erik           & 03.06.20 & 03.06.20 & 0,30  & removedcode/ExportDatabase                                              \\ \hline
  Überarbeitung Login                                                   & Heldt, Erik           & 04.06.20 & 04.06.20 & 1     & verschiedene Stellen                                            \\ \hline
  DB Button in Home Menu nicht korrekt angebunden                       & Heldt, Erik           & 03.06.20 & 03.06.20 & 1     & NewGraph.vue                                            \\ \hline
  Testen der Applikation                                                & Heldt, Erik           & 27.05.20 & 27.05.20 & 0,30  &                                                    \\ \hline
  Konsistenz bei der Positionierung ähnlicher Komponenten               & Heldt, Erik           & 01.06.20 & 03.06.20 & 0,50  & verschiedene Stellen                                      \\ \hline

  III.1 Definition of Done                                              & Henning, Tim          & 03.06.20 & 03.06.20 & 1     & projektdokumentation.tex                                             \\ \hline
  Bug: Start und Endzustandsanzeige                                     & Henning, Tim          & 29.05.20 & 29.05.20 & 2     & vargraph/graph/optimizations.js                                                   \\ \hline
  Selection-Felder bei Optimierungs-Einstellungen                       & Henning, Tim          & 31.05.20 & 31.05.20 & 4     & GraphInfo.vue $+$ vargraph/graph/optimizations.js                                          \\ \hline

  II.3 Überblick über Architektur                                       & Herterich, Linus      & 03.06.20 & 04.06.20 & 1,45  & projektdokumentation.tex                                             \\ \hline
  Testen der Applikation                                                & Herterich, Linus      & 03.06.20 & 03.06.20 & 0,5   &                                           \\ \hline
  Drag-n-Drop Button verschwindet nach drüberhovern nicht               & Herterich, Linus      & 28.05.20 & 03.06.20 & 2,10  & verschiedene Stellen                                          \\ \hline
  
  Bug: Falsche Anzeige der Gesamtkosten und Zeit nach Importieren       & Hohlfeld, Julius      & 25.05.20 & 28.05.20 & 1     & vargraph/JSonPersistence.js                                                    \\ \hline         
  Datenbank GUI wechselt nicht automatisch die Seite                    & Hohlfeld, Julius      & 27.05.20 & 27.05.20 & 1     & verschiedene Stellen                                          \\ \hline
  Schutz vor SQL-Injections                                             & Hohlfeld, Julius      & 03.06.20 & 03.06.20 & 1,30  & api.js                                                      \\ \hline
  II.4 Definierte Schnittstellen                                        & Hohlfeld, Julius      & 03.06.20 & 04.06.20 & 3,30  & projektdokumentation.tex                                                      \\ \hline
  Backend Test Recherche                                                & Hohlfeld, Julius      & 03.06.20 & 03.06.20 & 0,5   & Recherche                                                    \\ \hline

  Bugs: Persistenz des Graphen                                           & Karkoutli, Alaa Aldin & 02.06.20 & 02.06.20 & 3     & router/index.js                                              \\ \hline
  Sprint-Doku                                                           & Karkoutli, Alaa Aldin & 01.06.20 & 05.06.20 & 4     & documentation/Sprint\_8.tex                                            \\ \hline
  Graph wird nicht heruntergeladen,"Neuer Graph"->"Speichern"           & Karkoutli, Alaa Aldin & 01.06.20 & 01.06.20 & 1     & ExportDownload.vue $+$ HomeMenu.vue                                          \\ \hline
  Überarbeitung Login                                                   & Karkoutli, Alaa Aldin & 31.05.20 & 31.05.20 & 4     & LoginForm.vue $+$ store/store.js                                          \\ \hline
  
  
  Selection-Felder bei Optimierungs-Einstellungen                       & Koch, David           & 31.05.20 & 31.05.20 & 2     & GraphInfo.vue $+$ vargraph/graph/optimizations.js                                                   \\ \hline
  X.1 Handbuch                                                          & Koch, David           & 04.06.20 & 04.06.20 & 2     & projektdokumentation.tex                                         
\\ \hline
\end{longtable}

\subsection{Konkrete Code-Qualität im Sprint}
{\small Autor: Alaa Aldin Karkoutli}

Die Codequatlität war weiterhin gut.

\subsection{Ergebnisse des Reviews}
{\small Autor: Alaa Aldin Karkoutli}
        
Anwesend: Alaa Aldin, Erik, Jonas, Julius H., Lennart, David, Manuel, Alex, Julius J., Matthias, Linus, Tim \\

Im Review stellte jeder Anwesende seine Arbeit vor.\\
Die Ergebnisse und Erkenntnisse wurden ausgewertet.\\


\textbf{David:}
\begin{itemize}
\item Handbuch dokumentiert
\item Optimierung nicht viel gemacht
\end{itemize}

\textbf{Erik:}
\begin{itemize}
\item Dokumentationen geschrieben
\item Datenbank Preview-Bilder anzeigen
\item Testdatabase.js entfernt
\end{itemize}

\textbf{Jonas:}
\begin{itemize}
\item Springen der Knoten gefixt
\item Bechreibung in der Kanten lesen
\item Tests für Darkmode
\item Dokumentation geschrieben
\end{itemize}

\textbf{Julius H.:}
\begin{itemize}
\item Dokumentation geschrieben.
\item Sicher vor SQL-Injection
\item Bugs (Importiern und Datenbank) gefixet
\end{itemize}

\textbf{Linus:}
\begin{itemize}
\item Bug gefixt (Darg und Drop Knoten)
\item Dokumentation geschrieben
\end{itemize}

\textbf{Alaa Aldin, Lennart und Matthias (Login-Team):}
\begin{itemize}
\item Login-Überarbeitung vorbereitet (Code auskommentiert)
\item Rollenmanagement 
\end{itemize}

\textbf{Alaa Aldin:}
\begin{itemize}
\item Bugs(NeuerGraph -> Speichern, Persistenz der Graphen, Logout) gefixt
\item Knoten und Kanten in der gespeicherten Positionen laden
\item Änderungen auf importierten Graphen in der LocalStorage speichern
\end{itemize}

\textbf{Bugs}\\
Bugs sollen im Laufe des kommenden Sprints gefixt werden.\\

\subsection{Ergebnisse der Retrospektive}
{\small Autor: Alaa Aldin Karkoutli}

Anwesend: Alaa Aldin, Erik, Jonas, Julius H., Lennart, David, Manuel, Alex, Julius J. \\

Das Team war zufrieden mit der erreichten Leistung und hat eine positve Ansicht zum Sprint. Angesichts des baldigem Ende des Softwareprojekts wurde auch die Verwendung der restlichen Zeit kurz diskutiert.\\

\begin{center}
\begin{tabular}{ |c|c|c|c| }
\hline
 Keep & Add & Less & More\\
\hline 
-Teamwork & -Motivation & -unkompilierbare Doku & -Code Kommentare \\
-Bugfixing & -Sprint & & -Branches aufräumen \\
 & -funktionierender Login & & -Infos über Produktions-Version \\
 & -Console Errors entfernen & & \\
\hline     
\end{tabular}
\end{center}

\subsection{Abschließende Einschätzung des Product-Owners}
{\small Autor: Manuel Eckert}

Da dies der vorletzte Sprint war, wurde Wert darauf gelegt, möglichst keine große neuen Funktionalitäten zu entwickeln, sonder möglichst die bestehenden zu finalisieren. Weiterhin galt es Programmfehler die sich beim Entwickeln ergeben haben zu beseitigen. Als letzten großen Punkt wurde die Projektdokumentation in diesem Sprint bearbeitet. Da oft noch keine ausführliche Dokumentation zu den einzelnen Sprints geschrieben wurde, hat dies auch einen großen Teil des Sprints eingenommen. Trotzdem ist es uns gelungen alle User-Stories erfolgreich abzuschließen. \\
Da wir bis zu diesem Zeitpunkt vom ITSZ immer noch keine Nachricht bekommen haben, ob beziehungsweise wann wir einen Serverplatz bekommen könnten, haben wir uns nach anderen Möglichkeiten umgesehen. Glücklicherweise hat sich der Studierendenrat sofort bereit erklärt, uns zu helfen und ein Teil ihres Servers, welcher sich auch im HTWK-Netz befindet, bereitzustellen.

\subsection{Abschließende Einschätzung des Software-Architekten}
{\small Autor: Julius Jolig}

Das Projekt neigt sich dem Ende. Um dem Kunden das Projekt entsprechend übergeben zu können hat das Finden einer Lösung zum Hosting der Andwendung höchste Priorität. 

\subsection{Abschließende Einschätzung des Team-Managers}
{\small Autor: Alex Hofmann}

Es ist vereinzelt an der Motivation zu spüren, dass sich das Semester und somit auch das Softwareprojekt dem Ende zuneigt.
Dies liegt vermutlich auch daran, dass aufgrund von nur noch einem verbleibenden Sprint einige Komponenten, woran die Bachelor jetzt schon mehrere Wochen dran gearbeitet haben, auf der Strecke bleiben werden.
